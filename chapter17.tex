Глава 17
        ЧЕЛОВЕК-СМЕРЧ ПОЯВЛЯЕТСЯ СНОВА

    Питер простился с Лованой у ее комнаты и, выйдя из замка, поспешил 
к большому дереву, где его уже ждали Кривой Мик и Буньип. Серая Шкурка 
отправилась за Мунлайт, которая паслась у дороги примерно в миле 
отсюда.
    В тени дерева на спине лежал Буньип, сложив лапы на брюхе и 
размышляя, как бы выполнить второе задание.
    Питер сидел рядом.
    - Я придумал, как найти Фаерфакса, - сказал он. - Там в горах 
сотни долин, и он может пастись в любой. Наши поиски могут продлиться 
многие месяцы, но так ничего и не дать. Зато я знаю человека, который 
нам точно скажет, где он.
    - Это сильно облегчит нашу задачу, - отозвался Кривой Мик. - Как 
его звать?
    - Человек-смерч, Вилли-Вилли. Мы с Серой Шкуркой встретились с 
ним, когда шли по Пустыне Одиночества. У него есть подзорная труба, в 
которую он различает предметы за сотни миль.
    - А как мы с ним свяжемся? - спросил Кривой Мик.
    - Это непросто, - ответил Питер. - Действительно, как? Живет он 
далеко.
    - В Пустыне Одиночества?
    - Да.
    - Что ж, у нас хорошие лошади. Поскакали к нему. Это не займет 
много времени.
    - А как ты собираешься усмирить Фаерфакса, когда поймаешь? - 
спросил Буньип. - Я знаю двоих рыцарей, которые пытались его изловить, 
но у них ничего не вышло. Фаерфакс - очень сильный и норовистый конь.
    - Когда мы его поймаем, Питер его оседлает и на нем поедет: 
объезжать лошадей я его научил.
    - Я уверен, что смогу на нем проехать, - с некоторым сомнением 
произнес Питер. Однако уверенности в его голосе не было.
    - И я уверен, - сказал Кривой Мик. - В тебе есть мужество и 
отвага. Тебе, конечно, надо переодеться. Когда твоя голова начнет 
дергаться во все стороны, то перу на шляпе не удержаться, оно упадет, 
как тростинка. Когда станешь объезжать Фаерфакса, будь внимателен. 
Стоит немного расслабиться, и ты пропал. Я дам тебе одежду, в которой 
сам объезжаю лошадей - брюки и старую красную рубашку. Седло и уздечку 
можешь взять свои. Загони его. Крепко держи ноги в стременах и весь 
свой вес перенеси на ноги. Дай ему волю, но будь осторожен. Если он 
вздумает прыгнуть до небес, упреди его. Но не сиди как изваяние, 
расслабляйся. Больше непринужденности - и он твой. Не тереби его - 
пусть этим занимаются на родео. Помни, что в конце концов он 
выдохнется и захрипит, а отдохнув, начнет сначала. Лови этот момент, 
не давай ему отдышаться. Ты его усмиришь. Тебе нет равных.
    Кривой Мик похлопал Питера по спине.
    - А теперь в путь. Сперва нам придется хорошенько потрудиться, 
чтобы добраться до этой Пустыни Одиночества. А когда Вилли-Вилли 
подскажет, где этот Фаерфакс, мы быстро найдем его.
    - А на какой лошади ты поскачешь? - спросил Питер.
    - Да, на какой лошади я поскачу? - спросил Кривой Мик Буньипа.
    - Если пойти вдоль рва, то с другой стороны замка будет видна 
конюшня, - ответил Буньип. - Там стоят несколько лошадей, оставшихся 
от тех рыцарей и принцев, которых я убил когда-то. Возьми гнедого с 
белой звездочкой на лбу. По-моему, он неплох.
    Когда Кривой Мик вернулся, Серая Шкурка и Мунлайт уже ждали его. 
Мунлайт была под седлом, взнузданная и готовая к путешествию. Конь, 
выбранный Кривым Миком, уступал ей и в силе, и в изяществе. И выглядел 
совсем не так уверенно и гордо. Было непохоже, чтобы он рвался в бой, 
нетерпеливо прислушиваясь к зову трубы или шуму битвы.
    Мунлайт изменилась и стала более статной. Ее длинные узкие плечи 
говорили, что она не знает, что такое усталость. Круп был мускулист, и 
она скакала с такой легкостью, будто парила над землей. Каждый раз, 
когда Питер делал кого-то счастливее, давая Волшебный Лист, он не 
только сам менялся, превращаясь в благородного принца: менялась и 
Мунлайт, становясь достойной принца по силе и красоте.
    Кривой Мик подогнал стремена и вскочил в седло. Гнедой закрутился, 
и Мик пробормотал: "Конь-то совсем не выезжен". Потом он натянул 
поводья, и гнедой замер.
    Кривой Мик дал коню шенкеля, и тот рванулся вперед. "До встречи!" 
- крикнул он Буньипу. Питер последовал за ним, и две лошади бок о бок 
помчались по бушу, за которым находилась Пустыня Одиночества.
    - Мы идем к своей цели! - крикнула Серая Шкурка и запрыгала вслед. 
Она то поднималась, то опускалась, подобно морской волне, и вскоре 
догнала всадников; а когда те уже скрылись в зелени, головка Серой 
Шкурки то появлялась над кустами, то исчезала внизу.
    {whisp12.gif}
    Они скакали весь день и к вечеру достигли места, где росла высокая 
трава. Кривой Мик пришел в восторг, увидев это идеальное пастбище.
    - Здесь мы и остановимся, - сказал он. - За этим холмом будет 
озеро. Ты, Питер, напоишь лошадей, а мы с Серой Шкуркой устроим 
привал.
    Когда Питер привел лошадей обратно, в сложенном из камней очаге 
уже горел костер, и Кривой Мик жарил мясо. Питер расседлал лошадей и 
пустил их пастись, а потом сел к огню рядом с Серой Шкуркой и Миком.
    Утром они выехали еще до рассвета, чтобы успеть добраться до 
Пустыни Одиночества к обеду. По пути они преодолели несколько песчаных 
барханов, и вскоре вся пустыня раскинулась перед ними. Она показалась 
им еще более грустной и зловещей, чем когда бы то ни было.
    Они сдержали лошадей и остановились. Нигде на огромных безводных 
пространствах не было видно Вилли-Вилли.
    - Давайте пошлем ему сигнал дымом, как делают аборигены, - 
предложила Серая Шкурка. - Когда Вилли-Вилли увидит отдельные облака 
дыма, он обязательно примчится выяснить, в чем дело.
    Питер и Кривой Мик соскочили на землю и, собрав зеленых листьев 
эвкалипта, свалили их в кучу на открытом участке. Серая Шкурка 
отломила от наклонившегося эвкалипта большую ветку и заявила, что с ее 
помощью сделает целые облака дыма.
    Кривой Мик поджег листья, и из них повалил густой дым. Однако 
Серая Шкурка тут же закрыла костер большой веткой, чтобы дым 
накапливался. Время от времени она отдергивала ветку, и тогда дым 
вырывался наверх огромными клубами. Вскоре над пустыней уже плыла 
цепочка облаков.
    - Ну вот, теперь он обязательно появится, - сказала Серая Шкурка, 
останавливаясь, чтобы перевести дух.
    Она всматривалась вдаль, сощурив от солнца свои зоркие глазки.
    - Вот он! - воскликнула она наконец. - Ну и быстро же несется!
    - Теперь и я вижу, - подтвердил Питер и показал Кривому Мику, 
зрение которого уже немного испортилось, на далекий столб пыли.
    Приближаясь, смерч увеличивался в размерах. Оказавшись рядом, он 
замедлил скорость, потом остановился и опустился. Из оседающей пыли 
появились человек-смерч Вилли-Вилли и Том и зашагали к ним.
    - Я подумал, может, это Серая Шкурка посылает нам сигналы, - 
сказал Вилли-Вилли. - Как дела, Питер? С тех пор как мы встретились, я 
стал другим человеком. Том сделал мне капитальный ремонт. Никогда еще 
я так хорошо не работал. А благодаря тому листу, что ты мне подарил, 
моя жизнь стала намного интересней.
    Все уселись на земле и начали разговаривать. Том рассказал, как 
хорошо он живет с тех пор как встретил Вилли-Вилли, а Питер поведал о 
своих приключениях и о том, что в данный момент они ищут Фаерфакса.
    - Я верил, что если мы тебя встретим, ты нам поможешь его найти, - 
сказал Питер.
    - Ты обратился именно к тому, к кому нужно, - важно ответил Вилли-
Вилли. - Том, сходи туда, где мы остановились, и принеси подзорную 
трубу.
    Когда Том вернулся, Вилли-Вилли раздвинул трубу на полную длину и 
положил ее на плечо Питера.
    - Так, давай посмотрим. Только не дергайся. Что там такое? А, это 
часы на ратуше в Мельбурне. Показывают ровно двенадцать. Но это нам не 
надо. Развернемся к горам. Отлично. Немного приподними плечо. Хорошо. 
Так и держи. Вижу Фаерфакса. Он с табуном кобыл пасется в Долине 
Ключей. Я сейчас расскажу вам, как туда попасть.
    Он сложил подзорную трубу и указал на видневшийся у горизонта 
горный кряж.
    - Видите те две вершины? Одна чуть выше другой?
    - Вижу, - ответил Питер.
    - Долина Ключей лежит как раз между ними. Поезжайте в этом 
направлении, пока не выйдете на берег речки. Идите по ее течению, и вы 
придете в Долину Ключей. А теперь мне пора. Удачи вам всем. Заводи 
меня, Том.
    Том обмотал его за пояс шнуром и дернул. Вилли-Вилли завертелся 
мгновенно. Том впрыгнул во вращающийся круг и вместе с Вилли-Вилли 
понесся по пустыне.
    - Хороший он парень, - сказал Кривой Мик. Потом вскочил на коня и 
направил его в сторону гор. Питер и Серая Шкурка пустились следом.
    Только через два дня они добрались до речки, о которой говорил 
Вилли-Вилли, и пошли вдоль нее, пока она не привела в Долину Ключей.
    Кривой Мик развернул своего гнедого и направил его к небольшому 
холму: он надеялся оттуда обозреть всю долину. Поднимались они по 
дороге, очевидно, проложенной лошадьми. Около деревьев на самой 
вершине они остановились и окинули взглядом долину. Там росла густая 
трава и полевые цветы.
    Прямо под ними на широком поле пасся табун. Чуть поодаль они 
увидели Фаерфакса - могучего жеребца, рядом с которым другие кони 
выглядели просто карликовыми. В лучах заходящего солнца его рыжая 
шерсть отливала золотом. Даже издали бросался в глаза его дикий, 
необузданный нрав: копь рыл землю копытом и галопом носился вокруг 
своего табуна, оберегая его.
    Кривой Мик понял, что конь хотя и не видит их, но уже что-то 
заподозрил, и поэтому кивком приказал Питеру и Серой Шкурке следовать 
за ним - вниз по дороге. У ручья они нашли скалу, которая защищала их 
от ветра, и расположились на ночлег.
