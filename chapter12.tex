Глава 12
        СЕРАЯ ШКУРКА СРАЖАЕТСЯ С БУНЬИПОМ

    Они вскипятили котелок, и Буньип продолжил рассказ.
    "- Случилось так, что в тот день, когда я, изгнанный из школы, шел 
по дороге, Король выехал на охоту. Люди из его свиты окружили меня. 
Это были королевские придворные, они постоянно раскланивались и 
расшаркивались. Одеты они были богато - в длинные бархатные плащи и 
узкие в обтяжку брюки. Ехали все на прекрасных конях, но лучше всех 
был вороной жеребец короля, - он ни секунды не мог стоять спокойно. 
Когда король натягивал поводья, тот начинал гарцевать кругами.
    - Эй, малыш! - окликнул меня король, когда конь повернул головой в 
мою сторону. - Кто ты такой?
    Но конь уже встал ко мне боком, и мне пришлось ждать, прежде чем 
ответить.
    - Буньип, Ваше Величество.
    - Не слыхал о таких.
    Конь продолжал гарцевать и поворачивал короля то туда, то сюда, 
так что я снова ловил момент, когда он окажется ко мне лицом.
    - Мы живем в болоте возле дворца.
    - О боже! - воскликнул король. - Придется мне это место 
продезинфицировать.
    Тут жеребец снова отвернул короля в сторону. Ему это явно надоело, 
и он крикнул одному из придворных:
    - Остановите эту чертову лошадь!
    Человек, к которому он обращался, спрыгнул со своего коня и 
схватил под уздцы королевского жеребца. Было видно, что придворный 
здорово напуган, и не без оснований - король нагнулся и хлестнул его 
кнутом.
    - Как ты смел подавать королю такого коня? - набросился он на 
несчастного.
    - Но вы же сами просили гарцующего, - запинаясь, пробормотал 
придворный.
    - Гарцующего - да. Но я не просил, чтобы меня непрестанно кружили 
в вальсе.
    Король соскочил на землю и поправил корону, которая во время всей 
этой кутерьмы успела сползти набок.
    - Итак, - обратился он ко мне, - ты говорить, что ты Буньип. Я 
видел, как ты расправился с рыцарями. Я как раз находился у забора и 
видел твой оригинальный метод... Несомненно, ты мастерски выбросил всю 
эту банду учителей со Святым Георгием во главе с их коней.
    - Сбросил с коней, - поправил я его. Король изумленно уставился на 
меня.
    - Да ты интеллектуал, - презрительно бросил он. - Но ничего, эту 
дурь мы из тебя выбьем.
    Король мне не понравился. Мой отец говорил, что тот - чистейший 
деляга. Я не мог этого понять. Деньги для него значили все. И все же 
именно он был отцом Прекрасной Принцессы, которую держал в заточении в 
башне, и это знали все.
    - Послушай, малыш, - обратился ко мне король. - Хочешь получить 
работу? За сколько ты согласился бы потрудиться на меня?
    - А что я должен делать? - спросил я.
    - У меня есть дочь, Прекрасная Принцесса, - отвечал король. - Ее 
надо надежно сторожить от всяких рыцарей и принцев из других земель, 
которые придут просить ее руки. Она еще слишком молода для этого, и, 
естественно, я желаю, чтобы все ее поклонники тут же уничтожались, 
причем умело и эффективно. Если ты будешь набирать в себя побольше 
воды, я думаю, ты сможешь ликвидировать их вполне успешно. Вода, 
разумеется, бесплатно. Итак, - продолжал он, - какова наименьшая цена?
    - Десять долларов в неделю плюс содержание.
    Я знал, что всякий настоящий деляга обязательно сбавляет 
запрашиваемую сумму, и поэтому запросил больше, чем рассчитывал 
получить.
    - Хм, - задумался король. - Я сделаю вот что. Я дам тебе пять 
долларов в неделю плюс содержание.
    - Подходит, - сказал я. - Согласен.
    Я ответил мгновенно, потому что увидел, как он тут же пожалел, что 
не предложил мне четыре. Он был хорошим дельцом и без нужды деньгами 
не сорил.
    - Это очень солидная плата, - внушал он мне. - Надеюсь, ты не 
обжора.
    - Нет, я не обжора, - заверил я его.
    Когда доходит до дела, я тоже умею поторговаться. Как бы то ни 
было, я получил работу и несу здесь службу уже многие годы".
    - Ну как, интересная история? - добавил он.
    - О да, - согласилась Серая Шкурка. - А хотел бы ты услышать 
историю моей жизни?
    - Нет, - ответил Буньип, поднимаясь на лапы. - Я слишком занят. 
После того как я вас убью, мне надо будет организовать могильщиков, 
чтобы вас похоронили. Так что давайте не будем терять времени.
    Он осмотрел Питера и Серую Шкурку с головы до ног, потом перевел 
взгляд на Мунлайт, которая паслась неподалеку.
    - Лошадку я сохраню для Прекрасной Принцессы, - сказал Буньип. - 
Она сможет ездить на ней во время наших ежедневных прогулок вокруг 
замка. Да! Как вы предпочитаете быть убитыми - вместе или порознь? 
Гораздо более зрелищно, если вы вдвоем броситесь на меня с расстояния 
в сто ярдов. Тебе я дам копье, - обратился он к Питеру. - Только будь 
с ним осторожен. Оно принадлежало одному известному рыцарю, который 
умер в высшей степени благородно. Он сумел выбраться из доспехов, 
когда еще был жив, и потонул в одном белье. Эх, бедняга! Я потом не 
мог ужинать.
    - Вот что я тебе скажу, - ответила Серая Шкурка. - Разберись 
сначала со мной. Я нападу на тебя одна, такая, какая я есть. Мне не 
нужно ни копья, ни лошади. Я попробую только смеха ради.
    Питер остолбенел.
    - Ты же утонешь! - воскликнул он. - Не делай этого. - Он подбежал 
и горячо зашептал Серой Шкурке на ухо:
    - Я сейчас дам ему Волшебный Лист. Подожди, пока я не заведу 
разговор о подарках. Нам пока ничто не грозит.
    - Хочется его проучить, - шепотом ответила Серая Шкурка. - Он 
зазнался. Я видела, как он сбивал тех коров. Я запросто увернусь от 
его струй, вот увидишь.
    Одним прыжком она очутилась рядом с Буньипом.
    - Где мне встать?
    - Встань вон там на дороге, около большого дерева, и оттуда 
нападай на меня.
    - Согласна, - ответила Серая Шкурка.
    Она отошла туда, где росло большое дерево, и на несколько 
мгновений задержалась, плотно закрывая сумку, чтобы в нее не попала 
вода. Потом крикнула Буньипу, который в это время делал глубокие 
вдохи:
    - Я готова! Когда захочешь начать, крикни "Давай!"
    - Хорошо! - ответил Буньип. И, подождав, крикнул: - Давай!
    Сразу он выбросил из правой ноздри струю воды почти в фут 
толщиной. Он целился кенгуру в грудь, но она так быстро отпрянула в 
сторону, что он промахнулся. Тогда Буньип повернул голову, чтобы 
изменить направление струи, но Серая Шкурка вновь отпрыгнула. Буньип 
мотал головой из стороны в сторону, но Серая Шкурка все время 
увертывалась. С каждым прыжком она приближалась к извергающему воду 
Буньипу. Она отпрыгивала и проскальзывала под струями с такой 
скоростью, что Буньип подключил и вторую ноздрю, послав в кенгуру две 
струи сразу. Она впрыгнула между ними, потом отпрыгнула назад, 
перепрыгнула сразу через обе, проскользнула снизу и снова подскочила 
вверх между струями. Буньипу никак не удавалось попасть в нее. Он свел 
оба потока вместе, так что образовалась могучая струя толщиной в два 
фута, от которой Серой Шкурке пришлось увертываться чудовищными 
прыжками.
    Буньип выдохся. Он остановился, чтобы набрать побольше воздуха в 
легкие - без этого он не мог выдувать воду. Серая Шкурка 
воспользовалась моментом и бросилась к нему напрямик. Сделав последний 
мощный прыжок, она опустилась прямо на спину Буньипа.
    Эффект был потрясающий. Буньип закачался, закинул голову и издал 
такой рев, что деревья задрожали и в страхе сбросили листья. По склону 
холма прокатился сорвавшийся камень, а страшный порыв ветра всколыхнул 
воды рва.
    Огромные зубы Буньипа клацали, словно мечи, когда он рвал мнимых 
нападавших справа и слева. Тут Серая Шкурка с силой вонзила острые 
когти задних лап в бока Буньипа, и тот взревел уже от боли.
    Буньип запрыгал по дороге, и хотя он ничуть не походил на 
норовистую лошадь, но, как и она, изо всех сил старался сбросить с 
себя кенгуру, неуклюже подпрыгивая и виляя туловищем из стороны в 
сторону. Но кенгуру крепко сжимала лапами его бока, представляя себя 
на родео верхом на быке, и даже, помахав передней ланкой, закричала 
"Но! Но!", настолько унизив этим Буньипа, что он чуть не вывернулся 
наизнанку в попытках сбросить ее. Он развернул свою голову назад и 
вниз и набрал побольше воздуху с явным намерением смыть кенгуру со 
спины. Но Серая Шкурка скользнула вперед, уселась на его шее, где она 
соединялась с плечами, и, схватив его голову, повернула ее так, что 
выброшенная Буньипом струя попала в его собственный хвост. Омываемый 
водой, хвост дрожал, как тростинка в несущемся потоке.
    Буньип споткнулся. Теперь на него нападали и спереди, и сзади. В 
отчаянии он громко взревел, сделал еще один глубокий вздох, но Серая 
Шкурка набросила ему на шею петлю и перекрыла воду. Буньип задыхался. 
Он опустился на землю и успел только прохрипеть: "Сдаюсь".
    Серая Шкурка спрыгнула с его спины, и они с Питером стали ждать, 
когда он придет в себя.
    - Пожалуй, ты обошлась с ним слишком грубо, - сказал Питер, с 
беспокойством глядя на тяжело дышащего Буньипа.
    - Это он хотел обойтись с нами грубо, - ответила Серая Шкурка. - Я 
же только преподала ему урок. Кенгуру дерутся лучше, чем многие 
воображают. В конце концов он ведь хотел нас убить.
    - Это совершенно неважно. Ты нанесла удар по его гордости, ты его 
унизила, а это ужасно. Я собирался дать ему Волшебный Лист, и он стал 
бы добрым. А погляди на него теперь.
    Буньип действительно выглядел удрученным, но он быстро пришел в 
себя и начал упрекать Серую Шкурку за неспортивное поведение.
    - Ты не должна была использовать всякие там штучки с 
увертываниями, когда пошла на меня, - недовольно выговаривал ей 
Буньип. - Все рыцари настолько хорошо воспитаны, что они несутся прямо 
на тебя, издавая боевой клич. А вспрыгивать мне на спину! Им бы и в 
голову не пришло такое ребячество. Никто так не поступает, - ни 
рыцари, ни принцы.
    - Но я кенгуру.
    - Ты права, - согласился Буньип. - Сражаться с кенгуру меня не 
учили. Если бы я использовал вместо воды огонь, я бы, пожалуй, с тобой 
справился.
    Серая Шкурка хотела возразить, но Питер не дал ей открыть рот.
    - Возьми у меня подарок, - предложил он Буньипу. - Мне бы хотелось 
подарить тебе кое-что, прежде чем ты нас убьешь.
    Буньип колебался.
    - Дело в том, что когда я был маленьким, моя мама часто говорила 
мне: "Никогда не бери подарков от незнакомых людей на велосипедах". 
Да, именно так. И я не уверен, стоит ли мне брать подарок.
    - Но ведь у меня нет велосипеда, - возразил Питер.
    - Да, действительно. Хорошо, я возьму. Но это не значит, что я вас 
пощажу. Мне ведь платят пять долларов в неделю, чтобы я убивал всех 
принцев и рыцарей. Я не должен допустить, чтобы принцессу спасли. И 
что же это за подарок, который ты хочешь мне дать? Так и быть, я убью 
вас после того, как поблагодарю за него.
    - Это всего только лист, но Лист Волшебный, - сказал Питер, - 
Возьми, и ты почувствуешь себя хорошим и добрым.
    И Питер протянул Лист Буньипу.
    Тот взял ого и стал с удивлением рассматривать. Подарок показался 
ему пустяковым. Но потом он задумался. Он уже не выглядел таким 
свирепым, злобные глазки смягчились. Было видно, что внутри его 
нелепой головы происходили какие-то изменения, и он озирался, как 
будто в окружающем мире что-то изменилось.
    - Ты счастлив? - спросил Питер.
    - Таким счастливым я еще никогда не был. - Он с удивлением 
посмотрел на Питера. - Ты самый замечательный принц из всех, кого я 
когда-либо видел. И с чего это я хотел тебя убить? Да я бы просто не 
смог! Мне даже жаль, что я пытался убить Серую Шкурку. Я ее очень 
люблю.
    - Правда? - Серая Шкурка достала из сумки зеркальце и погляделась 
в него. Она подняла лапу и отбросила назад мех, закрывавший ей ухо. - 
Очень может быть. По-моему, я всегда была хорошенькой.
    - Ты знаешь, каким делает тебя этот лист? - спросил Питер, не 
обращая внимания на Серую Шкурку.
    - Нет, не знаю.
    - Теперь люди будут тебя любить, а не бояться.
    - Меня никто никогда не любил, - задумчиво произнес Буньип.
    - А теперь будут.
    - Я бы тоже хотел сделать вам подарок, но у меня ничего нет, - 
сказал Буньип. - Может быть, я могу вам чем-нибудь помочь?
    - Да, - ответил Питер. - Я хочу узнать, почему Прекрасная 
Принцесса сидит в заточении. Я хочу с ней поговорить. Помоги нам это 
устроить.
    На лице Буньипа мгновенно проступил страх, но в руке у него был 
Волшебный Лист, и его сила постепенно изгнала страх.
    - Прекрасная Принцесса будет рада с тобой познакомиться, я знаю 
точно. Она очень несчастна. Я раньше никому об этом не говорил. Но 
проникнуть в замок будет крайне трудно. Говорить с ней можно только в 
ее комнате. Это единственный способ. Я часто сопровождаю ее в верховых 
прогулках вокруг замка, но с нами всегда едут солдаты. Дай-ка я 
подумаю, как туда проникнуть.
    Буньип почесал затылок, потом сказал:
    - План готов. Каждое утро в семь часов ворота открываются, выходят 
два солдата и трубят в горн, чтобы все вставали. Затем они опускают 
мост и остаются охранять. После завтрака я подхожу к дверям и прошу 
позволения видеть короля, - я ежедневно отчитываюсь о числе рыцарей и 
принцев, пытавшихся спасти Прекрасную Принцессу. Солдаты идут к королю 
и докладывают ему о моем приходе. Потом они возвращаются и говорят, 
что король готов меня принять. Иногда, конечно, он бывает занят, и мне 
приходится уходить. Но такое случается редко.
    Солдаты отсутствуют минут десять. Этого времени вам хватит, чтобы 
перебежать мост и пройти ворота. Внутри замка вам придется самим 
искать дорогу по всем переходам и лестницам, ведущим к комнате 
Прекрасной Принцессы.
    Ее комната на самом верху башни - вон там, с голубой занавеской, - 
Буньип хвостом показал на то окно, которое Питер отметил, когда они 
только подходили к замку.
    - Прежде чем вы доберетесь до ее комнаты. вам придется миновать 
множество переходов. Старайтесь, чтобы вас не заметили. Если король 
или королева поймают вас в замке, вам отрубят голову. При любом звуке 
шагов - прячьтесь. Выбираться вам придется самим. Не забудьте, что 
после обеда ворота закрывают и поднимают мост.
    - Как-нибудь да выберемся, - сказал Питер. - А теперь мы вернемся 
в буш и поищем место для стоянки. В семь утра мы будем здесь.
    - Я буду вас ждать, - пообещал Буньип.
