\chapter{ПИТЕР ВСТРЕЧАЕТ СЕРУЮ ШКУРКУ}
\par\parЮжный Ветер разогнал облака, освободив небо от их ватного 
покрывала, и тотчас же сквозь ветви деревьев хлынуло солнце. Трава 
заколыхалась, запела птица. Питер вскочил на Мунлайт и галопом 
помчался домой. Кривой Мик сидел на пороге хижины, свивая кожаные 
ремешки в плеть. Лицо его напоминало грецкий орех, а лоб был так 
сморщен, что шапку ему приходилось не надевать, а чуть ли не 
навинчивать на голову. Объезжая диких лошадей, он неистово кричал и 
щелкал длинным бичом. Чем выше подпрыгивала лошадь, тем громче он 
кричал, а чем громче кричал он, тем выше подпрыгивала лошадь, и так 
продолжалось без конца. Зрелище прямо-таки восхитительное.
\parПитер осадил Мунлайт, а камни брызнули из-под копыт. Закудахтали и 
разбежались куры, гулявшие перед хижиной. Мальчик спрыгнул на землю и 
подбежал к Кривому Мику.
\par- Ты знаешь, - крикнул он, - у меня есть Волшебный Лист, а это 
значит, что меня любят и я нужен людям! Мне подарил его Южный Ветер. А 
еще он сказал, где я могу найти Прекрасную Принцессу. Он обещал, что 
Лист защитит меня. Я отправляюсь в путь сейчас же, сию минуту, а когда 
найду Принцессу, то привезу ее сюда, и мы будем здесь жить и разводить 
белых лошадей. Вот так.
\par- Отправиться "сию минуту" ты не можешь по той причине, что эта 
минута уже прошла, - отвечал Кривой Мик. - Тебе придется отправиться 
через пять минут, вот тогда-то и настанет твое "сейчас же".
\parПитер улыбнулся.
\par- Не говори так. Мне от таких разговоров только трудней тронуться 
с места.
\par- Знаю, - согласился Кривой Мик. - Трогаться всегда трудно. Я бы 
сказал, что нет ничего труднее, чем тронуться, и нет ничего легче, чем 
остаться. Будет лучше, если ты тронешься, как только я соберу тебе 
еду.
\parСтарик достал из сарая походный мешок и стал складывать в него 
еду. Он положил отбивные котлеты, колбасу, две буханки хлеба, три 
яблока, пачку чая, перец, соль и бананы.
\par- Котлеты жарь над углями, - наставлял Мик мальчика, затягивая 
тесемки метка и перебрасывая его через седло. - А перед тем как 
снимать котлеты, брось на угли парочку эвкалиптовых листьев. Запах от 
них пропитает котлеты и придает им аромат. А для чая я дам тебе 
котелок.
\par- И сколько эти котлеты жарить?
\par- Пока не потемнеют, - ответил Мик. Он поднял пастуший кнут, 
который только что сделал, и стал описывать им большие круги над 
головой. - Хорош кнут, хорош, - приговаривал он. - Ты только посмотри, 
как он ложится.
\parМик опустил кнут, и тот улегся на землю изящным завитком.
\par- Я хочу дать этот кнут тебе, - сказал Мик. - Он волшебный, самый 
лучший из всех, какие я когда-либо делал. Если к тебе придет беда, 
щелкни им разок, и я тут же появлюсь, будь ты даже на другом краю 
света. Смотри, как надо щелкать!
\parОн сделал шаг назад и раскрутил кнут над головой. Кнут вращался 
все быстрее и быстрее, и тут старик вдруг резко опустил руку вниз. 
Раздался оглушительный хлопок. Деревья закачались, с них шумно 
посыпались листья. Резкий звук пронесся через буш и отразился от гор, 
заполнив собой все пространство, словно ветер.
\par- Ай да кнут! - восторгался старик. - Мы назовем его Громобой. На, 
держи.
\parПитер был так взволнован, что не смог выдавить из себя "спасибо". 
Вместо этого он схватил руку старика, задержал ее в своих, - и Кривой 
Мик все понял. Потом Питер взял кнут и намотал его на руку. Всю жизнь 
он мечтал иметь собственный кнут и вот наконец сбылось.
\par- Хорошее ты имя придумал - Громобой, - сказал мальчик, потом 
добавил: - А ты думаешь, мне когда-нибудь понадобится твоя помощь?
\par- Еще бы! Каждому, кто ищет Прекрасную Принцессу, приходится 
выполнять три задания, одно другого сложнее, без этого Принцессу не 
освободить. Может быть, тебе придется сочинить самую большую на свете 
небылицу, или усмирить самого дикого на свете коня, или сражаться с 
самым сильным на свете человеком либо драконом. Вот тогда-то я и 
появлюсь. Щелкни кнутом, - и я буду тут как тут. Никто не может так 
лихо ездить на конях, так храбро сражаться, так ловко наплести с три 
короба, как я. В один прекрасный день ты найдешь Свою Принцессу и 
женишься на ней. А теперь ступай.
\parПитер каблуками тронул Мунлайт.
\par- Прощай, Мик! - прокричал он.
\parМунлайт сразу бросилась в галоп. Она мчалась стрелой, едва касаясь 
земли.
\par- Следуй туда, куда показывает моя тень, - сказал старый эвкалипт, 
когда Питер поравнялся с ним. Эвкалипту было уже пятьсот лет, и на 
пятьсот вопросов у него были приготовлены пятьсот ответов; ему 
приносили их птицы, находившие приют в кроне. Однажды перелетные 
кроншнепы рассказали ему, где живет Прекрасная Принцесса.
\parТень от мощного ствола красного эвкалипта указывала в сторону 
горного хребта, на голубых пиках которого покоился край небосвода. 
Питер направился туда по тропе, проложенной дикими собаками динго. 
Тропа огибала горные отроги, пересекала равнины. Питеру приходилось 
продираться сквозь заросли древовидного папоротника, который хлестал 
его по щекам.
\parОн ехал все утро. Он заставлял Мунлайт спускаться с одного берега 
реки и взбираться на противоположный просто потому, что ему это 
нравилось. Там, где реки были неглубокие и на дне виднелись желто-
коричневые камни, Мунлайт наклоняла голову и пила. Подкрепив силы, 
Мунлайт снова рвалась вперед, запрокинув голову и закусив удила.
\parПитер проголодался. Он остановился на берегу реки, где лежало 
несколько валунов и росла длинная и сочная трава, и решил половить 
рыбу, пока Мунлайт пасется. Он слез с нее, снял седло и уздечку и 
вместе с Громобоем положил их на камень. Потом Питер пошарил в кармане 
в поисках лески, которую всегда носил с собой.
\par- Прошу прощения, - раздался чей-то голос. Питер поднял голову и 
увидел, что из-за камня на него смотрит кенгуру.
\par- Меня зовут Серая Шкурка. Я спала. Я здесь живу, по крайней мере, 
сплю. А ты, конечно же, - продолжала она, - спишь по ночам. Из-за 
этого ты столько теряешь, что я бы посоветовала тебе переменить эту 
привычку.
\par- Но ночью немного увидишь, - возразил Питер.
\par- Зато намного больше услышишь! - воскликнула Серая Шкурка. - Ночь 
- самое подходящее время для того, чтобы слушать. Хотя все равно, у 
тебя такие маленькие уши, что я удивляюсь, как ты вообще что-либо 
слышишь.
\parОна выпрыгнула из-за валуна и остановилась перед Питером, а затем 
оперлась на хвост и стала на нем раскачиваться взад-вперед, словно в 
кресле-качалке.
\par- Видишь, какая я везучая, - объяснила она. - Кресла-качалки нынче 
в большой моде, а у меня есть свое собственное, не отделимое от меня. 
Если хочешь - можешь сесть ко мне на колено и покачаться вместе со 
мной, - дружелюбно предложила она.
\par- На твоей коленке не очень-то посидишь, - отвечал Питер. - У тех, 
у кого колени вывернуты наоборот, сидеть просто не на чем.
\par- Это верно, - согласилась Серая Шкурка. - Невозможно иметь все. 
Но по крайней мере я была достаточно учтива, и предложила тебе 
покачаться. А теперь скажи, чем ты так озабочен?
\par- Да, в общем-то, ничем.
\par- А вот и нет, озабочен. Я знаю, ты - голоден.
\par- О, это - да.
\par- А что бы ты хотел съесть? Назови любое блюдо. Вспомни свое 
любимое кушанье.
\parПитер подумал: "жареные колбаски с картофельным пюре, политые 
томатным соусом, да побольше. Чашку чая с тремя ложками сахара и 
мороженое".
\par- Пожалуйста, - тотчас отозвалась Серая Шкурка. Она опустила лапу 
в свою сумку и достала оттуда стол и стул. Потом она извлекла оттуда 
скатерть, потом - ножи, ложки, вилки, перец и соль, потом бутылку 
томатного соуса. Наконец, махнув лапкой и отвесив церемонный поклон, 
достала тарелку с колбасками и пюре, чашку дымящегося чая и розетку 
мороженого.
\par- А себе, - сказала она, - я, пожалуй, возьму пару пучков травы, 
которую мы, кенгуру, очень любим, и несколько листочков Acacia 
dumrosa.
\parОна два раза опустила лапку в сумку, и ее лакомство тоже оказалось 
на столе.
\par- А теперь придвинь стул и начинай есть, - сказала она.
\parПитер был потрясен.
\par- Вот это да! Даже не верится!
\par- А я вовсе и не собираюсь тебе ничего доказывать, - сказала Серая 
Шкурка. - Постарайся просто поверить мне.
\par- Я тебе верю, - отвечал Питер. - Ты совсем не похожа на врунишку.
\par- Ты прав, я не люблю врать. Но у меня была такая тяжелая жизнь, 
что я перестала доверять людям.
\par- От этого я тебя вмиг вылечу, - произнес Питер и вручил Серой 
Шкурке Волшебный Лист из маленькой сумочки, которая висела у него на 
шее. - Ну, и что ты теперь чувствуешь?
\par- Я чувствую, что меня словно наполнило радостью, - ответила Серая 
Шкурка. - И еще - гордостью... Я отнюдь не стала гордячкой, просто я 
испытываю чувство гордости, понимаешь? - Она взглянула на Волшебный 
Лист и улыбнулась. - Кроме того, я чувствую себя очень важной особой. 
- Секунду поколебавшись, она закончила: - Как будто меня многие любят.
\par- Именно это ты и должна была почувствовать, - подтвердил Питер.
\par- А где ты взял такой Лист?
\par- У Южного Ветра. Он сказал, что Лист поможет мне в пути.
\par- Как это мило с твоей стороны, что ты дал этот Лист мне, - 
сказала Серая Шкурка. - Хочешь добавки?
\par- Я еще и этого не съел.
\par- Ну, как знаешь! А то ведь оттуда можно еще много чего достать, - 
заверила его Серая Шкурка и принялась жевать любимую траву кенгуру.
\parНа тарелке Питера оставалось еще порядочно, и когда он наконец 
съел все, то заволновался: "Пожалуй, не стоит больше есть колбасок, а 
то не останется места для мороженого".
\par- Вот незадача, - посочувствовала Серая Шкурка. - Я очень хорошо 
тебя понимаю. Попробуй попрыгать вокруг стола. Тогда колбаски 
утрясутся и сверху появится место для мороженого.
\parПитер пропрыгал вокруг стола три раза.
\par- Ну как, помогло? - спросила Серая Шкурка, когда он снова сел.
\par- Еще как, - ответил Питер и без всякого труда съел мороженое.
\par- Наука - это, без сомнения, замечательная вещь, - заключила Серая 
Шкурка. - А теперь, - добавила она, - надо все перемыть. Я буду мыть, 
а ты - вытирать.
\parОна достала из сумки тазик с горячей водой, потом - кухонное 
полотенце и протянула его Питеру: - Вытирай все, что я буду тебе 
передавать, а потом бросай в мою сумку, - говорила она. - Ножи опускай 
ручкой вперед, вилки тоже. В конце концов, я ведь живое существо и 
могу порезаться и уколоться так же, как и ты.
\par- Я буду осторожен, - пообещал Питер, совершенно уверенный, что 
все эти предметы никак не поместятся в одной сумке. Но и ножи, и 
вилки, и чашки, и тарелка исчезали там одна за другой, проваливаясь в 
сумку, как письма в почтовый ящик. Даже стол со стулом съежились и 
исчезли, словно их никогда и не было.
\par- Не могу понять, почему ты не толстеешь? - удивился Питер. - И 
куда все это девается?
\par- Ты задал интересный вопрос, - сказала Серая Шкурка, приоткрыв 
лапками сумку и заглянув внутрь. - Я часто сама ломаю голову: 
действительно, куда? Но прелесть волшебства в том и состоит, что не 
все в нем должно быть нам понятно. Волшебство просто происходит - и 
все.
\par- А когда оно у тебя появилось? Когда ты была еще совсем маленькой 
девочкой? - спросил Питер. Он почему-то все время считал Серую Шкурку 
существом, подобным ему, то есть человеком.
\par- Хочешь, я расскажу тебе о своей жизни?
\par- Хочу.
\par- Тогда слушай, - начала Серая Шкурка и снова принялась 
раскачиваться на хвосте. - Я родилась в Кастлмейне, в бедной, но 
честной семье.
\par- Кажется, что-то похожее я уже где-то слышал. - Питер 
сосредоточился, пытаясь вспомнить, где.
\par- Очень может быть, ничто не ново под луной. Бедные, но честные 
родители встречаются довольно часто, и в том же Кастлмейне рождаются 
многие кенгурята. - Тут Серая Шкурка махнула лапкой, чтобы ее больше 
не перебивали.
\par- Родители гордились мной, потому что я была такой, как все 
малыши. Вот если бы я была похожа на лошадь, например, они бы мной 
вовсе не гордились.
\par- Это точно, - кивнул Питер.
\par- Я ничем не выделялась, и никто меня не сторонился, пока однажды, 
запустив руку в свою сумку, я не вытащила оттуда пучок раннего лука. С 
тех пор я, как белая ворона, всеми отвержена и несчастна.
\par- А что ты сделала с тем луком? - заинтересовался Питер.
\par- Съела.
\par- Я тоже люблю ранний лук, - признался Питер.
\parСерая Шкурка опустила лапку в сумку и достала пучок лука.
\par- На, возьми, - сказала она. - Попался серебристый. А вообще-то он 
бывает разный.
\parПитер сунул лук в карман.
\par- Мне пришлось переселиться в другой район, - продолжала Серая 
Шкурка. - Другие кенгуру не могли простить, что у меня волшебная сумка 
и я могу достать из нее все, что пожелаю, даже рояль. Я так от всех 
них отличалась, что они стали поговаривать, будто я не от мира сего. А 
когда я стала ходить в школу, мне не пришлось покупать ни карандаши, 
ни резинки, ни учебники. Я просто доставала их из своей сумки. А 
другие дети думали, что я ворую их и прячу. Вот они и сказали 
родителям, что я воришка. Мама-то знала, что я никогда ничего не 
украду, но и она, бывало, поговаривала: "Жаль, что ты не такая, как 
все". Потом кенгурята стали бросать в меня камнями, и никто не хотел 
со мной играть...
\par- Бедняжка! - посочувствовал Питер.
\par- Да! - всхлипнула Серая Шкурка. - Мне тоже стало себя жалко, и в 
конце концов я решила уйти. Кенгуру не любят, когда кто-то из них 
отличается от остальных. А я отличалась.
\par- Скажи, а ты из своей сумки можешь вытащить все-все? - 
недоверчиво спросил Питер.
\par- Абсолютно все.
\par- И слона?
\par- Смотри! - Серая Шкурка по локоть запустила лапку в сумку, изо 
всех сил за что-то дернула, и оттуда вылетел слон. Несколько секунд 
он, пошатываясь, в полнейшем замешательстве крутился на месте, затем 
прислонился к дереву, все еще тяжело дыша.
\par- Что это значит? Где я? - снова и снова повторял он. - Мне плохо! 
Меня тошнит!
\par- Ты в Австралии, - объяснила Серая Шкурка.
\par- Нет! - взревел слон. - Только не в Австралии! Это же на другой 
стороне Земли! Немедленно отправьте меня домой!
\par- Кончай распускать нюни! - строго приказала Серая Шкурка. - Где 
твое достоинство? Что бы сказала твоя мама, если б она тебя сейчас 
увидела?
\par- Она бы спросила, как меня угораздило сюда попасть, - ответил 
Слон. - Я никогда никуда не хожу без спроса. Вы же знаете, что мамы 
вечно беспокоятся. Я преспокойно гулял себе по Африке и срывал в лесу 
бананы...
\par- Но бананы в африканских лесах не растут, - возразил Питер.
\parСлон вдохнул побольше воздуха. Он был единственным ребенком в 
семье, и мог гулять допоздна, а уж чтоб ему возражали...
\par- Ты-то что знаешь об этом? - завопил он грозно. - Ты сам-то был 
там хоть когда-нибудь? - Он отошел от дерева и теперь стоял прямо 
перед Питером и Серой Шкуркой.
\par- Нет, ни разу, - вынужден был признать Питер, уже жалея, что 
вообще ввязался в спор.
\par- То-то и оно! - торжествующе воскликнул слон. - Итак, повторяю, я 
ел бананы, когда вдруг огромная волосатая, когтистая лапища схватила 
меня за ногу и потащила.
\parСерая Шкурка поднесла свою лапку к глазам и, хмурясь, осмотрела 
ее. "Идиотское описание", - пробормотала она.
\par- Я отбивался как мог, - продолжал слон, - но не успел и глазом 
моргнуть, как меня протащили сквозь какой-то мешок, покрытый изнутри 
мехом, и вот я здесь. Это просто безобразие! До чего же мы все 
докатимся в таком случае, хотел бы я знать? Я всегда был честным и 
послушным! Возвращайте меня немедленно домой!
\par- Это естественное желание, - согласилась Серая Шкурка. Протянув 
лапку, она схватила слона за ногу и что есть силы дернула, так что тот 
головой вперед упал в ее сумку и исчез.
\par- Надеюсь, это послужит тебе уроком, - сказала она, обернувшись к 
Питеру. - С волшебными сумками лучше не шутить. Если бы ты попросил 
меня достать из сумки льва, то мы бы уже давно переваривались в его 
желудке.
\parЭти слова испугали Питера.
\par- Не смей даже думать об этом! - закричал он. - Может быть, одной 
твоей мысли о льве достаточно, чтобы он выскочил оттуда. Иметь такую 
сумку просто опасно.
\par- Вовсе нет, - возразила Серая Шкурка, - из моей сумки ничто не 
выходит само по себе, - я все должна вытаскивать своими лапками. 
Лучшего способа защиты и не придумать. Ну, да ладно, а теперь расскажи 
мне о своей жизни.
\parПитер рассказал о старом Мике, который так здорово объезжает диких 
коней, о своей Мунлайт и о том, как они носились, обгоняя бурю. 
Рассказал, что сам он ищет Прекрасную Принцессу.
\par- Для мальчика твоего возраста ты рассказываешь просто 
превосходно, - заметила Серая Шкурка. - Ты должен как-нибудь 
использовать этот дар. Из тебя вышел бы, например, замечательный 
погонщик скота.
\par- Почему именно погонщик?
\par- А почему бы и нет?
\par- Но ведь погонщики зарабатывают не тем, что рассказывают 
удивительные истории.
\par- Конечно, не тем, - согласилась Серая Шкурка, - но ты представь, 
как счастлив был бы погонщик, умей он так хорошо рассказывать. А чем 
плохо быть счастливым погонщиком?
\par- Ты права, - задумчиво проговорил Питер, которому раньше такая 
мысль в голову не приходила. - Я мог бы стать погонщиком. Я вполне мог 
бы стать счастливым погонщиком. Только сначала я должен найти свою 
Прекрасную Принцессу.
\par- Тогда пошли, - предложила Серая Шкурка. - Мы только теряем время 
на разговоры.
\par- Разве ты пойдешь со мной? - удивился Питер.
\par- А как же, - сказала Серая Шкурка. - Ты дал мне Волшебный Лист, и 
теперь я знаю, что нужна тебе. Я буду тебе другом до гробовой доски, 
так что лови Мунлайт и поехали.
\parПитер пошел за своей любимой старой фетровой шляпой, которую 
оставил на земле около камня, но там ее не оказалось. На этом месте 
лежала изумительная голубая шляпа с лентой и страусовым пером, которое 
изящно изгибалось назад, как на шляпах принцев.
\par- Это же шляпа Принца! - воскликнула Серая Шкурка. - Но принцев 
здесь нет...
\parПитер надел ее, и она пришлась ему как раз впору.
\par- Знаешь, - сказал он, - я уверен, что эта шляпа предназначена для 
меня. Суди сама: я подарил тебе Волшебный Лист и сделал тебя 
счастливой. А ведь когда даришь счастье другому, всегда получаешь что-
нибудь сам. По-моему, это что-то вроде награды.
\par- По-моему, тоже, - согласилась Серая Шкурка.
