Алан МАРШАЛЛ

                        ШЕПОТ НА ВЕТРУ

                       Сказочная повесть
__________________________________________________________________

Перевод А. СЛОБОЖАНА
Иллюстрации художника ДЖЕКА НЬЮНБАМА
OCR Dauphin, IX/2002
__________________________________________________________________

                ОГЛАВЛЕНИЕ

    Глава  1. Поиски начинаются
    Глава  2. Питер встречает Серую Шкурку
    Глава  3. Долина Цепляющейся Травы
    Глава  4. Серая Шкурка сражается с Великаном.
    Глава  5. Гроза в замке
    Глава  6. Питер и Серая Шкурка в плену у Бледной Колдуньи
    Глава  7. Питер и Колдунья летят на Луну
    Глава  8. Человек-Смерч Вилли-Вилли
    Глава  9. Битва с Котами Сомнений
    Глава 10. Появление Буньипа
    Глава 11. Рассказ Буньипа
    Глава 12. Серая Шкурка сражается с Буньипом
    Глава 13. В замке
    Глава 14. Прекрасная Принцесса
    Глава 15. Три задания
    Глава 16. Состязание лгунов
    Глава 17. Человек-смерч появляется снова
    Глава 18. Фаерфакс
    Глава 19. Возвращение в замок
    Глава 20. Последнее задание
    Глава 21. Питер женится на Прекрасной Принцессе

__________________________________________________________________

    Я написал "Шепот на ветру" после того, как один старый русский 
писатель объяснил мне, что я уже достаточно вырос, чтобы писать для 
детей. Ему не нравилось, когда молодые писатели начинали с детских 
произведений, а потом переходили к взрослым. По его мнению, все должно 
быть наоборот. Это был Самуил Маршак, очень известный писатель, 
который, однако, сам свою первую книжку написал именно для детей.
    По возвращении домой мне пришла в голову одна идея. Я задумал 
написать сказку на австралийском материале, в которой перемешались бы 
колдуньи, эльфы и драконы - и, конечно, ее героем не мог быть никто 
другой, кроме этого ужасного лжеца, Кривого Мика.
    Я никогда ничего не писал специально для детей, мои произведения 
никогда не были только плодом воображения, я всегда придерживался 
реальной жизни. Но когда я начал работать над сказкой и почувствовал 
всю свою свободу, я подумал, как это замечательно. От работы над 
книгой я получил истинное удовольствие...

                                                     Алан Маршалл
                                (Из книги "Алан Маршалл рассказывает")
__________________________________________________________________

                                              АННЕ БЕЧЕРВАЙЗ -
                                              ПРЕКРАСНОЙ ПРИНЦЕССЕ

        Глава 1
        ПОИСКИ НАЧИНАЮТСЯ

    Когда-то, давным-давно, жили-были старик и маленький мальчик. 
Старик был такой скрюченный, что его прозвали Кривой Мик. Мальчика 
звали Питер. В хижине с трубой и двумя окнами, где они жили, день-
деньской не смолкал собачий лай. Дело в том, что заросли буша вокруг 
хижины были настолько густы, что собаке просто не хватало места 
повилять хвостом, а поскольку лаять без этого совершенно невозможно, 
она лаяла в доме.
    В хижине стояли два стула, стол и две кровати. Над открытым очагом 
свисал на цепях с потолка закопченный походный котелок. На деревянных 
гвоздях, вбитых в стену, были развешаны уздечки и седла. На стене 
висела картина, где была изображена брыкающаяся лошадь со всадником. 
Это и был Кривой Мик. На книжной полке стояли две книги: одна о 
лошадях, другая - о Прекрасных Принцессах. Между балок под крышей 
домика бегали опоссумы, а в норках под полом жили бандикуты.
    Кривой Мик был лучшим в мире наездником. Он мог стоять на лошади, 
когда она брыкалась, он мог лежать на лошади, когда она брыкалась, он 
мог сидеть на ней и пить кофе, когда она брыкалась. Он ездил на 
лошадях, которые взбрыкивали так высоко, что он мог шапкой поймать 
несколько звезд, он ездил на лошадях, которые взбрыкивали с такой 
скоростью, что даже сбрасывали с себя выжженное тавро. Он ездил на 
белых лошадях, он ездил на черных лошадях, он ездил на пегих лошадях, 
- но ни одна из них ни разу не сбросила его на землю. Кривой Мик и 
Питера научил ездить верхом.
    У Питера была белая лошадка, способная мчаться быстрее ветра. А 
красотой она напоминала птицу, и казалось даже, что в полете на ногах 
у нее вырастают крылья. Когда бури и грозы ломились через заросли 
буша, а летящие облака с грохотом скрещивали мечи молний, озаряя 
небосвод, Питер несся сквозь непогоду, перескакивая через каждую 
молнию раньше, чем она успевала погаснуть. При этом белая лошадка 
задирала голову и ржала навстречу ветру, а ее длинный хвост и лохматая 
грива развевались как знамена. Она мчала Питера так стремительно, что 
на него не успевала упасть ни единая капля дождя и так легко касалась 
копытами земли, что не приминала ни одного цветка, не сбивала ни одной 
капельки, которые бисером покрывали траву.
    Как-то раз их окликнул Южный Ветер, и голос его прогремел подобно 
грому: "Эй, мальчик на белом коне, что скачет через молнии! 
Остановись-ка! Я хочу поговорить с тобой". Он вышел из бури, 
закутанный в плащ из облаков, и сел на бревно. Это был старик с 
длинной седой бородой, в которой искрились дождевые капли, и молодыми 
глазами - яркими и блестящими, как у юноши.
    Подскакав к Южному Ветру, Питер натянул поводья, и лошадка встала 
на дыбы, изогнула шею.
    Ей, дочери ветра и солнца, хотелось мчаться наперегонки с бурей по 
горам и широким долинам, где она могла взять настоящий разбег. Но 
Питер потрепал ее по холке, приговаривая "постой, постой", и она 
остановилась.
    - А ты прекрасный наездник! - начал Южный Ветер. - Здорово же твоя 
лошадка прыгает через облака! Как ее звать?
    - Мунлайт, Лунное сияние.
    - Мунлайт! - повторил Южный Ветер. - Красивое имя, а главное - 
точное, ведь ее шерсть блестит, как луна. Не хочешь ли ты и Мунлайт 
поработать со мной? Я мою Землю и поддерживаю на ней чистоту: сдуваю 
мертвые листья и поливаю дождем. Если вы согласитесь, жизнь ваша будет 
прекрасной! Я познакомлю вас со своими братьями - Северным Ветром, 
Восточным Ветром и Западным Ветром, и вы будете помогать нам в наших 
странствиях по свету. Вы полетите впереди нас и поведете нас через 
горы и равнины. Мы привяжем к твоему седлу облака, и вы будете 
доставлять их в Мулгийские пустыни, в земли Спинифекса, которые лежат 
за Прибрежными горами. Вы принесете дождь в пустыни, где живут лишь 
песчаные бури, и в пустынях этих расцветут сады.
    - А найду ли я где-нибудь там Прекрасную Принцессу? - спросил 
Питер.
    - Прекрасную Принцессу? - рассмеялся Южный Ветер, да так, что от 
его смеха закачались деревья. - Нет. Я, например, уже тысячу лет не 
видел ни одной Прекрасной Принцессы. Это раньше я встречал их, когда 
они стояли у окон своих замков и расчесывали золотые волосы. Но 
сейчас... Нет, ты не найдешь Принцессы в тех краях, над которыми я 
пролетаю. А зачем она тебе?
    - Я хочу ее спасти.
    - Спасти от кого?
    - От дракона.
    - Ха! Так это, пожалуй, будет труднее всего. Драконы! Дай-ка 
подумать. Когда же я видел драконов в последний раз? Может быть, в 
Китае? Нет, не помню. Прекрасная Принцесса, которую стережет Дракон! 
Вот уж действительно задачка!
    - Кривой Мик сказал мне, что, если я поищу получше, я найду 
Прекрасную Принцессу. Еще он сказал, что всех Прекрасных Принцесс 
стерегут драконы.
    - Кривой Мик! - воскликнул Южный Ветер. - Это не тот ли, что 
вступил в борьбу с бураном и сломал ему шею?
    - Он самый. Он мне об этом рассказывал.
    - Тогда я его знаю, - обрадовался Южный Ветер. - Однажды я нес его 
пятьдесят миль на листе кровельного железа.
    - Да, он как раз чинил крышу овчарни, когда ты его сдул, - сказал 
Питер. - Он вспоминал, что эти пятьдесят миль промчался за десять 
минут.
    - О, то был настоящий ураган! - воскликнул Южный Ветер, потирая 
руки от удовольствия при одном воспоминании. - Мик привязал повозку с 
волами к красному эвкалипту, а я заставил ее трястись, как бумажный 
листок. Уж если он говорит, что ты найдешь Прекрасную Принцессу, 
значит, хоть одна Принцесса в Австралии да есть. Подожди здесь 
немного, я мигом облечу Землю с севера на юг и с запада на восток и 
скажу тебе, где я ее видел.
    Он поднялся и завернулся в свой громадный белый плащ. Раздался 
удар грома, пронесся мощный порыв ветра, и старик исчез. Ветви 
деревьев, мимо которых он промчался, хлестали по воздуху, а упавшие с 
них листья, покачиваясь, опускались на землю. Не успел Питер и глазом 
моргнуть, как старик уже был в облаках и погнал их на Север, крутя и 
вертя.
    Вернулся он минут через двадцать. Сначала в воздухе раздался 
легкий шелест, потом вихрем взметнулись сухие листья; и вот уже старик 
опять сидит на своем бревне.
    - Быстро же вы обернулись! - удивился Питер.
    - Да неужели? Я летел со скоростью ветра. Я пронесся над пустынями 
и долинами, где живут великаны, заглянул во все пещеры и каньоны, 
обыскал пляжи, где растут пальмы, но мне так и не удалось найти 
Прекрасную Принцессу. Потом я встретил своего брата, Северного Ветра, 
и загнал дым от раздутых им лесных пожаров прямо ему в глаза. Затем я 
вонзил в полыхающее пламя струи дождя, а самого братца отогнал в те 
земли, где рождаются ветры. Но нигде я не увидел ни одной Прекрасной 
Принцессы. Я сбросил колдунью с метлы, я заставил волосы в бороде 
великана щелкать, как кнуты, а потом полетел в землю аборигенов. Я 
осыпал брызгами водопада игравших на мелководье детишек, тронул за 
плечо их отца, который собирался пронзить рыбу острогой, и спросил 
его: "Где найти Прекрасную Принцессу, которую стережет Дракон?"
    "Принцессы жили в золотой век моего народа, - ответил он мне, - но 
потом все принцессы стали птицами, а драконы - ящерицами или змеями. 
Во всей Австралии осталась только одна принцесса. Она живет в 
заточении у своего жадного отца-короля, и ее стережет Буньип. Мне о 
ней поведало Солнце. Каждый день оно золотит ее волосы, и они начинают 
так ярко сиять, что в комнате тает темнота, а к окну подлетают птицы, 
надеясь согреться в холодные зимние дни. Она самая прекрасная из 
Прекрасных Принцесс со дня сотворения мира, и, когда она улыбается, 
страх покидает Землю, и все звери оставляют свои убежища и выходят к 
солнцу".
    "Как же я ее прозевал, - не понимаю", - удивился я.
    "Зато твои братья ее знают", - ответил рыбак.
    "А сможет ли найти ее маленький мальчик на белой лошадке?" - 
спросил я.
    "Да, сможет. Но только если он храбрый и добрый. Ему придется 
проехать по Долине Цепляющейся Травы к замку Великана, пересечь лес, 
где живет Бледная Колдунья, миновать Пустыню Одиночества и идти все 
вперед и вперед, пока не дойдет до Последнего Холма. Там растет 
эвкалипт, такой старый, что помнит начало мира. Мальчик должен сесть 
под него и сидеть до тех пор, пока луна не покажется из-за горизонта. 
А когда она целиком усядется на кромке земли, на ее фоне возникнет 
силуэт громадного замка. Тогда дерево заговорит и расскажет мальчику, 
как найти этот замок и Принцессу".
    Потом он что-то вспомнил и добавил: "Но оно расскажет это только 
доброму человеку!"
    Окончив рассказ, Южный Ветер поднялся с бревна и расправил белый 
плащ.
    - Итак, Питер, в путь! - воскликнул он. - Тебя ждет Прекрасная 
Принцесса! Не будь жадиной и эгоистом, думай сперва о других, а после 
- о себе, и ты найдешь ее, какие бы трудности и опасности ни 
встретились тебе. Мы с братьями попросим наших друзей не упускать тебя 
из виду и помогать во всем. А теперь на прощанье я хочу тебе кое-что 
подарить.
    Южный Ветер опустил руку в карман плаща и достал оттуда маленькую 
кожаную сумочку на серебряной цепочке. Склонившись, он повесил ее на 
шею Питеру.
    - В этой сумочке - Волшебный Лист, - сказал Южный Ветер. - Из всех 
подарков, которые я мог бы тебе преподнести, этот - самый ценный. И 
самый полезный для тебя. Он тебе очень поможет. Всякий раз, как ты 
повстречаешься с бедой, увидишь несчастных людей или, наоборот, людей, 
способных помешать тебе, способных убить и других, и тебя, отдай им 
этот Лист, и они изменятся. Расстаться с Листом не бойся. Как только 
ты его отдашь, в твоей сумке тут же появится новый.
    - Спасибо, - сказал Питер, - ты меня очень выручил.
    - А теперь мне пора, - заявил Южный Ветер. - Меня ждут три грозы, 
их нужно перенести в те долины, которые уже давно просят дождя. 
Прощай!
    Он умчался в небо, волоча за собой свое пышное одеяние.
    - Мистер Южный Ветер! - крикнул Питер вверх, в то место, где 
собирались облака. - А что означает этот Лист?
    Из крохотного окошечка ясного неба прогрохотало: "Он означает: 
тебя любят, ты нужен людям".

        Глава 2
        ПИТЕР ВСТРЕЧАЕТ СЕРУЮ ШКУРКУ

    Южный Ветер разогнал облака, освободив небо от их ватного 
покрывала, и тотчас же сквозь ветви деревьев хлынуло солнце. Трава 
заколыхалась, запела птица. Питер вскочил на Мунлайт и галопом 
помчался домой. Кривой Мик сидел на пороге хижины, свивая кожаные 
ремешки в плеть. Лицо его напоминало грецкий орех, а лоб был так 
сморщен, что шапку ему приходилось не надевать, а чуть ли не 
навинчивать на голову. Объезжая диких лошадей, он неистово кричал и 
щелкал длинным бичом. Чем выше подпрыгивала лошадь, тем громче он 
кричал, а чем громче кричал он, тем выше подпрыгивала лошадь, и так 
продолжалось без конца. Зрелище прямо-таки восхитительное.
    Питер осадил Мунлайт, а камни брызнули из-под копыт. Закудахтали и 
разбежались куры, гулявшие перед хижиной. Мальчик спрыгнул на землю и 
подбежал к Кривому Мику.
    - Ты знаешь, - крикнул он, - у меня есть Волшебный Лист, а это 
значит, что меня любят и я нужен людям! Мне подарил его Южный Ветер. А 
еще он сказал, где я могу найти Прекрасную Принцессу. Он обещал, что 
Лист защитит меня. Я отправляюсь в путь сейчас же, сию минуту, а когда 
найду Принцессу, то привезу ее сюда, и мы будем здесь жить и разводить 
белых лошадей. Вот так.
    - Отправиться "сию минуту" ты не можешь по той причине, что эта 
минута уже прошла, - отвечал Кривой Мик. - Тебе придется отправиться 
через пять минут, вот тогда-то и настанет твое "сейчас же".
    Питер улыбнулся.
    - Не говори так. Мне от таких разговоров только трудней тронуться 
с места.
    - Знаю, - согласился Кривой Мик. - Трогаться всегда трудно. Я бы 
сказал, что нет ничего труднее, чем тронуться, и нет ничего легче, чем 
остаться. Будет лучше, если ты тронешься, как только я соберу тебе 
еду.
    Старик достал из сарая походный мешок и стал складывать в него 
еду. Он положил отбивные котлеты, колбасу, две буханки хлеба, три 
яблока, пачку чая, перец, соль и бананы.
    - Котлеты жарь над углями, - наставлял Мик мальчика, затягивая 
тесемки метка и перебрасывая его через седло. - А перед тем как 
снимать котлеты, брось на угли парочку эвкалиптовых листьев. Запах от 
них пропитает котлеты и придает им аромат. А для чая я дам тебе 
котелок.
    - И сколько эти котлеты жарить?
    - Пока не потемнеют, - ответил Мик. Он поднял пастуший кнут, 
который только что сделал, и стал описывать им большие круги над 
головой. - Хорош кнут, хорош, - приговаривал он. - Ты только посмотри, 
как он ложится.
    Мик опустил кнут, и тот улегся на землю изящным завитком.
    - Я хочу дать этот кнут тебе, - сказал Мик. - Он волшебный, самый 
лучший из всех, какие я когда-либо делал. Если к тебе придет беда, 
щелкни им разок, и я тут же появлюсь, будь ты даже на другом краю 
света. Смотри, как надо щелкать!
    Он сделал шаг назад и раскрутил кнут над головой. Кнут вращался 
все быстрее и быстрее, и тут старик вдруг резко опустил руку вниз. 
Раздался оглушительный хлопок. Деревья закачались, с них шумно 
посыпались листья. Резкий звук пронесся через буш и отразился от гор, 
заполнив собой все пространство, словно ветер.
    - Ай да кнут! - восторгался старик. - Мы назовем его Громобой. На, 
держи.
    Питер был так взволнован, что не смог выдавить из себя "спасибо". 
Вместо этого он схватил руку старика, задержал ее в своих, - и Кривой 
Мик все понял. Потом Питер взял кнут и намотал его на руку. Всю жизнь 
он мечтал иметь собственный кнут и вот наконец сбылось.
    - Хорошее ты имя придумал - Громобой, - сказал мальчик, потом 
добавил: - А ты думаешь, мне когда-нибудь понадобится твоя помощь?
    - Еще бы! Каждому, кто ищет Прекрасную Принцессу, приходится 
выполнять три задания, одно другого сложнее, без этого Принцессу не 
освободить. Может быть, тебе придется сочинить самую большую на свете 
небылицу, или усмирить самого дикого на свете коня, или сражаться с 
самым сильным на свете человеком либо драконом. Вот тогда-то я и 
появлюсь. Щелкни кнутом, - и я буду тут как тут. Никто не может так 
лихо ездить на конях, так храбро сражаться, так ловко наплести с три 
короба, как я. В один прекрасный день ты найдешь Свою Принцессу и 
женишься на ней. А теперь ступай.
    Питер каблуками тронул Мунлайт.
    - Прощай, Мик! - прокричал он.
    Мунлайт сразу бросилась в галоп. Она мчалась стрелой, едва касаясь 
земли.
    - Следуй туда, куда показывает моя тень, - сказал старый эвкалипт, 
когда Питер поравнялся с ним. Эвкалипту было уже пятьсот лет, и на 
пятьсот вопросов у него были приготовлены пятьсот ответов; ему 
приносили их птицы, находившие приют в кроне. Однажды перелетные 
кроншнепы рассказали ему, где живет Прекрасная Принцесса.
    Тень от мощного ствола красного эвкалипта указывала в сторону 
горного хребта, на голубых пиках которого покоился край небосвода. 
Питер направился туда по тропе, проложенной дикими собаками динго. 
Тропа огибала горные отроги, пересекала равнины. Питеру приходилось 
продираться сквозь заросли древовидного папоротника, который хлестал 
его по щекам.
    Он ехал все утро. Он заставлял Мунлайт спускаться с одного берега 
реки и взбираться на противоположный просто потому, что ему это 
нравилось. Там, где реки были неглубокие и на дне виднелись желто-
коричневые камни, Мунлайт наклоняла голову и пила. Подкрепив силы, 
Мунлайт снова рвалась вперед, запрокинув голову и закусив удила.
    Питер проголодался. Он остановился на берегу реки, где лежало 
несколько валунов и росла длинная и сочная трава, и решил половить 
рыбу, пока Мунлайт пасется. Он слез с нее, снял седло и уздечку и 
вместе с Громобоем положил их на камень. Потом Питер пошарил в кармане 
в поисках лески, которую всегда носил с собой.
    - Прошу прощения, - раздался чей-то голос. Питер поднял голову и 
увидел, что из-за камня на него смотрит кенгуру.
    - Меня зовут Серая Шкурка. Я спала. Я здесь живу, по крайней мере, 
сплю. А ты, конечно же, - продолжала она, - спишь по ночам. Из-за 
этого ты столько теряешь, что я бы посоветовала тебе переменить эту 
привычку.
    - Но ночью немного увидишь, - возразил Питер.
    - Зато намного больше услышишь! - воскликнула Серая Шкурка. - Ночь 
- самое подходящее время для того, чтобы слушать. Хотя все равно, у 
тебя такие маленькие уши, что я удивляюсь, как ты вообще что-либо 
слышишь.
    Она выпрыгнула из-за валуна и остановилась перед Питером, а затем 
оперлась на хвост и стала на нем раскачиваться взад-вперед, словно в 
кресле-качалке.
    - Видишь, какая я везучая, - объяснила она. - Кресла-качалки нынче 
в большой моде, а у меня есть свое собственное, не отделимое от меня. 
Если хочешь - можешь сесть ко мне на колено и покачаться вместе со 
мной, - дружелюбно предложила она.
    - На твоей коленке не очень-то посидишь, - отвечал Питер. - У тех, 
у кого колени вывернуты наоборот, сидеть просто не на чем.
    - Это верно, - согласилась Серая Шкурка. - Невозможно иметь все. 
Но по крайней мере я была достаточно учтива, и предложила тебе 
покачаться. А теперь скажи, чем ты так озабочен?
    - Да, в общем-то, ничем.
    - А вот и нет, озабочен. Я знаю, ты - голоден.
    - О, это - да.
    - А что бы ты хотел съесть? Назови любое блюдо. Вспомни свое 
любимое кушанье.
    Питер подумал: "жареные колбаски с картофельным пюре, политые 
томатным соусом, да побольше. Чашку чая с тремя ложками сахара и 
мороженое".
    - Пожалуйста, - тотчас отозвалась Серая Шкурка. Она опустила лапу 
в свою сумку и достала оттуда стол и стул. Потом она извлекла оттуда 
скатерть, потом - ножи, ложки, вилки, перец и соль, потом бутылку 
томатного соуса. Наконец, махнув лапкой и отвесив церемонный поклон, 
достала тарелку с колбасками и пюре, чашку дымящегося чая и розетку 
мороженого.
    - А себе, - сказала она, - я, пожалуй, возьму пару пучков травы, 
которую мы, кенгуру, очень любим, и несколько листочков Acacia 
dumrosa.
    Она два раза опустила лапку в сумку, и ее лакомство тоже оказалось 
на столе.
    - А теперь придвинь стул и начинай есть, - сказала она.
    Питер был потрясен.
    - Вот это да! Даже не верится!
    - А я вовсе и не собираюсь тебе ничего доказывать, - сказала Серая 
Шкурка. - Постарайся просто поверить мне.
    - Я тебе верю, - отвечал Питер. - Ты совсем не похожа на врунишку.
    - Ты прав, я не люблю врать. Но у меня была такая тяжелая жизнь, 
что я перестала доверять людям.
    - От этого я тебя вмиг вылечу, - произнес Питер и вручил Серой 
Шкурке Волшебный Лист из маленькой сумочки, которая висела у него на 
шее. - Ну, и что ты теперь чувствуешь?
    - Я чувствую, что меня словно наполнило радостью, - ответила Серая 
Шкурка. - И еще - гордостью... Я отнюдь не стала гордячкой, просто я 
испытываю чувство гордости, понимаешь? - Она взглянула на Волшебный 
Лист и улыбнулась. - Кроме того, я чувствую себя очень важной особой. 
- Секунду поколебавшись, она закончила: - Как будто меня многие любят.
    - Именно это ты и должна была почувствовать, - подтвердил Питер.
    - А где ты взял такой Лист?
    - У Южного Ветра. Он сказал, что Лист поможет мне в пути.
    - Как это мило с твоей стороны, что ты дал этот Лист мне, - 
сказала Серая Шкурка. - Хочешь добавки?
    - Я еще и этого не съел.
    - Ну, как знаешь! А то ведь оттуда можно еще много чего достать, - 
заверила его Серая Шкурка и принялась жевать любимую траву кенгуру.
    На тарелке Питера оставалось еще порядочно, и когда он наконец 
съел все, то заволновался: "Пожалуй, не стоит больше есть колбасок, а 
то не останется места для мороженого".
    - Вот незадача, - посочувствовала Серая Шкурка. - Я очень хорошо 
тебя понимаю. Попробуй попрыгать вокруг стола. Тогда колбаски 
утрясутся и сверху появится место для мороженого.
    Питер пропрыгал вокруг стола три раза.
    - Ну как, помогло? - спросила Серая Шкурка, когда он снова сел.
    - Еще как, - ответил Питер и без всякого труда съел мороженое.
    - Наука - это, без сомнения, замечательная вещь, - заключила Серая 
Шкурка. - А теперь, - добавила она, - надо все перемыть. Я буду мыть, 
а ты - вытирать.
    Она достала из сумки тазик с горячей водой, потом - кухонное 
полотенце и протянула его Питеру: - Вытирай все, что я буду тебе 
передавать, а потом бросай в мою сумку, - говорила она. - Ножи опускай 
ручкой вперед, вилки тоже. В конце концов, я ведь живое существо и 
могу порезаться и уколоться так же, как и ты.
    - Я буду осторожен, - пообещал Питер, совершенно уверенный, что 
все эти предметы никак не поместятся в одной сумке. Но и ножи, и 
вилки, и чашки, и тарелка исчезали там одна за другой, проваливаясь в 
сумку, как письма в почтовый ящик. Даже стол со стулом съежились и 
исчезли, словно их никогда и не было.
    - Не могу понять, почему ты не толстеешь? - удивился Питер. - И 
куда все это девается?
    - Ты задал интересный вопрос, - сказала Серая Шкурка, приоткрыв 
лапками сумку и заглянув внутрь. - Я часто сама ломаю голову: 
действительно, куда? Но прелесть волшебства в том и состоит, что не 
все в нем должно быть нам понятно. Волшебство просто происходит - и 
все.
    - А когда оно у тебя появилось? Когда ты была еще совсем маленькой 
девочкой? - спросил Питер. Он почему-то все время считал Серую Шкурку 
существом, подобным ему, то есть человеком.
    - Хочешь, я расскажу тебе о своей жизни?
    - Хочу.
    - Тогда слушай, - начала Серая Шкурка и снова принялась 
раскачиваться на хвосте. - Я родилась в Кастлмейне, в бедной, но 
честной семье.
    - Кажется, что-то похожее я уже где-то слышал. - Питер 
сосредоточился, пытаясь вспомнить, где.
    - Очень может быть, ничто не ново под луной. Бедные, но честные 
родители встречаются довольно часто, и в том же Кастлмейне рождаются 
многие кенгурята. - Тут Серая Шкурка махнула лапкой, чтобы ее больше 
не перебивали.
    - Родители гордились мной, потому что я была такой, как все 
малыши. Вот если бы я была похожа на лошадь, например, они бы мной 
вовсе не гордились.
    - Это точно, - кивнул Питер.
    - Я ничем не выделялась, и никто меня не сторонился, пока однажды, 
запустив руку в свою сумку, я не вытащила оттуда пучок раннего лука. С 
тех пор я, как белая ворона, всеми отвержена и несчастна.
    - А что ты сделала с тем луком? - заинтересовался Питер.
    - Съела.
    - Я тоже люблю ранний лук, - признался Питер.
    Серая Шкурка опустила лапку в сумку и достала пучок лука.
    - На, возьми, - сказала она. - Попался серебристый. А вообще-то он 
бывает разный.
    Питер сунул лук в карман.
    - Мне пришлось переселиться в другой район, - продолжала Серая 
Шкурка. - Другие кенгуру не могли простить, что у меня волшебная сумка 
и я могу достать из нее все, что пожелаю, даже рояль. Я так от всех 
них отличалась, что они стали поговаривать, будто я не от мира сего. А 
когда я стала ходить в школу, мне не пришлось покупать ни карандаши, 
ни резинки, ни учебники. Я просто доставала их из своей сумки. А 
другие дети думали, что я ворую их и прячу. Вот они и сказали 
родителям, что я воришка. Мама-то знала, что я никогда ничего не 
украду, но и она, бывало, поговаривала: "Жаль, что ты не такая, как 
все". Потом кенгурята стали бросать в меня камнями, и никто не хотел 
со мной играть...
    - Бедняжка! - посочувствовал Питер.
    - Да! - всхлипнула Серая Шкурка. - Мне тоже стало себя жалко, и в 
конце концов я решила уйти. Кенгуру не любят, когда кто-то из них 
отличается от остальных. А я отличалась.
    - Скажи, а ты из своей сумки можешь вытащить все-все? - 
недоверчиво спросил Питер.
    - Абсолютно все.
    - И слона?
    - Смотри! - Серая Шкурка по локоть запустила лапку в сумку, изо 
всех сил за что-то дернула, и оттуда вылетел слон. Несколько секунд 
он, пошатываясь, в полнейшем замешательстве крутился на месте, затем 
прислонился к дереву, все еще тяжело дыша.
    - Что это значит? Где я? - снова и снова повторял он. - Мне плохо! 
Меня тошнит!
    - Ты в Австралии, - объяснила Серая Шкурка.
    - Нет! - взревел слон. - Только не в Австралии! Это же на другой 
стороне Земли! Немедленно отправьте меня домой!
    - Кончай распускать нюни! - строго приказала Серая Шкурка. - Где 
твое достоинство? Что бы сказала твоя мама, если б она тебя сейчас 
увидела?
    - Она бы спросила, как меня угораздило сюда попасть, - ответил 
Слон. - Я никогда никуда не хожу без спроса. Вы же знаете, что мамы 
вечно беспокоятся. Я преспокойно гулял себе по Африке и срывал в лесу 
бананы...
    - Но бананы в африканских лесах не растут, - возразил Питер.
    Слон вдохнул побольше воздуха. Он был единственным ребенком в 
семье, и мог гулять допоздна, а уж чтоб ему возражали...
    - Ты-то что знаешь об этом? - завопил он грозно. - Ты сам-то был 
там хоть когда-нибудь? - Он отошел от дерева и теперь стоял прямо 
перед Питером и Серой Шкуркой.
    - Нет, ни разу, - вынужден был признать Питер, уже жалея, что 
вообще ввязался в спор.
    - То-то и оно! - торжествующе воскликнул слон. - Итак, повторяю, я 
ел бананы, когда вдруг огромная волосатая, когтистая лапища схватила 
меня за ногу и потащила.
    Серая Шкурка поднесла свою лапку к глазам и, хмурясь, осмотрела 
ее. "Идиотское описание", - пробормотала она.
    - Я отбивался как мог, - продолжал слон, - но не успел и глазом 
моргнуть, как меня протащили сквозь какой-то мешок, покрытый изнутри 
мехом, и вот я здесь. Это просто безобразие! До чего же мы все 
докатимся в таком случае, хотел бы я знать? Я всегда был честным и 
послушным! Возвращайте меня немедленно домой!
    - Это естественное желание, - согласилась Серая Шкурка. Протянув 
лапку, она схватила слона за ногу и что есть силы дернула, так что тот 
головой вперед упал в ее сумку и исчез.
    - Надеюсь, это послужит тебе уроком, - сказала она, обернувшись к 
Питеру. - С волшебными сумками лучше не шутить. Если бы ты попросил 
меня достать из сумки льва, то мы бы уже давно переваривались в его 
желудке.
    Эти слова испугали Питера.
    - Не смей даже думать об этом! - закричал он. - Может быть, одной 
твоей мысли о льве достаточно, чтобы он выскочил оттуда. Иметь такую 
сумку просто опасно.
    - Вовсе нет, - возразила Серая Шкурка, - из моей сумки ничто не 
выходит само по себе, - я все должна вытаскивать своими лапками. 
Лучшего способа защиты и не придумать. Ну, да ладно, а теперь расскажи 
мне о своей жизни.
    Питер рассказал о старом Мике, который так здорово объезжает диких 
коней, о своей Мунлайт и о том, как они носились, обгоняя бурю. 
Рассказал, что сам он ищет Прекрасную Принцессу.
    - Для мальчика твоего возраста ты рассказываешь просто 
превосходно, - заметила Серая Шкурка. - Ты должен как-нибудь 
использовать этот дар. Из тебя вышел бы, например, замечательный 
погонщик скота.
    - Почему именно погонщик?
    - А почему бы и нет?
    - Но ведь погонщики зарабатывают не тем, что рассказывают 
удивительные истории.
    - Конечно, не тем, - согласилась Серая Шкурка, - но ты представь, 
как счастлив был бы погонщик, умей он так хорошо рассказывать. А чем 
плохо быть счастливым погонщиком?
    - Ты права, - задумчиво проговорил Питер, которому раньше такая 
мысль в голову не приходила. - Я мог бы стать погонщиком. Я вполне мог 
бы стать счастливым погонщиком. Только сначала я должен найти свою 
Прекрасную Принцессу.
    - Тогда пошли, - предложила Серая Шкурка. - Мы только теряем время 
на разговоры.
    - Разве ты пойдешь со мной? - удивился Питер.
    - А как же, - сказала Серая Шкурка. - Ты дал мне Волшебный Лист, и 
теперь я знаю, что нужна тебе. Я буду тебе другом до гробовой доски, 
так что лови Мунлайт и поехали.
    Питер пошел за своей любимой старой фетровой шляпой, которую 
оставил на земле около камня, но там ее не оказалось. На этом месте 
лежала изумительная голубая шляпа с лентой и страусовым пером, которое 
изящно изгибалось назад, как на шляпах принцев.
    - Это же шляпа Принца! - воскликнула Серая Шкурка. - Но принцев 
здесь нет...
    Питер надел ее, и она пришлась ему как раз впору.
    - Знаешь, - сказал он, - я уверен, что эта шляпа предназначена для 
меня. Суди сама: я подарил тебе Волшебный Лист и сделал тебя 
счастливой. А ведь когда даришь счастье другому, всегда получаешь что-
нибудь сам. По-моему, это что-то вроде награды.
    - По-моему, тоже, - согласилась Серая Шкурка.

        Глава 3
        ДОЛИНА ЦЕПЛЯЮЩЕЙСЯ ТРАВЫ

    - До чего же здорово вот так путешествовать, - говорила Серая 
Шкурка, прыгая рядом с Питером, который ехал верхом. - Прыгая вверх, я 
делаю вдох, приземляюсь - выдох, когда я вверху - я вижу все на многие 
мили вокруг. Смотри! - она подпрыгнула так высоко, что оказалась выше 
Питера, который натянул поводья, придерживая лошадь.
    - У тебя это прекрасно получается! - крикнул он. - А теперь 
поскачем быстрее.
    - Давай! - согласилась Серая Шкурка. - Мы с тобой летим. Я вижу на 
милю вперед. Я вижу, что там за холмом.
    Они взлетели на холм и остановились на вершине.
    - Что-то у меня тут закололо, - пожаловалась Серая Шкурка, держась 
за бок. - Я запыхалась. Надо передохнуть. Мы давали не меньше сорока 
миль в час, и это в гору!
    - Что за странная долина! - воскликнул Питер, глядя вперед, где 
перед ними лежало широкое поле сухой травы. Трава колыхалась от ветра, 
по ней скользили какие-то тени. До холма, где стояли Питер и Серая 
Шкурка, доносилось ее холодное, режущее слух шуршание, похожее на 
шипение змей или на чей-то неустанный шепот.
    Серая Шкурка поежилась.
    - Не нравится мне это место, - сказала она. - Я здесь уже была. 
Это Долина Цепляющейся Травы, которая непрестанно что-то шепчет. Дети 
всего мира, перед тем как стать взрослыми, обязательно должны пройти 
через эту долину. Те, у кого родители умные, проходят ее легко. Но 
большинству здесь приходится туго.
    - Давай как-нибудь увильнем, - предложил Питер. - Давай обойдем 
ее.
    - Не получится, - ответила Серая Шкурка. - Прекрасная Принцесса 
живет как раз по ту сторону Долины, так что нам, хочешь, не хочешь, 
придется ее пересечь. Трава будет цепляться за ноги, пытаться повалить 
на землю, а если ты схватишь ее рукой, поранит тебе пальцы. Знаешь, 
есть такая трава, о которую можно порезаться, если попытаться ее 
сорвать. Так вот, эта - такая же. Она ужасная. К тому же вечно что-то 
шепчет.
    - Что же она шепчет?
    - Это зависит от того, с кем она говорит. Особенно она жестока к 
детям несчастным, обиженным, одиноким и к тем, кто не верит в себя. 
Она тянет их вниз, режет им коленки, да так, что шрамы остаются на всю 
жизнь. Если прислушаться, можно различить ее шепот: "Ты - слишком 
толстый, ты - слишком худой, ты - дылда, ты - коротышка. Расправь 
плечи. Почему ты такой глупый? Подожди, дома я тебе задам. Вот дойдет 
до отца, тогда узнаешь. Ты лентяй. Ты эгоист. И врун. Почему ты так 
плохо учишься? Почему ты хуже соседских детей? Сделай то, сделай это, 
иди сюда, пойди туда, чтоб тебя было видно, но не слышно, слушай 
старших, подчиняйся, соглашайся..."
    - Трава начинает шептать, стоит ребенку лишь подойти к ней, - 
продолжала Серая Шкурка. - Она сводит детей с ума, режет им ноги, и 
следы от этих порезов не затягиваются и не исчезают.
    Пока они разговаривали, из-за кустов у подножья холма вышла 
группка детей. Все они были в школьной форме, и у каждого за спиной 
висел ранец с учебниками. Маленьким было лет по семь, старшим - лет по 
четырнадцать. Они остановились у края Долины и прислушивались к шепоту 
травы. Идти по полю они боялись, оно казалось им просто бескрайним.
    - Знать бы, как им помочь, - сказал Питер, Его внезапно охватила 
ненависть к траве, и он живо представил себе, как скосил бы 
газонокосилкой все это поле, чтобы трава никогда больше не цеплялась 
за ноги ребят.
    Серая Шкурка догадалась, о чем он думает,
    - Ее не скосить, - сказала она. - Я однажды пыталась, но трава 
вырастает быстрее, чем ты ее срезаешь, а уж я-то умею обращаться с 
косой.
    - Стоя тут, мы вряд ли им поможем, - сказал Питер.
    Он подбежал к лошади, вскочил на нее и галопом погнал вниз по 
склону. Серая Шкурка - за ним. Они мчались меж деревьев и скал, 
перепрыгнули через мелкий ручей и устремились к детям, стоявшим на 
краю поля под красным эвкалиптом. Многие дети плакали. Самый маленький 
мальчик порезал об острую траву руку и обмотал ее носовым платком: на 
платке виднелись пятна крови. Другой мальчик сильно порезал себе 
коленки, а у одной девочки на лбу был синяк - она ударилась, 
споткнувшись о невидимый в траве камень. Дети были напуганы и стояли, 
сбившись в кучку. Питер осадил лошадь возле них; они смотрели на 
кенгуру, как на врага.
    Вдалеке, посреди долины виднелась еще одна группа детей. Те 
решились проложить себе путь через Цепляющуюся Траву, но, чем дальше 
они углублялись в сплетенную гущу стеблей и листьев, тем труднее 
становилось им идти. Трава неистово колыхалась вокруг них, из ее 
сплошной переплетенной массы вытягивались маленькие серые пальцы и 
раздирали детям руки и ноги. Постоянный шепот все нарастал, пока не 
перешел в громкое шипение, как у тысячи змей, когда они то бросаются 
вперед, то отступают.
    По траве, поднимаясь и опускаясь серыми волнами, проносились 
темные тени, напоминающие тени от облаков. Становилось дурно от одного 
взгляда на эти сухопутные волны, которые, в отличие от волн морских, 
не освежали душу.
    "Почему ты не можешь сдать экзамены? - шептала детям потревоженная 
трава. - Надо больше заниматься и меньше играть. Надо трудиться 
упорней. Кто не сдаст экзамены - останется без работы. Ты уже слишком 
большая, чтобы играть в куклы и прочие детские игры. Подумай о 
будущем".
    Трава становилась все злее и злее, и дети стали падать и 
барахтаться в объятиях листьев и стеблей. Питер не мог больше ждать. 
Детям нужно было дать Волшебный Лист. Мальчик пустил лошадь в галоп, и 
она помчалась вперед длинными прыжками. В руке Питер держал Волшебный 
Лист, и трава, тянувшаяся к нему, съеживалась и увядала. Колышащиеся 
стебли расступались под копытами Мунлайт, увядая и шипя, словно их 
что-то сжигало.
    Когда Питер нагнал детей, трава отпрянула от них и безжизненными 
плетьми легла у их ног. Ее шепот умолк.
    Питер соскочил с лошади и вручил каждому ребенку по Волшебному 
Листу. Дети прекратили плакать и стали улыбаться.
    - Вам больше нечего бояться, - сказал Питер. - Продолжайте свой 
путь. Пока у вас в руках будут эти Волшебные Листики, ничего плохого с 
вами не случится.
    Дети побежали вперед, смеясь и танцуя. Питер проводил их взглядом, 
пока они не достигли края ноля, а потом вскочил на Мунлайт и вернулся 
к детям, которых оставил под красным эвкалиптом.
    - Скажите, вы добрые, и ты, и кенгуру? - шепотом спросила одна 
девочка, державшая за руку брата.
    - Пожалуй, да, - ответил Питер. - Во всяком случае, мы пришли сюда 
помочь вам.
    Тут все дети заулыбались и перестали бояться.
    - А мне ты дашь такой лист? - спросила девочка, у которой все лицо 
было покрыто веснушками.
    - Конечно. Я всем дам по листу.
    - Скажи мне, - спросила Серая Шкурка девочку, - мама тебя любит?
    - Ну, конечно! Но она любила бы меня еще больше, если бы не эти 
веснушки. Она все время мне говорит: "Как жалко, что у тебя веснушки!"
    - Ах, вот оно что! - сказала Серая Шкурка, и, немного подумав, 
добавила: - По-моему, веснушки - это даже очень красиво.
    - По-моему, тоже, - согласился Питер и дал девочке Волшебный Лист.
    Она зажала Лист в руке. И тут же веснушки на ее лице стали совсем 
незаметными, а само оно так изменилось, будто с души слетела тень, и 
вместо нее засверкало солнце, отражаясь в глазах.
    - И я хочу такой лист, - произнес мальчик, который стоял, понурив 
голову, - он стеснялся смотреть на Питера.
    - А что говорит твой отец?
    - Он все время твердит, что я неудачник, а я не знаю, что это 
такое. По-моему, он жалеет, что я расту не таким, как он. Я хочу быть 
художником, а он твердит, что все художники не от мира сего.
    - О боже! - воскликнула Серая Шкурка и зашептала Питеру на ухо: - 
Именно так и про меня все говорили, когда я вынула из сумки пианино. 
Понимаешь, я люблю играть на пианино. Меня уверяли, что я играю Шопена 
с большим чувством, уверяли люди, которые любят музыку.
    - А я умею играть "Дом родной" на губной гармошке, - сказал Питер. 
- Старина Мик говорил, что у него от моей игры даже слезы на глаза 
навертываются.
    - Вот это успех! - воскликнула Серая Шкурка. - Я бы тоже хотела 
делать что-нибудь такое, что заставляло бы людей смеяться, плакать или 
танцевать.
    - А лист-то мне дадите? - спросил мальчик, который уже начал 
беспокоиться.
    - О, прости меня, - сказал Питер. Он вынул из маленькой сумки 
Волшебный Лист и вручил его мальчику, который вдруг поднял голову и 
улыбнулся. - Больше ты не будешь ходить, понурив голову.
    - А у меня для тебя тоже есть подарок, - сказала Серая Шкурка. Она 
засунула лапку в карман и вытащила оттуда альбом и набор красок. В нем 
были краски двадцати четырех цветов и еще две кисточки.
    - Теперь, когда у тебя есть Волшебный Лист, - сказала она, - ты 
нарисуешь удивительные картины. Покажи их завтра отцу, они ему 
понравятся.
    Все дети просили Волшебный Лист. Питер быстро раздал листья, и 
вскоре ребята почувствовали, что больше не боятся Цепляющейся Травы.
    - Теперь вы сможете пройти по этому полю, - сказал Питер. - Я 
провожу вас.
    С гиканьем дети побежали по траве, и бледные стебли в ужасе 
расступались перед ними, образуя проход. Он протянулся через все поле, 
а там, по ту сторону поля, высились зеленые холмы, протекали речки и 
светило яркое солнце. Дети бросились бежать по свободному проходу, 
потом остановились и помахали Питеру и Серой Шкурке.
    Именно в этот момент Питер заметил, что у него на ногах появились 
замечательные сапоги, каких он никогда в жизни не видел. Они были из 
тончайшей кожи, а широкие отвороты доходили до колен. Питер с 
удивлением их разглядывал.
    - Не вижу ничего удивительного, - сказала Серая Шкурка. - 
Волшебство постепенно превращает тебя в принца. Это Волшебный Лист 
награждает тебя, когда ты приносишь людям добро. Интересно, что бы 
случилось, если бы ты вдруг стал зазнайкой и эгоистом?
    Питеру тоже было интересно - что.

        Глава 4
        СЕРАЯ ШКУРКА СРАЖАЕТСЯ С ВЕЛИКАНОМ

    Серая Шкурка и Питер прошли совсем немного, когда далеко на юге 
стали собираться черные тучи. Раздались отдаленные раскаты грома, 
усилился ветер. Когда тучи приблизились, ударила молния. Огненные 
зигзаги ринулись в провалы между облаками и устремились к деревьям. За 
этим последовали оглушительные удары грома, и деревья в смятении 
зашумели кронами.
    - Дождь может пойти с минуты на минуту, - произнес Питер. - 
Поскакали вон к тем скалам впереди. Может быть, там мы найдем пещеру.
    - А не найдем, так придется помокнуть, - сказала Серая Шкурка.
    Они свернули с дороги и, продираясь через густые заросли 
вечнозеленых кустарников, напрямик помчались к стеной возвышающимся 
скалам. К счастью, им удалось обнаружить тропу, проложенную вомбатами. 
По ней они дошли до прогалины у подножья скал и прямо перед собой 
увидели огромную пещеру, в которой, наверное, жили многие поколения 
вомбатов.
    Когда они уже входили в укрытие, начали падать первые крупные 
капли дождя. В пещере было сухо, на песчаном полу валялись камни, на 
которых можно было сидеть.
    - Мы успели как раз вовремя, - сказала Серая Шкурка, встряхиваясь. 
- Пожалуй, здесь можно заночевать.
    Мощный удар грома заглушил ее слова, и молния осветила пещеру, 
словно в нее вдруг прорвалось солнце. В пещеру залетали капли дождя, и 
вместе с ними влетел Южный Ветер, стряхивая брызги с волос.
    - Я так и думал, что найду вас где-нибудь здесь, - сказал он, 
усаживаясь.
    - Честное слово, я рад тебя видеть! - воскликнул Питер.
    - Я следил за тобой, - сказал Южный Ветер. - Да и мои братья тоже 
о тебе рассказывали. Деревья в этих местах давно уже жаждут дождя, вот 
я и решил навестить тебя, заодно напоив буш. Видал, какой я принес 
ливень, а? Деревьям он понравится. Но скоро мне придется уйти. Нужно 
промчаться еще сотни миль, прежде чем эта громада облаков иссякнет. 
Послушай-ка, я скажу тебе кое-что. Ты вступил в земли великана Ярраха.
    - А кто это?
    - Это исполин, о котором я тебе говорил. Он нападает на всех, кто 
проходит по его землям.
    - И что же он с ними делает - убивает?
    - Нет, не убивает. Он всех хватает и запихивает в маленькие 
коробки, а продержав там несколько месяцев, выпускает. Однако к этому 
времени все люди становятся одинаковыми. Они разучиваются думать сами, 
и могут только повторять, что они читали или слышали. Нет ничего 
ужаснее, чем стать такими, поэтому я пришел предупредить тебя: будь 
осторожен.
    - А он очень большой?
    - Ростом он с гору, а ботинки у него длиной в пять футов.
    - А что он делает с кенгуру? - поинтересовалась Серая Шкурка. - 
Неужели их тоже запихивает в коробки?
    - К сожалению, да.
    - Но меня нельзя запихнуть в маленькую коробку, - не соглашалась 
Серая Шкурка, - ведь тогда сломается мой хвост. - Ее стала беспокоить 
подобная перспектива. - Никакому великану не позволено ломать мой 
хвост.
    - Можно сделать отверстие внизу коробки и просунуть его туда, - 
предложил Питер.
    - Мне это совсем не нравится, - ответила Серая Шкурка. - Хорошо 
тебе говорить. Ты можешь запросто подтянуть колени к подбородку и 
легко поместиться в коробке, а у меня нет коленей.
    - Что ж, придется как-нибудь обмануть этого великана, - сказал 
Питер и обратился к Южному Ветру:
    - А Волшебный Лист разве не поможет нам превратить этого великана 
в доброго?
    - Помочь-то он поможет, - ответил Южный Ветер, - но тебе будет не 
просто заставить великана взять этот лист. Вспомни о его росте. А 
когда ты окажешься у него в коробке, так и подавно не сможешь это 
сделать. Тебе надо постараться поставить его в такое положение, чтобы 
он не смог отказаться от твоего подарка.
    - Попробую, - сказал Питер.
    - Да, попробуй, и по-моему, ты сумеешь это сделать. Конечно, 
сумеешь. Тот, кто дерзает и не сдается, всегда своего добивается. А 
теперь мне пора.
    Ударил гром, блеснула молния, Южный Ветер снова оказался среди 
облаков и погнал их на север. Через несколько минут уже светило 
солнце.
    И Питер, и кенгуру проголодались. Мунлайт тоже нужен был отдых и 
еда. Серая Шкурка достала из сумки несколько охапок сена и бросила их 
лошадке, которая тут же принялась его жевать. Питеру Серая Шкурка 
достала кусок мяса и пирог с почками, себе - тоже немного сена.
    Через час они собрались покинуть пещеру, чтобы продолжить 
путешествие. Питер подвел свою лошадку к выходу, но вдруг остановился. 
Две мощные ноги загородили ему путь.
    - Великан! - прошептала Серая Шкурка, тоже увидевшая ноги. Они 
отступили в глубь пещеры и затаились. Было слышно, как великан топает 
снаружи. Он нюхал воздух и говорил сам с собой.
    - Я чую, чую людей. Они наверняка прячутся где-то здесь. Следы, 
ведущие в пещеру, принадлежат лошади. А вот следы кенгуру.
    Было слышно, как великан сопит, опускаясь та землю. Его пальцы 
начали отдирать куски земли, чтобы увеличить отверстие, потом в пещеру 
просунулась огромная рука, она начала медленно шарить в пустоте. 
Могучие пальцы ощупывали потолок и стены пещеры, словно что-то 
отыскивая.
    Раньше Питер часто засовывал руку в кроличьи норы и ловил 
кроликов. Теперь он понял, что должны были чувствовать кролики. Он 
приготовил и распрямил уложенный кольцами Громобой. Описав им над 
головой несколько кругов, Питер резко опустил кнут, так что он щелкнул 
прямо по большому пальцу великана, содрав с него кожу. Показалась 
кровь.
    Великан взвыл от боли и быстро убрал руку. Было слышно, как он 
сосет палец и шепчет: "Наверное, там змеи".
    Потом стало слышно, что великан опускается на землю, чтобы 
заглянуть в пещеру, и вот уже его глаз загородил весь проем. Глаз был 
огромный, его ресницы напоминали камыши у берега озера. Питер мог бы 
выбить ему глаз одним ударом Громобоя, но он не мог заставить себя 
поступить так жестоко. Глаз, не мигая, смотрел прямо на них.
    - Ага! - с удовлетворением произнес великан.
    Но это радостное восклицание вдруг перешло в крик от боли. Это 
Питер, раскрутив кнут, ударил им великана по щеке, так что выступила 
кровь. Великан отпрянул, затем, судя по звукам, поднялся и поплелся к 
ручью.
    - Он хочет промыть щеку, - сказала Серая Шкурка.
    - Наверное, - согласился Питер, - но ручей отсюда недалеко. Мы 
пройдем в глубь пещеры и будем устраиваться на ночлег. Становится 
поздно, а в темноте вручить великану Волшебный Лист все равно 
невозможно.
    Они крепко заснули в глубине пещеры. Когда рассвело и запели 
первые птицы, мальчик и кенгуру уже завтракали, собираясь продолжить 
путешествие.
    Серая Шкурка волновалась.
    - Побудь здесь, пока я огляжусь, - сказала она. Выбравшись на 
солнце, она остановилась, принюхалась, потом пропрыгала взад-вперед 
перед пещерой и вернулась.
    - Кажется, все в порядке, - сказала она. - Пожалуй, можно рискнуть 
идти дальше.
    Питер вскочил в седло, и они тронулись. Они уже почти прошли 
поляну, как вдруг услышали глухой удар и в испуге оглянулись. Великан 
Яррах спрыгнул со скалы, за которой он прятался всю ночь. Увидев 
беглецов, он захохотал от удовольствия. Великан отрезал им путь в 
пещеру, а спасаться в зарослях буша было бесполезно - великан догнал 
бы их, сделав всего несколько шагов. Так Питер и Серая Шкурка попали в 
ловушку. Великан подошел к ним, расставив руки и скрючив пальцы.
    - Кто посмел войти в мой лес? - прогремел он. Схватив за корень 
древовидную акацию, он вырвал ее из земли и швырнул в Питера и Серую 
Шкурку. Дерево просвистело в воздухе, крутясь и переворачиваясь. 
Ударившись о землю недалеко от них, оно разлетелось на куски, щепки и 
ветки. Большая ветка пролетела над головой кенгуру. Серая Шкурка едва 
успела пригнуться, и ветка воткнулась в землю.
    - Видишь? - крикнула она Питеру. - Он хотел убить нас. - Кенгуру 
не на шутку разозлилась. - Смотри за мной. Я ему сейчас покажу.
    Большими прыжками она поскакала к великану, каждый раз подпрыгивая 
на тридцать футов. Приблизившись настолько, что ее почти доставали 
скрюченные руки великана, она с огромной силой оттолкнулась от земли. 
Она взлетала все выше и выше, пока не допрыгнула до его груди, и тогда 
она вытянула вперед свои могучие задние лапы и со страшной силой 
ударила великана в грудь, причем ее пальцы глубоко вонзились в плотную 
ткань его рубашки. Оттолкнувшись от мощной груди великана, как от 
трамплина, она еще раз подпрыгнула и сделала в воздухе сальто. При 
этом ее тяжелый хвост, описав вокруг нее дугу, ударил великана по лицу 
с силой брошенного бревна.
    От удара голова великана качнулась назад. Он взвыл от боли и 
растопырил руки, пытаясь удержать равновесие, потерянное еще после 
первого удара в грудь, но не смог устоять на ногах и рухнул назад. Он 
повалился, как валится могучее дерево, и, ударившись огромной головой 
о вершину скалы, растянулся на лужайке лицом вверх. С вершины с 
грохотом скатились два валуна, выбитые со своих мест ударом мощного 
затылка. Сам великан неподвижно лежал на земле, оглушенный ударом.
    - Надо сковать ему руки, пока он не пришел в себя, - крикнула 
Серая Шкурка.
    Питер бросился ей помогать. Соскочив с седла, он схватился за 
гигантские наручники, которые Серая Шкурка как раз вытаскивала из 
своей сумки. Наручники со звоном упали на землю и тут же, на глазах, 
стали так быстро увеличиваться в размерах, что вскоре ни Питер, ни 
Серая Шкурка уже не могли их поднять и надеть на запястья великана, - 
такими тяжелыми они стали.
    Питер посмотрел на могучую руку, лежащую в пяти ярдах от него. Она 
напоминала ствол дерева, и была такой же тяжелой. Ладонь была 
раскрыта, и пальцы торчали вверх, как шесты. По такой ладони Питер мог 
бы проехать на коне.
    Положение было безнадежным. Казалось, не было никакой возможности 
сковать великана, пока он не придет в себя, а до тех пор Волшебный 
Лист ему никак не вручить.
    - Нам может помочь только один человек на свете, - сказал Питер, - 
крановщик. Он знает, как обращаться с тяжелыми грузами, и сумеет 
заковать великана в кандалы.
    - А это идея! - воскликнула Серая Шкурка. Она засунула лапку в 
карман и рывком вытащила оттуда портового крановщика. Он был весь в 
муке: очевидно, его выхватили прямо из трюма.
    - Как меня сюда занесло? - сердито заговорил он. - Почему какая-то 
кенгуру дергает меня за ногу?
    - Сейчас некогда заниматься объяснениями, - ответила Серая Шкурка. 
- Ты вернешься на свой корабль, как только сделаешь для нас одно 
дельце. Вон видишь лежащего великана?
    - Я в великанов не верю, - сказал крановщик. - И этот меня не 
убеждает.
    - Посмотри, вон на земле рядом с тобой лежит кисть его руки. А 
дальше - его локоть.
    - Это пластик, - упорствовал крановщик. - Вы, наверное, сделали 
этого великана для какого-нибудь представления или чего-то в этом 
роде.
    - А ты дотронься до его руки, - не унималась Серая Шкурка.
    Крановщик подошел к руке великана и дотронулся до нее.
    - Разрази меня гром, - воскликнул он, - но этот пластик теплый!
    В это время великан глубоко вздохнул и снова замер.
    - Что это? - спросил крановщик.
    - Это великан просыпается, - ответил Питер. - Если ты не поспешишь 
надеть на него наручники, он всех нас убьет.
    - Тогда за дело, - сказал крановщик. - Мне нужен кран.
    Серая Шкурка вытащила из кармана маленький кран. Он начал быстро 
увеличиваться в размерах, пока его стрела не оказалась выше великана.
    - Подними обе его руки и сложи их на груди, - скомандовала Серая 
Шкурка. - Но сначала подними туда меня.
    Крановщик обвязал Серую Шкурку веревкой и поднял ее так, что она 
смогла дотянуться до рубашки великана и по ней залезть ему на грудь. 
Потом крановщик одну за другой поднял обе руки великана и сложил их на 
груди, после чего доставил туда же наручники, которые Серая Шкурка тут 
же защелкнула на гигантских запястьях.
    {whisp01.gif}
    Тем же способом они сковали кандалами и ноги великана. Затем Серая 
Шкурка достала из сумки цепь и затянула ее петлей на его шее. Цепь 
сразу же стала увеличиваться в размерах и в весе. Свободный конец 
крановщик обмотал вокруг скалы, а Серая Шкурка продела в одно из 
звеньев огромный замок и закрыла его.
    Великан Яррах шевельнулся и попытался сесть.
    - Мне пора, - сказал крановщик, прыгнул к Серой Шкурке в сумку и 
тотчас исчез. Кран стал уменьшаться и последовал туда же.
    Великан попытался освободиться.
    - Откуда взялись эти наручники? - ревел он. - Снимите их с меня.
Он дергал руками в разные стороны, пытаясь разорвать наручники. Он 
хотел подняться, но цепь удерживала его в лежачем положении. Он дергал 
руками и ногами, вопил, бился о землю - и так до тех пор, пока силы не 
покинули его. Тогда он затих и только глубоко дышал.
    Этого-то момента Питер и дожидался. Он положил Волшебный Лист на 
ладонь великану, отскочил и стал смотреть. Великан сжал ладонь и взял 
Лист. И тут же с ним начали происходить перемены. Свирепое выражение 
его лица смягчилось, а когда он заговорил, то голос уже не вызывал ни 
у кого страха.
    - Вы меня боитесь? - обратился он к Питеру.
    - Да.
    - Почему?
    - Потому что ты пытался нас убить.
    - Наверное, на меня что-то нашло. Я никогда больше никого не убью. 
Сейчас я хочу помочь вам, вот и все.
    - А засунуть нас в маленькие коробки ты больше не хочешь, нет? - 
спросил Питер.
    - С этим я покончил навсегда, - пообещал великан. - Пойдем вместе 
в замок и выпустим всех, кто там томится.
    Серая Шкурка отомкнула наручники и кандалы, сняла с великана цепь. 
Он сел, потянулся, размял руки и ноги, затем медленно встал и 
осмотрелся.
    - "Как же нам добраться до замка?" - пробормотал он про себя, а 
вслух предложил: - Прыгай на коня, Питер. Я вас всех понесу. Самим вам 
никогда на эти скалы не забраться.
    Когда Питер вскочил на Мунлайт, великан наклонился, поднял их 
обоих и опустил в карман рубахи. Потом он поднял Серую Шкурку и 
опустил ее рядом с ними. Поднявшись на скалу, великан пошел по дороге, 
которая вела к его замку.

        Глава 5
        ГРОЗА В ЗАМКЕ

     Великан шел, тяжело переваливаясь с ноги на ногу. Он был неуклюж 
и время от времени спотыкался. Его корпус раскачивался из стороны в 
сторону, и у Питера и Серой Шкурки возникло ощущение, что они плывут 
на корабле по бурному морю. Они сидели в кармане рубахи, сшитой из 
грубого, как брезент, и такого же прочного материала.
    Питер продолжал сидеть на Мунлайт, которая крепко упиралась всеми 
четырьмя копытами в дно кармана. Серая Шкурка стояла рядом, держась за 
кожаное стремя.
    Питер привстал в стременах, ухватился за верхний край кармана, и, 
оттянув его вниз, смог высунуть голову наружу. Вдали виднелся замок 
великана, вознесший к самому небу свои мощные башни и зубчатые стены. 
Деревья рядом с ним казались кустами.
    Когда они подошли к замку, Питер увидел ворота, к которым они как 
раз направлялись. Ворота имели форму арки, причем эта арка была такой 
огромной, что вполне могла бы служить мостом через морской пролив.
    Питер отцепился от верхней кромки кармана и опустился в седло.
    - Мы подходим к такому огромному замку, какого я в жизни не 
встречал, - сказал он Серой Шкурке. - Когда ты его увидишь, ты не 
поверишь, что это замок.
    - Я уже видела, каков великан Яррах, и теперь поверю чему угодно.
    Пока они так разговаривали, великан вошел в замок, прошел по 
коридору на кухню, где достал из кармана своих пассажиров и опустил на 
пол. Потом он сунул в карман Волшебный Лист, который держал в руке.
    Помещение, в котором они оказались, было похоже на огромную 
пещеру. Как темное небо, раскинулся над ними сводчатый потолок. 
Кухонная плита была размером с дом, в ее топке горели стволы деревьев. 
На полке, сложенной из огромных кусков скалы, обтесанных и уложенных 
как кирпичи, стояли гигантские кастрюли и чайники.
    Мебель в кухне была под стать огромному росту великана. Ножки 
стульев были намного выше Питера, хотя он все еще сидел верхом. Стол, 
сколоченный из струганных стволов деревьев, стоял на ножках толщиной в 
три фута.
    Питер слез с Мунлайт и привязал поводок к одной из этих ножек.
    - Посидите на стуле, пока я готовлю вам еду, - сказал великан. - Я 
хочу потушить говядину с картошкой, и уже припас необходимые продукты 
- двух отличнейших быков и полтонны картошки.
    Серая Шкурка брезгливо поморщилась.
    - Обо мне не беспокойся, - сказала она. - Мне хватит охапки травы.
    - Хорошо, - сказал великан. - Я брошу охапку в жаркое.
    Серой Шкурке чуть было не стало дурно.
    - Я предпочитаю траву в свежем виде, - объяснила она. - Хорошо бы 
только обрызнуть ее холодной водой.
    - А мне жаркого совсем чуть-чуть, - попросил Питер, аппетит 
которого стал понемногу пропадать.
    Питер был бы не прочь сесть на стул, но он просто не мог на него 
забраться. Серая Шкурка догадалась, о чем он думает. Она достала из 
сумки небольшую лесенку и прислонила ее к ножке стула. По ней они оба 
поднялись на сиденье, обитое красным плюшем. Им показалось, что они 
погрузились в шерстяное море или в высокую траву.
    - Этот стул пора косить, - сказала Серая Шкурка. - Мне так и 
хочется достать из сумки газонокосилку и хорошенько здесь пройтись.
    - Смотри, - прошептал Питер. - Великан помешивает жаркое. Меня от 
этого тошнит.
    Великан поставил на плиту огромный котел и помешивал в нем 
деревянным черпаком. Скоро из котла начали подниматься клубы пара. 
Вытяжной трубы в кухне не было, и пар собирался в тучи под сводами 
потолка. Постепенно облачный покров окутал весь потолок, и тучи стали 
яростно носиться по кухне взад-вперед, ища выход.
    Великан Яррах с тревогой посмотрел наверх.
    - Похоже, погода у нас на кухне портится, - объявил он. - В 
комнате сухо, но мы не можем перейти туда, пока мясо не дотушится. 
Правда, - добавил он, - дождя, по-моему, не будет.
    Грянул гром. Замок содрогнулся. Ударившись друг о друга, зазвенели 
чашки на кухонном шкафу. Одновременно с громом блеснул зигзаг молнии и 
ударил в каменную полку рядом с великаном. От удара вывалился один из 
огромных камней, сверкнуло, затем погасло голубое пламя.
    Питер испугался, что молния может попасть в стул, на котором они с 
Серой Шкуркой сидели, и крикнул великану: "У тебя в кухне есть 
громоотвод?"
    - Нет, - ответил великан. - Я никак не ожидал, что из-за моего 
жаркого начнется гроза. Но громоотвод сейчас будет.
    Он отошел от плиты и крикнул в коридор:
    - Джордж, по-моему, я опять вызвал грозу! Ты можешь сейчас же 
придти сюда?
    - Конечно, - ответил низкий голос из какой-то дальней комнаты. - 
Уже иду.
    Яррах вернулся к котлу и гигантской ложкой попробовал жаркое. "Еще 
не вполне готово", - пробормотал он и снова принялся помешивать в 
котле, отчего вверх метнулись новые клубы пара.
    Из коридора послышались тяжелые шаги, и в кухню вошел великан 
среднего роста, в парадном вечернем костюме. Он был в белых резиновых 
перчатках и держал длинный медный стержень с тремя разветвлениями на 
концах.
    - Это Джордж, - представил его Яррах. - Он и есть мой громоотвод. 
Когда я занимаюсь стряпней, он всегда готов придти и немного 
поотводить молнии.
    Джордж поклонился, улыбаясь. Гром загрохотал как барабанная дробь, 
и Джордж стал размахивать ему в такт своим медным стержнем, как 
дирижерской палочкой. Когда из туч под потолком ударила молния, Джордж 
успел поймать ее медным стержнем. Пройдя через стержень, молния с 
треском вышла с другого конца, вылетела за дверь, пронеслась по 
коридору, повернула за угол и, оказавшись на свободе, уткнулась в 
землю.
    Джордж, улыбаясь, повернулся и отвесил поклон Питеру и Серой 
Шкурке, которые смотрели на него с изумлением.
    - Как ты думаешь, надо ему похлопать? - спросила Серая Шкурка.
    - Похоже, он этого ждет, - ответил Питер.
    Они оба захлопали в ладоши, и Джордж поклонился еще раз.
    {whisp02.gif}
    Вдруг блеснула новая молния и опалила Джорджу брюки. Он быстро 
повернулся и стал с бешеной скоростью отводить молнии, которые 
посыпались одна за другой. Джордж рывками переводил свой инструмент из 
стороны в сторону, перехватывая молнии, которые, не касаясь пола, 
вылетали в дверь, проносились по коридору и, повернув за угол, уходили 
в землю.
    - Как он вам нравится? - прокричал великан Яррах, перекрывая 
грохот грома. - Я еще не встречал никого, кто отводил бы молнии лучше 
Джорджа.
    От этой похвалы Джордж пришел в полный восторг. Гроза затихала. В 
качестве заключительного аккорда Джордж взмахнул своим мерным стержнем 
и одним широким движением поймал две молнии сразу. Они голубым 
пламенем блеснули на конце стержня и исчезли в коридоре.
    Упало несколько капель дождя. Если Джордж чего и боялся на свете, 
так это намочить свой вечерний костюм. Он посмотрел на небо, 
повернулся и побежал.
    - Держитесь! - крикнул он Питеру и Серой Шкурке. - Сейчас хлынет 
ливень.
    Он промчался по коридору с такой скоростью, что, наверное, обогнал 
бы молнию, хлопнул дверью своей комнаты и заперся изнутри.
    Из туч хлынули целые потоки. Вода лилась на плиту, в котел с 
мясом, в чашки на кухонном шкафу. Сбегая со стола, она образовывала 
четыре водопада. Вскоре по всему полу уже несся бурный поток, 
стремившийся к выходу.
    - Мунлайт утонет! - вскричал Питер. - Она привязана к ножке стола.
    Он быстро полез вниз по лестнице. Серая Шкурка бросилась за ним. 
Вода на полу уже доходила Питеру до пояса, и ему пришлось бороться с 
течением, чтобы добраться до лошади, которая стояла рядом с одним из 
водопадов. Под столом вода неслась особенно быстро, и Мунлайт было 
очень трудно удержаться на месте.
    Питер прошел сквозь льющийся со стола водопад и на секунду 
прижался к ножке стола, чтобы перевести дыхание. Поток воды несся мимо 
него, разбивался о другую ножку на два потока и снова сливался вместе 
под кухонным шкафом. В воду падали смытые с открытых полок кастрюли и 
миски и тоже неслись к двери, словно какой-то затейливый флот.
    Питеру и Серой Шкурке удалось добраться до Мунлайт, которую поток 
воды уже сбил с ног. Она погружалась в воду и била передними ногами, 
отчаянно пытаясь разорвать поводок. Серая Шкурка достала из сумки нож 
и, перерезав поводок, освободила лошадь.
    Теперь они плыли все. Сверху гремел голос великана: "Где вы? Я 
спасу вас. Я не дам вам утонуть". Волшебный Лист настолько изменил 
его, что он думал о безопасности своих друзей больше, чем о 
собственной. "Где вы? Где?" - не переставая кричал он.
    Питера и его друзей течение унесло под стол, и там великан не мог 
их увидеть. Ему самому приходилось несладко. Огонь в топке погас, и 
великан все время натыкался на разные затопленные предметы. Он 
рассадил себе ногу о помойное ведро и ударился головой о прибитую к 
стене кофемолку. Несколько раз он спотыкался и падал, сердито ругая 
эту проклятую кухонную погоду.
    Питер окликнул его, но крик потонул в реве водяного потока. Вода 
носила Питера и его друзей по всей кухне, швыряла из стороны в 
сторону, мчала по каким-то речным порогам. Вдруг они очутились в 
гигантском водовороте и стали вращаться по огромному кругу, который с 
каждым витком становился все меньше, приближаясь к центру воронки, 
откуда доносилось ужасающее бульканье.
    - Мы в раковине! - крикнул Питер, сообразив, что бульканье 
исходило из сливного отверстия. Тут волна отбросила их к краю 
раковины, и Мунлайт, зацепившись за него своими мощными ногами, сумела 
выкарабкаться сама и вытащить за собой и Питера, и Серую Шкурку, 
которые ухватились за кожаные стремена. Они спрыгнули в воду по другую 
сторону раковины и были подхвачены течением. Оно потащило их к двери, 
куда со всех сторон стекалась вода, устремившаяся в коридор.
    Теперь это был даже не поток, а могучая река, тяжелой массой 
переливающаяся через порог кухни. Мунлайт, высоко задрав голову и 
отфыркиваясъ, стремительно протащила Питера и Серую Шкурку через 
водоворот и увлекла в коридор. Подгоняемые разъяренной рекой, они 
ринулись вперед и уже через мгновение оказались выброшенными на ровную 
землю, по которой вода растекалась и исчезала в траве, пробивая себе 
путь к ручью.
    Серая Шкурка встряхнулась, освобождаясь от воды, которая пропитала 
ее шубку. Питер промок насквозь, но, едва его коснулись солнечные 
лучи, мгновенно высох. На нем появилась голубая бархатная куртка с 
золотыми пуговицами, на голове снова оказалась шляпа с перьями, на 
ногах - сапоги, сухие и начищенные, хотя Питер и опасался, что вода 
может испортить их.
    Такое изменение внешнего вида Питера нисколько не удивило Серую 
Шкурку.
    - Тебе надо еще обзавестись новыми брюками, - сказала она, оглядев 
его. - Эти старые портят весь вид. Тебе предстоит еще долгий путь, 
прежде чем ты станешь принцем.
    Ее речь прервало появление великана, который вышел из замка, 
выжимая бороду. В его ботинках хлюпала вода. Он стал извиняться за 
причиненное беспокойство. Он жалел о погубленном мясе и хотел 
приготовить что-нибудь другое.
    - Не беспокойся о нас, - сказала Серая Шкурка. - Нашу еду я ношу в 
своей сумке. Уже становится темно, и нам пора идти.
    - Но прежде чем мы уйдем, - сказал Питер, - ты должен освободить 
всех, кого засунул в коробки. Теперь, когда ты стал добрым великаном, 
ты не должен больше никого захватывать. Ты должен помогать всем, кто 
идет этой дорогой, а не пытаться делать всех одинаковыми.
    - Я хотел освободить всех сразу после обеда, - сказал великан 
Яррах. - Мы сделаем это сейчас. Идите за мной.
    Он подошел к маленькой двери в стене замка. Сразу за ней вниз 
уходила узкая лестница. Она привела в огромное подземелье, где рядами 
стояли коробки. Мальчики, девочки, мужчины и женщины были втиснуты в 
них в самых неудобных позах. Увидев Питера и Серую Шкурку, пленники 
стали кричать: "Помогите нам! Вызволите нас отсюда!"
    - Сейчас я вас освобожу, - сказал великан. - Потом я всех накормлю 
и выведу отсюда.
    Питер и кенгуру помогали великану открывать ящички. Они бегали от 
одного ящичка к другому, отодвигали задвижки и помогали людям ступить 
на землю. Сначала люди не могли ни согнуться, ни разогнуться, словно 
они одеревенели, и им пришлось долго махать руками и топать ногами, 
прежде чем они почувствовали, что снова могут ходить. Великан вывел их 
по лестнице на воздух. Люди смотрели на солнце, на облака, на вершины 
деревьев, и мысль о свободе возвращала румянец на их бледные щеки.
    - Мы должны вас сейчас покинуть, - сказал Питер. - Нам еще 
предстоит долгий путь. Великан Яррах позаботится о вас. Он накормит 
вас и проводит.
    - Да, - подтвердил великан. - Вам больше незачем меня бояться.
    Питер вскочил на Мунлайт и обернулся, чтобы помахать на прощанье 
великану и стоящим вокруг него людям.
    - Прощайте! - крикнул он.
    - Прощайте! - ответили они.
    - Будь осторожен! - добавил великан Яррах. Питер тронул поводья, и 
Мунлайт потрусила прочь, сопровождаемая Серой Шкуркой.

        Глава 6
        ПИТЕР И СЕРАЯ ШКУРКА
        В ПЛЕНУ У БЛЕДНОЙ КОЛДУНЬИ

    Некоторое время спустя, проезжая по широкой, поросшей кустарниками 
долине, Питер почувствовал усталость. Он оглянулся и увидел, что Серая 
Шкурка немного отстала.
    - Что случилось? - крикнул он ей.
    Серая Шкурка допрыгала по тропе до дерева, стоявшего на краю 
поляны, и села под ним.
    - Сама не знаю, - ответила она. - Мне не хочется ничего делать. Я 
хочу лишь одного: сидеть под этим деревом и играть на гитаре.
    Она достала из сумки гитару и принялась наигрывать песенку. Питер 
соскочил на землю и сел рядом, опершись спиной о лежащее бревно.
    - У меня тоже пропало желание что-либо делать, - сказал он. - Я не 
хочу искать Прекрасную Принцессу. Мне бы хотелось остаться здесь и 
ничего не делать, мне бы хотелось поставить под этими деревьями хижину 
и просто проводить в ней все время. Интересно, что это на нас нашло?
    Вдруг из кустов вылетела кукабарра и уселась на ветку над головой 
Питера. Она начала смеяться, и Питеру и Серой Шкурке показалось, что 
она смеется над ними.
    - Почему ты смеешься? - спросил Питер.
    - Хм, - усмехнулась кукабарра. - Многие люди шли этой дорогой, и 
со всеми случалось одно и то же. Я сама видела. Им не хотелось 
работать. Не хотелось идти вперед. Они просто садились на землю и 
ничто их больше не волновало. И все оттого, что вы вступили во 
владения Бледной Колдуньи, и она вас околдовала. Теперь Питер 
позабудет о Прекрасной Принцессе. Вы перестанете к чему-либо 
стремиться. А потом придет Бледная Колдунья. Сначала она покажется вам 
доброй. Она пригласит вас к себе в гости, а затем убьет вас обоих и 
съест на ужин. Ясно вам? Она поедает людей. Это очень злая колдунья.
    Серая Шкурка швырнула гитару обратно в сумку и вскочила. Она очень 
сильно рассердилась.
    - Не желаю быть съеденной на ужин кем бы то ни было! Мы должны 
поскорее избавиться от этого ужасного чувства, которое нашло на нас. 
Давай поделаем упражнения для дыхания и попрыгаем, чтобы снова 
почувствовать силу. Нам необходимо сосредоточиться.
    - А я не хочу сосредоточиваться, - ответил Питер. - Я вообще 
ничего не хочу. Сейчас слишком жарко.
    Он снял бархатную куртку и бросил ее рядом на землю. Потом снял и 
бросил около бревна цепочку, на которой висела кожаная сумочка с 
Волшебным Листом. Сумочка упала в высокую траву.
    Лишившись Волшебного Листа, который придавал ему силы, Питер стал 
раздражительным, ему все теперь действовало на нервы. Он сердито 
смотрел на Серую Шкурку, которая прыгала взад-вперед, пытаясь 
избавиться от непонятной усталости.
    - Почему бы тебе не посидеть? - раздраженно сказал Питер. - Не 
будь дурой, это прыганье ничем не поможет.
    - Если я сяду, то засну, - объяснила Серая Шкурка. - А я не хочу, 
чтобы Бледная Колдунья застала нас врасплох.
    - Прыгай - не прыгай, все равно толку не будет, - пробормотал 
Питер, уже начиная клевать носом.
    Серая Шкурка с беспокойством посмотрела на него, потом пошла к 
лошади и отвела на крошечный островок зеленой травы. Погладив ее, 
сказала: "Никуда не уходи отсюда, Мунлайт. Что бы ни случилось, 
оставайся где-нибудь поблизости".
    Затем она вернулась к Питеру, который уже спал крепким сном.
    Серая Шкурка понимала, что колдунья должна быть где-то рядом, если 
чары ее усиливаются. Сама Серая Шкурка решила бодрствовать во что бы 
то ни стало. Она принялась трясти Питера, пытаясь его разбудить, потом 
выпрямилась и застыла, прислушиваясь. Она услышала, как кто-то 
пробирается к ним сквозь буш.
    Кукабарра, увидев с высоты колдунью, улетела прочь и растаяла в 
кронах деревьев. Колдунья, наклонив голову, медленно шла по узкой 
тропинке, проложенной в буше. Подойдя к Питеру и Серой Шкурке, она 
подняла голову и посмотрела на них. Глаза ее горели злобой. На ней 
была высокая черная шляпа и черный плащ. Нос был крючком, подбородок 
выгнут навстречу носу. В руках она держала метлу, а на плече у нее 
сидела ворона.
    - О, мои прелестные, - тихо сказала она, - пойдемте ко мне, я 
покажу вам кое-что очень интересное, приготовленное специально для 
вас.
    Колдунья внимательно их осмотрела. Пощупав руки и ноги Питера, она 
прошептала:
    - Очень пухленький, пойдет на пирог с мясом...
    - Идемте же, идемте, пообедаем вместе, - уговаривала она.
    - Неохота нам с тобой обедать, - сказала Серая Шкурка.
    - А я уверена, что твой приятель был бы не прочь, - ответила 
колдунья. Она коснулась плеча Питера, в тот открыл глаза. Сначала он 
смотрел ничего не понимающим взглядом и не замечая склонившейся над 
ним колдуньи. Потом у него в глазах появился страх, и он рывком сел.
    - Поднимайся, - заговорила колдунья вкрадчивым голосом, - 
поднимайся и иди за мной. Я твой друг. Я не причиню тебе зла. Идем же.
    Питер встал на ноги и стоял, глядя на колдунью. Его слегка 
покачивало.
    - Иди! Иди за мной! - приказала колдунья.
    Питер пошел за колдуньей словно через силу. Сначала он качался, 
как пьяный, но потом настойчивый голос колдуньи словно бы успокоил 
его, и он пошел увереннее и тверже.
    Серая Шкурка колебалась. Она не поддалась чарам колдуньи и поэтому 
могла бы остаться на месте, но ей не хотелось оставлять Питера одного, 
и поэтому она пошла тоже. По пути Серая Шкурка оставляла отметины, по 
которым можно было бы найти дорогу назад. Она не сомневалась, что 
Мунлайт никуда не уйдет и будет пастись недалеко от бревна.
    Следуя за колдуньей, Питер и кенгуру подошли к маленькой хижине, 
прятавшейся в зарослях акаций. Она выглядела довольно непритязательно: 
острая крыша, крошечные грязные оконца, на одной стороне - труба, на 
другой - крыльцо с навесом.
    Поднявшись на крыльцо, Питер вслед за колдуньей вошел в 
единственную комнату хижины. Это была странная комната. У огня сидел 
черный кот, на балке под потолком - сова, над огнем на цепях висел 
котел, из которого поднимался пар.
    В углу комнаты стояла большая клетка с распахнутой дверцей. 
Колдунья вошла в клетку, Питер - за ней. Серая Шкурка, оцепенев и 
словно не понимая, что делает, вошла тоже.
    Именно этого колдунья и ждала. Быстро повернувшись, она 
выскользнула за дверь и захлопнула ее за собой. Потом она заперла ее 
на замок, и Питер с Серой Шкуркой оказались в плену.
    Они, должно быть, сразу же заснули, потому что когда они вновь 
открыли глаза, был уже вечер. Колдунья сидела на стуле и чинила свою 
метлу. Питер и Серая Шкурка поняли, что чары колдуньи перестали 
действовать, потому что оба они соображали ясно.
    - Ага, проснулись, - сказала колдунья, поднимая голову, чтобы 
посмотреть на них. - Очень хорошо. Вы только не волнуйтесь. Чувствуйте 
себя, как дома. Я съем вас не раньше, чем через несколько дней, а пока 
хочу, чтобы вы хорошо ели и поправлялись. Сейчас мне надо лететь на 
Луну. Мне приходится подметать ее каждую ночь, потому что американцы и 
русские продолжают забрасывать туда всякую дрянь, работы там ужасно 
много. А пока сова, кот и ворона составят вам компанию. Я постараюсь 
обернуться побыстрей.
    С этими словами колдунья сунула метлу под мышку и вышла из 
комнаты. И тут же с улицы раздался свист рассекаемого воздуха, - это 
она полетела на Луну.
    - Ну и попали мы с тобой в переделку, - сказал Питер. - Я оставил 
Волшебный Лист рядом с курткой, и теперь мы не сможем превратить эту 
колдунью в добрую, пока не достанем его.
    - Отсюда не выбраться, - сказала Серая Шкурка. - Нам суждено 
набрать вес, а потом быть съеденными. Колдунья нас ни за что не 
выпустит, пока не раскормит, а тогда отправит прямехонько в котел.
    Питер и Серая Шкурка стали обсуждать разные способы спасения, но 
все они сводились к тому, чтобы завоевать дружбу колдуньи, а ведь это 
было невозможно...
    - Зачем тебе понадобилось снимать с шеи цепочку? - раздраженно 
спросила Серая Шкурка.
    - Даже не знаю. Я, наверное, спятил, - ответит Питер. - А почему 
ты меня не остановила? Тебе ничего не стоило это сделать, - сердито 
набросился он на Серую Шкурку.
    - Лист-то ведь был дан тебе, а не мне!
    - Ну и что! Все равно ты виновата не меньше!
    Серая Шкурка ничего не ответила.
    Через некоторое время Питер снова заговорил:
    - Интересно, откуда она взяла все эти фотоаппараты? Посмотри, вон 
сколько их валяется у стены.
    У стены на полу лежало свыше десятка фотоаппаратов и кинокамер, 
причем большая часть - со штативами. Тут же был и маленький 
механический экскаватор, какие обычно работают на батарейках.
    - Она, наверное, увлекается фотографией, - предположила Серая 
Шкурка. - Я спрошу, когда она вернется. Нам надо стараться быть с ней 
поласковей. Мне бы очень не хотелось разозлить ее.
    Колдунья вернулась лишь под утро. Когда она прилетела, Питер и 
Серая Шкурка спали. Она забегала по комнате, готовя им еду, но Питер и 
Серая Шкурка уже поужинали, воспользовавшись сумкой кенгуру, и поэтому 
от еды отказались. Это разозлило колдунью. Она пробормотала что-то 
вроде того, что добьется от них послушания, даже если ей придется 
морить их голодом.
    Серая Шкурка попыталась наладить контакт.
    - Вы интересуетесь фотографией? - спросила она.
    - Вовсе нет, - ответила колдунья. - Все эти фотоаппараты я 
подобрала на Луне. Их туда забрасывают ракеты, а я, когда ее подметаю, 
беру их себе. Этот экскаватор, когда я его нашла, сам рыл яму. Я 
принесла его сюда, но здесь он почему-то не работает. Зато у меня есть 
замечательная коллекция фотографий. Если хотите, могу показать. Я 
проявляла пленки, которые были в фотоаппаратах, и печатала фотографии.
    - Как-нибудь в другой раз, - сказала Серая Шкурка, она терпеть не 
могла разглядывать фотографии.
    Питер все время сидел, подперев голову руками, и думал, как бы 
обмануть колдунью. Неожиданно его осенило.
    - Какое блюдо вы собираетесь из меня приготовить? - спросил он.
    - А как ты сам хочешь быть приготовленным? - вопросом на вопрос 
ответила колдунья.
    - Пожалуй, я бы хотел быть поджаренным с приправой из травы бо-бо.
    - Я о такой не слыхала, - бросила колдунья, однако травой 
заинтересовалась.
    - Не слыхала о траве бо-бо! - воскликнул Питер. - Да ведь за этой 
травой люди гоняются по всему миру! Когда ею посыпают жаркое, оно 
приобретает удивительный аромат. Мне рассказал о ней Старина Мик, и он 
же научил меня, как ее найти. У нее такой изумительный вкус, что люди 
отдавали жизнь, чтобы только найти ее. Я бы не хотел быть изжаренным 
без травы бо-бо.
    - А где она растет? - заинтересовалась колдунья.
    - Я видел ее недалеко от того места, где ты нас захватила. Если 
хочешь, могу принести ее.
    - Скажи, как она выглядит, я принесу сама.
    - Ее трудно описать. Лучше я покажу ее тебе.
    Колдунья колебалась. Она была жадной и очень хотела заполучить эту 
травку, но боялась, как бы Питер не обманул ее.
    - Ты уверен, что от этой травки ты станешь вкуснее?
    - Абсолютно! Посыпь меня ею, и у тебя получится самое прекрасное 
блюдо, какое тебе только доводилось есть.
    - Хорошо, - согласилась колдунья, наконец. - Отведи меня туда, где 
она растет, но если попытаешься меня обмануть, я превращу тебя в жабу, 
и ты будешь ею до конца дней своих!
    Она открыла замок и выпустила Питера. Серая Шкурка, не 
перестававшая удивляться всему этому разговору, осталась в клетке 
одна. Но Питер, уходя, подмигнул ей, и она успокоилась.
    Оказавшись на свободе, Питер пошел к тому месту, где колдунья 
впервые напустила на них свои чары. Он хотел найти сумочку с Волшебным 
Листом, он точно знал, где ее оставил: около бревна, у которого они с 
Серой Шкуркой отдыхали.
    Питер шел впереди, - колдунья - за ним. Вот он увидел Мунлайт, 
которая паслась на островке зеленой травы, потом разглядел и бревно. 
Питер пошел медленнее, осматривая землю, словно выискивал траву бо-бо.
    - Здесь ли она? - забеспокоилась колдунья.
    - Где-то здесь.
    Около бревна Питер заметил блеснувшую в траве цепочку и шагнул к 
ней. Но колдунья тоже ее увидела. Она бросилась вперед и, обогнав 
Питера, склонилась над ней.
    Питер рассвирепел - из-за этой поганой колдуньи рушился его 
замечательный план! Колдунья, нагнувшись за цепочкой, оказалась перед 
ним в позе, которая вдохновила его. Он со всей силы дал ей пинка.
    {whisp03.gif}
    Удар оторвал ошарашенную колдунью от земли. Она перелетела через 
бревно и приземлилась на голову. Питер схватил сумочку, надел на шею 
цепочку и достал Волшебный Лист.
    Колдунья была вне себя от ярости. С диким криком, по-кошачьи она 
вскочила на ноги. Лицо ее исказила злоба. Растопырив руки и загребая 
воздух скрюченными пальцами, она завизжала, что навсегда превратит 
Питера в жабу.
    Питер испугался, но тем не менее протянул ей Лист и сказал:
    - Вот травка бо-бо. Возьми ее, прежде чем меня заколдовать.
    Колдунья замешкалась. Проклятье, уже готовое сорваться с ее губ, 
замерло. Она вырвала Волшебный Лист из рук Питера.
    - Давай сюда! - взвизгнула она.
    Как только ее рука коснулась Листа, с ней стали происходить 
поразительные перемены. Однако она так долго пребывала во власти зла, 
что смогла воспротивиться волшебной силе Листа. Она не хотела быть 
нужной, не хотела, чтобы ее любили. Она хотела только одного - 
ненавидеть.
    Колдунья бросилась на землю и, казалось, стала с кем-то сражаться. 
На губах у нее выкупила пена. Она молотила по земле сжатыми кулаками.
    - Не хочу! Не хочу! - кричала она. - Не хочу, чтобы меня любили!
    Но сила Волшебного Листа была все-таки больше силы колдуньи. Она 
долго каталась по земле, наконец, замерла, и еще через несколько 
секунд медленно встала. Отвратительное выражение исчезло с ее лица, на 
нем появилось сострадание и доброта. Она ласково посмотрела на Питера.
    - Прости меня, - сказала она совсем другим голосом.
    А Питер, вернув Волшебный Лист, только и думал, как бы помочь 
Колдунье. Он обнял ее за плечи и сказал:
    - Теперь ты станешь счастливее. Забудь прошлое. Ты научишься 
любить людей, а они полюбят тебя.
    - Наверное, я привыкну быть хорошей, - сказала колдунья, - но 
поначалу мне будет трудно. Обратно к хижине они шли рядом.
    - Хочешь, я сегодня помогу тебе подметать Луну? - предложил Питер.
    - Хочу, - обрадовалась колдунья.
    Когда они вошли в хижину, Серая Шкурка поразилась происшедшей 
перемене. Колдунья выпустила ее из клетки и попросила чувствовать себя 
как дома.
    Питер рассказал Серой Шкурке, как все произошло, а потом добавил:
    - Завтра мы с тобой отправимся дальше, а сегодня я хочу помочь 
колдунье подмести Луну. Чтобы доказать, что отныне мы - друзья.
    - Неплохая мысль, - одобрила Серая Шкурка и шепотом добавила:
    - А бархатную куртку ты взял?
    - Нет, она пропала.
    - Так я и думала, - огорчилась Серая Шкурка.

        Глава 7
        ПИТЕР И КОЛДУНЬЯ ЛЕТЯТ НА ЛУНУ

    Когда Питер и колдунья собрались лететь, на небе висела полная 
Луна. Они стояли у порога хижины и смотрели вверх, на светящийся 
лунный диск, по которому им вскоре предстояло прогуляться.
    - Отсюда Луна выглядит чистой, - сказала колдунья. - Но когда ты 
попадешь туда, ты обомлеешь, - столько там мусора. Просто ужас!
    - На Луне уже побывали люди, - сказал Питер.
    - Знаю, знаю. После них я целую неделю наводила порядок. А ты бы 
видел, как этот парень Армстронг делал первый шаг! Смех! Я чуть живот 
не надорвала!
    Колдунья села верхом на метлу, Питер пристроился сзади. Он 
вцепился в ее черный плащ и закрыл глаза. Тонкими, костлявыми пальцами 
колдунья сжала метлу, и она ракетой взвилась в небо. Земля таяла 
позади, по скорости они не замечали. Вскоре они почувствовали 
невесомость, и сидеть на метле стало легко.
    - Сегодня метла хорошо летит, - сообщила колдунья. - Пожалуй, с 
двумя пассажирами на борту она идет лучше. Мы уже превысили вторую 
космическую скорость, семь миль в секунду, но я разгоню ее до 240 
тысяч миль в час. Значит на Луне мы будем через 55 минут. Держись 
крепче!
    Метла действительно летела быстрее. Лунный диск впереди быстро 
увеличивался.
    Вдруг колдунья так резко вильнула в сторону, что Питер чуть не 
слетел с метлы. Мимо них промелькнул какой-то космический корабль с 
освещенными иллюминаторами.
    - Куда тебя несет? Смотреть надо! - закричала колдунья. Потом 
обернулась к Питеру и сказала:
    - Хорошенькое дело! Этот дурак чуть меня не сбил. Сейчас я ему 
покажу!
    Она резко бросила метлу в вираж, развернулась и погналась за 
кораблем, летавшим вокруг земли. Сманеврировав, она пристроилась прямо 
к его иллюминатору. Из-за толстого стекла на них ошарашенно смотрел 
человек.
    - Эй, балбес, права-то у тебя есть? - вопила колдунья, радуясь, 
что снова может быть грубой. - Ты нас чуть не сбил. Коляской тебе надо 
управлять, детской коляской - и ничем больше!
    Космонавт смотрел на них широко раскрытыми от изумления глазами. 
Он пытался что-то сказать в микрофон, но не мог выдавить ни слова. 
Вдруг он заговорил с необыкновенной скоростью, словно хотел сообщить 
что-то необычайно срочное.
    - Земля, Земля! Прием.
    - Что случилось? - спросил голос из приемника.
    - В иллюминатор на меня смотрят ведьма и мальчик. Они верхом на 
метле. Прием.
    - Ты болен. Повторяю: ты болен. Слишком устал. Подбавь кислорода. 
Потяни рычаг Б, потом задвинь его назад. Выпей кофе из термоса 3-а. 
Сфотографируй ведьму и немедленно отправь фотографию нам. Повторяю: ты 
нездоров. Возьми упаковку таблеток на полке слева. Прими сразу две 
таблетки. После этого ты уснешь. Повторяю: ты болен.
    Колдунья постучала рукой по стеклу.
    - Не обращай на них внимания! - крикнула она космонавту. - Ты 
вполне здоров.
    Но космонавт уже проглотил две таблетки и заснул.
    - Готов, - сказала колдунья. - Теперь он проспит несколько 
часов... - Она заглянула в иллюминатор. - Что там перед ним в термосе? 
Не кофе ли?
    - Наверное.
    - Как ты думаешь, для космонавта будет большая потеря, если мы его 
выпьем? Вон сверху люк. Через него мы могли бы легко попасть внутрь.
    - Лично я не отказался бы от чашечки кофе.
    Колдунья подогнала метлу к люку корабля. Там они совершили 
посадку, открыли люк и быстро скользнули вниз, причем колдунья 
прихватила с собой метлу. Питер закрыл люк. Колдунья налила две 
чашечки кофе из термоса и передала одну Питеру. Они уселись и 
принялись пить маленькими глотками.
    - Хороший кофе у них в Америке, - сказала колдунья. - Но нам надо 
спешить: на Луне еще уйма работы.
    - Бедняга этот парень, - заговорил Питер. - Ты думаешь, на Земле 
ему кто-нибудь поверит, что он видел в иллюминаторе ведьму и мальчика 
верхом на метле?
    - Конечно, не поверят. Ему скажут, что все это ему приснилось или 
просто он все это выдумал. Когда ты встречаешь что-то удивительное и 
рассказываешь об этом другим, тебе никто не верит. Если же ты хочешь, 
чтобы тебе верили - говори неправду. Скажи, что во время путешествия 
не случилось ничего интересного, и тебе поверят.
    - Может быть, мы оставим ему записку, подпишемся и приколем к его 
куртке?
    - Давай, только побыстрее. У тебя есть бумага и ручка?
    - Я вырву листок из его блокнота, - сказал Питер. Блокнот и ручка 
лежали рядом.
    Питер взял листок и написал: "Удостоверяем, что ведьма и мальчик 
посетили этот корабль и выпили кофе космонавта".
    Он передал листок колдунье: "Подписывай!" Та нацарапала внизу свое 
имя, Питер подписался тоже. Они прикололи листок к куртке космонавта и 
вылезли через люк наружу.
    - Хотелось бы мне видеть его лицо, когда он проснется и прочтет 
записку, - мечтательно произнесла колдунья.
    Они снова сели верхом на метлу.
    - Оттолкнись ногами от корабля, когда я дам знак, - сказала 
колдунья. - Это придаст нам начальное ускорение. - Она уселась 
поудобней, скомандовала: "Давай!", Питер оттолкнулся, и они взлетели.
    Некоторое время метлу водило из стороны в сторону, но потом она 
понеслась ровно, как машина. Колдунья гнала свою метлу быстрее, чем 
обычно. Они мчались с такой скоростью, что казалось, будто Луна 
несется им навстречу. Скоро уже можно было различить ее высокие крутые 
горы и огромные сухие моря. Колдунья и Питер ринулись вниз с 
безоблачного неба и нырнули в огромный кратер, на дне которого лежало 
множество камней.
    Там они остановились. Колдунья слезла с метлы и принялась яростно 
подметать. Она двигалась с такой скоростью, что почти исчезла из вида. 
Питер понял, что помочь ей ничем не сможет и принялся бродить вокруг, 
перепрыгивая через скалы и перемахивая через расселины. На Луне это 
оказалось очень легко, ведь его вес не превышал там нескольких фунтов
    Он услышал, как колдунья, стоя где-то внизу на ровной площадке, 
зовет его. Оказывается, она обнаружила кинокамеру, которая 
поворачивалась на 180 градусов туда и обратно.
    - Давай позировать, - объявила она Питеру, как только он 
спустился. - Становись рядом, и когда камера повернется к нам, говори 
"Сы-ыр", и растягивай губы в стороны. Тогда получится, будто мы 
улыбаемся.
    Камера медленно поворачивалась.
    - Давай! - скомандовала колдунья.
    - Сы-ыр! - сказали они разом.
    {whisp04.gif}
    - То-то на земле будут ошарашены, - сказала колдунья. - Это у них 
станет главной сенсацией дня. А камеру мы оставим здесь. У меня их и 
так скопилось слишком много. Слышишь попискивание? Это, наверное, наше 
изображение передается на землю.
    Взяв метлу, колдунья подмела вокруг, передвинула несколько камней, 
чтобы был опрятный вид. Потом села и принялась копаться в метле, 
готовя ее к полету.
    - Сегодня мы опаздываем, - объяснила она, - и мне хотелось бы в 
этом полете установить рекорд. Если замерзнешь, обмотайся моим плащом. 
Стартовать, пожалуй, будем вон с той скалы, - она указала на 
остроконечную вершину, которая возвышалась над ними.
    Они легко полезли наверх, - ведь на Луне их вес уменьшился, - и 
уже через несколько минут вскарабкались на вершину. Питеру не давала 
покоя мысль о камнях, которые привозили с Луны астронавты, и он сунул 
в карман горсть камешков, что валялись под ногами. "Раздам их 
друзьям", - подумал он.
    - Послушай, - обратился он к колдунье, - а почему Армстронг и 
Олдрин не могли ходить по Луне нормально? Им надо было всего лишь 
набить карманы свинцом. Поехали!
    Не успели они оторваться от скалы, как метла провалилась вниз, 
нырнула в широкую впадину и понеслась над одним из сухих морей. Питер 
видел, как внизу, на ровной поверхности, громоздились скалы. В глубине 
одной из них был виден яркий красный отсвет: там горел вулкан.
    Метла разогналась, повернула к Земле и вскоре достигла скорости 
полутора миль в секунду, необходимой, чтобы преодолеть притяжение 
Луны. Скорость ее нарастала, пока не остановилась на какой-то 
постоянной цифре. К метле был привинчен спидометр, и колдунья часто с 
ним сверялась.
    - Сейчас мы даем 250 тысяч миль в час, - сообщила она. - Плюс-
минус полмили.
    Питер стал замерзать. Он замотался в плащ колдуньи и закрыл глаза.
    Вдруг он услышал, как она крикнула: "Держись!" Метла стала 
дергаться, словно брыкающаяся лошадь. Хорошо еще, что Питер был 
неплохим наездником, а то его бы сбросило.
    - Мы входим в земную атмосферу, - объяснила колдунья, - а скорость 
слишком большая. Приходится тормозить, иначе сгорим. Вот мы и 
подпрыгиваем, как камешек на поверхности пруда.
    Переднюю часть метлы покрыла полоска голубого пламени. Огонь 
опалил брови колдуньи. Питер чувствовал вокруг себя жар, но он крепко 
прижался к спине колдуньи, и огонь миновал его. Метла прыгала подобно 
дельфину.
    - Мне придется сделать несколько витков вокруг Земли, чтобы сбить 
скорость! - крикнула колдунья.
    Они обогнули Землю три раза и, когда погасили скорость, колдунья 
расстегнула пуговицу на плаще, который раздулся позади них, словно 
парашют.
    - Вот коридор. Мы спускаемся.
    Метла, как метеор, пронзила атмосферу. Позади нее тянулся длинный 
огненный след. Колдунья промчалась над крышей хижины, блестяще 
выполнила петлю и приземлилась у порога.
    - Вот что получается от увлечения скоростью, - сказала она, ступая 
на Землю. - Никогда не думала, что так получится, больше я этого 
делать не буду. Желание пофасонить дорого обошлось мне: я сожгла верх 
шляпы и опалила брови. Тебе еще повезло, что ты сидел сзади.
    - Жар чувствовался и там, - сказал Питер. - Мне обожгло палец, он 
и сейчас немного болит.
    Навстречу из хижины выпрыгнула Серая Шкурка.
    - Я подумала, что это падает комета, - сказала она. - Что 
случилось?
    - Тормоза отказали, - объяснила колдунья.
    - Эта метла для космоса не годится, - сказала Серая Шкурка. - Надо 
ее поменять на новую модель.
    - Чепуха! - бросила колдунья. Она вошла в хижину, где уже был 
готов чай.
    Рассвет только разгорался, небо на востоке порозовело.
    - Надо трогаться в путь, как только попьем чаю, - сказал Питер.
    - А куда вы торопитесь? - забеспокоилась колдунья. - Можете 
оставаться у меня сколько хотите, вы же знаете.
    - Я ищу Прекрасную Принцессу, - объяснил Питер. - Путь еще очень 
длинный, нам надо поторапливаться. Я надеюсь, мы с тобой когда-нибудь 
встретимся.
    - Я тоже надеюсь. - Колдунья выглядела сейчас добродушной пожилой 
женщиной, так сильно изменил ее Волшебный Лист.
    - А ты когда-нибудь слышала о Прекрасной Принцессе? - обратился к 
ней Питер.
    - Нет. В своей жизни я встретила только одну принцессу. Она 
пыталась столкнуть меня с подоконника, поэтому я сорвала у нее с 
головы корону и забросила в озеро. И она не сможет выйти замуж, пока 
не найдет корону.
    - Как это жестоко!
    - Я же тогда была злой колдуньей и ужас что вытворяла. Это было не 
так давно, но сейчас я стала совсем другой. Я никогда не буду делать 
ничего подобного. Если же я что-нибудь узнаю о твоей Прекрасной 
Принцессе, я дам тебе знать.
    На прощанье Питер поцеловал колдунью в щеку и вдруг почувствовал, 
что бархатная куртка снова на нем. Серая Шкурка тоже ее заметила и 
улыбнулась.
    - Мунлайт привязана неподалеку, - сообщила она.

        Глава 8
        ЧЕЛОВЕК-СМЕРЧ ВИЛЛИ-ВИЛЛИ

    В эту ночь Питер и Серая Шкурка устроили привал на берегу реки. 
Шершавые стволы эвкалиптов словно разглядывали себя в водной глади; ее 
зеркальную неподвижность нарушали только платипусы - их горбатые спины 
то и дело высовывались на поверхность. Серая Шкурка достала из сумки 
жареное мясо, пирог и пудинг.
    Утром друзья двинулись дальше по неровной, холмистой местности. По 
тропам меж скал медленно передвигались вомбаты, а с дерева на дерево 
перелетали малиновые попугаи. Возле одной каменистой горы заросли буша 
вдруг поредели, и жаркое солнце принялось без помех палить 
путешественников. Мунлайт вспотела: в том месте, где сбруя терлась о 
шею, выступила испарина. Ручьи и речки сбегали с противоположной 
стороны горы, а здесь перед путешественниками простиралась только 
голая равнина, слепящая глаза.
    - Это Пустыня Одиночества, - объяснила Серая Шкурка. - Мой папа 
рассказывал, что люди, переходящие ее, часто умирают от жажды. Здесь 
никого не встретишь. Это глухое место, и перейти его можно только 
вместе с друзьями. Когда люди идут компанией, они разговаривают, 
смеются и забывают, в какую даль забрели и как трудно тут добыть воду. 
Как ни странно, именно они часто ее находят. Те же, кто идет в 
одиночку, не находят никогда.
    Неожиданно на горизонте появился высокий пылевой столб. Вращаясь в 
воздухе, он двигался прямо на Питера и Серую Шкурку. Когда он 
приблизился еще, они разглядели листья и целые веточки, втянутые в 
столб. Если он проходил по сухой траве, то она, недвижно лежавшая на 
земле, вдруг бешено подскакивала и начинала носиться по кругу, а затем 
ныряла в столб и выбрасывалась наверх, где продолжала судорожно 
вращаться. Сухие листья и трава образовывали нечто вроде шапки над 
безумно пляшущим пылевым столбом, который поднимал вверх руки и 
опирался на землю ногами.
    - Это человек-смерч Вилли-Вилли! - воскликнула Серая Шкурка, 
которая, похоже, знала все на свете. - Он - дядя четырех ветров и 
частенько здесь носится. Он любит принимать зримый облик, поэтому 
живет в Пустыне Одиночества, где много пыли и сухой травы, которые 
легко сорвать с места и поднять в воздух. А если он пляшет по зеленой 
траве, его почти не видно, - только трава машет, как бы приветствует 
его, зная, что это он идет.
    {whisp05.gif}
    Смерч обошел их вокруг. Поднятые песчинки больно хлестали Питера и 
Серую Шкурку, так что им пришлось прикрыть глаза и отступить. Тогда 
вращающийся пылевой столб дернулся и остановился, и откуда-то из его 
глубины выпрыгнул маленький человечек. Он было пошатнулся, по тут же 
выпрямился.
    - Добрый день! - заговорил он. - Простите, что меня немного 
качает. Просто я двигался со слишком большой скоростью, чтобы успеть к 
вам.
    Освобожденный от хватки человека-смерча, пылевой столб распался, 
превратившись в легкий туман, и скоро совсем исчез.
    У человека-смерча Вилли-Вилли было улыбающееся загорелое лицо. 
Носил он куртку и брюки цвета красной земли, как песок пустыни, как 
краска, которой разрисовывают себя аборигены. Он приподнялся на носках 
в своих гибких ботинках и, раскинув руки в стороны, несколько раз 
обернулся вокруг себя.
    - Предпочитаю останавливаться постепенно, - пояснил он. - Вращение 
с такой скоростью, какая была у меня, требует большого внимания. - Он 
потер руки, очищая их от пыли, и продолжал: - Я уже говорил, вращение 
- это счастье движения. Вопрос лишь в том, танцую ли я потому, что 
счастлив, или же счастлив потому, что танцую!
    - А кому ты это говорил? - удивился Питер.
    - Одному дубу. Он вечно вздыхает. Он вздыхает, когда нет ветра, он 
вздыхает, когда есть ветер... Я посоветовал ему танцевать и махать 
своими руками-ветками. Любой из нас ведь может загрустить, когда 
просто стоит и ничего не делает, верно?
    - Пожалуй, - согласился Питер. - Но я никогда не слышал, чтобы 
дерево танцевало.
    - Ну как же! - воскликнула Серая Шкурка, шокированная невежеством 
друга. - Что же, по-твоему, деревья делают, когда ты спишь? А спишь ты 
треть своей жизни. Деревья же в это время танцуют.
    - Никогда не думал об этом, - признался Питер.
    - Они танцуют, покачивая ветвями, - продолжал Вилли-Вилли. - Они 
раскачиваются вместе с тенью и скользят вместе с Луной. Между прочим, 
о тебе мне тоже рассказало дерево. - Маленький человечек улыбнулся. - 
Оно сказало: "Когда Питер подойдет к Пустыне Одиночества, он будет 
нуждаться в друге. Встреть его и перенеси через пустыню, хорошо?" И 
вот я здесь.
    - Разве она так велика, что мы не сможем перейти ее сами? - 
удивился Питер. - Я поеду верхом, а Серая Шкурка поскачет рядом.
    - Конечно, на самом деле пустыня не такая уж большая. Но тем 
детям, которые идут по ней в одиночку, без друга, она представляется 
бесконечной, и они легко сбиваются с пути. Я тебя перенесу через нее, 
не беспокойся, и для этого сделаю не простой смерч, а смерч-громадину! 
Это будет здорово!
    - Меня при вращении тошнит, - заявил Питер.
    - Меня тоже, - поддержала Серая Шкурка. - Больше шести вращений я 
не выдерживаю. После этого я отключаюсь.
    - Правда? - удивился Вилли-Вилли. - Но внутри смерча будет 
совершенно спокойно, и вы совсем не будете вращаться. Вы будете сидеть 
на воздушной подушке, а пыль и листья будут вращаться вокруг вас. - Он 
огляделся. - Я сейчас попробую завестись и протанцую несколько миль по 
пустыне, пока не наберу двадцати ярдов в диаметре, а потом вернусь и 
подхвачу вас.
    - Тревожно мне что-то, - шепнула Серая Шкурка. - Я понимаю, что 
друзья необходимы, но не такие, от которых на тебя накатывает дурнота.
    - Вилли-Вилли ведь обещал, что мы не будем вращаться, - успокоил 
ее Питер. - Он наш друг, а друзья никогда не делают ничего во вред и 
не пугают.
    - Ладно уж, рискнем, - согласилась Серая Шкурка. - Когда мы 
отправляемся? - обратилась она к Вилли-Вилли.
    - Жди, я скажу, - ответил человек-смерч. - Ты, Питер, садись на 
лошадь, а ты, - он обернулся к Серой Шкурке, - стой рядом и держись за 
стремя.
    Они так и сделали: Питер сел верхом, а Серая Шкурка стала рядом. 
Она крепко сжала кожаный ремень и закрыла глаза.
    Питеру стало страшно. Он позвал Вилли-Вилли:
    - Послушай, прежде чем мы отправимся, я хотел бы подарить тебе 
этот Лист, - и он вручил Вилли-Вилли Волшебный Лист.
    Маленький человечек был тронут.
    - Какой чудесный подарок! - воскликнул он. - Не беспокойся, я буду 
осторожен и внимателен, ведь ты - мой лучший друг. Я не стану делать 
смерч таким огромным, как говорил. Если честно, то я чуть-чуть 
прихвастнул, двадцать ярдов в диаметре мне не осилить.
    Он сунул Лист в карман, приподнялся на носки и, вытянув руки в 
стороны, начал вращаться. Где-то внутри пего несколько раз чихнул 
мотор, но пыль в воздух не поднялась. Человек-смерч остановился и сел.
    - Видите ли, - стал он оправдываться. - Мотор у меня двухтактный, 
а двухтактные всегда заводятся плохо. Знаете, как трудно завести 
газонокосилку? У вас случайно нет при себе шнура?
    - Сейчас дам, - отозвалась Серая Шкурка. Она пошарила в сумке и 
вытащила пусковой шнур с деревянной ручкой на конце.
    - Ничего себе! - изумился Вилли-Вилли. - Будь у меня такая сумка, 
я сколотил бы целое состояние. В любом приличном магазине такой шнур 
продается не меньше, чем за полдоллара. Ты не могла бы поставлять их 
мне крупными партиями? Беру двенадцать дюжин, естественно, с 
десятипроцентной скидкой за оплату наличными.
    - Не говори глупостей, - рассердилась Серая Шкурка. - У меня тут 
не лавка запчастей. Я люблю людей и просто дарю то, что им нужно.
    - Конечно, конечно, - сконфузился Вилли-Вилли. - Прошу простить 
меня, - и он погладил Лист.
    Человек-смерч попытался еще раз запустить себя. Он обмотал 
пусковой шнур вокруг талии и попросил Питера дернуть за него.
    - Лучше всего я завожусь, если дернуть сильно и резко - одним 
рывком, но изо всей силы, - объяснил он.
    Питер дергал несколько раз, затем попробовала Серая Шкурка, но 
мотор все не подхватывал и даже ни разу не чихнул. Человек-смерч 
вращался, пока не кончался шнур, а потом останавливался.
    - Не понимаю, что с ним случилось, - озабоченно произнес Вилли-
Вилли. - Я прошел капитальный ремонт всего месяц назад. Надо серьезно 
подумать о переходе на четырехтактный.
    - Наверно, загрязнилась свеча зажигания, - предположил Питер. - А 
как зазор между контактами - нормальный?
    Он все знал о двухтактных моторах, потому что у Кривого Мика была 
газонокосилка, которая тоже плохо заводилась.
    - Сейчас посмотрим, - с надеждой сказал Вилли-Вилли.
    Он сунул руку в карман, достал оттуда свечу и принялся пристально 
ее разглядывать. Потом лизнул ее, потер о рукав, снова лизнул, снова 
потер о рукав, и после этого поднес к глазам.
    - Кажется, чистая... - решил он.
    - А зазор между контактами? - снова спросил Питер.
    Вилли-Вилли протянул свечу Питеру и тот постучал ею о камень.
    - Попробуй еще раз.
    Вилли-Вилли отправил ее в карман, и Питер снова дернул за шнур.
    - Та-та-та, - затянул человек-смерч, - та-та-тр-тр-тр-тр, - и он 
начал вращаться.
    Питер, все еще со шнуром в руке, отскочил в сторону, и человек-
смерч пронесся мимо, увлекая за собой пыль.
    - Стойте вместе, я через минуту вернусь и заберу вас! - крикнул он 
на ходу.
    Он понесся по пустыне, подбирая на ходу пыль. Она вилась в 
воздухе, напоминая спущенную с небес гигантскую веревку, которая к 
тому же раскачивалась туда-сюда.
    Потом он начал гудеть, гул разрастался и наконец вся пустыня 
наполнилась неистовым ревом. Всосанная пыль бешено крутилась огромными 
кругами, быстро поднимаясь и превращая веревку в столб, а столб - в 
кошмарную движущуюся башню. Чахлые деревца, над которыми она 
проносилась, неистово вскидывали головы. Смерч срывал с них листья и 
бросал вверх, чтобы они кружились вместе с пылью на высоте пятьсот 
футов. Смерч отошел примерно на милю, а потом по кругу стал 
возвращаться назад. От его грома дрожала земля.
    Перед собой смерч гнал стаю кенгуру, которые затем свернули куда-
то в сторону. С его пути удирали, также на полной скорости, несколько 
эму, вытянув вперед длинные, похожие на копья, шеи, и поднимая лапками 
фонтанчики пыли.
    - Не нравится мне это, - сказала Серая Шкурка. - Совсем не 
нравится.
    - Мне тоже, - признался Питер. - Мне даже немножко страшно.
    Он вскочил на Мунлайт, а Серая Шкурка стала рядом, держась за 
стремя.
    - Я же дал ему Волшебный Лист, - сказал Питер, - так что зла он, 
нам, конечно, не причинит. Но я все равно волнуюсь
    - А я не боюсь, - решительно заявила Серая Шкурка, - просто Вилли-
Вилли себя не контролирует. Он плохой водитель, вот что! Держись! - 
неожиданно крикнула она.
    Смерч развернулся и двигался теперь прямо на них. Воздух вокруг 
заколыхался и грива Мунлайт забилась на ветру. Серая Шкурка спрятала 
голову в седло. Питер нагнулся и уткнулся в гриву Мунлайт.
    Завывающий столб пыли возвышался над ними. Листья и веточки 
кружились в вихре, как мотыльки вокруг света. Смерч грохотал, 
скрежетал, мотаясь из стороны в сторону на всей своей страшной высоте, 
словно пораженный болью гигантский удав. Внезапно он накрыл Питера и 
Серую Шкурку, после чего они ощущали только вой ветра.
    Потом Питер и Серая Шкурка почувствовали, как невидимые руки 
подхватили их и втянули внутрь. Мунлайт рванулась, встала на дыбы, 
вместе с ней поднялась и Серая Шкурка, инстинктивно сжимавшая стремя. 
Она изо всех сил держалась за него, хотя лапы ее вдруг ослабли и она 
чуть не разжала пальцы. Питеру казалось, будто он сидит на 
необъезженной лошади, но он лишь сжал колени, и продолжал сидеть в 
прежней позе; он чувствовал, как вертит под ним лошадь. Неожиданно их 
обоих выхватило из тьмы и пыли и перенесло в спокойное недвижное место 
- это было высоченное пространство с гладкими вращающимися стенами. 
Впечатление было такое, словно они попали в середину гибкой трубы, 
уходящей в необозримую высоту и теряющуюся среди грозовых туч. Здесь 
царил покой, снаружи сюда еле доносился, словно из-за стены, рев 
могучего смерча.
    Сам Вилли-Вилли сидел на стуле и пил чай, налив его из фляги. Его 
стул неподвижно висел в воздухе. Рядом стояли еще два.
    - Садитесь, - пригласил Вилли-Вилли, жестом указывая на стулья. - 
Я всегда держу их под открытым небом, чтобы они были рядом, если мне 
вдруг захочется проехаться с комфортом.
    - А ты уверен, что они нас выдержат? - спросила Серая Шкурка. - 
Они же ни на чем не держатся.
    - А в самолетах? - вопросом на вопрос ответил Вилли-Вилли. - В 
самолетах ведь есть кресла.
    - Да, но там все иначе, - не соглашалась Серая Шкурка.
    - Да эти стулья слона выдержат! - уверил ее Вилли-Вилли.
    - Что ж, - сказала Серая Шкурка, - посмотрим. - И она выдернула из 
сумки слона. Это был тот самый слон, которого она вытащила, когда 
впервые встретила Питера. Слон был взбешен.
    - Послушай, ты! Уж не собираешься ли ты вытаскивать меня из своей 
проклятой сумки каждый раз, когда вам приспичит разрешить спор? - 
раздраженно взревел он. - Это уже переходит всякие границы! Мне это 
надоело!
    - Я хочу лишь, чтобы ты сел на этот стул, - спокойно объяснила 
Серая Шкурка. - Давай, дружище, садись!
    Слон, недовольно ворча, сел.
    - Ну вот видите! - воскликнул Вилли-Вилли. - Не шелохнулся!
    - Ты прав, - сдалась Серая Шкурка. Затем обратилась к слону: - 
Спасибо, дружище. Извини, что побеспокоили.
    - Чтоб это было в последний раз! - по-прежнему сердился слон. Он 
хотел добавить что-то еще, но Серая Шкурка схватила его и головой вниз 
бросила в сумку, где он и исчез. Потом она опустилась на стул.
    - Имея такой талант, следует соблюдать осторожность, - сказал 
Вилли-Вилли. - Ведь смерч не предназначен для переноски слонов. Он мог 
бы вообще развалиться, и мы сломали бы себе шеи. Да и так у нас упала 
скорость. Надо перейти на пониженную передачу. Сейчас я переключу на 
вторую.
    Он достал из кармана рычаг переключения передач и прикрепив к 
ботинку, стал крутить туда-сюда, пока смерч вновь не начал набирать 
обороты.
    - Тр-тр, тр-тр, - громыхал он. Вилли-Вилли опять поднял рычаг в 
верхнее положение и смерч стал вращаться быстрее и быстрее.
    Питер сел на другой стул и устроился поудобней. Мунлайт стояла 
рядом, равнодушная и к окружающему гулу, и к тому, что стоит на 
воздухе. Она только задирала голову и настороженно шевелила ушами.
    Питер погладил и похлопал ее по шее.
    - Внимание! - неожиданно произнес Вилли-Вилли подчеркнуто 
серьезным тоном. - Желаю вам приятного путешествия! Мы летим на высоте 
пятьсот футов, подгоняемые попутным ветром. Координаты нашего 
местонахождения: 65 минут долготы и 23 минуты широты. К югу и северу 
от гор местами туман. Температура в Дарвине 89 градусов, в Мельбурне 
52 градуса. В аварийной ситуации пристегните ремни безопасности и 
надуйте плот, находящийся в сиденье под вами. Сохраняйте его в надутом 
виде во время приземления. Непосредственно перед посадкой поместите 
плот под себя. Благодарю за участие и сотрудничество.
    Напоследок он с важным видом кашлянул, чтобы все поняли, какой он 
блестящий оратор.
    - Ничего не понимаю. Что-то у тебя все с ног на голову поставлено, 
- отозвался Питер. Речь эта показалась ему верхом глупости.
    - Никто тебя и не просит искать ноги и головы в том, что я сказал, 
- обиделся Вилли-Вилли. - Новые головы и ноги никому не нужны - у всех 
есть свои. А говорил я от чистого сердца, - с чувством закончил он.
    На Серую Шкурку речь, наоборот, произвела впечатление.
    - Очень интересно, что в Дарвине температура 89 градусов, - 
сказала она. - Но меня беспокоит надувание плотов при приземлении. Это 
очень ответственно, особенно в случае аварии.
    - Все это ерунда! - гнул свое Питер.
    Неожиданно труба, в которой они сидели, наполнилась сеном. Оно 
летало над головами, кружилось внизу под их стульями, - там его было 
особенно много. Оно щекотало им кожу, вилось вокруг ног, а потом 
уходило вверх и в конце концов припечатывалось к стенам. Где-то далеко 
внизу, на земле, сердито кричали люди, и труба усиливала их крики.
    - О боже! - воскликнул Вилли-Вилли. - Мы переехали стог сена! Оно 
принадлежит четырем карликам, что живут в пещере. Слышите, как они 
кричат? Надо будет вернуть сено, когда я вас высажу. Ой-ей-ей! 
Настоящий грабеж!
    - Отличное сено, - сказал Питер, разглядывая колосок. - В нем 
много овса, то, что нужно для лошадей. - Ему вдруг очень захотелось 
показать это сено Кривому Мику.
    - Надо определить, где мы находимся, - озабоченно сказал Вилли-
Вилли. - Я несколько отклонился от курса.
    Он вытащил из-под стула подзорную трубу и раздвинул ее - она 
оказалась довольно-таки длинной. Просунув один конец сквозь стену 
смерча, он приложился к окуляру.
    - Мы идем на северо-юго-восток со скоростью сто узлов, - объявил 
он. - В заливе Порт-Филип ветер свежий.
    - Но отсюда до Порт-Филипа тысячи миль, - сказал Питер. Он уже 
начал сомневаться, знает ли вообще Вилли-Вилли, куда несется.
    - Но я и вижу на тысячи миль в свою подзорную трубу, - ответил 
Вилли-Вилли. - Сейчас мы проходим мимо высоких скал. Вижу огромного 
кенгуру-валлаби. А сейчас вижу одинокого человека, бредущего по 
пустыне, он сбился с пути.
    Вилли-Вилли резко сложил трубу и повел смерч по дуге, чтобы 
выравнять курс и взять заблудившегося. Смерч подхватил его, как 
листок, втащил наверх и поставил рядом со всеми.
    - Хочу пить, - сказал человек. - Я сбился с пути, и не мог идти 
дальше.
    Серая Шкурка достала из сумки бутылку лимонада и протянула ему; он 
жадно его выпил,
    - Нельзя переходить Пустыню Одиночества без друга, - сказал Вилли-
Вилли. - Тебе повезло, что мы оказались неподалеку.
    - Это верно. Я увидел смерч, но никак не подумал, что в нем может 
кто-нибудь находиться. Признаться, я даже испугался, когда он меня 
подхватил. - Внезапно он замолк, прислушиваясь, как постукивает смерч, 
затем заявил: - Надо подрегулировать мотор. Он неправильно дает искру.
    - А ты разбираешься в двухтактных? - изумился Вилли-Вилли. - Кто 
ты по профессии?
    - Механик-моторист.
    - Двухтактные заводить можешь?
    - Запросто.
    - Тогда могу предложить тебе работу, - сказал Вилли-Вилли, 
довольный собой. - Как тебя звать?
    - Том.
    - Хорошее имя. Короткое и простое. Мы с тобой будем друзьями, Том.
    Неожиданно труба выбросила вверх листья, которые покружились 
вокруг и прилипли к стене.
    - Мы подошли к опушке Недреманного Леса, - объяснил Вилли-Вилли. - 
Я, наверное, зацепил эвкалипт. Но я ему сейчас все возмещу. Готовьтесь 
к посадке. Пристегните ремни!
    - Но на стульях нет никаких ремней! - воскликнул Питер, торопливо 
шаря по стулу.
    - Ты прав, их нет, - нехотя согласился Вилли-Вилли. - Тогда не 
пристегивайтесь. Держитесь - опускаемся! Вууу-у! - И он закрыл глаза.
    Смерч сделал круг и остановился. Вращение прекратилось, и все 
стали медленно, как листья, опускаться. Они опускались все ниже и 
ниже, пока не почувствовали под ногами землю. В неподвижном воздухе на 
них начало падать сено, листья и пыль, и скоро вокруг них вырос целый 
стог.
    Очутились они на опушке Недреманного Леса. Здесь густо росли 
эвкалипты, которые умели подслушивать мысли и шепотом передавать 
дальше, так что вскоре уже весь лес знал о появлении гостей и об их 
планах.
    - Только добрые люди могут пересечь этот лес, - начал человек-
смерч. - Волшебный Лист защитит вас. Я перенес вас через Пустыню 
Одиночества, как и обещал. Больше вы никогда не будете одиноки. Тем, 
кто хоть раз пересек эту пустыню, одиночество не страшно. Видите вон 
ту гору? - Он указал в сторону пика, торчавшего из-за деревьев на 
горизонте. - Это Последняя Гора. Идите к ней. Там вы узнаете ответы на 
все ваши вопросы. - Он повернулся к Тому. - А теперь заводи меня, Том. 
Такого человека, как ты, я искал всю жизнь. Обмотай меня вокруг пояса 
шнуром и дерни.
    - Где шнур? - спросил Том.
    - Ой-ей-ей, я, наверно, потерял его! - воскликнул Вилли-Вилли, 
ощупывая карманы.
    Серая Шкурка вытащила связку шнуров и протянула Тому:
    - Следи за ними, Том, он вечно их теряет. И за ним тоже следи. Он 
хороший человек, но ему необходим друг и помощник.
    - Не беспокойся, я послежу за ним, - пообещал Том.
    Все вместе они пролезли сквозь стог и выбрались на чистое место. 
Мунлайт по пути жадно хватала сено, поэтому Питер взял две охапки и 
отнес ей в сторонку под дерево, где она могла спокойно поесть.
    Человек-смерч стоял на клочке свободной земли, среди пыли, и Том 
обматывал его шнуром.
    - Я вернусь за тобой, Том, - сказал Вилли-Вилли. - Встань на копну 
и жди меня. Нам надо отдать сено карликам.
    - Понятно, всем отойти! - скомандовал Том и дернул за шнур. 
Человек-смерч несколько раз кашлянул, чихнул и завращался в облаке 
пыли. Он вернулся в пустыню, там начал раскачиваться и расти, пока не 
поднялся до небес. Вскоре он развернулся и устремился туда, где Том 
ждал его, стоя на сене. Поднимаясь в воздух на огромной копне, Том на 
прощанье махнул рукой Питеру и Серой Шкурке и исчез в пыли. Смерч 
полетел прочь, становясь все меньше и меньше, пока не исчез совсем.

        Глава 9
        БИТВА С КОТАМИ СОМНЕНИЙ

    Эту ночь Питер и Серая Шкурка провели под деревом на опушке 
Недреманного Леса и утром, едва солнце стало всходить, были на ногах. 
Свои спальные мешки они аккуратно свернули, и Серая Шкурка опустила их 
в сумку. Расположившись на берегу ручья позавтракать, Питер и Серая 
Шкурка наблюдали, как черные утки плыли вверх по течению, направляясь 
к тихим заводям, скрытым в кустах. Время от времени в воздухе 
раздавалось хлопанье крыльев - это пролетали чирок или свиязь.
    - Вот бы здесь жить, - вздохнула Серая Шкурка.
    - Да, я бы тоже не отказался, - согласился Питер.
    Ветер шевелил листья старого дерева, под которым они сидели, и 
Питеру показалось, что оно стало что-то нашептывать. Питер поднял 
голову, пытаясь определить, откуда идет звук. Он пристально смотрел на 
колышащиеся листья и ждал, когда шепот усилится.
    - Послушай, - обратился он к Серой Шкурке, - по-моему, это дерево 
пытается что-то нам сообщить.
    Серая Шкурка тоже посмотрела наверх и, замерев, стала ждать.
    Шепот усиливался.
    "Недреманный Лес очень жесток, - шептала крона. - Каждое дерево, 
что растет здесь, подслушивает и подсматривает. Вам будет казаться, 
что за вами наблюдают тысячи глаз. Идите прямо по тропе и не сходите с 
нее, чтобы поглазеть на ручейки и водопады, которые вам встретятся. В 
центре низины на вас набросятся Коты Сомнений. Размером они с 
леопардов, а полосы на шкуре, как у тигров. Они будут выть и реветь и 
испугают вас. Но вы должны сразиться с ними и идти вперед, пока не 
дойдете до Последнего Холма".
    Шелест листвы прекратился, и шепот затих. Питер поднялся, похлопал 
дерево по стволу, затем, обернувшись к Серой Шкурке, сказал:
    - Пора в путь. Через этот лес нам ехать долго, может быть, 
несколько дней.
    - Я готова, - сказала Серая Шкурка. Она подождала, пока Питер 
вскочит на Мунлайт, и запрыгала следом. Переходя неглубокий ручей, 
Мунлайт подняла фонтан брызг и остановилась попить. Потом они 
выбрались на противоположный берег и пустились по узкой тропинке, 
которая и привела их в лес.
    Вокруг них недвижно высились огромные стройные стволы эвкалиптов, 
среди которых то тут, то там попадались акация и черное дерево. Лесные 
цветы уже начали пробиваться из-под прошлогодних листьев, вовсю цвел 
вереск. Воздух был настолько чудесен, что Питер сделал несколько 
глубоких вдохов, перевел Мунлайт на легкий галоп и ехал, раскачиваясь 
из стороны в сторону. Серая Шкурка прыгала следом, ее хвост то 
поднимался, то опускался подобно рукоятке водяного насоса.
    Они двигались весь день, а когда начали опускаться сумерки, нашли 
старый эвкалипт с огромным дуплом, дно которого было устлано сухой 
корой, а поверху лепились сделанные из глины гнезда ласточек. В этом 
дупле Питер и Серая Шкурка провели ночь, пока утром их не разбудили 
солнечные лучи.
    На второй день идти стало намного труднее. Местами тропинка совсем 
заросла, а иногда они наталкивались на упавшее дерево, которое 
перегораживало дорогу. Такие преграды они легко преодолевали. Мунлайт 
разгонялась, взвивалась в воздух и перелетала через них, как на 
крыльях. Серая Шкурка прыгала следом, так что иногда в воздухе они 
оказывались бок о бок. Как-то раз в полете Питер протянул руку и 
дотронулся до кенгуру. Так они и приземлились вместе. Это было 
здорово!
    И все же некоторое время спустя они ощутили усталость. Питера, 
который обычно ни на что не жаловался, охватило уныние. Стала 
недовольно ворчать и Серая Шкурка.
    - Похоже, нам ни в жизнь отсюда не выбраться, - захныкала она.
    - И Принцессу нам не найти, - вторил ей Питер. - Мы все идем и 
идем, а замка все нет и нет.
    - Я знаю, что надо делать! - заявила Серая Шкурка. - Надо 
поворачивать назад. Я устала от того, что все здесь подслушивает тебя, 
все подсматривает за тобой. Нам надо...
    Ее прервал ужасающий, невообразимый вой. Он раздался из густых 
зарослей в низине, куда сбегала тропинка. Это был даже не вой, а почти 
вопль, и едва он стих, как ему ответил другой, затем - еще один, и вот 
уже вся долина огласилась таким кошачьим концертом, что Питер и Серая 
Шкурка остановились как вкопанные и в испуге посмотрели друг на друга.
    - Наверное, это Коты Сомнений, - сказала Серая Шкурка. - Помнишь, 
что говорило дерево? Хорошо бы выпутаться отсюда без всяких сражений.
    - Хорошо бы выпутаться хоть как-нибудь! - поправил ее Питер.
    Тропа, по которой они спускались, шла наискось по склону холма. 
Справа круто вверх уходил склон, весь покрытый адиантумом и мхом. 
Слева край тропы нависал над плотными колючими зарослями кизила. В 
конце склона дорогу пересекал ручей, с трудом продиравшийся сквозь 
кустарник. Противоположный склон был почти голым, и на нем стоял 
высокий, обожженный эвкалипт, раскинув на фоне неба растопыренные 
ветви.
    Мимо этого дерева пробежал один из Котов, подобный тигру. 
Огромными скачками он мчался к тропинке наперерез Питеру и Серой 
Шкурке. Следом за ним бежали другие Коты, а неистово раскачивающийся 
кустарник говорил о том, что сквозь него продираются новые и новые 
зверюги. Они яростно выли, и Питеру пришлось изо всех сил натянуть 
поводья, чтобы сдержать Мунлайт.
    - Нас спасут только ноги, - сказала Серая Шкурка. - Вперед! От 
собак я уже бегала, но чтобы удирать от гигантских котов... Вот уж 
никогда не думала, что доживу до такого.
    - Надеюсь, их когти не достанут до Мунлайт.
    - А ты угости их Громобоем, - посоветовала кенгуру. - Разверни-ка 
кнут и повращай его. Не подпускай их к лошади. А если кто из котов 
прыгнет, постарайся, чтоб Громобой подпортил ему шкуру. Ну что, готов?
    - Готов, - ответил Питер. В правой руке он держал кнут, в левой - 
поводья. Он привстал в седле, слегка наклонился вперед и пустился в 
галоп. Мунлайт неслась вниз по тропе с бешеной скоростью, ее уши были 
наставлены вперед, грива развевалась по ветру. Рядом скакала Серая 
Шкурка. При каждом скачке она взлетала так высоко, что равнялась с 
Питером. А опустившись, тут же отталкивалась от земли мощными лапами и 
через мгновение снова была в воздухе.
    Мунлайт бежала настолько уверенно, что ни разу не оступилась в ямы 
и рытвины, которые образовались в земле от дождей. Она то огибала их, 
то перепрыгивала.
    Один Кот, с полосами, как у обычной кошки, выскочил на тропу 
впереди них. Глухо рыча, он повернулся и прыгнул на Серую Шкурку, 
которая, однако, успела приготовиться к встрече: задними лапами она 
сильно ударила Кота в грудь да еще вдобавок разодрала ему шкуру своими 
длинными когтями.
    От кошачьего воя у всех заложило уши. Коты вылетали из кустов, 
словно полосатые снаряды. Кнут Питера щелкал без остановки, и каждый 
щелчок вырывал из кошачьей шкуры клок шерсти, который плавно опускался 
на землю. Получив удар кнутом, Коты валились, а когда приходили в 
чувство, Питер был далеко. Догнать его они уже были не в силах.
    {whisp06.gif}
    Никто из Котов так и не дотянулся до Питера, хотя одному удалось 
зацепить лапой седло и разорвать Питеру брюки. Питер огрел его по 
голове рукояткой кнута, в которую был залит свинец, и оглушенный Кот 
брякнулся на землю. Других Котов это, правда, не остановило, и они 
продолжали выскакивать из кустов. Несколько раз они царапнули когтями 
бока Мунлайт, и та, разъярившись, с такой силой лягала их подкованными 
копытами, что они еще в воздухе нагнали корчиться и вопить, а затем 
исчезали позади.
    Серая Шкурка тем временем сцепилась с одним особенно свирепым 
животным, которому удалось избежать удара ее задних лап. Обхватив его 
передними лапами, она со страшной силой сдавила его и не отпускала до 
тех пор, пока тот не обмяк. Тогда она бросила его на землю и поскакала 
дальше.
    Подлетев к ручью, который пересекал дорогу, Мунлайт и Серая Шкурка 
одновременно изо всех сил оттолкнулись от земли, чтобы разом 
перемахнуть через Котов, поджидавших их у берега. Коты тянулись вверх, 
загребая лапами воздух, но достать Питера и Серую Шкурку так и не 
смогли.
    Теперь дорога пошла в гору, так что Мунлайт и Серой Шкурке 
пришлось напрячь все свои мощные мускулы, чтобы уйти от преследования. 
Справа и слева на них выскакивали новые и новые Коты, но наши друзья 
успевали промчаться мимо них прежде чем те прыгали.
    Сейчас все решала выносливость, а уж по этой части Мунлайт и Серая 
Шкурка могли потягаться с кем угодно. Наконец, вопли их 
преследователей затихли вдали, и Питер перевел Мунлайт на рысь, чтобы 
она могла немного отдохнуть.
    - Больше они до нас не доберутся, - сказала Серая Шкурка. Теперь, 
когда опасность миновала, она очень гордилась собой.
    Питер не ответил. Он слишком устал, ему хотелось только одного: 
поскорей найти место для ночной стоянки.
    Они поднялись из низины, взобрались на гребень невысокого холма, с 
которого открывался вид на многие мили вокруг, и застыли. Перед ними 
высился Последний Холм, а на его вершине рос старый красный эвкалипт. 
Его искореженный ствол, казалось, согнулся под тяжестью своих же 
ветвей, которые переплетались на фоне неба, вытягивая из воздуха 
солнечное тепло и влагу.
    Питер и Серая Шкурка поспешили к нему. Доскакав до зеленой 
травянистой лужайки, Питер спрыгнул на землю. Серая Шкурка догнала его 
и остановилась. Вот, наконец, оно перед ними, это старое как мир 
дерево, - так назвал его старик-абориген. Мощные корни дерева словно 
могучими пальцами стискивали землю. Ни один ураган не мог повалить 
его. Питер и Серая Шкурка чувствовали, что рядом с этим великаном в 
них самих как бы вливались небывалые силы. Любое дело теперь казалось 
им по плечу. Они найдут Прекрасную Принцессу, обязательно найдут.
    Они уже успели проголодаться, и потому Серая Шкурка достала из 
сумки стул и стол и накрыла его как для принца. Они принялись за еду, 
и им показалось, что ничего более вкусного они никогда еще не едали. 
Потом Серая Шкурка убрала все обратно в сумку и достала два спальных 
мешка, которые они развернули изголовьем к дереву. Питер расседлал 
Мунлайт, снял с нее уздечку и пустил пастись на лужайке с высокой 
травой.
    Так лежали они, ожидая восхода луны, ведь именно на ее фоне, как 
сказал Южный Ветер, Питер должен был увидеть замок Прекрасной 
Принцессы.
    Вот на востоке появился еле заметный отсвет. Из тьмы проступили 
силуэты неподвижных деревьев, молча ожидавших взрыва лунного света, 
что возвещало восход луны. Наконец, из-за горизонта показалась 
изогнутая полоска желтого света. Она росла, росла и в конце концов 
превратилась в сияющий диск, который лежал на кромке земли: темным 
силуэтом на нем проступили башни и укрепления огромного замка.
    И тут зашелестели листья старого эвкалипта, что-то нашептывая. 
Звуки сливались в один глухой, по мягкий голос. Дерево заговорило.
    "Перед тобой замок Прекрасной Принцессы, Питер. Твой поиск 
подходит к концу. Впереди еще много трудностей, но ты доказал, что 
храбр и любишь людей. Волшебный Лист и в будущем защитит тебя и 
придаст сил справиться со всеми испытаниями. Он уже изменил жизнь тех, 
кто встретился тебе на пути; те, что были прежде жестокими и злыми, 
сейчас будут помогать путникам, которых встретят, а не убивать их.
    Завтра тебе предстоит трудный день, но замок уже недалеко, и путь 
к нему открыт. Ты сам найдешь способ встретиться с Прекрасной 
Принцессой. А теперь спи. Тебе ничто не причинит зла, пока ты спишь 
под моей кроной".
    Сверху склонилась темная ветка и листьями слегка прикоснулась к 
Питеру. Это был дружеский жест.
    Питеру дерево понравилось.
    - Ты слышала, что оно сказало? - шепотом спросил он у Серой 
Шкурки. - Оно коснулось меня своими листьями. Оно замечательное!
    - Похоже, это славное существо, - ответила Серая Шкурка уже в 
полусне.
    Когда они крепко уснули, тень от дерева полностью накрыла их, и в 
темноте одежда Питера изменилась. Теперь на нем был костюм принца, и 
выглядел он как настоящей принц.

        Глава 10
        ПОЯВЛЕНИЕ БУНЬИПА

    Рано утром Питер и Серая Шкурка собрались и тронулись в путь. 
Деревья, подсвеченные первыми лучами солнца, отбрасывали длинные тени, 
которые, играя, падали на тропинку, на Питера и Серую Шкурку.
    Пройдя около двух миль, они увидели замок: за поворотом тропинки 
открылось могучее строение с башнями, бастионами и зубчатыми стенами. 
В каменных стенах были прорезаны узкие окна-бойницы, зарешеченные, как 
в тюрьме. Без решетки оставалось лишь одно окно - под самой крышей 
огромной башни, оно было узким и скругленным сверху. Сотни птиц сидели 
на его подоконнике и кружились рядом, пытаясь добраться до еды, 
насыпанной на каменном карнизе. Питер подумал, что там-то, наверное, и 
находится комната Прекрасной Принцессы. Но ее он так и не увидел, 
должно быть, она была чем-то занята и не подходила к окну.
    Замок окружал глубокий ров с темной, неподвижной водой. 
Перебраться через него можно было только по подъемному мосту, который 
соединял ворота замка с дорогой, по которой пришли Питер и Серая 
Шкурка. В этот момент мост был поднят и удерживался наверху двумя 
мощными цепями, которые уходили в стену замка и где-то внутри 
соединялись с механизмом, опускавшим и поднимавшим мост.
    Позади моста можно было разглядеть огромные сводчатые ворота, 
окованные медью. Металлические шарниры выделялись на фоне деревьев как 
солнечные шары. Ворота были такой ширины, что в них могли бы въехать 
бок о бок четыре всадника, и такой высоты, что сквозь них можно было 
пронести поднятые флаги и знамена. Сейчас ворота были заперты и 
прошибить их невозможно было никаким тараном.
    Вокруг всего замка шла вытоптанная темная дорожка. Иногда она 
отходила ото рва и исчезала среди кустарников, потом появлялась снова 
и шла все дальше и дальше по кругу, пока не соединялась со своим 
началом у ворот замка. У обочины этой дороги, примерно в ста ярдах от 
моста, рос старый эвкалипт: его широкие, раскидистые ветви 
образовывали пятно густой тени посреди жаркого света уже поднявшегося 
солнца.
    Под деревом, сложив когтистые лапы на жирном брюхе, лежало на 
спине самое удивительное животное, какое когда-либо видел Питер. Если 
бы оно встало, то, наверное, всем своим видом и размером напомнило бы 
динозавра, но сейчас, спящее, оно выглядело не очень страшным.
    Питер и Серая Шкурка вышли из кустов и, подойдя поближе к 
чудовищу, принялись его рассматривать. Оно было все, от носа до 
хвоста, покрыто шерстью. Тело его напоминало громадного вомбата, 
толстый, малоподвижный хвост - кенгуру, длинная шея - жирафа, а голова 
- дракона. Но вместо панцирной чешуи и острого гребня, какие бывают у 
драконов, у него была густая шерсть. На голове шерсть была длинной и 
неопрятной, она даже свисала ему на глаза, и было видно, что ее ни 
разу не расчесывали. Во сне чудовище храпело, и в такт храпу 
поднимались и опускались его скрещенные на брюхе передние лапы.
    {whisp07.gif}
    - Кто бы это мог быть? - в изумлении спросил Питер.
    - Это Буньип, который день и ночь сторожит Прекрасную Принцессу. 
Нам про него говорили.
    - Ах да, я и забыл.
    - Он очень злобный, - продолжала Серая Шкурка, но в голосе ее 
появились нотки неуверенности, так как спокойное похрапывание 
продолжалось. - По крайней мере, так говорят... Рыцарей и принцев он 
убивает десятками. Он их подпаливает и поджаривает, обдавая огнем из 
ноздрей. Посмотри-ка: не ноздри, а печные трубы. Давай лучше отойдем, 
да обсудим, как с ним бороться. Не нравятся мне его ноздри. Вмиг 
сделают из тебя поджарку. Идем.
    Она попятилась, но Буньип неожиданно вскочил и в изумлении 
уставился на них.
    - Что вы тут делаете? - проревел он. - Ваши имена? Кто вы такие? 
Стоять смирно! Произнесите по буквам "фантасмагория". Проводимые за 
обычную плату экскурсии для туристов начинаются в два часа. Вы входите 
через передние ворота, а выносят вас через заднюю калитку на носилках: 
а там уже наготове доктор! Теперь отвечайте, или вы умолкнете навеки!
    - Ну, - начал Питер, - если говорить о твоем замечательном 
приветствии, то оно показалось мне несколько путанным.
    - Твоя правда, - согласился Буньип. - Такой уж я путаник. Что 
дальше?
    - Ты стережешь Прекрасную Принцессу?
    - Да. А в чем дело?
    - Я пришел спасти ее!
    - Вот как! Жаль. Очень, очень жаль... Не люблю и никогда не любил 
убивать дружелюбных юношей. Но это моя работа. Я убиваю рыцарей на 
черных конях и на белых конях, убиваю любых принцев, - для меня нет 
разницы - убиваю без малейшего снисхождения. Прекрасная Принцесса 
должна быть защищена от всякого, кто захочет ее спасти.
    Она - если можно употребить такое словцо, - неспасабельна. И не 
забудьте, - поспешил он добавить, - что принцы и рыцари все равно 
погибнут, ведь самым настойчивым из них король дает три задания, и ни 
одно из них невыполнимо. А ваша смерть будет совершенно 
безболезненной. Гарантирую, вы ничего не почувствуете. Надеюсь, у вас 
не останется ко мне дурных чувств. Я вас опрысну, и все. Чистая, 
здоровая смерть, и никакого беспорядка.
    - Что значит - "опрыснешь"? Что это ты задумал? - возмутилась 
Серая Шкурка. - Ни я, ни Питер не дадим себя опрыскивать. И вообще - 
будь поосторожнее, если решил угрожать нам... Кое-кто из моих знакомых 
за это уже получил по носу.
    - Нет, вы только ее послушайте! - презрительно воскликнул Буньип. 
- Одна струя из моей ноздри, и ты отлетишь по этой дороге на сотню 
ярдов. О, женщина! Я в своем деле - чемпион мира, а ты говоришь, что я 
получу щелчок по носу. Ха-ха-ха-ха-а! - Буньип запрокинул голову и 
затрясся от смеха.
    - Я думала, ты выдыхаешь огонь, - сказала Серая Шкурка смущенно. - 
Как же ты можешь сторожить Прекрасную Принцессу, если ты не 
огнедышащий?
    - Что ж, сделаем так, - сказал Буньип. - Садитесь сюда и 
отдыхайте. Я вас не трону, пока не позавтракаю, так что можете 
расслабиться и чувствовать себя как дома. Я же пока обегу вокруг замка 
и прогоню людей и зверей, если кто-нибудь подошел слишком близко. Я 
вернусь быстро - минут через двадцать, а потом мы вместе позавтракаем. 
У меня приготовлены печеные лягушки - пальчики оближете. - Он 
облизнулся. - Когда вернусь, я расскажу вам о себе, а затем вы сможете 
бежать. Терпеть не могу убивать людей, которые не пытаются убежать. Я 
даю каждому фору в пятьдесят ярдов, прежде чем пускаю воду. Нельзя 
требовать от меня большего.
    Буньип поднялся и принялся подпрыгивать, чтобы разогреть мускулы. 
Выглядел он действительно очень необычно. Длинная шея поднимала его 
голову на высоту двадцать футов над землей. Он мог вертеть ею по 
сторонам, и ему не составляло труда даже так развернуть ее, чтобы 
посмотреть назад. Тело его было громоздким и тяжелым. Свой 
кенгуруподобный хвост при ходьбе он держал на весу, а во время бега 
покачивался и переваливался с ноги на ногу, как перегруженный корабль 
во время сильного шторма.
    Сейчас он пустился галопом по той широкой дороге, что огибала 
замок. Когда он поднимался в воздух, его шея подавалась вперед, а 
когда лапы снова касались земли, шея возвращалась в прежнее положение. 
Он бежал подобно жирафу, хотя ноги у него были короткими, как у 
медведя. На бегу Буньип выстреливал из ноздрей водяные струи и громко 
ревел. Эти струи сверкающими дугами вылетали из его ноздрей и 
подбрасывали в воздух случайно забредших сюда коров. При желании 
Буньип был способен посылать струи из любой ноздри, причем с такой 
силой, что коров по несколько раз переворачивало, прежде чем они, 
чихая и отплевываясь, могли подняться на дрожащие ноги.
    Если бы он продолжал выпускать свои струи еще некоторое время, 
животные бы просто утонули, но он удовлетворялся тем, что только валил 
их с ног. Когда же против него выходили рыцари и принцы, он делал вдох 
поглубже и посылал более длинную струю, которая ударяла во всадника, и 
тот с грохотом и лязганьем падал на землю. После этого можно было 
разглядеть только копья, мечи и дергающиеся ноги - все это торчало из 
воды, которая быстро спадала и растекалась. Мало кто из рыцарей 
выдерживал такие водяные удары. Они тонули вместе со всем своим 
снаряжением.
    Но в этот пригожий день ни принцев, ни рыцарей Буньипу не 
встретилось, и вскоре он тяжело подбежал к Питеру и Серой Шкурке. На 
его лице играла самодовольная ухмылка, а где-то сзади отфыркивались 
несколько коров.
    - Ну, как я? - крикнул он, едва остановившись. - Видели, как я 
подбросил вон ту корову одной несильной струей из правой ноздри?
    - По-моему, это жестоко, - сказал Питер. - Та корова не 
представляла ни для кого никакой опасности.
    - Опасность представляет всякий, кто приближается к замку, - 
заявил Буньип. - Если бы вы знали, сколько рыцарей и принцев мечтают 
жениться на Прекрасной Принцессе, вы бы ужаснулись. Сегодня еще 
спокойно. Лишь вы двое и подвернулись. А то этих спасителей набегает 
со всего света! Прекрасные Принцессы ныне - редкость. Наши могильщики 
копают для убитых могилы все дни напролет, кроме субботнего вечера и 
воскресенья, когда они ходят в церковь.
    Буньип уселся под деревом и развернул свой завтрак.
    - Нет ли у тебя перца и соли? - спросил он Серую Шкурку. - 
Королевская кухарка так невнимательна. А печеные лягушки без соли 
просто в горло не лезут.
    Серая Шкурка достала из сумки солонку и перечницу и передала 
Буньипу.
    - Удобная у тебя сумка, - похвалил он и продолжал: - Нет смысла 
предлагать вам разделить со мной трапезу, потому что я убью вас 
примерно через час. Если же у вас есть свой завтрак, то оставьте его 
мне: я съем его с чаем в три часа. А теперь я буду завтракать и 
рассказывать вам о своей жизни.

        Глава 11
        РАССКАЗ БУНЬИПА

    - Вы удивляетесь, почему я не выдыхаю дым и огонь, - начал Буньип, 
проглотив порцию лягушачьих лапок. - Потому что это не нравится, вот и 
все.
    Понимаете, дело вот в чем. Я родился в болоте - там рождаются все 
буньипы. Мои родители были обычными буньипами без каких-либо особых 
талантов. Папа любил реветь по ночам, пугая людей до смерти. В 
остальное время он ловил рыбу и охотился на животных, спускавшихся к 
водопою. Мы жили хорошо. Но папу волновало наше общественное 
положение. Другие буньипы жили в просторных пещерах, пол которых от 
стены до стены был покрыт болотной травой. А мы жили в норе, где на 
полу всегда стояла слякоть. Папа и мама считали, что единственный 
способ поднять наше общественное положение - эхо добиться для меня 
должности охранника Прекрасной Принцессы. Это была ответственная 
работа, и отец смог бы этим хвастаться.
    И вот он послал меня в Школу Драконов, где драконят обучали 
охранять Прекрасных Принцесс. Директором школы был старый рыцарь по 
имени Святой Георгий. Вы случайно не слыхали о таком? Когда-то он 
здорово сразился с драконом и якобы прикончил его. Я не верил ни 
одному его слову начиная с того момента, как он в первый же раз бросил 
нам "Привет".
    - Я о нем слышал, - сказал Питер. - Он действительно убил дракона. 
Есть такой рассказ - "Святой Георгий и Дракон". Это был замечательный 
человек.
    - Да, именно так он себя и называет, - презрительно бросил Буньип.
    Питер решил с ним не спорить. Не потому, что боялся, - ведь в руке 
он держал Волшебный Лист, - просто не любил спорить.
    - Нашими учителями были рыцари, которые прежде сражались с 
драконами, - продолжал Буньип. - Они должны были обучить нас бороться 
с рыцарями так, чтоб никто из них спасти Прекрасную Принцессу не мог.
    Это была не школа, а чистый ужас. Нас поднимали ни свет ни заря, 
выстраивали колонной и вели на завтрак. Подавали нам расплавленную 
лаву и горящие угли. Считалось, что это самая подходящая еда для 
драконят, которым предстоит научиться пускать на врагов пламя с дымом. 
Но у меня от такой еды было только несварение желудка.
    После завтрака нас выстраивали на площадке в шеренгу, напротив 
становились рыцари на конях. Они были закованы в лучшие доспехи, 
которые так сияли на солнце, что слепили нам глаза. Их кони радостно 
ржали, задирали морды и били копытами. Рыцари наставляли свои копья и 
мчались на нас с криками: "Ату их, ату! Защищайтесь, сэр! Иду на вас, 
мошенник! Руки вверх!" - и тому подобными. Конечно, мы старались. Мы 
выдыхали пламя и страшно много дыма. Но мы ведь были еще дети и могли 
дохнуть пламенем только футов на десять. После хорошего завтрака кое-
кому из учеников удавалось достичь и двенадцати, но у меня дальше 
шести никогда не выходило. Так что оставалось просто увертываться от 
коней. О, боже! Как мы вертелись! Мы путались у них под ногами, 
хлестали их хвостами и рычали, скалили зубы, кусались и даже делали 
сальто, лишь бы увернуться от копья. Из-за моего меха мне доставалось 
больше всех. Драконята целиком покрыты чешуйками, и копья с них просто 
соскальзывали. Иногда им удавалось даже перекусить копье пополам. Мне 
было хуже. Острие копья вырывало у меня шерсть клочьями и продирало 
между ребрами целые борозды. Я ревел как бык, опустивший голову под 
воду - такой боевой клич буньипов - но на рыцарей он не производил ни 
малейшего впечатления.
    Если мы ломали им копья, они лупили нас мечом плашмя. От нашего 
огня их спасали доспехи. А чтобы в них не изжариться, они с нами 
подолгу не возились и в конце концов прогоняли с площадки.
    Рыцари любили греть походный котелок и, поливая чай из жестяных 
кружек, хвастаться друг перед дружкой тем, по скольку драконов они 
уничтожили, прежде чем стали учителями. Они сидели на траве и 
потешались, а нам приходилось выслушивать эти россказни из-за забора.
    Меня они просто бесили. И однажды ночью, наслушавшись стонов и 
тяжелых вздохов драконят в нашей спальне, я кое-что придумал. На 
следующее утро, когда мы собрались на площадке, я бросился к озеру 
перед нашей школой и втянул в себя столько воды, что уровень озера 
упал на два фута. Знаете, как иногда булькает в животе? Вот так 
булькало и у меня. Это шипели раскаленные угли, когда вода гасила во 
мне огонь. Я ничего не сказал, а вернулся и встал в строй вместе с 
другими драконятами. Когда же рыцари на нас поскакали, я послал в них 
такую струю воды из каждой ноздри, что всех их повалило в одну кучу. 
Дело в том, что я сумел пустить одну длинную струю. Они решили, что 
прорвалась плотина. Они вопили, бранились, пытались снова взобраться 
на коней, но вода сбивала их с ног и они снова падали, растеряв в 
конце концов все свое оружие.
    Драконята ликовали. Все хотели научиться пускать воду вместо огня, 
но я не мог их этому научить. У драконов желудки изнутри выложены 
железом, чтобы там мог гореть огонь. От воды оно бы стало ржаветь, и у 
них образовалась бы язва. А согласитесь, что дракон с язвой от воды - 
жалкое зрелище...
    Как бы то ни было, один из учителей кинулся в школу и привел 
Святого Георгия. Как же! Ученики посмели бросать вызов установленному 
порядку. Это же бунт! Если бы драконы переделали свои желудки и стали 
извергать воду вместо дыма и огня, Святой Георгий остался бы без 
работы. Поэтому он надел новый шлем с таким огромным забралом, что 
когда оно было опущено, он уже ничего ни видел, ни слышал... Он надел 
новые латы со специальным приспособлением для закрепления конца копья 
и сел на боевого коня, на котором было навешано столько доспехов, что 
он скакал с превеликим трудом.
    Святой Георгий опустил забрало и крикнул нам:
    - Отправляйтесь по своим комнатам и выдыхайте дым два часа.
    Никто не шелохнулся.
    У меня в желудке еще оставалась тысяча галлонов воды, так что я 
молчал и ждал, когда он нападет, но он не спешил. Он повторил свое 
требование: всем разойтись по комнатам и выдыхать дым. Но драконята 
сгрудились вокруг меня. Они были напуганы, и один из них признался: 
"Во мне не осталось ни одной искорки. Я не смогу дохнуть огнем. Он же 
убьет нас всех, чтоб другим было неповадно. Ты точно сможешь его 
остановить?"
    - Увидишь, - ответил я.
    И вот Святой Георгий двинулся на меня. Его могучий конь медленно 
разгонялся и, подгоняемый шпорами, громыхнул галопом. Святой Георгий 
выпрямил в стременах ноги и подался вперед в седле. Привстав на 
стременах и закрепив копье, он медленно опускал его до тех пор, пока 
оно не оказалось направленным прямо в меня. Все рыцари его 
подбадривали. "Берегись! Это сам Святой Георгий, победитель Дракона!"
    Когда между нами оставалось пятьдесят ярдов, я попробовал дальний 
удар и пустил ему в наколенники две струи из каждой ноздри. Потом я 
немного поднял голову и пустил несколько струй ему в набедренники, 
наручники, грудной панцирь, оплечье и забрало. Он покачнулся. В конце 
концов я выпустил ему в нагрудник все полтораста галлонов. О, видели 
бы вы это зрелище! Он отлетел к лошадиному хвосту, раскорячив ноги и 
молотя по воде руками, затем с треском и грохотом бухнулся наземь. 
Гром раздался такой, будто я перевернул повозку с жестяными котелками. 
Конь же его продолжал приближаться. Я ударил из правой ноздри прямой 
наводкой ему в грудь, он встал на дыбы и понесся прочь.
    {whisp08.gif}
    Видели бы вы Святого Георгия! Он поднимался по частям, словно 
складной метр, и грозил мне железным кулаком. "Ты исключен!" - крикнул 
он. Его так и подмывало броситься на меня даже без коня, но он был 
трусоват. "Пошел вон из моих владений!" - продолжал он вопить.
    Так я и сделал и отправился к тому месту, где забор шел вплотную к 
дороге. Но я не мог не думать о папе с мамой. Родители, которые 
надеются улучшить свое положение с помощью успехов детей, очень 
сердятся, когда с детьми происходит нечто подобное. Ты просто не 
имеешь права их разочаровывать, вот в чем беда. Я никогда об этом не 
забывал. Понятно, что мне было невесело, и я загрустил.
    Буньип и в самом деле загрустил даже при воспоминании об этом.
    - Теперь все позади, - сказал Питер. - Сейчас мы с Серой Шкуркой 
вскипятим котелок. Чай пить будешь?
    - Буду что? - поразился Буньип.

        Глава 12
        СЕРАЯ ШКУРКА СРАЖАЕТСЯ С БУНЬИПОМ

    Они вскипятили котелок, и Буньип продолжил рассказ.
    "- Случилось так, что в тот день, когда я, изгнанный из школы, шел 
по дороге, Король выехал на охоту. Люди из его свиты окружили меня. 
Это были королевские придворные, они постоянно раскланивались и 
расшаркивались. Одеты они были богато - в длинные бархатные плащи и 
узкие в обтяжку брюки. Ехали все на прекрасных конях, но лучше всех 
был вороной жеребец короля, - он ни секунды не мог стоять спокойно. 
Когда король натягивал поводья, тот начинал гарцевать кругами.
    - Эй, малыш! - окликнул меня король, когда конь повернул головой в 
мою сторону. - Кто ты такой?
    Но конь уже встал ко мне боком, и мне пришлось ждать, прежде чем 
ответить.
    - Буньип, Ваше Величество.
    - Не слыхал о таких.
    Конь продолжал гарцевать и поворачивал короля то туда, то сюда, 
так что я снова ловил момент, когда он окажется ко мне лицом.
    - Мы живем в болоте возле дворца.
    - О боже! - воскликнул король. - Придется мне это место 
продезинфицировать.
    Тут жеребец снова отвернул короля в сторону. Ему это явно надоело, 
и он крикнул одному из придворных:
    - Остановите эту чертову лошадь!
    Человек, к которому он обращался, спрыгнул со своего коня и 
схватил под уздцы королевского жеребца. Было видно, что придворный 
здорово напуган, и не без оснований - король нагнулся и хлестнул его 
кнутом.
    - Как ты смел подавать королю такого коня? - набросился он на 
несчастного.
    - Но вы же сами просили гарцующего, - запинаясь, пробормотал 
придворный.
    - Гарцующего - да. Но я не просил, чтобы меня непрестанно кружили 
в вальсе.
    Король соскочил на землю и поправил корону, которая во время всей 
этой кутерьмы успела сползти набок.
    - Итак, - обратился он ко мне, - ты говорить, что ты Буньип. Я 
видел, как ты расправился с рыцарями. Я как раз находился у забора и 
видел твой оригинальный метод... Несомненно, ты мастерски выбросил всю 
эту банду учителей со Святым Георгием во главе с их коней.
    - Сбросил с коней, - поправил я его. Король изумленно уставился на 
меня.
    - Да ты интеллектуал, - презрительно бросил он. - Но ничего, эту 
дурь мы из тебя выбьем.
    Король мне не понравился. Мой отец говорил, что тот - чистейший 
деляга. Я не мог этого понять. Деньги для него значили все. И все же 
именно он был отцом Прекрасной Принцессы, которую держал в заточении в 
башне, и это знали все.
    - Послушай, малыш, - обратился ко мне король. - Хочешь получить 
работу? За сколько ты согласился бы потрудиться на меня?
    - А что я должен делать? - спросил я.
    - У меня есть дочь, Прекрасная Принцесса, - отвечал король. - Ее 
надо надежно сторожить от всяких рыцарей и принцев из других земель, 
которые придут просить ее руки. Она еще слишком молода для этого, и, 
естественно, я желаю, чтобы все ее поклонники тут же уничтожались, 
причем умело и эффективно. Если ты будешь набирать в себя побольше 
воды, я думаю, ты сможешь ликвидировать их вполне успешно. Вода, 
разумеется, бесплатно. Итак, - продолжал он, - какова наименьшая цена?
    - Десять долларов в неделю плюс содержание.
    Я знал, что всякий настоящий деляга обязательно сбавляет 
запрашиваемую сумму, и поэтому запросил больше, чем рассчитывал 
получить.
    - Хм, - задумался король. - Я сделаю вот что. Я дам тебе пять 
долларов в неделю плюс содержание.
    - Подходит, - сказал я. - Согласен.
    Я ответил мгновенно, потому что увидел, как он тут же пожалел, что 
не предложил мне четыре. Он был хорошим дельцом и без нужды деньгами 
не сорил.
    - Это очень солидная плата, - внушал он мне. - Надеюсь, ты не 
обжора.
    - Нет, я не обжора, - заверил я его.
    Когда доходит до дела, я тоже умею поторговаться. Как бы то ни 
было, я получил работу и несу здесь службу уже многие годы".
    - Ну как, интересная история? - добавил он.
    - О да, - согласилась Серая Шкурка. - А хотел бы ты услышать 
историю моей жизни?
    - Нет, - ответил Буньип, поднимаясь на лапы. - Я слишком занят. 
После того как я вас убью, мне надо будет организовать могильщиков, 
чтобы вас похоронили. Так что давайте не будем терять времени.
    Он осмотрел Питера и Серую Шкурку с головы до ног, потом перевел 
взгляд на Мунлайт, которая паслась неподалеку.
    - Лошадку я сохраню для Прекрасной Принцессы, - сказал Буньип. - 
Она сможет ездить на ней во время наших ежедневных прогулок вокруг 
замка. Да! Как вы предпочитаете быть убитыми - вместе или порознь? 
Гораздо более зрелищно, если вы вдвоем броситесь на меня с расстояния 
в сто ярдов. Тебе я дам копье, - обратился он к Питеру. - Только будь 
с ним осторожен. Оно принадлежало одному известному рыцарю, который 
умер в высшей степени благородно. Он сумел выбраться из доспехов, 
когда еще был жив, и потонул в одном белье. Эх, бедняга! Я потом не 
мог ужинать.
    - Вот что я тебе скажу, - ответила Серая Шкурка. - Разберись 
сначала со мной. Я нападу на тебя одна, такая, какая я есть. Мне не 
нужно ни копья, ни лошади. Я попробую только смеха ради.
    Питер остолбенел.
    - Ты же утонешь! - воскликнул он. - Не делай этого. - Он подбежал 
и горячо зашептал Серой Шкурке на ухо:
    - Я сейчас дам ему Волшебный Лист. Подожди, пока я не заведу 
разговор о подарках. Нам пока ничто не грозит.
    - Хочется его проучить, - шепотом ответила Серая Шкурка. - Он 
зазнался. Я видела, как он сбивал тех коров. Я запросто увернусь от 
его струй, вот увидишь.
    Одним прыжком она очутилась рядом с Буньипом.
    - Где мне встать?
    - Встань вон там на дороге, около большого дерева, и оттуда 
нападай на меня.
    - Согласна, - ответила Серая Шкурка.
    Она отошла туда, где росло большое дерево, и на несколько 
мгновений задержалась, плотно закрывая сумку, чтобы в нее не попала 
вода. Потом крикнула Буньипу, который в это время делал глубокие 
вдохи:
    - Я готова! Когда захочешь начать, крикни "Давай!"
    - Хорошо! - ответил Буньип. И, подождав, крикнул: - Давай!
    Сразу он выбросил из правой ноздри струю воды почти в фут 
толщиной. Он целился кенгуру в грудь, но она так быстро отпрянула в 
сторону, что он промахнулся. Тогда Буньип повернул голову, чтобы 
изменить направление струи, но Серая Шкурка вновь отпрыгнула. Буньип 
мотал головой из стороны в сторону, но Серая Шкурка все время 
увертывалась. С каждым прыжком она приближалась к извергающему воду 
Буньипу. Она отпрыгивала и проскальзывала под струями с такой 
скоростью, что Буньип подключил и вторую ноздрю, послав в кенгуру две 
струи сразу. Она впрыгнула между ними, потом отпрыгнула назад, 
перепрыгнула сразу через обе, проскользнула снизу и снова подскочила 
вверх между струями. Буньипу никак не удавалось попасть в нее. Он свел 
оба потока вместе, так что образовалась могучая струя толщиной в два 
фута, от которой Серой Шкурке пришлось увертываться чудовищными 
прыжками.
    Буньип выдохся. Он остановился, чтобы набрать побольше воздуха в 
легкие - без этого он не мог выдувать воду. Серая Шкурка 
воспользовалась моментом и бросилась к нему напрямик. Сделав последний 
мощный прыжок, она опустилась прямо на спину Буньипа.
    Эффект был потрясающий. Буньип закачался, закинул голову и издал 
такой рев, что деревья задрожали и в страхе сбросили листья. По склону 
холма прокатился сорвавшийся камень, а страшный порыв ветра всколыхнул 
воды рва.
    Огромные зубы Буньипа клацали, словно мечи, когда он рвал мнимых 
нападавших справа и слева. Тут Серая Шкурка с силой вонзила острые 
когти задних лап в бока Буньипа, и тот взревел уже от боли.
    Буньип запрыгал по дороге, и хотя он ничуть не походил на 
норовистую лошадь, но, как и она, изо всех сил старался сбросить с 
себя кенгуру, неуклюже подпрыгивая и виляя туловищем из стороны в 
сторону. Но кенгуру крепко сжимала лапами его бока, представляя себя 
на родео верхом на быке, и даже, помахав передней ланкой, закричала 
"Но! Но!", настолько унизив этим Буньипа, что он чуть не вывернулся 
наизнанку в попытках сбросить ее. Он развернул свою голову назад и 
вниз и набрал побольше воздуху с явным намерением смыть кенгуру со 
спины. Но Серая Шкурка скользнула вперед, уселась на его шее, где она 
соединялась с плечами, и, схватив его голову, повернула ее так, что 
выброшенная Буньипом струя попала в его собственный хвост. Омываемый 
водой, хвост дрожал, как тростинка в несущемся потоке.
    Буньип споткнулся. Теперь на него нападали и спереди, и сзади. В 
отчаянии он громко взревел, сделал еще один глубокий вздох, но Серая 
Шкурка набросила ему на шею петлю и перекрыла воду. Буньип задыхался. 
Он опустился на землю и успел только прохрипеть: "Сдаюсь".
    Серая Шкурка спрыгнула с его спины, и они с Питером стали ждать, 
когда он придет в себя.
    - Пожалуй, ты обошлась с ним слишком грубо, - сказал Питер, с 
беспокойством глядя на тяжело дышащего Буньипа.
    - Это он хотел обойтись с нами грубо, - ответила Серая Шкурка. - Я 
же только преподала ему урок. Кенгуру дерутся лучше, чем многие 
воображают. В конце концов он ведь хотел нас убить.
    - Это совершенно неважно. Ты нанесла удар по его гордости, ты его 
унизила, а это ужасно. Я собирался дать ему Волшебный Лист, и он стал 
бы добрым. А погляди на него теперь.
    Буньип действительно выглядел удрученным, но он быстро пришел в 
себя и начал упрекать Серую Шкурку за неспортивное поведение.
    - Ты не должна была использовать всякие там штучки с 
увертываниями, когда пошла на меня, - недовольно выговаривал ей 
Буньип. - Все рыцари настолько хорошо воспитаны, что они несутся прямо 
на тебя, издавая боевой клич. А вспрыгивать мне на спину! Им бы и в 
голову не пришло такое ребячество. Никто так не поступает, - ни 
рыцари, ни принцы.
    - Но я кенгуру.
    - Ты права, - согласился Буньип. - Сражаться с кенгуру меня не 
учили. Если бы я использовал вместо воды огонь, я бы, пожалуй, с тобой 
справился.
    Серая Шкурка хотела возразить, но Питер не дал ей открыть рот.
    - Возьми у меня подарок, - предложил он Буньипу. - Мне бы хотелось 
подарить тебе кое-что, прежде чем ты нас убьешь.
    Буньип колебался.
    - Дело в том, что когда я был маленьким, моя мама часто говорила 
мне: "Никогда не бери подарков от незнакомых людей на велосипедах". 
Да, именно так. И я не уверен, стоит ли мне брать подарок.
    - Но ведь у меня нет велосипеда, - возразил Питер.
    - Да, действительно. Хорошо, я возьму. Но это не значит, что я вас 
пощажу. Мне ведь платят пять долларов в неделю, чтобы я убивал всех 
принцев и рыцарей. Я не должен допустить, чтобы принцессу спасли. И 
что же это за подарок, который ты хочешь мне дать? Так и быть, я убью 
вас после того, как поблагодарю за него.
    - Это всего только лист, но Лист Волшебный, - сказал Питер, - 
Возьми, и ты почувствуешь себя хорошим и добрым.
    И Питер протянул Лист Буньипу.
    Тот взял ого и стал с удивлением рассматривать. Подарок показался 
ему пустяковым. Но потом он задумался. Он уже не выглядел таким 
свирепым, злобные глазки смягчились. Было видно, что внутри его 
нелепой головы происходили какие-то изменения, и он озирался, как 
будто в окружающем мире что-то изменилось.
    - Ты счастлив? - спросил Питер.
    - Таким счастливым я еще никогда не был. - Он с удивлением 
посмотрел на Питера. - Ты самый замечательный принц из всех, кого я 
когда-либо видел. И с чего это я хотел тебя убить? Да я бы просто не 
смог! Мне даже жаль, что я пытался убить Серую Шкурку. Я ее очень 
люблю.
    - Правда? - Серая Шкурка достала из сумки зеркальце и погляделась 
в него. Она подняла лапу и отбросила назад мех, закрывавший ей ухо. - 
Очень может быть. По-моему, я всегда была хорошенькой.
    - Ты знаешь, каким делает тебя этот лист? - спросил Питер, не 
обращая внимания на Серую Шкурку.
    - Нет, не знаю.
    - Теперь люди будут тебя любить, а не бояться.
    - Меня никто никогда не любил, - задумчиво произнес Буньип.
    - А теперь будут.
    - Я бы тоже хотел сделать вам подарок, но у меня ничего нет, - 
сказал Буньип. - Может быть, я могу вам чем-нибудь помочь?
    - Да, - ответил Питер. - Я хочу узнать, почему Прекрасная 
Принцесса сидит в заточении. Я хочу с ней поговорить. Помоги нам это 
устроить.
    На лице Буньипа мгновенно проступил страх, но в руке у него был 
Волшебный Лист, и его сила постепенно изгнала страх.
    - Прекрасная Принцесса будет рада с тобой познакомиться, я знаю 
точно. Она очень несчастна. Я раньше никому об этом не говорил. Но 
проникнуть в замок будет крайне трудно. Говорить с ней можно только в 
ее комнате. Это единственный способ. Я часто сопровождаю ее в верховых 
прогулках вокруг замка, но с нами всегда едут солдаты. Дай-ка я 
подумаю, как туда проникнуть.
    Буньип почесал затылок, потом сказал:
    - План готов. Каждое утро в семь часов ворота открываются, выходят 
два солдата и трубят в горн, чтобы все вставали. Затем они опускают 
мост и остаются охранять. После завтрака я подхожу к дверям и прошу 
позволения видеть короля, - я ежедневно отчитываюсь о числе рыцарей и 
принцев, пытавшихся спасти Прекрасную Принцессу. Солдаты идут к королю 
и докладывают ему о моем приходе. Потом они возвращаются и говорят, 
что король готов меня принять. Иногда, конечно, он бывает занят, и мне 
приходится уходить. Но такое случается редко.
    Солдаты отсутствуют минут десять. Этого времени вам хватит, чтобы 
перебежать мост и пройти ворота. Внутри замка вам придется самим 
искать дорогу по всем переходам и лестницам, ведущим к комнате 
Прекрасной Принцессы.
    Ее комната на самом верху башни - вон там, с голубой занавеской, - 
Буньип хвостом показал на то окно, которое Питер отметил, когда они 
только подходили к замку.
    - Прежде чем вы доберетесь до ее комнаты. вам придется миновать 
множество переходов. Старайтесь, чтобы вас не заметили. Если король 
или королева поймают вас в замке, вам отрубят голову. При любом звуке 
шагов - прячьтесь. Выбираться вам придется самим. Не забудьте, что 
после обеда ворота закрывают и поднимают мост.
    - Как-нибудь да выберемся, - сказал Питер. - А теперь мы вернемся 
в буш и поищем место для стоянки. В семь утра мы будем здесь.
    - Я буду вас ждать, - пообещал Буньип.

        Глава 13
        В ЗАМКЕ

    На следующее утро Питер и Серая Шкурка уже сидели под большим 
деревом, когда появился Буньип. Светило солнце, и растущая около рва 
ива, склонившись, опустила в воду свои длинные пальцы. Из буша 
раздавался смех кукабарры, а в долине за дорогой заливалась какая-то 
птица.
    Буньип сел на землю рядом с Питером.
    - Я всю ночь не спал, - пожаловался он. - Все размышлял. В замке 
полно солдат. Это вообще очень оживленное место - там и кухарки, и 
камеристки, фрейлины, и придворные. Только настоящий храбрец решится 
пробивать себе путь туда, где заточена Прекрасная Принцесса. Может 
быть, лучше это отложить на денек-другой? Меня терзает мысль, что кто-
нибудь из вас может погибнуть.
    - Кривой Мик наказал мне никогда ничего не откладывать, - ответил 
Питер. - Через несколько дней будет то же самое. Я хочу войти в замок 
сегодня. Не волнуйся. Я взял с собой Громобой, - и Питер показал 
Буньипу свернутый кольцом кнут, который лежал рядом с ним.
    - А я быстро бегаю! - ответила Серая Шкурка. - Этим мастиффам ни 
за что меня не поймать.
    Тут раздался скрежет ржавого блока, и гигантский подъемный мост 
медленно пришел в движение.
    - Сейчас откроют ворота, - прошептал Буньип. - Быстро прячьтесь за 
дерево!
    Огромные цепи, на которых удерживался мост, задрожали, приняв его 
вес, и тяжелая махина стала потихоньку опускаться, пока не коснулась 
берега рва неподалеку от большого дерева. Затем раздался звук 
отодвигаемых засовов, и ворота распахнулись.
    Из ворот вышли два солдата с горнами. Они были в коротких кожаных 
куртках, сбоку висели мечи. Ступив на мост, они поднесли к губам горны 
и затрубили с такой силой, что все кукабарры разом смолкли, а 
плававшие около ивы малые поганки нырнули под воду, оставив на 
поверхности только расходящиеся круги.
    - Я пойду сейчас, - быстро проговорил Буньип. - А вы пойдете, 
когда я дам вам знак.
    Он тяжело заковылял к мосту. Казалось, от каждого его шага дрожит 
земля, - такой он был огромный и массивный.
    - Интересно, выдержит его мост? - полюбопытствовала Серая Шкурка.
    Мост действительно содрогнулся, когда на него ступил Буньип, но 
сделан он был добротно, и к тому же Буньип уже сотни раз по нему 
ходил. Когда он подошел ко входу в замок, солдаты велели ему ждать и 
пошли известить короля. Как только они исчезли, Буньип махнул хвостом. 
Питер и Серая Шкурка перебежали мост и встали рядом. В руке Питер 
крепко сжимал Громобой.
    - А теперь в ворота - живо! - скомандовал Буньип. - Потом по 
коридору налево. По нему ходят реже, чем по другим, и оттуда вы 
найдете путь к комнате принцессы.
    Питер и Серая Шкурка бросились в ворота и повернули по галерее 
налево. Их никто не заметил, хотя у дальней стены большого помещения, 
начинавшегося сразу за воротами, Питер увидел группу солдат. Солдаты 
через маленькую дверь выходили во внутренний двор. Они смеялись и 
дурачились и не заметили двух посторонних.
    Из галереи они вошли в коридор, по которому сновало множество 
людей. Одни несли корзины с овощами и фруктами, другие - клетки с 
овцами для дворцовой кухни. Очевидно, это был день, когда местные 
фермеры, садоводы и мясники приносили свои продукты в замок на 
продажу.
    - Будь у меня корзина фруктов, - сказал Питер, - я бы мог сойти за 
садовода.
    - Только не в этом наряде, - поправила его Серая Шкурка. - Но мы 
можем его поменять. - Она вытащила из сумки какую-то ношеную-
переношеную робу из дерюги, всю латаную и перелатаную. Накинула ее на 
Питера, и он оказался укрытым ею полностью. Снова засунув лапу в 
сумку, Серая Шкурка достала старую шапку, которую Питер тут же надел. 
Свою шляпу он отдал кенгуру, которая бросила ее в сумку, где та и 
исчезла.
    - Я возвращу ее тебе позже, - пообещала она. - А теперь нам 
предстоит достать корзину фруктов.
    Она несколько секунд шарила в своей сумке и, вытащив оттуда мешок 
апельсинов, нахмурилась.
    - Я хотела разных фруктов, а получились одни апельсины.
    - И хорошо, - возразил Питер. - Я выращиваю одни апельсины. Идем 
же! Давай к кому-нибудь пристроимся!
    Они дождались, когда мимо проходил старик, толкавший перед собой 
тележку с картофелем. Питер ступил в коридор и тоже навалился на 
тележку. За ним с мешком апельсинов шла Серая Шкурка.
    - Спасибо, друг, - поблагодарил старик. - В наши дни мало кто 
помогает друг другу.
    Они уже очутились во внутреннем дворике и увидели марширующих 
солдат.
    - Теперь недалеко, - сказал старик. Сквозь широкую дверь он провел 
их в хозяйственное помещение, где был устроен базар. Это был огромный 
зал с рядами грубых лавок и массивных столов, на которые продавцы 
выкладывали свои товары.
    На один из таких столов Серая Шкурка положила мешок с апельсинами. 
Питер, попрощавшись со стариком, подошел и стал рядом.
    - Что дальше? - спросил он.
    Слова Питера услышал один придворный, который делал закупки для 
дворца, - толстяк со снисходительно-самонадеянным видом, - и подошел к 
нему.
    - Я тебе отвечу, что делать, - сказал он Питеру. - Ты можешь 
забрать своего кенгуру. Суп из кенгуриных хвостов на этой неделе мы не 
готовим. Понятно? А теперь убирайся.
    - А как насчет апельсинов? - запинаясь, спросил Питер. Он очень 
испугался, поняв, что кенгуру здесь посчитали выставленным на продажу.
    - А-а, апельсины, - толстяк взял один. - Это уже кое-что. Именно 
таких мы искали для Прекрасной Принцессы. - Он сжал апельсин. - 
Замечательный. Беру.
    - Можно, мы ей сами отнесем? - Питер решил испробовать такой 
способ добраться до принцессы.
    Придворный опешил, он даже покраснел от гнева.
    - Что такое ты мелешь? - вскричал он, оглядывая народ. - Вот 
человек, который посмел заявить, что хочет взглянуть на Прекрасную 
Принцессу. Наверное, он не тот, за кого себя выдает. Одет в лохмотья, 
а хочет нести апельсины принцессе! Позвать стражу! Сообщить королю! 
Отрубить ему голову!
    Он пришел в состояние неописуемой ярости и молотил руками воздух. 
Окружавшие в остолбенении смотрели на Питера. Они решили, что он сошел 
с ума. Но все-таки еще и побаивались.
    Серая Шкурка тоже испугалась. Она сообразила, что Питер сморозил 
глупость. Солдаты, маршировавшие во дворе, остановились и 
прислушивались к крикам в кухне, а один из них побежал сообщить 
королю.
    - Давай-ка сматываться отсюда, - быстро шепнула она Питеру, до 
которого только теперь дошло, что он натворил. - Прыгай ко мне на 
спину! Скорее! Не трать ни секунды!
    Питер вскочил ей на спину, одной рукой держась за ее мех, другой 
сжимая кнут. Серая Шкурка оттолкнулась и вылетела из комнаты, прежде 
чем ошарашенный придворный успел ее схватить.
    Она проскакала вдоль двора, причем с такой скоростью, какую еще 
никогда не развивала. Только опыт наездника помог Питеру не свалиться. 
Он крепко прижался к кенгуру, когда она резко повернула направо в 
коридор, наклонившись набок, как велосипедист во время виража. 
Промчавшись по коридору, она попала к узкой лестнице и запрыгала по 
ней вверх через четыре ступени.
    На первой лестничной площадке Серая Шкурка остановилась, чтобы 
отдышаться, а Питер тем временем сбросил робу, надетую поверх его 
роскошных одежд. Кенгуру вернула ему шляпу с перьями, и он снова стал 
принцем - и по одежде, и по манерам.
    Снизу до них доносились крики, топот ног и лай собак, рвущихся с 
поводка. Им надо было спешить, они понимали, что уже через несколько 
секунд по этой лестнице прогрохочут солдаты.
    - Туда, где заточена принцесса, должны вести несколько лестниц, - 
сказал Питер. - Солдаты и охрана все кинутся по ним, и это крыло замка 
они прочешут очень быстро. Нам лучше всего забраться как можно выше, 
чтобы оказаться над комнатой принцессы. Ты согласна?
    - По-моему, другого выхода нет, - согласилась Серая Шкурка. - 
Поскакали туда.
    Они преодолевали один пролет за другим, и в конце концов оказались 
в тесном помещении, которое явно использовали в качестве кладовки. 
Кругом громоздились горы оружия: лат, мечей, копий, тут же валялась 
сломанная прялка и множество стульев с отломанными ножками. Кругом 
лежала пыль, а на потолке висела грязная паутина.
    - Сюда уже давно никто не входил, - сказала Серая Шкурка. - 
Посмотри, сколько на полу пыли. Наши следы так и отпечатались.
    Питер подошел к окну. Выглянув, он увидел внизу знакомый ров, 
большое дерево на противоположной его стороне, спящего под кустом 
Буньипа и Мунлайт, пасущуюся на поляне неподалеку. Он даже узнал куст, 
в котором спрятал седло и уздечку.
    - Подойди-ка сюда. По-моему, окно Прекрасной Принцессы прямо под 
нами, на два этажа ниже.
    Серая Шкурка заглянула вниз, и прямо под собой увидела 
развевающуюся по ветру голубую занавеску.
    - Ну и везет же нам! - воскликнула она. - Это точно ее комната. 
Надо только придумать, как туда спуститься. Здесь мы еще в опасности. 
Скоро солдаты доберутся и до этих комнат, и нас схватят.
    Она засунула лапу в сумку, вытащила моток веревки и привязала ее 
конец к толстой балке, которая подпирала стену.
    - Зачем это? - спросил Питер.
    - Я сделаю в веревке петлю, ты в нее сядешь, я опущу тебя до окна 
принцессы, и ты в него влезешь. Мы знаем, что она добрая девушка и 
обязательно поможет тебе. Может быть, она спасет и меня.
    - Но мы будем вместе.
    - Нет, я останусь здесь.
    - Это невозможно. Я тебя не брошу. Разве ты не понимаешь, что 
скоро сюда вломятся солдаты? Мы с тобой друзья. Если тебя убьют, я не 
знаю, что сделаю. Мне в жизни не будет ни минуты счастья.
    - Это единственный выход, - настаивала Серая Шкурка. - Тебя я 
спустить могу, но самой мне спускаться нельзя. Ведь когда ты 
заберешься в окно, мне надо будет вытащить назад веревку, чтобы ее не 
заметили. Когда солдаты найдут меня здесь, они решат, что я просто 
животное, которое продавалось на суп. Они меня не тронут. Они ведь не 
подозревают, что кенгуру могут разговаривать, а я буду молчать. Если 
мы останемся здесь вдвоем, то вдвоем и погибнем. А если ты попадешь к 
принцессе, ты сможешь ей все рассказать, и она придумает, как нам 
помочь.
    Питер понимал, что Серая Шкурка права, хотя ему казалось 
преступлением бросать ее одну перед разъяренными солдатами. В конце 
концов он, однако, согласился, взобрался на подоконник и сел в петлю, 
завязанную Серой Шкуркой на конце веревки.
    - Держись крепче! - напутствовала его Серая Шкурка. - Оттолкнись 
от стены, чтобы не поцарапаться об нее, а теперь - с богом! - и она 
столкнула Питера с подоконника.
    Питер начал медленно опускаться. Это было страшно - висеть высоко 
над деревьями и рвом. Когда Питер достаточно опустился, его стало 
раскачивать из стороны в сторону, и он ободрался о каменную стену, но 
тут же уперся в нее рукой и стабилизировал спуск. Посмотрев наверх, он 
увидел голову Серой Шкурки, и это его успокоило.
    Питер видел, как внизу под ним полощется на ветру голубая 
занавеска. Когда он очутился прямо против окна, веревка замерла. 
Схватившись за подоконник, Питер подтянулся, забрался на него, 
освободил веревку, которая тотчас же исчезла наверху, и ступил внутрь.
    Серая Шкурка тем временем спрятала веревку в сумку. Заперев дверь 
комнаты, она укрылась за коробками, сваленными в углу. Хотя она и 
убеждала Питера, что ей ничто не угрожает, на самом же деле она очень 
боялась. Боялась собак, лай которых доносился с первого этажа, потому 
что здесь, в закрытом помещении, она не могла использовать свою 
скорость. Скорчившись за коробками, она представляла себя несущейся по 
бушу, и в этот момент топот солдат на лестнице вернул ее мысли к 
запертой комнате и недавним страхам.
    Солдаты забарабанили в дверь чем-то железным и потребовали, чтобы 
ее немедленно открыли. Не получив ответа, они высадили ее и ворвались 
в комнату, выставив вперед длинные копья. Два огромных мастиффа 
кинулись обнюхивать всю рухлядь, и хотя они изо всех сил пытались 
освободиться от поводков, солдаты держали их крепко. Собаки скребли 
пол здоровенными лапами, задыхались от лая, из пасти у них текла 
слюна... - но поводки сдерживали их.
    - Обыскать комнату! - громко скомандовал капитан, верзила с грубым 
небритым лицом. - Выше комнат нет. Он должен быть здесь. Перебросьте 
все барахло в один угол. Он прячется где-то тут.
    Солдаты быстро добрались до сложенных в углу коробок, отодвинули 
их в сторону и увидели Серую Шкурку. Она не говорила и ничем не 
показывала, что кенгуру могут разговаривать с теми, кто несет в себе 
дух буша. Она казалась обычной кенгуру, простым животным, которых люди 
отстреливают, чтобы приготовить из них консервы для собак, а также 
ради того, что они называют спортивным интересом.
    Солдаты схватили Серую Шкурку, набросили ей на шею веревку и 
выволокли на середину комнаты, где стоял разъяренный капитан.
    - Что это? - проревел он. - И где тот фермер, что ее привел?
    - Похоже, он убежал, - сказал один из солдат. - Здесь его нет.
    - Он привел кенгуру в замок, чтобы продать на суп, - пояснил 
другой солдат. - Они бежали вместе. Фермер, наверно, прячется где-то 
еще. Он, очевидно, бросил кенгуру, и зверь забился сюда.
    - Да, так все и было, точно, - согласился капитан. - Как бы то ни 
было, кенгуру мы отведем к королю и узнаем, что он скажет.
    Солдаты потащили Серую Шкурку вниз по лестнице и привели в 
королевские покои, где на украшенном алмазами троне восседал король. 
Вокруг него толпились рыцари и придворные, наперебой старавшиеся 
сказать королю что-нибудь приятное.
    - Где тот фермер, который привел в замок это животное? - вопросил 
король.
    - Мы не смогли его сыскать, Ваше Величество, - ответил капитан. - 
Наверно, он покинул замок вместе с другими фермерами.
    - Мы должны остерегаться принцев и рыцарей, - молвил король и 
резко спросил: - Но ты не думаешь, что это мог быть переодетый рыцарь, 
а?
    - Нет, что вы! - воскликнул капитан, - то был жалкий бродяга в 
залатанных обносках. Он ни за что не мог быть принцем. По-моему, он 
просто хотел посмотреть на Прекрасную Принцессу. Наверно, он очень 
беден, раз пытался продать эту бедную животину на суп.
    - Это так, - подвел итог король, окидывая Серую Шкурку хищным, 
оценивающим взглядом. - На кенгуру можно что-нибудь заработать? - 
поинтересовался он, выпрямляясь и поправляя корону. - За сколько его 
покупают на суп?
    - Очень дешево, Вате Величество, - ответит капитан. - Можно 
выручить больше, если пустить его на консервы для собак. Да и шкура 
чего-то стоит.
    - Чудесно, - обрадовался король. - Построить во дворе для него 
клетку и держать там, пока я не свяжусь с доверенным человеком из 
фирмы "Собачьи консервы".
    Он сделал запись в блокноте: "Продать кенгуру и на вырученные 
деньги купить еще несколько кенгуру. Предположительная прибыль через 
шесть месяцев - сорок долларов".
    {whisp09.gif}
    Эта цифра ему понравилась. Серая Шкурка, даже под мехом 
похолодевшая при упоминании о консервах для собак, все же поняла, что 
по крайней мере несколько дней ее не тронут.
    Ей было очень грустно. Она жаждала мчаться по бушу вместе с 
Питером и Мунлайт, но ей казалось, что те дни никогда не вернутся. Она 
засунула лапу в сумку и сжала Волшебный Лист. Это ее успокоило.
    - Убрать кенгуру отсюда, - распорядился король, закончив подсчеты. 
- Поставить клетку и бросить туда травы.
    Солдаты потащили Серую Шкурку во двор, где по приказу капитана они 
соорудили маленький загон рядом со стеной замка. Они огородили 
небольшой участок высокой металлической сеткой, через которую без 
хорошего разбега Серой Шкурке было не перепрыгнуть. А разбегаться было 
негде. Ни крыши, ни навеса, чтобы укрываться от дождя или солнца, не 
было, как не было и соломы для постели, так что лежать приходилось 
прямо на булыжнике.
    Один солдат кинул ей поесть немного сена, и все ушли.
    Серая Шкурка качалась на хвосте и размышляла, сможет ли Питер 
спасти Прекрасную Принцессу.
    "Надеюсь, они меня не забудут", - думала она.

        Глава 14
        ПРЕКРАСНАЯ ПРИНЦЕССА

    Спрыгнув с подоконника, Питер оказался в просторной комнате, 
роскошное убранство которой потрясло его. Такого он еще не видал. На 
стенах висели бесценные гобелены и картины. Ковер с персидским 
орнаментом полностью покрывал пол. У стены стоял стол, на котором 
лежали прекрасные камни: опал и яшма, дымчатый кварц и болотный агат, 
янтарь, хризопраз и окаменевшее дерево. По ковру гонялись друг за 
другом два кролика. Под потолком свили гнездо ласточки и сновали туда-
сюда, нося червячков своим птенцам.
    У дальней стены стоял книжный шкаф. Питер разглядел три тома 
"Повелителя Колец", всю серию о Нарнии, "Алису в стране чудес", 
"Остров сокровищ" и несколько книг о лошадях. На шкафу стояла ваза с 
полевыми цветами, а рядом, свернувшись в клубок, спал опоссум. На 
диване - тоже спал - мальтийский терьер, похожий на шерстяной коврик.
    В центре комнаты за столом сидела самая красивая девушка, каких 
Питер когда-либо видел. Ее гладкие длинные волосы опускались почти до 
пояса и обрамляли лицо двумя золотистыми локонами, а голубые глаза 
были настолько глубоки, красивы и излучали столько света, что в 
комнате не оставалось ни одного темного уголка, и вся она освещалась 
ими, как солнцем.
    Серый дрозд, клевавший на подоконнике крошки, пытался воспеть ее 
красоту, но ему явно не хватало звуков. Даже дикие лебеди, голова к 
голове летевшие в вечернем небе, не могли сравниться красотой с 
принцессой. Не могли с ней сравниться и застенчивые орхидеи, 
пускавшиеся танцевать в лунном свете при малейшем дуновении ветерка.
    Один взгляд на принцессу делал человека счастливым.
    Принцесса писала сочинение, и на столе перед ней лежала раскрытая 
книга. Ни один человек ни разу не входил в ее комнату без 
предварительного доклада, и появление Питера, да еще из окна, поразило 
ее до чрезвычайности. Она поднялась и смотрела на него широко 
раскрытыми глазами. На ней было шелковое платье, сверкавшее голубым 
огнем, как крылья вьюрка и одновременно как лунное отражение на воде. 
Украшали платье алмазы. Только принцесса могла носить такой наряд.
    Питер не мог вымолвить ни слова. Он вообще не умел разговаривать с 
девчонками, а с принцессой и подавно. Он просто не знал, что сказать.
    Наконец, принцесса улыбнулась и спросила:
    - Ты - принц, который продирался сквозь леса и горы, чтобы меня 
спасти?
    - Я еще не совсем принц, - отвечал Питер. - Но уже наполовину им 
стал.
    - Одет ты как принц.
    - Да, и мой наряд с каждым днем становится все лучше. Как тебе 
нравится моя шляпа? Со страусиным пером?
    - Шляпа замечательная. Как хорошо, что перья именно страусиные. У 
некоторых принцев на шляпе перья цапли. Камилла, моя служанка, 
объяснила мне, что из-за этих перьев цапель убивают, и я поклялась, 
что никогда не буду иметь дело с тем, кто носит перья цапли. А теперь 
расскажи мне о себе.
    Питер рассказал ей, как он жил с Кривым Миком, и как он отправился 
спасать Прекрасную Принцессу. Рассказал о Серой Шкурке и ее волшебной 
сумке. Он говорил и говорил, а принцесса слушала, подавшись вперед и 
боясь пропустить хоть слово. Рассказ Питера чрезвычайно ее изумил, а 
когда она услышала, как Питеру пришлось ради нее бросить Серую Шкурку, 
она вскочила и воскликнула: "Мы должны ее снасти!" А подумав, 
добавила: "Но я ведь здесь в заточении и выхожу отсюда только под 
охраной. Все равно я что-нибудь придумаю, надо только чуточку 
потерпеть".
    - Можно и потерпеть, лишь бы быть уверенным, что ей ничто не 
грозит, - сказал Питер.
    - Ей ничто не будет грозить, - пообещала принцесса.
    - А почему тебя держат как в тюрьме? - спросил Питер. - Я ведь 
пришел спасти тебя. Я должен знать твое имя и вообще все о тебе.
    - Меня зовут Лована. Тебе нравится? На языке аборигенов оно 
означает "первая дочь".
    - По-моему, чудесное имя.
    - Мне тоже нравится.
    - Расскажи о себе еще.
    - О, мне столько надо тебе рассказать! Как замечательно, когда 
есть кому рассказывать. Сначала я расскажу, как я ощущаю мир.
    И она поведала Питеру о том, что мир полон тайн и красоты, о том, 
что ей хочется окунуться в него, словно сразу за окном начинается 
волшебный бассейн.
    - Я знала, что только в юности можно открыть свою душу внешнему 
миру, - говорила принцесса. - Мне хотелось все подержать в руках, 
прислониться щекой ко всему, что растет. Но в юности надо и учиться. 
Меня заперли в четырех стенах и не давали смотреть, как в лесу растут 
орхидеи. Мне пришлось склоняться в этой комнате над книгой и учиться, 
но страниц я почти не видела - мысленно я танцевала среди деревьев и 
подбрасывала в воздух цветы. И вот король и королева - а они очень 
строгие родители - решили не выпускать меня отсюда, пока я не сдам 
экзамены и не получу матрикул, и каждое утро они читают мне нотации. 
Кривой Мик читает тебе нотации?
    - Нет, - ответил Питер. - Он меня любит. Он ни за что не стал бы 
так делать. Когда я был маленьким, он показал мне всех птиц в лесу, 
всех зверей, все цветы. О жизни в буше я узнал все, что мог. Потом он 
читал мне книги обо всем на свете, а когда я подрос, я стал читать 
сам. Вот так я и выучился.
    - Мой папа все делает иначе, - сказала Лована. - Он любит стоять 
надо мной и говорить: "Я не желаю, чтобы моя дочь была более 
невежественной, чем другие девочки. Все дети в моем замке сдали 
экзамены и получили матрикул. Мне хочется тобой гордиться, а когда 
другие говорят мне, что их дочки уже сдали экзамены, мне от стыда 
некуда глаза девать. Я хотел бы иметь право ответить им, что моя дочь 
тоже получила матрикул. Как ты сможешь существовать в этом мире, если 
не будешь знать больше других? Твоя беда в том, что ты ленива и не 
хочешь заниматься. Мы с твоей матерью не можем на равных общаться с 
другими людьми, зная, что у нас глупая дочь. Как только ты сдашь 
экзамены, я тебя освобождаю. А до тех пор принцам и рыцарям бесполезно 
добиваться твоей руки. Я поставлю перед ними невыполнимые задачи и 
отрублю им головы".
    "Именно так, - говорит и моя мать - королева. - Подумать только, 
дочь сэра Реджинальда Малтраверса уже получила матрикул, а ей всего 
пятнадцать лет! И теперь весьма состоятельный принц предлагает ей руку 
и сердце. А он принадлежит к одной из лучших семей".
    "Это позор! - подводит итог король".
    - Как я рыдала, - продолжала Лована. - Мне никогда не сдать этот 
экзамен, и никогда не суждено выйти отсюда, если ты меня не спасешь.
    - По-моему, ты совсем не глупа, - сказал Питер. - Стоит на тебя 
взглянуть, чтобы это понять. Я тебя заберу отсюда, и, когда ты станешь 
взрослой, мы поженимся. Ты будешь жить у нас с Кривым Миком, помогать 
нам готовить завтрак, кормить птиц и животных. Мы будем убирать дом, 
застилать постели и угощать гостей чаем. Мы купим тебе небольшую 
лошадку, такую, как Мунлайт, и каждый день мы будем ездить по бушу. Я 
научу тебя колоть дрова для печки и жарить на углях котлеты. И еще мы 
будем читать сотни книг.
    - Как чудесно! - воскликнула Лована. - Мне здесь не разрешают 
делать ничего похожего, а мне бы хотелось этого больше всего на свете. 
Я с удовольствием буду колоть дрова и жарить на углях котлеты. А 
сейчас, если я проголодалась, мне стоит позвонить в колокольчики. 
Видишь, вон они стоят.
    Рядом с ее постелью висела полка, и на ней Питер увидел ряд 
колокольчиков.
    - Для разных людей разные колокольчики, - продолжала Лована. - 
Большой вызывает кухарку, следующий - горничную, и так до самого 
маленького, которым я зову учителя. Сегодня я должна ему позвонить 
после того как выучу исследователей Австралии. Утром мне надо будет 
перечислить их всех моему отцу. Скажи, если мы сможем пожениться, ты 
действительно разрешишь готовить вам с Кривым Миком еду, ухаживать за 
вами, когда вы будете болеть, кормить птиц и животных?
    - Ты сможешь все делать, но мы будем тебе помогать. Мы всегда 
помогаем друг другу. Если я заболею, то помогать не смогу. Тогда уже 
тебе придется делать все самой.
    - Как вы, наверно, замечательно живете! - сказала Лована. - Вы, 
наверно, такие счастливые!
    - Да, мы счастливы, - ответил Питер. - Но сейчас я беспокоюсь за 
Серую Шкурку. Хорошо бы ты сдала экзамен прямо сегодня. Тогда тебя бы 
отпустили, и мы смогли бы найти ее.
    - Я не очень умна, - грустно сказала Лована, - в этом вся беда. 
Эти экзамены я никогда не сдам.
    - Нет, сдашь, - сказал Питер. - Гляди, у меня для тебя подарок. - 
Он вынул из маленькой сумочки, висевший у него на шее, Волшебный Лист 
и протянул его принцессе. - Возьми его. Сожми в руке. Ну, пожалуйста. 
Это удивительный Лист. Вот увидишь.
    Лована взяла Лист и крепко его сжала. И понемногу она начала 
меняться. Она стала еще более прекрасной, чем раньше. Улыбка, 
озарившая ее лицо, была так непередаваемо хороша, что Питер чуть не 
закричал.
    Подарок изменил и Питера: он стал благороднее и мужественнее.
    - Ты вдруг так изменился... - сказала Лована.
    - Мне тоже это показалось... А ты сама что-нибудь чувствуешь?
    - Невероятно, но мне хочется сдавать экзамены прямо сейчас! Не 
понимаю, что со мной случилось?
    - Тебя любят, ты нужна людям, - ответил Питер. - Ты сдашь любой 
экзамен. Звони своему учителю.
    - Звоню. Спрячься в моей спальне, и я его позову.
    Она закрыла за Питером дверь спальни и позвонила в самый маленький 
колокольчик. Учитель явился мгновенно. Он уже стучал в дверь, когда 
принцесса еще не кончила звонить.
    - Войдите! - отозвалась принцесса. В комнату вошел сутулый человек 
в перепачканном мелом черном кителе и в очках со стальной оправой, 
причем очки сидели на кончике его носа так, что он мог смотреть поверх 
них.
    - Ваше Высочество закончили заниматься? - спросил он.
    - Да, закончила, и закончила навсегда, - ответила Лована. - Я хочу 
прямо сейчас сдавать экзамены. Принесите мне вопросы, пожалуйста.
    - Но, Ваше Высочество, - пытался возразить учитель. - Вы отлично 
знаете, что три раза вы уже проваливались. Вы еще совсем не готовы 
сдавать ни один экзамен. Понимаете, сидеть над раскрытой книгой - еще 
не значит выучить ее. Необходимо собраться. Кроме того, я уверен, что 
вы не помните дату казни Карла I. Или Уота Тайлера. Или разницу между 
полуостровом и мысом.
    - Я знаю абсолютно точно, в каком году он был казнен, и знаю, что 
казнить людей жестоко. Я помню все до единой книги, которые читала о 
мысах и об Уоте Тайлере. Немедленно дайте мне экзаменационные вопросы.
    Лована верила в Волшебный Лист.
    - Хорошо, хорошо, - проворчал учитель. - Но король будет страшно 
недоволен, если вы снова провалитесь.
    Он ушел за вопросами и через некоторое время вернулся, держа в 
руке несколько больших листов.
    - Итак, перед вами экзаменационные вопросы, - высокопарно начал 
он. - Садитесь за стол и отвечайте на них. Разговаривать запрещается. 
Я буду сидеть и следить за временем.
    Лована взяла вопросы и разложила перед собой на столе. Она ни 
капли не волновалась, как раньше. Она нисколько не боялась, поскольку 
знала, что сможет ответить на любой вопрос.
    Учитель достал из кармана жилетки массивные часы и посмотрел на 
них.
    - Я даю вам два часа на всю работу, - сказал он. - Приступайте к 
работе, когда я скажу "Начали".
    Несколько секунд он неотрывно смотрел на часы и затем скомандовал: 
"Начали!"
    Никогда еще Лована так быстро не писала. Ее перо скользило по 
бумаге без остановки. Ни один вопрос не поставил ее в тупик. Она то и 
дело бросала взгляд на Лист, сжатый в ее левой руке. Гора исписанных 
листков рядом с ней быстро росла, и, когда учитель крикнул: "Время!", 
Лована успела ответить на все вопросы.
    Учитель собрал листки и сел их проверять, а Лована стала наблюдать 
за его лицом. Возгласы, которые он издавал при чтении, не тревожили 
ее: "А-а! О-о! Ох!" - восклицал он.
    Закончив, учитель положил листы на стол и сказал:
    - Я просто не могу поверить. Вы сдали экзамен с наивысшими 
оценками. Поразительно! Вы были совершенной тупицей, Ваше Высочество! 
Я думал, у вас только красота и совсем нет ума. Что с вами произошло? 
Я обязан доложить королю. Теперь мне полагается повышение.
    Он убежал искать короля, а Лована пошла в спальню сказать Питеру о 
своем успехе.
    - Если это и есть образование, то я его получила, - сказала она. - 
Там не было ни одного вопроса о доброте и бескорыстии. От меня не 
требовалось проявить свое внимание и заботу. Ни слова о полевых цветах 
и птицах. Ни слова о жизни. Всему этому мне придется учиться у тебя и 
Кривого Мика.
    - Ты уже выучилась, - ответил Питер. А затем спросил: - Теперь ты 
свободна? - Он спешил спасти Серую Шкурку.
    - Да. Всем известно, почему я сидела взаперти, так что королю 
придется держать свое слово. Они с учителем сейчас придут. Встань 
рядом со мной, и я не буду бояться. Он попытается избавиться от тебя, 
дав три невыполнимых задания.
    - Мы не побоимся, - ответил Питер. - Волшебный Лист нам поможет.

        Глава 15
        ТРИ ЗАДАНИЯ

    Крики команд и топот охраны возвестили о приближении короля. Дверь 
рывком распахнулась, и вошел король, а следом за ним и королева. Она 
тут же бросилась обнимать Ловану, - ее прямо-таки распирало от 
гордости.
    - Дорогая, - восклицала она. - Я так счастлива. Леди Малтраверс 
умрет от зависти, когда узнает о твоих оценках. Ты сдала лучше, чем 
обе ее дочери.
    Но король был взбешен. Он увидел Питера и отметил его великолепную 
одежду и гордую осанку.
    - Кто посмел сюда вторгнуться? - вскричал он. - Взять его!
    Он махнул охране, солдаты рванулись с места, но Лована шагнула 
вперед и подняла руку.
    - Я призываю вас выполнить ваше же обещание, - обратилась она к 
королю. - Вы объявили, что когда я получу матрикул и стану свободной, 
я буду вправе сама выбирать друзей.
    - Пропади ты пропадом! - выругался король. - Разве я такое обещал? 
Наверно, у меня помутился рассудок.
    - А слово короля - закон, - напомнила принцесса.
    - Все ясно, - быстро произнес король, - но как бы то ни было, 
нельзя требовать от короля так много. Прошу тебя никогда больше не 
напоминать мне ни о чем подобном. Матрикул ты, может быть, и получила, 
но вот тактичности тебе явно не хватает. Как человеку деловому, мне бы 
хотелось, чтобы переговоры велись на моих условиях. Не волнуйся, я не 
трону этого принца, или кто он там есть, но ты должна знать, что 
согласие на брак с тобой я дам только тому, кто сможет выполнить три 
задания, которые я назову.
    - Я их выполню, какие бы они ни были, - сказал Питер. - Назовите 
их.
    - Хорошо, - ответил король. - Во-первых, ты должен привести 
человека, который сможет солгать лучше меня.
    - Боже мой! - прошептала Лована на ухо Питеру. - Мой отец - 
величайший лгун всех времен. Нет такого человека, кто мог бы его 
превзойти.
    - Такой человек есть, - прошептал Питер в ответ. А королю он 
заявил: - Я приведу такого человека, при условии, что судьей будет 
Буньип.
    Королю это понравилось. Буньип всегда был верен ему, и он не 
сомневался, кого тот объявит победителем в состязании лгунов.
    - Идет, - ответил король. - Я согласен назначить Буньипа судьей.
    - А какими будут другие задания? - спросил Питер.
    - Если тебе удастся найти человека, который превзойдет меня во 
лжи, твоим следующим заданием будет усмирить Фаерфакса, дикого коня из 
Глухих Гор. Я хочу, чтобы его привели ко мне, оседланного, 
взнузданного и пригодного для езды.
    - Не слыхал о таком, - сказал Питер.
    Король рассмеялся.
    - Услышишь, - сказал он, потирая руки. - Он опаснее дракона, 
быстрее, чем удар молнии, и подпрыгивает выше деревьев. Еще никому и 
никогда не удавалось набросить на него уздечку. Если он тебя и не 
убьет, то подбросит так высоко, что друзья успеют подготовиться к 
твоим похоронам, прежде чем ты упадешь на землю.
    Питер не мог представить такого коня, но он знал, что с помощью 
Кривого Мика он его усмирит.
    - А третье задание? - спросил он.
    - Третье же задание, - сказал король, - будет самым главным. 
Корона, которую должна надеть Лована, если ей когда-нибудь суждено 
выйти за тебя замуж, лежит на дне озера. Туда ее забросила колдунья, 
которая подметает луну. Она ее украла, когда меня не было дома. Со 
мной бы у нее такие штучки не прошли, это точно. В наши дни никому 
нельзя верить.
    - Это та самая колдунья, что летает с кинокамерой?
    - Да, с чрезвычайно дорогой камерой.
    - Это последнее задание, которое я должен выполнить, чтобы 
получить руку принцессы? - спросил Питер, сообразив, что речь идет о 
той самой колдунье, с которой он встретился на пути в замок.
    - Да, последнее. Боюсь, однако, что тебе и жизни не хватит, чтобы 
достать золотую корону со дна озера, даже если тебе удастся выполнить 
первые два. Ну, а коли ты все выполнишь... принцесса - твоя.
    Король смахнул пушинку со своего камзола и объявил:
    - Отныне вы оба имеете право на свободное передвижение в пределах 
замка до тех пор, пока не придет время начать первое испытание.
    Он кашлянул в кулак, чтобы подчеркнуть собственную значительность, 
и приказал страже удалиться.
    - Идем, - позвал он королеву. - Надо обговорить с Буньипом условия 
состязания лгунов.
    Обернувшись к Питеру, он бросил:
    - Состязание состоится завтра днем в Большом Зале. До этого срока 
покидать замок тебе не разрешается.
    Когда король с королевой ушли, Лована сказала:
    - А теперь пойдем искать Серую Шкурку.
    Они побежали наверх, где Питер оставил Серую Шкурку, но там никого 
не было. Тогда они спустились и пошли по коридорам, минуя стражу, 
которая получила приказание их не трогать, и потому беспрепятственно 
пропускала. Серую Шкурку они нашли в саду, запертую в загончике из 
металлической сетки, натянутой на столбах. Выглядела она вполне 
сносно, поскольку верила, что Питер и принцесса ее спасут.
    - Я знала, что вы придете, - сказала она. - И надеялась, что это 
случится скоро. Кормят меня здесь ужасно - плесневелым сеном. И я не 
могла разговаривать - с тех самых пор, как меня схватили. Через день-
другой меня бы пустили на консервы для собак. А я пыталась думать о 
чем-нибудь другом. Но ответь, почему и ты, и принцесса свободно 
разгуливаете?
    - Ее зовут Лована. - сказал Питер.
    - Это имя мне правится, - сказала Серая Шкурка. - Оно звучит, 
словно дуб что-то шепчет на ветру...
    - Я рада, что оно тебе понравилось, - отозвалась Лована.
    Питер поведал Серой Шкурке обо всем, что произошло. Он рассказал о 
трех заданиях, которые ему предстоит выполнить, и о том, как он 
собирается ударом Громобоя вызвать на помощь Кривого Мика.
    - Что ж, не стоит терять время, - посоветовала Серая Шкурка. - 
Если состязание лгунов состоится завтра, то Кривому Мику надо дать 
время придумать какую-нибудь потрясающую историю, которая бы короля 
припечатала.
    Они вышли на середину двора и Питер развернул длинную плеть 
Громобоя. Собравшись с силами, он начал описывать им над головой 
круговые движения, пока тот не образовал вращающееся в воздухе колесо. 
Тогда Питер резко бросил руку вниз и назад, и в ответ громыхнуло так, 
словно выстрелила пушка. Солдаты и офицеры, которые стояли оперевшись 
о стену, подпрыгнули от испуга, но когда обернулись, чтобы узнать, в 
чем дело, то увидели только морщинистого погонщика, который стоял 
перед Лованой и ее другом.
    {whisp10.gif}
    Кривой Мик был удивлен не меньше солдат. Он сидел у дома и чинил 
седло, как вдруг в одно мгновение очутился во дворе замка. Но Питер 
быстро ему все объяснил, и Мик сел на скамьи у стены обдумать 
создавшееся положение.
    - В состязании лгунов я без труда уложу короля на обе лопатки, - 
промолвил он. - Но этот Фаерфакс меня беспокоит. Как мы его поймаем, 
ума не приложу.
    - Я догоню его на Мунлайт, - сказал Питер. - Не может он бежать 
быстрее нее. А догнав, я вскочу на него.
    - А где он обычно пасется?
    - Живет он в Глухих Горах, - ответил Питер. - Мы его запросто 
найдем.
    - Это может оказаться не так-то просто, - ответил Кривой Мик. - Но 
пока отложим это. Как насчет третьего задания? Озеро, наверно, 
глубокое, а ныряльщик я никудышний.
    - Об этом не беспокойся, - сказал Питер. - Я знаю, кто мне 
достанет корону. Двух первых заданий я боюсь больше.
    - Состязание лгунов, считай, у нас в кармане, - заверил его Кривой 
Мик. - Куда королю со мной тягаться! Завтра щелкни кнутом, когда надо 
будет идти в Большой Зал, и я буду тут как тут. А пока я вернусь домой 
и дочиню седло.
    Он встал и исчез, подняв за собой вихрь из пылинок.

        Глава 16
        СОСТЯЗАНИЕ ЛГУНОВ

    На следующий день в Большом Зале собрались люди со всех четырех 
концов королевства. Пришли фермеры и садоводы, скотоводы и бродяги-
золотоискатели, а Директор акционерной компании по добыче олова, меди, 
цинка, свинца, золота и нефти привел с собой жену ж детей. Король 
всячески его обхаживал, так как надеялся найти нефть на принадлежащих 
компании землях, и предоставил им всем места в первом ряду.
    Директор и сам был лгун что надо.
    - Надеюсь, Ваше Величество, почерпнуть из ваших историй что-нибудь 
полезное, - заявил он.
    - Мне нечему вас учить, - заскромничал король. - Я всего лишь 
любитель, у меня нет вашего опыта. Но я, без сомнения, одолею 
соперника, кого бы они ни выставили. - И он кивнул в сторону Питера и 
Серой Шкурки, которые вместе с Кривым Миком входили в зал.
    Лована была не с ними. Она сидела рядом с королевой на одном из 
тронов, поставленных в зале.
    Король разговаривал с Буньипом, который был чрезвычайно доволен 
тем, что его хитрость удалась и он сумел тайно впустить своих друзей в 
замок. Он сидел на огромном бревне, которое выкатили на середину 
площадки. Рядом с ним стоял трон, так обильно утыканный алмазами, что 
просто резало глаза. Он предназначался для короля. Кривой Мик сел на 
бревне рядом с Буньипом, который разглядывал какие-то бумаги. Питер 
расположился слева от королевского трона.
    Мгновение спустя король вошел и опустился на трон. Он расправил 
мантию, поправил корону и придал лицу важное выражение.
    - Удобно ли Вам, Ваше Величество? - спросил Кривой Мик.
    - Вполне, - ответил король. - Когда я удобно устраиваюсь, я удачно 
лгу. Поэтому на состязании я считаю очень важным сесть как следует, а 
не присесть кое-как.
    - Согласно правилам, - начал Буньип, открывая состязания и читая 
скрепленный сургучом пергамент, - вы вправе усаживаться и 
присаживаться как пожелаете. Вы вправе лгать с лошади или с любого 
другого животного, избранного вами для передвижения. Вы вправе лгать, 
двигаясь по направлению к судье, - то есть ко мне. Но не вправе лгать, 
двигаясь от судьи, так как это оскорбительно для меня. Вы вправе 
лгать...
    - Хватит болтать! - вспылил король. - Приступим к состязаниям. 
Правила я знаю - сам писал.
    - Это уже пошла первая ложь, Ваше Величество? - спросил Кривой 
Мик, который считал короля весьма посредственным лгуном. Он был 
уверен, что сумеет одолеть его с помощью преувеличения, в крайнем 
случае с помощью выдумки или легкой неправды.
    - Я еще не начинал, - ответил король.
    - Тишина! - проревел Буньип.
    Слушатели поудобней уселись в креслах. Все они надеялись на 
поражение короля, поскольку любили Ловану и желали ей выйти замуж за 
Питера.
    - Состязание объявляется открытым. Первый свою ложь представляет 
король. В зависимости от качества лжи я начисляю от одного очка за 
преувеличение до десяти за ложь наглую и бессовестную.
    Король откашлялся и приступил.
    - В юности я зарабатывал немало денег тем, что вырубал из земли 
заброшенные шахты и продавал их на юге для колодцев.
    - Как давно это было? - спросил Буньип.
    - Очень давно. Задолго до того, как я родился, - ответил король.
    - Что ж, такое красивое начало лжи не часто услышишь, - сказал 
Буньип. - Продолжайте.
    - Для такой работы требовалось большое искусство, - говорил 
король. - Мне надо было сохранять кристальную честность и иметь 
обширные познания в математике.
    - Зачем?
    - Мне приходилось немало складывать, умножать и вычитать.
    - Понятно. А сколько вы брали за колодец?
    - Сто долларов.
    - Одно очко, только одно, - решил Буньип.
    - Четыреста долларов.
    - Уже лучше. Еще очко.
    - Каждый колодец я грузил в фургон, запряженный быками, - 
продолжал король, - и затем трогался в Мельбурн. В общем, задавал я 
быкам работенку.
    - А сколько их было?
    - Двести двадцать.
    - Ничего себе упряжка! - воскликнул Буньип.
    - Да, не маленькая. Я протянул телефон от фургона до передних в 
упряжке быков, и посадил погонщика управлять ими. Когда я хотел 
сделать привал, я звонил из фургона погонщику и просил его 
остановиться. Тот останавливал передних быков. За ними постепенно 
останавливались другие, когда их цепь ослабевала. В результате задние 
останавливались через полчаса после передних. Такая система всю дорогу 
работала неплохо, пока как-то раз я случайно не набрал неправильный 
номер, а когда исправил ошибку, быки прошли уже десять миль.
    - Полагаю, что при данных обстоятельствах указанное расстояние 
является весьма вероятным, - сказал Буньип. - Продолжайте.
    - Самая большая неприятность произошла в то утро, когда мы 
двинулись в путь, - объяснил король. - Передние тронулись и лишь через 
полчаса задние почувствовали, что цепь натянулась и тронулись тоже. Мы 
теряли так много времени на то, чтобы тронуться с места и остановку, 
что до Мельбурна добирались целых шесть месяцев. Мне удалось всю шахту 
продать одному человеку, который хотел соорудить в своем саду колодец. 
И надо сказать, хороший колодец получился. Но я в конце концов 
забросил это дело, поскольку оно не окупало расходы.
    - Это все? - спросил Буньип.
    - Да.
    - Что ж, ложь хорошая, качественная, но лично мне труднее всего 
поверить в то, что вы возили шахты на быках вместо того, чтобы 
отправить их поездом. Почему вы так не сделали?
    - Шахты находились в необжитых районах, поезда туда не ходили.
    - А, все ясно. Теперь послушаем, что скажет Кривой Мик.
    Кривой Мик поднялся с бревна и начал.
    - Однажды я вез на фургоне с быками из Бурка пять тонн оловянных 
дудочек.
    - Неплохое начало, - отметил Буньип, делая какие-то пометки в 
блокноте.
    - А куда ты их вез? - спросил Питер, которого этот рассказ вдруг 
заинтересовал.
    - Никуда, - ответил Кривой Мик.
    - Но ты ведь должен был везти их куда-нибудь, - встрял король.
    - Я вез их никуда, - повторил Кривой Мик. - Если бы я вез их куда-
то, я бы в конце концов туда и привез, и тогда оловянных дудок у меня 
больше не было бы. А поскольку я не вез их никуда, они всегда были при 
мне.
    - Но какая от них польза? - спросил Питер.
    - Ни малейшей, - ответил Кривой Мик, - хотя они могли бы 
пригодиться для подарков.
    - Мне почему-то кажется, что ты лжешь, - проговорил Буньип.
    - А я что должен делать? - поинтересовался Кривой Мик.
    Буньип выглядел огорошенным.
    - Да, конечно, - быстро добавил он. - Я стал забывать. Никогда еще 
ложь так тесно не переплеталась с правдой. Продолжай.
    - Налетел порыв ветра, - рассказывал Кривой Мик, - который 
всколыхнул деревья и поднял пыль.
    - Утверждение правдоподобно, - перебил его Буньип. - Ветер всегда 
колышет деревья и поднимает пыль. Боюсь, мне придется тебя оштрафовать 
на десять очков. На правду не переходить. Продолжай.
    - Мне следовало бы сказать, что ветер не колыхал деревья и не 
поднимал пыли, - поправился Кривой Мик, - хотя на расстоянии вытянутой 
руки уже не было ничего видно. Ветер с такой силой дул в 
противоположных направлениях, что он сталкивался сам с собой и 
оставался в полнейшей неподвижности, пытаясь преодолеть сам себя. 
Понятно, что я хочу сказать?
    - Нет, - ответил Буньип.
    - На это я и рассчитывал, - сказал Кривой Мик.
    - Это как два быка, которые столкнулись лбами и не могут друг 
друга сдвинуть? - спросил Питер.
    - Да, - ответил Кривой Мик.
    Буньипу это стало надоедать и он строго посмотрел на Питера.
    - Здесь сужу я. Быки не имеют никакого отношения к рассказу. - 
Затем Кривому Мику: - Пожалуйста, продолжай.
    - Я тащил быков вверх по песчаному склону, - продолжал Кривой Мик. 
- Я уже находился на уровне колес, а у быков голова вытянулась к 
коленям. Я щелкал кнутом и вопил: "Но-о! Но-о!", как вдруг услышал 
прекраснейшую музыку. Я оперся о кнутовище и прислушался. Музыка 
продолжалась, и я осознал, что слушаю Пятую симфонию Бетховена, 
исполняемую на оловянных дудочках. Понимаете, дудочки лежали 
мундштуками в сторону ветра. Веревки, которыми они были связаны, 
закрывали одни отверстия на дудочках, и открывали другие. Когда фургон 
раскачивался от рывков, одни отверстия открывались, другие 
закрывались, причем таким образом, что ветер, влетающий в мундштуки, 
вылетая, сотворял маленькое чудо. Знаете, меня это изумило.
    - И меня это изумляет, - угрюмо проворчал король.
    Буньип в блокноте суммировал очки. Он то бесцельно смотрел на 
крышу, то стучал карандашом по зубам, на какой-то странице нарисовал 
лошадиную голову, тут же ее перечеркнул и произнес:
    - Приняв во внимание правила состязания, оговоренные в третьем 
параграфе меморандума, а также условия, нижеименуемые... - впрочем, 
довольно. Победителем объявляю Кривого Мика.
    Зрители захлопали, закричали "Ура!", а одна фрейлина даже упала в 
обморок. Не потому, что для этого был какой-то повод. Просто ей 
нравилось падать в обморок на подобных состязаниях. Она считала, что 
ее утонченное воспитание не позволяет ей выносить ложь.
    Король был вне себя. Вскочив, он отбросил мантию вверх и в 
сторону, так что она намоталась на его плечи как крыло. На нижних 
ступеньках лестницы, ведущей к трону, он задержался и гневно произнес:
    - Возмутительно! Того и гляди, начнут ставить под сомнение 
обещания, которые дают высокопоставленные лжецы! Я обязан извиниться 
перед Директором акционерной компании по добыче олова, меди, цинка, 
свинца, золота и нефти за оскорбление, нанесенное его прозорливости.
    - Полно, полно, - поспешно отозвался Директор. - Не надо ломать 
комедию.
    Он нацарапал какую-то записку на бумаге и протянул ее королю.
    - Вы можете получить работу в нашем совете директоров, как только 
захотите.
    Король прочел записку. Настроение его поднялось, по гнев не 
прошел.
    - К счастью, остаются еще два задания, чтобы испытать этого 
самозванца, который выдает себя за принца. Пусть сначала приведет мне 
Фаерфакса, Дикого коня с Глухих Гор - оседланного, укрощенного и 
годного для езды, а потом готовится к последнему заданию.
    Король покинул зал. Последовавшая за ним королева, спускаясь во 
двор, наступила пяткой на собственный шлейф и разорвала его по шву. 
Одна из фрейлин, однако, скрепила края булавкой. Это злополучное 
происшествие случилось как нельзя некстати, поскольку королю пришлось 
стоять и ждать королеву, а делать это с подобающим величием крайне 
затруднительно.
    {whisp11.gif}
    Питер пошел проводить принцессу Ловану. Она взяла его под руку, и 
они так и шли по длинным коридорам. Когда кто-нибудь встречался им, 
Питер снимал шляпу с перьями и кланялся, а принцесса улыбалась и 
кланялась тоже. Те, кому она улыбалась, говорили: "Словно запели вдруг 
все лесные птицы".

        Глава 17
        ЧЕЛОВЕК-СМЕРЧ ПОЯВЛЯЕТСЯ СНОВА

    Питер простился с Лованой у ее комнаты и, выйдя из замка, поспешил 
к большому дереву, где его уже ждали Кривой Мик и Буньип. Серая Шкурка 
отправилась за Мунлайт, которая паслась у дороги примерно в миле 
отсюда.
    В тени дерева на спине лежал Буньип, сложив лапы на брюхе и 
размышляя, как бы выполнить второе задание.
    Питер сидел рядом.
    - Я придумал, как найти Фаерфакса, - сказал он. - Там в горах 
сотни долин, и он может пастись в любой. Наши поиски могут продлиться 
многие месяцы, но так ничего и не дать. Зато я знаю человека, который 
нам точно скажет, где он.
    - Это сильно облегчит нашу задачу, - отозвался Кривой Мик. - Как 
его звать?
    - Человек-смерч, Вилли-Вилли. Мы с Серой Шкуркой встретились с 
ним, когда шли по Пустыне Одиночества. У него есть подзорная труба, в 
которую он различает предметы за сотни миль.
    - А как мы с ним свяжемся? - спросил Кривой Мик.
    - Это непросто, - ответил Питер. - Действительно, как? Живет он 
далеко.
    - В Пустыне Одиночества?
    - Да.
    - Что ж, у нас хорошие лошади. Поскакали к нему. Это не займет 
много времени.
    - А как ты собираешься усмирить Фаерфакса, когда поймаешь? - 
спросил Буньип. - Я знаю двоих рыцарей, которые пытались его изловить, 
но у них ничего не вышло. Фаерфакс - очень сильный и норовистый конь.
    - Когда мы его поймаем, Питер его оседлает и на нем поедет: 
объезжать лошадей я его научил.
    - Я уверен, что смогу на нем проехать, - с некоторым сомнением 
произнес Питер. Однако уверенности в его голосе не было.
    - И я уверен, - сказал Кривой Мик. - В тебе есть мужество и 
отвага. Тебе, конечно, надо переодеться. Когда твоя голова начнет 
дергаться во все стороны, то перу на шляпе не удержаться, оно упадет, 
как тростинка. Когда станешь объезжать Фаерфакса, будь внимателен. 
Стоит немного расслабиться, и ты пропал. Я дам тебе одежду, в которой 
сам объезжаю лошадей - брюки и старую красную рубашку. Седло и уздечку 
можешь взять свои. Загони его. Крепко держи ноги в стременах и весь 
свой вес перенеси на ноги. Дай ему волю, но будь осторожен. Если он 
вздумает прыгнуть до небес, упреди его. Но не сиди как изваяние, 
расслабляйся. Больше непринужденности - и он твой. Не тереби его - 
пусть этим занимаются на родео. Помни, что в конце концов он 
выдохнется и захрипит, а отдохнув, начнет сначала. Лови этот момент, 
не давай ему отдышаться. Ты его усмиришь. Тебе нет равных.
    Кривой Мик похлопал Питера по спине.
    - А теперь в путь. Сперва нам придется хорошенько потрудиться, 
чтобы добраться до этой Пустыни Одиночества. А когда Вилли-Вилли 
подскажет, где этот Фаерфакс, мы быстро найдем его.
    - А на какой лошади ты поскачешь? - спросил Питер.
    - Да, на какой лошади я поскачу? - спросил Кривой Мик Буньипа.
    - Если пойти вдоль рва, то с другой стороны замка будет видна 
конюшня, - ответил Буньип. - Там стоят несколько лошадей, оставшихся 
от тех рыцарей и принцев, которых я убил когда-то. Возьми гнедого с 
белой звездочкой на лбу. По-моему, он неплох.
    Когда Кривой Мик вернулся, Серая Шкурка и Мунлайт уже ждали его. 
Мунлайт была под седлом, взнузданная и готовая к путешествию. Конь, 
выбранный Кривым Миком, уступал ей и в силе, и в изяществе. И выглядел 
совсем не так уверенно и гордо. Было непохоже, чтобы он рвался в бой, 
нетерпеливо прислушиваясь к зову трубы или шуму битвы.
    Мунлайт изменилась и стала более статной. Ее длинные узкие плечи 
говорили, что она не знает, что такое усталость. Круп был мускулист, и 
она скакала с такой легкостью, будто парила над землей. Каждый раз, 
когда Питер делал кого-то счастливее, давая Волшебный Лист, он не 
только сам менялся, превращаясь в благородного принца: менялась и 
Мунлайт, становясь достойной принца по силе и красоте.
    Кривой Мик подогнал стремена и вскочил в седло. Гнедой закрутился, 
и Мик пробормотал: "Конь-то совсем не выезжен". Потом он натянул 
поводья, и гнедой замер.
    Кривой Мик дал коню шенкеля, и тот рванулся вперед. "До встречи!" 
- крикнул он Буньипу. Питер последовал за ним, и две лошади бок о бок 
помчались по бушу, за которым находилась Пустыня Одиночества.
    - Мы идем к своей цели! - крикнула Серая Шкурка и запрыгала вслед. 
Она то поднималась, то опускалась, подобно морской волне, и вскоре 
догнала всадников; а когда те уже скрылись в зелени, головка Серой 
Шкурки то появлялась над кустами, то исчезала внизу.
    {whisp12.gif}
    Они скакали весь день и к вечеру достигли места, где росла высокая 
трава. Кривой Мик пришел в восторг, увидев это идеальное пастбище.
    - Здесь мы и остановимся, - сказал он. - За этим холмом будет 
озеро. Ты, Питер, напоишь лошадей, а мы с Серой Шкуркой устроим 
привал.
    Когда Питер привел лошадей обратно, в сложенном из камней очаге 
уже горел костер, и Кривой Мик жарил мясо. Питер расседлал лошадей и 
пустил их пастись, а потом сел к огню рядом с Серой Шкуркой и Миком.
    Утром они выехали еще до рассвета, чтобы успеть добраться до 
Пустыни Одиночества к обеду. По пути они преодолели несколько песчаных 
барханов, и вскоре вся пустыня раскинулась перед ними. Она показалась 
им еще более грустной и зловещей, чем когда бы то ни было.
    Они сдержали лошадей и остановились. Нигде на огромных безводных 
пространствах не было видно Вилли-Вилли.
    - Давайте пошлем ему сигнал дымом, как делают аборигены, - 
предложила Серая Шкурка. - Когда Вилли-Вилли увидит отдельные облака 
дыма, он обязательно примчится выяснить, в чем дело.
    Питер и Кривой Мик соскочили на землю и, собрав зеленых листьев 
эвкалипта, свалили их в кучу на открытом участке. Серая Шкурка 
отломила от наклонившегося эвкалипта большую ветку и заявила, что с ее 
помощью сделает целые облака дыма.
    Кривой Мик поджег листья, и из них повалил густой дым. Однако 
Серая Шкурка тут же закрыла костер большой веткой, чтобы дым 
накапливался. Время от времени она отдергивала ветку, и тогда дым 
вырывался наверх огромными клубами. Вскоре над пустыней уже плыла 
цепочка облаков.
    - Ну вот, теперь он обязательно появится, - сказала Серая Шкурка, 
останавливаясь, чтобы перевести дух.
    Она всматривалась вдаль, сощурив от солнца свои зоркие глазки.
    - Вот он! - воскликнула она наконец. - Ну и быстро же несется!
    - Теперь и я вижу, - подтвердил Питер и показал Кривому Мику, 
зрение которого уже немного испортилось, на далекий столб пыли.
    Приближаясь, смерч увеличивался в размерах. Оказавшись рядом, он 
замедлил скорость, потом остановился и опустился. Из оседающей пыли 
появились человек-смерч Вилли-Вилли и Том и зашагали к ним.
    - Я подумал, может, это Серая Шкурка посылает нам сигналы, - 
сказал Вилли-Вилли. - Как дела, Питер? С тех пор как мы встретились, я 
стал другим человеком. Том сделал мне капитальный ремонт. Никогда еще 
я так хорошо не работал. А благодаря тому листу, что ты мне подарил, 
моя жизнь стала намного интересней.
    Все уселись на земле и начали разговаривать. Том рассказал, как 
хорошо он живет с тех пор как встретил Вилли-Вилли, а Питер поведал о 
своих приключениях и о том, что в данный момент они ищут Фаерфакса.
    - Я верил, что если мы тебя встретим, ты нам поможешь его найти, - 
сказал Питер.
    - Ты обратился именно к тому, к кому нужно, - важно ответил Вилли-
Вилли. - Том, сходи туда, где мы остановились, и принеси подзорную 
трубу.
    Когда Том вернулся, Вилли-Вилли раздвинул трубу на полную длину и 
положил ее на плечо Питера.
    - Так, давай посмотрим. Только не дергайся. Что там такое? А, это 
часы на ратуше в Мельбурне. Показывают ровно двенадцать. Но это нам не 
надо. Развернемся к горам. Отлично. Немного приподними плечо. Хорошо. 
Так и держи. Вижу Фаерфакса. Он с табуном кобыл пасется в Долине 
Ключей. Я сейчас расскажу вам, как туда попасть.
    Он сложил подзорную трубу и указал на видневшийся у горизонта 
горный кряж.
    - Видите те две вершины? Одна чуть выше другой?
    - Вижу, - ответил Питер.
    - Долина Ключей лежит как раз между ними. Поезжайте в этом 
направлении, пока не выйдете на берег речки. Идите по ее течению, и вы 
придете в Долину Ключей. А теперь мне пора. Удачи вам всем. Заводи 
меня, Том.
    Том обмотал его за пояс шнуром и дернул. Вилли-Вилли завертелся 
мгновенно. Том впрыгнул во вращающийся круг и вместе с Вилли-Вилли 
понесся по пустыне.
    - Хороший он парень, - сказал Кривой Мик. Потом вскочил на коня и 
направил его в сторону гор. Питер и Серая Шкурка пустились следом.
    Только через два дня они добрались до речки, о которой говорил 
Вилли-Вилли, и пошли вдоль нее, пока она не привела в Долину Ключей.
    Кривой Мик развернул своего гнедого и направил его к небольшому 
холму: он надеялся оттуда обозреть всю долину. Поднимались они по 
дороге, очевидно, проложенной лошадьми. Около деревьев на самой 
вершине они остановились и окинули взглядом долину. Там росла густая 
трава и полевые цветы.
    Прямо под ними на широком поле пасся табун. Чуть поодаль они 
увидели Фаерфакса - могучего жеребца, рядом с которым другие кони 
выглядели просто карликовыми. В лучах заходящего солнца его рыжая 
шерсть отливала золотом. Даже издали бросался в глаза его дикий, 
необузданный нрав: копь рыл землю копытом и галопом носился вокруг 
своего табуна, оберегая его.
    Кривой Мик понял, что конь хотя и не видит их, но уже что-то 
заподозрил, и поэтому кивком приказал Питеру и Серой Шкурке следовать 
за ним - вниз по дороге. У ручья они нашли скалу, которая защищала их 
от ветра, и расположились на ночлег.

        Глава 18
        ФАЕРФАКС

    На следующее утро Кривой Мик, Питер и Серая Шкурка вновь поднялись 
на вершину холма и там, прячась за деревьями, разработали план поимки 
Фаерфакса. Долина была открытая, ровная. Речка, вдоль которой они шли, 
извивалась вдоль долины и исчезала в тесном ущелье. По берегам речки, 
повторяя каждый ее изгиб, стояла, словно охрана, стена деревьев.
    Табун лошадей пасся в дальнем конце долины. Кривого Мика, однако, 
больше заинтересовала дорога, которая вела от табуна к началу ущелья. 
Очевидно ее проложили лошади, когда переходили из этой долины на 
другие пастбища. Она-то и надоумила Кривого Мика, как поймать 
Фаерфакса.
    Он поделился своим планом с Питером и Серой Шкуркой, и скоро они 
отправились в обход к тому ущелью, где исчезала дорога. Кривой Мик 
намеревался погнать лошадей по этой дороге, которая была единственным 
известным им путем к спасению.
    Достигнув начала ущелья, они вошли в него и шли до тех пор, пока 
оно не сузилось и не превратилось в узкую теснину с уходящими ввысь 
стенами. То тут, то там посреди дороги лежали огромные валуны, 
сорвавшиеся с отвесных стен каньона, а сама дорога была изрыта 
копытами лошадей, проходивших по ней то в одну, то в другую сторону.
    Здесь все остановились. Кривой Мик решил перегородить ущелье 
высоким забором, который должен был остановить лошадей, когда они 
помчатся прочь из долины. Речка в этом месте была совсем мелкой и 
перегородить ее не составляло труда.
    - Прежде всего мы должны построить забор, - сказал Кривой Мик. - А 
для этого пусть Серая Шкурка достанет нам необходимые инструменты. 
Серая Шкурка оправдала надежды Кривого Мика. Ей льстило, что успех 
операции зависел и от нее. Она вытащила из сумки экскаватор для рытья 
ям, потом, улыбаясь, извлекла лом, два мотка колючей проволоки, 
лопату, бензопилу, бурав и краску для железа.
    - Теперь у нас есть все, что необходимо, - сказал Кривой Мик, не 
замечая довольного вида Серой Шкурки. - Мы спилим вон те молодые 
деревья, что впереди. Ты, Питер, возьмешь пилы и приготовишь шесты. 
Ты, Серая Шкурка, выроешь шесть ям для столбов на дороге и в речке на 
мелководье, а я заберусь на скалу как можно выше и натяну от одной 
стены до другой колючую проволоку.
    Проработав весь день, они плотно поужинали жареным мясом и 
улеглись в свои спальные мешки. Утром они вернулись немного назад по 
ущелью и построили еще один забор, с мощными, сделанными из бревен 
воротами. Ворота Кривой Мик оставил открытыми, чтобы лошади, убегая из 
долины, могли в них проскочить.
    - Теперь, Питер, вы с Серой Шкуркой возвращайтесь к тому месту, 
откуда мы вошли в долину, и оттуда гоните лошадей вдоль дороги. Да, и 
поработай как следует своим кнутом, Питер, - пусть несутся в ущелье, - 
когда же попадут в загон, я закрою ворота, а потом заарканю Фаерфакса. 
Вперед!
    До начала долины было не очень далеко. Питер прижался к холке 
Мунлайт. и та обогнула долину, пройдя у самых гор. Серая Шкурка 
скакала рядом.
    Увидев, что к нему галопом приближается всадник, Фаерфакс вскинул 
голову и фыркнул. Скакавшую рядом кенгуру он не опасался - кенгуру для 
лошади не страшны, по всадник, размахивающий кнутом, - совсем другое 
дело.
    Он рысью пустился навстречу Питеру, высоко вскидывая ноги и тяжело 
ударяя копытами о землю. Ноздри его раздувались от гнева. Он выгнул 
шею, вскинул голову и грозно заржал, бросая вызов непрошенному гостю. 
Рыжая шерсть Фаерфакса золотилась на солнце, и Питер с восхищением 
разглядывал его. Расстояние между ними все сокращалось.
    Раскрутив над головой кнут, Питер с такой силой щелкнул им, что 
ответное эхо громыхнуло как пушечный выстрел. Фаерфакс отпрянул назад, 
тут же развернулся и помчался к табуну, который от страха сбился в 
кучу. Он ворвался в ее середину и стал кусать лошадей за бока и холки, 
принуждая бежать. Пока они набирали скорость, он нетерпеливо крутился 
вокруг них, подгоняя, но когда они разогнались вовсю, он возглавил 
табун и повел его по дороге к ущелью. Его грива билась на ветру, как 
пламя, а длинный хвост стлался над землей.
    Питер следил за Мунлайт. Хотя она без труда могла бы перегнать 
любую лошадь из табуна, она, словно понимая свою роль, ровно, без 
усилий бежала за ними.
    Когда табун вошел в ущелье, Питер догнал задних лошадей и обрушил 
на них град ударов кнута. Лошади обезумели, задние стали наседать на 
передних, а те в свою очередь просто втолкнули Фаерфакса в загон. За 
ним потоком понеслись остальные. Кривой Мик выскочил из-за дерева, за 
которым прятался, и захлопнул ворота.
    Фаерфакс добежал до забора и остановился на всем скаку. Он пытался 
повернуться, но обезумевшие кобылы окружили его, не давая одним мощным 
прыжком перемахнуть через забор, который отделял его от свободы.
    Кривой Мик подождал, пока Фаерфакс пробьется назад к воротам, и 
набросил на него аркан. Конь-великан взревел от ярости. Он рванулся 
вверх, взмахнув передними копытами. Но Кривой Мик для упора намотал 
веревку на шест, так что Фаерфакс только туже затянул петлю на 
собственной шее. Полузадушенный, он рухнул на землю посреди ошалевших 
от страха кобыл, которые теперь шарахались от него во все стороны.
    - Открывай ворота! - крикнул Кривой Мик.
    Питер, верхом на Мунлайт сдавший у ворот, спрыгнул на землю, 
бросился к воротам и мгновенно их распахнул. Кобылы, увидев выход, 
кучей ринулись туда, тесня друг друга, и помчались назад к долине с ее 
просторами и травой.
    В загоне остались один Кривой Мик и лежащий у его ног жеребец. Мик 
быстро расслабил петлю на его шее и через голову надел ему недоуздок, 
а повод привязал к столбу в заборе. Фаерфакс с трудом встал на ноги и, 
почувствовав узду, стал биться, как рыба на крючке. Но повод держал 
крепко, и через некоторое время конь затих, дрожа всем телом. Повод 
оставался натянутым.
    Кривой Мик набросил ему на голову мешок, и, оказавшись в темноте, 
Фаерфакс перестал тянуть повод и стоял спокойно.
    Подошли Питер и Серая Шкурка и вместе укрепили на его спине 
тяжелое седло и затянули пахву. Не снимая мешка, надели уздечку. Когда 
все было готово, Кривой Мик скомандовал:
    - Садись на него, Питер!
    И Питер вскочил в седло, прежде чем Фаерфакс успел сообразить, что 
с ним произошло.
    Тогда Кривой Мик сорвал мешок с головы коня, и какую-то секунду 
конь-великан стоял не шелохнувшись. Потом он мощно оттолкнулся от 
земли и одним гигантским прыжком покрыл половину расстояния до ворот, 
но Питер словно прирос к седлу. Он улыбнулся, помахал в воздухе рукой 
и прокричал: "Но! Но!" Выгнув дугой спину и опустив морду между 
передних ног, Фаерфакс сделал второй мощный прыжок. Приземлился он на 
вытянутых, поставленных рядом ногах, и удар, сотрясший его при 
приземлении, перетряхнул все кости в теле Питера.
    Понемногу Питер привык к прыжкам коня. Он чувствовал под собой это 
могучее тело, предвидел дальнейшие движения задних ног. Питер был 
уверен, что Фаерфакс попытается сбросить его, прыгая вверх и в 
сторону. Так и случилось. Фаерфакс подпрыгнул и затем, уже в воздухе, 
с такой быстротой лягнул задними ногами вбок, что Питер неминуемо 
оказался бы на земле, если бы не ожидал этого фокуса. Он изо всей силы 
сжал бока коня ногами и удержался.
    Поняв свою неудачу, Фаерфакс прямо-таки завизжал от ярости. Он 
сорвался с места и, перейдя за несколько прыжков в карьер, с такой 
скоростью помчался по ущелью, что стволы деревьев вдоль дороги 
показались Питеру частоколом.
    Кривой Мик вскочил на Мунлайт, и они с Серой Шкуркой поспешили 
следом. Едва успели они добраться до долины, как увидели редчайшую 
картину. Фаерфакс задался целью во что бы то ни ехало сбросить Питера 
и продемонстрировал такой набор трюков, что Питер потом хвастался ими 
десять лет подряд. Фаерфакс взмывал в воздух выше деревьев, вертелся 
волчком, выгибал спину, угрожающе храпел, на что Питер в ответ издавал 
клич погонщика.
    {whisp13.gif}
    Фаерфакс пытался ободрать Питера о стволы деревьев, но Питер 
рывком высвободил ноги из стремени и подобрал их, так что конь только 
поранил сам себя. Тогда Фаерфакс вернулся на открытое пространство, 
где он мог без помех прыгать и брыкаться. Но он уже стал уставать, его 
гордая голова начала клониться к земле. Оставался еще один прием - 
встать на дыбы и упасть на спину, чтобы раздавить Питера своим весом. 
Это была его последняя, отчаянная попытка избежать поражения.
    Но Питер предвидел и ее и был наготове, когда огромный конь встал 
на дыбы. На несколько мгновений Питер прилип к спине Фаерфакса, пока 
тот, раскачиваясь на задних ногах, бил передними воздух и ревел широко 
открытым ртом, из которого падала пена. Вот конь на мгновение замер, 
стройный, как дерево, и тут же рухнул спиной назад. Питер, заранее 
освободивший ноги от стремян, успел оттолкнуться и соскочить с седла. 
Он мягко приземлился рядом с Фаерфаксом, держа поводья в руке. Когда 
же конь с трудом поднялся, Питер снова вскочил в седло, готовый к 
следующему трюку.
    Но сил у Фаерфакса больше не осталось: Питер чувствовал, как тот 
дрожит под седлом. Он осадил его, ласково потрепал по взмыленной шее и 
мягко заговорил. Потом он шагом провел коня по всей долине, пока все 
его опасения не рассеялись, и лишь тогда остановился рядом с Кривым 
Миком и сказал:
    - Что ж, я покорил его, но я раскаиваюсь в этом. Нельзя укрощать 
такого великолепного коня.
    - Я знаю, - согласился Кривой Мик. - Но ты не забывай, что сейчас, 
когда он сломлен, он к тебе привяжется. И не будет больше стремиться к 
свободе.
    - Я рад этому, - ответил Питер.
    Он спрыгнул на землю и держал уздечку, пока Кривой Мик садился в 
седло.
    - Я сделаю на нем круг по долине, чтобы посмотреть, как он идет.
    И он пустил коня легким галопом. Когда Кривой Мик вернулся, он 
улыбался.
    - Давай я все же поеду на Мунлайт и поведу гнедого. На Фаерфаксе в 
замок должен въехать ты сам.
    Питер сел верхом на Фаерфакса, и они тронулись в путь. Все очень 
устали.

        Глава 19
        ВОЗВРАЩЕНИЕ В ЗАМОК

    На обратном пути Серая Шкурка обогнала Кривого Мика и ускакала 
вперед. Она хотела предупредить короля об их приходе и сообщить 
Ловане, что Фаерфакс, Дикий конь из Глухих Гор, пойман и усмирен. 
Питеру хотелось, чтобы Лована первой узнала эту новость, поэтому Серая 
Шкурка, прискакав к замку, растолкала храпящего Буньипа и попросила 
отвести ее к принцессе.
    Буньип зевнул и сонно пробурчал:
    - Мне приказано убивать с наименьшей болью всех желающих увидеть 
Прекрасную Принцессу. Готовься, пока я вскипячу котелок.
    - Не дури! - рассердилась Серая Шкурка. - Со сном никак не 
расстаться? Давай, просыпайся поскорей.
    Буньип рывком сел.
    - А, это ты! - воскликнул он. - А где Питер? Поймал он дикого 
коня?
    - Об этом я расскажу только принцессе, - твердо сказала Серая 
Шкурка. - Отведи меня к ней.
    Буньип медленно встал и, еле переставляя ноги, побрел к замку, 
что-то бормоча себе под нос. У рва он позвонил в колокольчик, и 
подъемный мост, погромыхивая цепями, тут же опустился, глухо 
ударившись о берег.
    Буньип пересек мост, перебросился несколькими Фразами с 
приветствовавшей его охраной и повел Серую Шкурку к комнатам, где жила 
Прекрасная Принцесса. Они постучали в дверь и стали ждать.
    Когда Лована открыла дверь, на ее устах играла улыбка. Она слышала 
прыжки Серой Шкурки по коридору и догадалась, что та несет вести от 
Питера. Принцесса протянула к ней руки, и они крепко обнялись.
    Буньип смотрел на них с отвращением. Он не верил во всякие там 
поцелуи и объятья и потому пробурчал:
    - Женщины есть женщины! Когда они встречаются, они выставляют себя 
на посмешище. Только на нервы мне действуют, а на нервной почве у меня 
разыгрывается мигрень.
    У него и в самом деле разыгралась мигрень, и он поглаживал голову, 
входя в комнату следом за Лованой и Серой Шкуркой.
    Когда все расселись. Серая Шкурка рассказала с начала до конца 
историю укрощения Фаерфакса. И закончила следующими словами:
    - Они прибудут в замок с минуты на минуту. Зови короля, Лована. Он 
должен видеть их приезд, чтобы лично убедиться в успехе Питера. 
Наверно, он придет в ярость...
    Лована позвала слуг и отправила их в покои короля. Буньипу она 
велела погромче прореветь приказ: всем собраться под большим деревом.
    - Пусть твой голос разнесется на многие мили, - закончила она.
    Буньип остался доволен. Он вышел к подъемному мосту, откашлялся и 
с таким жаром несколько раз прокричал объявление, что разбудил всех 
обитателей замка.
    Лована надела свое лучшее платье, и они с Серой Шкуркой тоже 
вышли.
    - Это наверняка одно из лучших твоих платьев, - сказала Серая 
Шкурка.
    - Ты думаешь, Питеру оно понравится? - спросила Лована.
    - Да, - ответила Серая Шкурка.
    Когда они шли по мосту, люди уже собирались под деревом. Для 
короля и королевы поставили переносные троны, Лована и Серая Шкурка 
сидели на траве рядом с ними. Король с королевой выходили из замка, их 
приветствовали хлопками, но когда подъехали Кривой Мик и Питер, им 
устроили настоящую овацию.
    При выезде из буша Кривой Мик придержал Мунлайт, и последние сто 
ярдов Питер проехал на Фаерфаксе один. Жеребец выглядел как нельзя 
лучше. Его огненная шерсть, которую Питер тщательно расчесал при 
приближении к замку, сверкала, как начищенная медь. Фаерфакс шел 
легким галопом, раскачиваясь из стороны в сторону и выгнув шею, как на 
параде, и поднялся на дыбы, когда Питер остановил его перед королем. 
Король был заворожен его красотой. Он шепотом спросил у одного из 
придворных:
    - Сколько такая лошадь может стоить на свободном рынке?
    - Тысячу долларов, - ответил придворный почти не открывая рта. 
Король радостно потер руки.
    - Так значит, ты привел мне Дикую лошадь из Глухих Гор, укрощенную 
и пригодную к езде.
    - Да, он укрощен, Ваше Величество, - ответил Питер. - Именно в 
этом и было мое задание, но я не привел его вам в качестве подарка. Он 
принадлежит Кривому Мику, который его поймал.
    - Что за вздор! По какому такому праву конь принадлежит Кривому 
Мику?
    - По праву того, кто его поймал. Если бы не Кривой Мик, мы бы 
никогда его не поймали. Вы дали слово, что второе задание будет 
считаться выполненным, когда Фаерфакс будет приведен к вам оседланный, 
укрощенный и пригодный к езде. О том, что он будет принадлежать вам, 
ничего не говорилось.
    "Боже милостивый! - воскликнул про себя король. - Они так 
произносят это "слово короля", как будто мы не люди. Но ничего, третье 
задание его доконает, я отрублю ему голову и конь будет моим".
    А вслух он произнес:
    - Второе задание ты выполнил удовлетворительно, но на принцессе ты 
сможешь жениться только когда выполнишь третье и последнее задание, а 
если не выполнишь - я тебя обязательно казню, или, по крайней мере, 
это сделает Буньип, - добавил он самодовольно.
    - Э-э... да... конечно, - запинаясь, выдавил из себя Буньип, придя 
в замешательство. - Это запросто, да... да. Одной струи из правой 
ноздри и двух из левой ему должно хватить, - и Буньип подмигнул 
Питеру.
    - Ты должен выполнить третье задание, - продолжал король. - Ты 
должен принести мне золотую с алмазами корону, которую колдунья, 
подметающая Луну, забросила в озеро после того, как принцесса ее 
обидела.
    - Я только столкнула ее с подоконника, - оправдывалась принцесса. 
- Она царапалась в стекло и напугала меня.
    - Знаю, знаю, - раздраженно проговорил король. - Но она сорвала 
драгоценную корону с твоей головы, и прежде чем швырнуть ее в озеро, 
произнесла заклятье.
    - Какое заклятье? - спросил Питер. Лована смотрела на него с 
беспокойством.
    - А такое... Принцесса сможет выйти замуж, только если у нее на 
голове будет та самая корона с алмазами. - Вид у короля был печальный: 
корона стоила по меньшей мере полмиллиона долларов. Он вздохнул и, 
наклонившись к уху придворного, спросил: - А сколько бы она стоила 
сейчас?
    - Миллион долларов, - ответил тот.
    - Боже милостивый! - воскликнул король.
    Шатаясь, он сошел с трона и побрел в замок, оплакивая свою потерю.

        Глава 20
        ПОСЛЕДНЕЕ ЗАДАНИЕ

    Утром Питер и его друзья собрались под большим деревом обсудить, 
каким образом достать золотую корону со дна озера.
    Буньип предложил выпить всю воду из озера и затем выпустить ее в 
речку, но Кривой Мик заметил, что речка выйдет из берегов и затопит 
много земли. Поэтому от его предложения пришлось отказаться.
    Питер все время думал о колдунье, которую обидела принцесса. Он не 
сомневался, что это та самая колдунья, которая возила его на Луну, и 
стал ломать голову, как бы ее позвать. Если бы она показала точное 
место на озере, куда она бросила корону, они могли бы попросить 
великана Ярраха войти в воду и своей ручищей пошарить по дну.
    Питер поделился планом с Серой Шкуркой и Кривым Миком, и он им 
чрезвычайно понравился. Но как известить колдунью и великана?
    - Я на Фаерфаксе могу домчаться до избушки колдуньи и потом до 
замка великана, - предложил Кривой Мик. - Этой ночью конь передохнул и 
к утру должен быть свежим. Если он пойдет галопом, то к вечеру я уже 
снова буду здесь.
    - А вдруг не успеешь? - засомневался Питер. Он-то знал, сколько 
миль предстояло покрыть Кривому Мику.
    - Непременно успею. Если сам я к вечеру и не вернусь, то великан и 
колдунья придут точно. Метла колдуньи дает хорошую скорость, и она, 
наверно, прилетит первой, хотя и великан одним шагом перемахивает 
несколько миль. Оба они обязательно явятся. Но почему тебе так важно 
заполучить их именно к вечеру?
    - Хочу, чтобы они работали ночью, - ответил Питер. - Ведь днем 
великан всех перепугает, да и колдунье люди тоже не обрадуются.
    - Понятно. Значит, мне надо отправляться как можно скорее.
    Кривой Мик недолго седлал Фаерфакса. Он вскочил на коня, махнул на 
прощанье рукой и устремился по дороге. Фаерфакс шел широким галопом. 
Кривой Мик натягивал поводья, не давая коню опустить голову, Фаерфакс 
грыз удила. Они быстро скрылись из виду.
    Весь этот день Питер провел, ухаживая за Мунлайт. Буньип храпел 
под деревом. Ближе к вечеру, когда в ложбинах и под деревьями стали 
сгущаться тени. Серая Шкурка пошла в замок позвать Ловану. Питеру 
хотелось, чтобы она была с ним, когда придут ведьма и великан.
    Лована все еще боялась колдуньи. Ей не верилось, что та стала 
добродушной старушкой, какой обрисовал ее Питер.
    - Волшебный Лист изменил ее совершенно, - убеждал он ее.
    Но Лована не могла забыть, как та скреблась в окно, чтобы попасть 
в комнату, и сомнения продолжали мучить ее.
    Когда Серая Шкурка привела Ловану к дереву, она села рядом с 
Питером и спросила:
    - А что мне сказать колдунье, когда я ее увижу? Она ведь помнит, 
что я спихнула ее с подоконника.
    - Я не удивлюсь, даже если она тебя... поцелует. Она ведь уверена, 
что ты ее любишь. Вот что делает Волшебный Лист.
    Он взял ее руку в свою и не отпускал. И она позабыла о колдунье.
    - Мне нравится, когда ты держишь мою руку, - сказала она.
    - Мне тоже нравится держать ее. Ты выйдешь за меня, когда я 
выполню третье задание?
    - Да. Я буду всегда вместе с тобой и Кривым Миком, готовить вам 
завтраки и убирать дом. Я хочу делать все, что мне не разрешали делать 
здесь. Это так замечательно! - И Лована опустила голову на плечо 
Питеру.
    Буньипу стало противно.
    - Никогда такого не видывал, - бурчал он. - Буньипы никогда не 
держат девочек за руки. Так просто не принято. - Он резко запрокинул 
голову: - Что это?
    Полоска голубого пламени дугой прочертила небо. На мгновение она 
осветила призрачным светом деревья и нырнула вниз. Голубое пламя 
облетело вокруг дерева, сделало разворот, затем петлю и ударилось в 
землю рядом с Буньипом, который от испуга плюхнулся на спину.
    - Как поживаете? - спросила колдунья.
    - Мы очень рады тебя видеть, -- сказал Питер.
    - Вам повезло, что вы вообще меня видите, - ответила колдунья. - Я 
иногда увлекаюсь скоростью и, входя в атмосферу, имею массу 
неприятностей. А к вам я так торопилась, что при входе в плотные слои 
у меня образовался огненный хвост, как у кометы. Я немного опалилась, 
но ничего, пройдет.
    - Ты помнишь Ловану? - спросил Питер. - Принцессу, у которой ты 
украла корону?
    - Помню, помню, - сказала колдунья, подходя к принцессе и обнимая 
ее. - Мне всегда хотелось снова с тобой встретиться и извиниться за 
то, что я пыталась к тебе влезть. Да еще сорвала с тебя корону - это 
было ужасно! Кривой Мик сказал мне, что ты хочешь получить ее назад, и 
я тебе помогу. Я покажу тебе точное место, куда я ее зашвырнула.
    - Мы с Лованой собираемся пожениться, как только достанем со дна 
корону, - объяснил ей Питер.
    - Чудесно! - воскликнула колдунья. - Никогда не забуду, что я 
подумала, когда первый раз увидела тебя. А подумала я, какое из тебя 
получится прекрасное жаркое... - она вдруг остановилась и 
сконфузилась. - Ой-ой-ой! Что же это я говорю? Я хотела сказать, я 
подумали, какой замечательный муж получился бы для Прекрасной 
Принцессы.
    - А ты больше не сердишься на меня, что я столкнула тебя с 
подоконника? - спросила Лована.
    - Нет, дорогая, что ты! Мне не страшно, когда меня сталкивают с 
окон. У меня ведь всегда при себе метла, и я на ней улетаю. А теперь 
пойдем к озеру, и я покажу вам, где лежит корона. Я спешу, мне надо 
успеть на луну, пока она не зашла. Там опустился новый аппарат, и, 
может быть, я пополню свою коллекцию одной-двумя камерами. Пошли.
    И колдунья по петляющей тропинке повела всех к озеру, которое в 
лунном свете выглядело серебряным диском. Подойдя ближе, они увидели 
стаи уток, плавающих посредине озера.
    Колдунья только собиралась указать место рядом с одной из стай, 
как вдруг земля у них под ногами начала содрогаться, и они услышали 
глухие удары шагов великана, идущего по бушу. Это подходил великан 
Яррах. Было слышно, как трещат и ломаются деревья, которые ноги 
великана подминали, словно траву.
    Затем они увидели, как нога раздавила несколько деревьев не так 
далеко от них. Потом она поднялась, исчезла во тьме, и следующий шаг 
пришелся совсем рядом с ними.
    - Надо что-то предпринять, а то он на нас наступит, - сказала 
колдунья, глядя наверх во тьму, где огромная круглая тень заслонила 
луну. - Я поднимусь на метле и поговорю с ним.
    Она вскочила на метлу и рванулась во тьму. Секунду спустя где-то в 
вышине раздался ее сердитый голос:
    - Смотри, куда ступаешь. Ишь, вымахал, а ума-то с гулькин нос. Ты 
чуть меня не раздавил!
    - Прошу прощения, мадам, - извинился великан Яррах. - Я иду помочь 
своим друзьям.
    - Если тебе нужны Питер и Прекрасная Принцесса, то они у тебя под 
ногами.
    - Правда? Я хорошо его помню и рад, что он нашел свою принцессу, Я 
с ним поговорю и узнаю, что он от меня хочет.
    Великан оставил звездную вышину и наклонился. Его могучее тело 
показалось Питеру надвигающейся на небо безбрежной тенью. Питер вышел 
на открытое место, и пальцы великана охватили его со всех сторон. Его 
поднимали все выше и выше, и в конце концов он оказался на уровне лица 
великана, сидя на его ладони. Великан открыл рот, и Питер увидел 
громадный язык, который шевелился, как крыло дракона в пещере. Питер 
испугался и захотел снова очутиться на земле, рядом с Лованой. Но 
голос великана звучал мягко.
    - С того времени, как ты дал мне Волшебный Лист, - сказал он, - я 
сделался самым счастливым великаном на свете. Ты знаешь - скольким 
людям я помог с тех пор, как ты посетил мой замок.
    - На самом деле ты всегда был добрым, тебя только приучили быть 
злым. А Лист всего лишь сделал тебя тем, кем ты был с самого начала. Я 
знал, что ты мне поможешь.
    - Что я должен сделать?
    Колдунья, сидевшая на метле чуть выше Питера, не дала ему и рта 
раскрыть.
    - Дай я все ему расскажу, потому что я тоже хочу тебе помочь, - 
заявила она.
    - Судя по тому, как ты меня приветствовала, не похоже, что ты 
очень хочешь кому-то помочь, - тихо проговорил великан.
    - Иногда меня тянет на старое, - сконфузилась колдунья. - Забудь, 
что я сказала. Когда я была злой, я украла у принцессы корону и 
забросила в озеро. И еще произнесла заклятье, что без нее она не 
выйдет замуж. Так что видишь, они с Питером никогда не смогут 
пожениться, если мы не достанем корону со дна озера, а оно очень 
глубокое. Питер хочет, чтобы ты вошел в воду и достал корону - я 
покажу тебе точное место.
    - Это нетрудно, - сказал Великан, - когда начнем?
    - Сначала поставь меня на землю, - сказал Питер. Ему не хотелось 
путешествовать до середины озера на ладони, да если еще вдобавок 
именно она будет шарить под водой.
    {whisp14.gif}
    Великан Яррах нагнулся и опустил Питера на землю. Огромные пальцы, 
как решетка, окружили Питера, а когда он уже прочно стоял на земле, 
разомкнулись и освободили его.
    - Я достану корону для Лованы, - пообещал великан.
    Яррах стал выпрямляться, его огромная голова поднималась все выше 
и выше над деревьями, туда, где его уже ждала колдунья, чтобы указать 
точное место.
    - Видишь, - прямо посреди озера плавает стая уток, - сказала она. 
- Как черные точки на серебре.
    - Отлично вижу, - ответил великан.
    - Вот там и лежит корона. Прямо под ними. Они, конечно, взлетят, 
когда ты к ним приблизишься, но ты заметь место на воде и пощупай дно. 
Там в иле и лежит корона.
    - А какая там глубина? - спросил великан.
    - Тебе примерно до пояса, но я боюсь, что ты еще немного 
погрузишься в ил. Может быть, возьмешь что-нибудь вроде палки - ствол 
дерева, например?
    - Не нужна мне никакая палка!
    Великан Яррах сделал большой шаг, и его громадный ботинок ступил в 
воду. Он двигался вперед, пока вода не дошла ему до пояса. Стая уток с 
кряканьем перелетела на другое место.
    Он на мгновение замер, его силуэт резко обозначился на фоне 
залитой лунным светом воды. Потом он наклонился и опустил руку в воду. 
Но глубина была слишком большой, и дна он не достал. Тогда он сделал 
глубокий вдох и с головой ушел под воду. Минуту спустя он резко 
вырвался на поверхность и стал с такой силой отфыркиваться, что в 
воздух поднялись миллионы капель воды.
    Сначала Питеру даже показалось, что пошел дождь. Серая Шкурка 
бросила Питеру и Ловане плед, который она выхватила из своей сумки, и 
они в него закутались.
    - Ох! - воскликнула Серая Шкурка, глядя на великана. - 
Оказывается, это всего лишь отфыркивается наш большой друг. А сейчас 
он снова ушел под воду.
    На этот раз великан оставался под водой минуты две, а когда, 
наконец, вынырнул на поверхность, то сделал такой выдох, что все 
деревья вокруг озера закачались, словно сквозь них пронесся порыв 
ветра.
    Потом великан немного передвинулся и стал ощупывать дно в другом 
месте. На этот раз он нашел корону, выпрямился и поднял ее над 
головой. Корона сверкала, как скопление звезд. Выйдя на берег, он 
вручил ее Ловане, а она от волнения даже не смогла ее как следует 
надеть. Питер поправил на ней корону и отступил на шаг, чтобы 
посмотреть со стороны.
    Корона излучала собственный свет, и все пространство вокруг озера 
озарялось ее лучами. Но красивое лицо Лованы словно освещалось и 
внутренним огнем. А Волшебный Лист, спрятанный в медальоне, придавал 
ей чувство собственного достоинства и величественную осанку. Она 
больше не была маленькой запуганной девочкой. Она стала принцессой 
Лованой.
    Даже деревья, и те притихли и больше не шелестели листьями, как 
будто любое движение могло помешать им любоваться принцессой.
    В ярком свете короны изменился и Питер. Он вырос на два дюйма, 
раздался в плечах и превратился в статного юношу. Обернувшись к 
великану, он сказал:
    - От имени Лованы и от меня позволь поблагодарить тебя, Яррах! 
Скоро мы навестим тебя - я хочу показать Ловане твою кухню.
    - Вы всегда желанные гости в моем доме, - ответил великан. - 
Выберите только день, когда в кухне будет хорошая погода. А теперь 
давайте прощаться. Мне предстоит до рассвета проделать большой путь.
    Вскоре он зашагал прочь, и было слышно, как под его ногами трещат 
деревья. Когда же он переступил через ближнюю гряду гор, звук его 
шагов постепенно затих.
    Засобиралась и колдунья. Ей не терпелось до рассвета подмести чуть 
ли не пол-Луны.
    - Слушай, Питер! - обратилась она к нему. - Когда у вас с Лованой 
свадьба?
    - Послезавтра, - ответил Питер. - Мы надеемся, что ты почтишь нас 
своим присутствием.
    - Конечно! Но вот что я хочу предложить: если вы еще не наняли 
фотографа, я хотела бы быть вашим официальным фотографом... если вы, 
конечно, не возражаете. У меня лучшая в мире коллекция фотоаппаратов. 
Правда, некоторые из них годятся лишь на то, чтобы снимать звезды и 
тому подобную чепуху. Но я могла бы сгонять на Луну и сделать оттуда 
дистанционный снимок. Представляете: вы с Лованой идете по подъемному 
мосту! Это была бы самая оригинальная фотография в мире!
    - Мы с удовольствием назначим тебя нашим официальным фотографом, - 
сказала Лована, - и я буду рада иметь фотоснимок, сделанный с Луны.
    - Что ж, договорились! - обрадовалась колдунья. - И, оставив 
позади себя голубую полоску пламени, стартовала в сторону Луны.
    - Такой скоростной метлы я еще не видел, - проворчал Буньип, 
сладко спавший на берегу озера. - Настоящая спортивная модель, 
преемистость у нее просто страшная.
    Тут он огляделся и понял, что слушать его некому: Питер и Лована 
уже шли по дороге к замку. Бурча себе что-то под нос, Буньип поплелся 
следом.
    Когда они подходили к большому дереву, их догнал Кривой Мик. 
Фаерфакс был весь в мыле, но голову держал высоко.
    Кривому Мику не терпелось узнать, как дела с великаном и 
колдуньей. И пока он чистил коня, Питер рассказал ему обо всем, что 
случилось, и о том, как они достали корону.
    - Теперь ты выполнил все задания, - сказал Кривой Мик. - Осталось 
только сыграть свадьбу, а перед этим мне надо хорошенько выспаться.
    Он отвел Фаерфакса в конюшню, завернулся в одеяло и лег прямо под 
звездами, не обращая внимания на храп Буньипа. Питер проводил Ловану в 
ее комнату, тоже забрался в спальный мешок, который ему дала Серая 
Шкурка, и вскоре заснул.

        Глава 21
        ПИТЕР ЖЕНИТСЯ НА ПРЕКРАСНОЙ ПРИНЦЕССЕ

    Настал день свадьбы Питера и Лованы, и все жители окрестных 
селений шли или ехали к замку, разукрашенному флагами. Флаги были алые 
и золотые, голубые и желтые, и все они неистово бились на ветру, 
словно пели принцессе Ловане песнь радости.
    Лована в своей комнате готовилась к свадьбе. Вокруг нее суетились 
фрейлины, то затягивая ей платье в одном месте, то отпуская в другом. 
Само же платье, сшитое из наилучшего шелка цвета слоновой кости, 
сверкало алмазами, его украшал и золотой с драгоценными камнями 
медальон, в котором хранился Волшебный Лист. Лицо Лованы было 
настолько прекрасно, что все, кто ее видел, испытывали благоговейный 
восторг.
    Лована не была уже той грустной принцессой, которая дрожала при 
одной только мысли об экзаменах или о том, что король с королевой 
могут ее отругать. Теперь она была счастливой и свободной и выходила 
замуж за человека, которого любила. Не придется ей больше выслушивать 
нотации королевы, не придется подчиняться капризам короля-себялюбца...
    А король в это время сидел в кабинете и занимался арифметикой. 
Сначала он подсчитал общую сумму своего состояния. Оно достигало 
сорока миллионов долларов. Затем дошла очередь до страшного действия - 
вычитания. Ему надо было вычесть стоимость Фаерфакса и золотой с 
алмазами короны. Он давно считал то и другое своей собственностью, и 
мысль о том, что они потеряны для него навсегда, приводила его в 
бешенство.
    За минувшее время король постепенно свыкся с тем, что корона 
навеки погребена на дне озера, поэтому, когда Питер принес ее, он 
просто обомлел. Очищенная от грязи и натертая Лованой, корона с 
алмазами сверкала так ярко, что королю даже пришлось надеть солнечные 
очки, чтобы защититься от блеска. Ему нестерпимо хотелось оставить ее 
себе, и он всячески тянул время, пытаясь придумать Питеру еще какое-
нибудь задание, которое тот ни за что не выполнит.
    Но Питер стоял на своем:
    - Король дал слово, что для женитьбы на принцессе мне нужно 
выполнить всего три задания.
    Король в ярости стукнул кулаком по столу.
    - Проклятье! - вскричал он. - Неужели нельзя относиться к моим 
словам как к словам обычного дельца, а короля оставить в покое?
    - Конечно, нельзя, - ответил Питер.
    - О боже! - воскликнул король. - Ты один стоил мне дороже, чем все 
другие претенденты!
    - Еще бы! - усмехнулся Питер. - Всех остальных вы убивали, а их 
деньги присваивали. Буньип мне рассказал.
    - Этот Буньип - известный лгун и мошенник, - злобно прошипел 
король. - Я его изничтожу.
    - Вряд ли вы это сделаете, - ответил Питер. - Ведь он теперь наш 
друг.
    Больше королю сказать было нечего. Он в последний раз бросил 
завистливый взгляд на корону и пошел сообщить все королеве.
    Она же все это время смотрела в замочную скважину и восхищалась 
Питером. Он выглядел так величественно, что у нее исчезли всякие 
сомнения относительно его хорошего воспитания и того, какое 
впечатление он произведет на соседей. Когда король открыл дверь, она 
быстро отскочила в сторону и бросилась к Питеру, протянув к нему руки 
для объятья.
    - Дорогой мой! - сказала она, - ты выглядишь просто очаровательно. 
А теперь - извини меня - мне надо увидеть любимую дочь и проверить, 
что за платье она собирается надеть. - И королева упорхнула прежде, 
чем король успел отчитать ее за подглядывание.
    Таким образом Питер и Лована получили согласие ее родителей на 
брак и стали быстро готовить свадьбу.
    В день бракосочетания Питер почистил Мунлайт и подготовил ее к 
обратному путешествию. Для себя Лована выбрала в королевской конюшне 
самую красивую кобылку. Она была такой же белоснежной, как Мунлайт, и 
величиной с нее же. Седло сделали из тончайшей кожи и украсили 
драгоценными камнями.
    Кривой Мик собрался ехать на Фаерфаксе. Он надел на него простое 
пастушье седло, заявив, что на всякие "новомодные" седла у него просто 
нет времени.
    Три лошади ждали под большим деревом. К морде каждой была 
подвязана торба с сечкой, и они спокойно ели.
    Люди переходили подъемный мост и стекались во двор замка, где 
должно было происходить бракосочетание. Даже самый большой зал не смог 
бы вместить всех желающих. В основном это были крестьяне, жившие на 
обширных землях короля; они были бедны и поэтому одеты просто. Однако 
в толпе попадались и рыцари и принцы из других королевств, 
прослышавшие, что нашелся наконец такой претендент, который выполнил 
все задания отца принцессы. Въезжая во двор на роскошных лошадях и со 
знаменами в руках, они завидовали Питеру.
    Никто из них, однако, не подходил к Буньипу, который исполнял роль 
церемониймейстера и командовал гостям: "Пожалуйста, не толкайтесь. Не 
спешите. Не стоит торопиться. Сюда, пожалуйста". Ему хотелось 
вспомнить свое искусство и сбить этих зазнаек с коней струями из двух 
ноздрей сразу, но с тех пор, как Питер дал ему Волшебный Лист, он не 
мог больше причинять людям зло.
    В одном конце двора соорудили помост, на который лицом к алтарю 
поставили несколько тронов. Здесь же стояла группа епископов в длинных 
белых одеждах, с митрами на головах. Они говорили о бедности в их 
приходах, и о скудных сборах, которые они получали по воскресеньям.
    По бокам помоста стояли трубачи, и, когда все собрались, они с 
такой силой задули в трубы, что с некоторых гостей чуть не слетели 
шляпы.
    После того, как трубачи протрубили второй раз, Буньип проорал:
    - Король и королева!
    Один из рыцарей, который в свое время еле унес ноги после того, 
как Буньип сбил его наземь, от страха свалился с лошади.
    Король и королева вышли из двери, расположенной за помостом, и 
сели на два лучших трона. Собравшиеся приветствовали их, хотя и без 
особого энтузиазма; в ответ король поднял руки над головой, с 
достоинством сцепив их в рукопожатье.
    Затем Буньип проорал второй раз:
    - Принцесса Лована и принц Питер!
    На этот раз толпа ревела во всю мочь, пока они усаживались на два 
трона поскромнее.
    - Серая Шкурка и Кривой Мик! - проревел опять Буньип, и вошли 
друзья Питера. У Серой Шкурки за ухом был цветок, и выглядела она 
вполне благовоспитанной. Кривой Мик был в штанах наездника и со 
шпорами, и хотя благовоспитанным его назвать было нельзя, зато 
выглядел он величайшим наездником мира. Они сели на два расшатанных 
трона, давным-давно валявшихся среди разной рухляди. В этом ни Питер, 
ни Лована не были виноваты, потому что свадьбу организовывал и троны 
готовил король.
    Буньип уже собирался объявить торжество открытым, когда в воздухе 
что-то сверкнуло и прямо перед помостом неуклюже приземлилась 
колдунья. Волосы у нее были опалены. Со всех сторон на ней висело 
столько фотоаппаратов, что метла, наверно, с трудом ее подняла. Она 
уже сделала дистанционный снимок с Луны, а теперь желала снять ближним 
планом.
    Водрузив перед помостом на треноги несколько фотоаппаратов, 
колдунья стала с невероятной быстротой делать снимки со вспышкой. 
Лампы-вспышки у нее были какие-то громоздкие, и каждый раз, когда она 
нажимала на спуск, издавали громкий хлопок. Своей яркостью они 
затмевали солнце. Многие гости жаловались, что после такой вспышки они 
по пять минут ничего не видят. Это значило, что они вообще ничего не 
видели, поскольку колдунья щелкала не переставая.
    {whisp15.gif}
    Епископы не обращали на нее ни малейшего внимания. Они воздевали 
руки к небу и ходили туда-сюда, вперед-назад, в то время как Питер и 
Лована стояли перед алтарем. Каждый из епископов хотел самолично 
скрепить их брак, поскольку такая возможность предоставлялась нечасто. 
Они оказывались друг у друга на пути, сталкивались, и, как утверждали 
потом некоторые длинные языки, обвенчали Питера и Ловану два раза.
    Когда церемония подошла к концу, Питер взял слово. Он сказал:
    - Я хочу сделать подарок каждому из присутствующих здесь. 
Пожалуйста, подходите все к помосту. Но сначала я хочу преподнести 
подарки королю и королеве.
    Он подошел к подножью тронов и вручил родителям принцессы по 
Волшебному Листу. Сначала королева с презрением его отвергла, сердито 
проворчав: "Ты все превращаешь в балаган", но Питер буквально вложил 
Лист в ее руку, и она мгновенно изменила тон: "О! Как ты добр! - 
воскликнула она. - Спасибо тебе огромное!"
    Она улыбалась и рассыпала всем воздушные поцелуи, и стала просто 
очаровательной. На короля, однако, Лист подействовал иначе. Он по-
настоящему страдал. Всю жизнь он любил копить деньги. Теперь же ему 
хотелось помогать людям, отдать им часть своих богатств - ведь он 
чувствовал, что они его любят. И пока он сидел, скривившись от муки, 
Питер и Лована быстро раздавали Листы, так что вскоре лица у всех 
светились счастьем.
    У края помоста стояло несколько женщин с младенцами. Король 
неожиданно сошел с трона и перецеловал младенцев одного за другим. 
Матери были безмерно счастливы, хотя самим детям это не очень-то 
понравилось.
    Затем король вновь поднялся на помост и призвал стражу:
    - Возьмите эти ключи и откройте мою кладовую. Принесите два 
сундука с золотом. Я хочу подарить каждому из присутствующих по 
пятьсот долларов.
    Стражники в величайшем возбуждении убежали, а король плюхнулся на 
трон и сказал сам себе:
    - Какой комар меня укусил? Я, наверно, сошел с ума!
    Когда стражники принесли сундуки, король открыл их ключом, 
спрятанным у него в кармане, и раздал всем по пятьсот долларов.
    Бедняки обомлели. Некоторые плакали от счастья, поскольку им 
больше не надо было беспокоиться о ценах на фрукты и овощи. У них были 
деньги купить все необходимое.
    Король с каждой минутой чувствовал себя все более счастливым. 
Теперь он обратился к поварам:
    - Идите на кухню и принесите сюда каждому по куску мяса для 
воскресного обеда.
    Даже королева посчитала, что он зашел слишком далеко.
    - Хватит и по полкуска, - прошептала она, прикрыв рот рукой.
    - Нет, по два куска каждому! - крикнул король, не обращая внимания 
на слова королевы. Повара побежали на кухню и скоро стали раздавать 
мясо.
    Лована некоторое время смотрела, как они это делают, а потом 
наклонилась и прошептала Серой Шкурке:
    - Не могла бы ты пойти и принести два фунта сосисок? Я хочу их 
поджарить Питеру и Кривому Мику на ужин.
    - Не беспокойся, - ответила Серая Шкурка. - Нам можно не думать о 
еде. Из своей сумки я могу достать все, что тебе надо.
    - Ах, да! - восторженно воскликнула Лована. - Я и забыла.
    - Нам пора, - сказал Питер, по очереди пожимая всем руки.
    Когда гости увидели, что они уезжают, все проводили их до того 
места, где стояли привязанные лошади. Сюда пришли все - король и 
королева, Буньип, нянюшки, ухаживавшие за принцессой. Лована их всех 
поцеловала. Теперь, когда она уезжала, король с королевой загрустили.
    - Я буду навещать вас раз в месяц, - пообещала она.
    Питер подсадил ее на спину белой кобылки и вскочил на Мунлайт. 
Кривой Мик верхом на храпящем Фаерфаксе повел их через буш. Фаерфакс 
задрал голову и изогнул шею - горячий рыжий жеребец, он шел впереди 
двух белых как снег кобылиц, на которых ехали два прекрасных человека. 
Позади всех скакала Серая Шкурка.
    И вот в конце концов они приехали к избушке Кривого Мика в буше. 
Коровы приветливо мычали, козы блеяли, куры кудахтали, а собаки 
носились и лаяли. Такой счастливой Лована еще никогда не была.
    Она готовила еду, убирала избушку и вместе с Питером носилась по 
бушу. Обе лошадки летали быстрее ветра, и длинные волосы Лованы 
золотым потоком струились за ней.
    Так и жили они вместе с Кривым Миком и Серой Шкуркой и были самыми 
счастливыми на земле.
__________________________________________________________________

    Маршалл А. Шепот на ветру: Сказочная повесть / Оформл. худож. 
Стрельниковой Е. Б. - Л., 1991. - 192 с.

    Романтическая сказка о юноше, отправившемся на поиски Прекрасной 
Принцессы, построена на оригинальном австралийском фольклоре.
    В аллегорической форме автор проповедует высокую гуманистическую 
идею: только добро и любовь делают человека по-настоящему счастливым, 
помогают преодолеть любые трудности в жизни.

    Издание осуществлено за счет средств переводчика.
    (C) Перевод Слобожана А.В., 1991
__________________________________________________________________

    Алан Маршалл
    ШЕПОТ НА ВЕТРУ
    Сказочная повесть

    Редактор Е. В. Стукалин
    Художественный редактор В. И. Круговов
    Технический редактор Е. В. Траскевич
    Корректоры Т. В. Мережина, Л. Л. Бубнова

    Сдано в набор 29.10.90. Подписано к печати 17.12.90.
    Формат 70х108 1/32. Бумага типографская № 1.
    Гарн. обыкн.-нов. Печать высокая. Печ. л. 6.0.
    Тираж 50 000. Заказ № 915/5. Цена 3 руб.

    Ордена Трудового Красного Знамени издательство "Художественная 
литература", Ленинградское отделение. 191186, Ленинград, Невский пр., 
28; ЛИО "Редактор", 190008, Ленинград, кан. Грибоедова, 170.

    Ордена Трудового Краевого Знамени Первая типография издательства 
"Наука", 199034, Ленинград, В-34, 9 линия, 12
__________________________________________________________________

    OCR dauphin@ukr.net