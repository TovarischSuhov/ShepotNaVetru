Глава 13
        В ЗАМКЕ

    На следующее утро Питер и Серая Шкурка уже сидели под большим 
деревом, когда появился Буньип. Светило солнце, и растущая около рва 
ива, склонившись, опустила в воду свои длинные пальцы. Из буша 
раздавался смех кукабарры, а в долине за дорогой заливалась какая-то 
птица.
    Буньип сел на землю рядом с Питером.
    - Я всю ночь не спал, - пожаловался он. - Все размышлял. В замке 
полно солдат. Это вообще очень оживленное место - там и кухарки, и 
камеристки, фрейлины, и придворные. Только настоящий храбрец решится 
пробивать себе путь туда, где заточена Прекрасная Принцесса. Может 
быть, лучше это отложить на денек-другой? Меня терзает мысль, что кто-
нибудь из вас может погибнуть.
    - Кривой Мик наказал мне никогда ничего не откладывать, - ответил 
Питер. - Через несколько дней будет то же самое. Я хочу войти в замок 
сегодня. Не волнуйся. Я взял с собой Громобой, - и Питер показал 
Буньипу свернутый кольцом кнут, который лежал рядом с ним.
    - А я быстро бегаю! - ответила Серая Шкурка. - Этим мастиффам ни 
за что меня не поймать.
    Тут раздался скрежет ржавого блока, и гигантский подъемный мост 
медленно пришел в движение.
    - Сейчас откроют ворота, - прошептал Буньип. - Быстро прячьтесь за 
дерево!
    Огромные цепи, на которых удерживался мост, задрожали, приняв его 
вес, и тяжелая махина стала потихоньку опускаться, пока не коснулась 
берега рва неподалеку от большого дерева. Затем раздался звук 
отодвигаемых засовов, и ворота распахнулись.
    Из ворот вышли два солдата с горнами. Они были в коротких кожаных 
куртках, сбоку висели мечи. Ступив на мост, они поднесли к губам горны 
и затрубили с такой силой, что все кукабарры разом смолкли, а 
плававшие около ивы малые поганки нырнули под воду, оставив на 
поверхности только расходящиеся круги.
    - Я пойду сейчас, - быстро проговорил Буньип. - А вы пойдете, 
когда я дам вам знак.
    Он тяжело заковылял к мосту. Казалось, от каждого его шага дрожит 
земля, - такой он был огромный и массивный.
    - Интересно, выдержит его мост? - полюбопытствовала Серая Шкурка.
    Мост действительно содрогнулся, когда на него ступил Буньип, но 
сделан он был добротно, и к тому же Буньип уже сотни раз по нему 
ходил. Когда он подошел ко входу в замок, солдаты велели ему ждать и 
пошли известить короля. Как только они исчезли, Буньип махнул хвостом. 
Питер и Серая Шкурка перебежали мост и встали рядом. В руке Питер 
крепко сжимал Громобой.
    - А теперь в ворота - живо! - скомандовал Буньип. - Потом по 
коридору налево. По нему ходят реже, чем по другим, и оттуда вы 
найдете путь к комнате принцессы.
    Питер и Серая Шкурка бросились в ворота и повернули по галерее 
налево. Их никто не заметил, хотя у дальней стены большого помещения, 
начинавшегося сразу за воротами, Питер увидел группу солдат. Солдаты 
через маленькую дверь выходили во внутренний двор. Они смеялись и 
дурачились и не заметили двух посторонних.
    Из галереи они вошли в коридор, по которому сновало множество 
людей. Одни несли корзины с овощами и фруктами, другие - клетки с 
овцами для дворцовой кухни. Очевидно, это был день, когда местные 
фермеры, садоводы и мясники приносили свои продукты в замок на 
продажу.
    - Будь у меня корзина фруктов, - сказал Питер, - я бы мог сойти за 
садовода.
    - Только не в этом наряде, - поправила его Серая Шкурка. - Но мы 
можем его поменять. - Она вытащила из сумки какую-то ношеную-
переношеную робу из дерюги, всю латаную и перелатаную. Накинула ее на 
Питера, и он оказался укрытым ею полностью. Снова засунув лапу в 
сумку, Серая Шкурка достала старую шапку, которую Питер тут же надел. 
Свою шляпу он отдал кенгуру, которая бросила ее в сумку, где та и 
исчезла.
    - Я возвращу ее тебе позже, - пообещала она. - А теперь нам 
предстоит достать корзину фруктов.
    Она несколько секунд шарила в своей сумке и, вытащив оттуда мешок 
апельсинов, нахмурилась.
    - Я хотела разных фруктов, а получились одни апельсины.
    - И хорошо, - возразил Питер. - Я выращиваю одни апельсины. Идем 
же! Давай к кому-нибудь пристроимся!
    Они дождались, когда мимо проходил старик, толкавший перед собой 
тележку с картофелем. Питер ступил в коридор и тоже навалился на 
тележку. За ним с мешком апельсинов шла Серая Шкурка.
    - Спасибо, друг, - поблагодарил старик. - В наши дни мало кто 
помогает друг другу.
    Они уже очутились во внутреннем дворике и увидели марширующих 
солдат.
    - Теперь недалеко, - сказал старик. Сквозь широкую дверь он провел 
их в хозяйственное помещение, где был устроен базар. Это был огромный 
зал с рядами грубых лавок и массивных столов, на которые продавцы 
выкладывали свои товары.
    На один из таких столов Серая Шкурка положила мешок с апельсинами. 
Питер, попрощавшись со стариком, подошел и стал рядом.
    - Что дальше? - спросил он.
    Слова Питера услышал один придворный, который делал закупки для 
дворца, - толстяк со снисходительно-самонадеянным видом, - и подошел к 
нему.
    - Я тебе отвечу, что делать, - сказал он Питеру. - Ты можешь 
забрать своего кенгуру. Суп из кенгуриных хвостов на этой неделе мы не 
готовим. Понятно? А теперь убирайся.
    - А как насчет апельсинов? - запинаясь, спросил Питер. Он очень 
испугался, поняв, что кенгуру здесь посчитали выставленным на продажу.
    - А-а, апельсины, - толстяк взял один. - Это уже кое-что. Именно 
таких мы искали для Прекрасной Принцессы. - Он сжал апельсин. - 
Замечательный. Беру.
    - Можно, мы ей сами отнесем? - Питер решил испробовать такой 
способ добраться до принцессы.
    Придворный опешил, он даже покраснел от гнева.
    - Что такое ты мелешь? - вскричал он, оглядывая народ. - Вот 
человек, который посмел заявить, что хочет взглянуть на Прекрасную 
Принцессу. Наверное, он не тот, за кого себя выдает. Одет в лохмотья, 
а хочет нести апельсины принцессе! Позвать стражу! Сообщить королю! 
Отрубить ему голову!
    Он пришел в состояние неописуемой ярости и молотил руками воздух. 
Окружавшие в остолбенении смотрели на Питера. Они решили, что он сошел 
с ума. Но все-таки еще и побаивались.
    Серая Шкурка тоже испугалась. Она сообразила, что Питер сморозил 
глупость. Солдаты, маршировавшие во дворе, остановились и 
прислушивались к крикам в кухне, а один из них побежал сообщить 
королю.
    - Давай-ка сматываться отсюда, - быстро шепнула она Питеру, до 
которого только теперь дошло, что он натворил. - Прыгай ко мне на 
спину! Скорее! Не трать ни секунды!
    Питер вскочил ей на спину, одной рукой держась за ее мех, другой 
сжимая кнут. Серая Шкурка оттолкнулась и вылетела из комнаты, прежде 
чем ошарашенный придворный успел ее схватить.
    Она проскакала вдоль двора, причем с такой скоростью, какую еще 
никогда не развивала. Только опыт наездника помог Питеру не свалиться. 
Он крепко прижался к кенгуру, когда она резко повернула направо в 
коридор, наклонившись набок, как велосипедист во время виража. 
Промчавшись по коридору, она попала к узкой лестнице и запрыгала по 
ней вверх через четыре ступени.
    На первой лестничной площадке Серая Шкурка остановилась, чтобы 
отдышаться, а Питер тем временем сбросил робу, надетую поверх его 
роскошных одежд. Кенгуру вернула ему шляпу с перьями, и он снова стал 
принцем - и по одежде, и по манерам.
    Снизу до них доносились крики, топот ног и лай собак, рвущихся с 
поводка. Им надо было спешить, они понимали, что уже через несколько 
секунд по этой лестнице прогрохочут солдаты.
    - Туда, где заточена принцесса, должны вести несколько лестниц, - 
сказал Питер. - Солдаты и охрана все кинутся по ним, и это крыло замка 
они прочешут очень быстро. Нам лучше всего забраться как можно выше, 
чтобы оказаться над комнатой принцессы. Ты согласна?
    - По-моему, другого выхода нет, - согласилась Серая Шкурка. - 
Поскакали туда.
    Они преодолевали один пролет за другим, и в конце концов оказались 
в тесном помещении, которое явно использовали в качестве кладовки. 
Кругом громоздились горы оружия: лат, мечей, копий, тут же валялась 
сломанная прялка и множество стульев с отломанными ножками. Кругом 
лежала пыль, а на потолке висела грязная паутина.
    - Сюда уже давно никто не входил, - сказала Серая Шкурка. - 
Посмотри, сколько на полу пыли. Наши следы так и отпечатались.
    Питер подошел к окну. Выглянув, он увидел внизу знакомый ров, 
большое дерево на противоположной его стороне, спящего под кустом 
Буньипа и Мунлайт, пасущуюся на поляне неподалеку. Он даже узнал куст, 
в котором спрятал седло и уздечку.
    - Подойди-ка сюда. По-моему, окно Прекрасной Принцессы прямо под 
нами, на два этажа ниже.
    Серая Шкурка заглянула вниз, и прямо под собой увидела 
развевающуюся по ветру голубую занавеску.
    - Ну и везет же нам! - воскликнула она. - Это точно ее комната. 
Надо только придумать, как туда спуститься. Здесь мы еще в опасности. 
Скоро солдаты доберутся и до этих комнат, и нас схватят.
    Она засунула лапу в сумку, вытащила моток веревки и привязала ее 
конец к толстой балке, которая подпирала стену.
    - Зачем это? - спросил Питер.
    - Я сделаю в веревке петлю, ты в нее сядешь, я опущу тебя до окна 
принцессы, и ты в него влезешь. Мы знаем, что она добрая девушка и 
обязательно поможет тебе. Может быть, она спасет и меня.
    - Но мы будем вместе.
    - Нет, я останусь здесь.
    - Это невозможно. Я тебя не брошу. Разве ты не понимаешь, что 
скоро сюда вломятся солдаты? Мы с тобой друзья. Если тебя убьют, я не 
знаю, что сделаю. Мне в жизни не будет ни минуты счастья.
    - Это единственный выход, - настаивала Серая Шкурка. - Тебя я 
спустить могу, но самой мне спускаться нельзя. Ведь когда ты 
заберешься в окно, мне надо будет вытащить назад веревку, чтобы ее не 
заметили. Когда солдаты найдут меня здесь, они решат, что я просто 
животное, которое продавалось на суп. Они меня не тронут. Они ведь не 
подозревают, что кенгуру могут разговаривать, а я буду молчать. Если 
мы останемся здесь вдвоем, то вдвоем и погибнем. А если ты попадешь к 
принцессе, ты сможешь ей все рассказать, и она придумает, как нам 
помочь.
    Питер понимал, что Серая Шкурка права, хотя ему казалось 
преступлением бросать ее одну перед разъяренными солдатами. В конце 
концов он, однако, согласился, взобрался на подоконник и сел в петлю, 
завязанную Серой Шкуркой на конце веревки.
    - Держись крепче! - напутствовала его Серая Шкурка. - Оттолкнись 
от стены, чтобы не поцарапаться об нее, а теперь - с богом! - и она 
столкнула Питера с подоконника.
    Питер начал медленно опускаться. Это было страшно - висеть высоко 
над деревьями и рвом. Когда Питер достаточно опустился, его стало 
раскачивать из стороны в сторону, и он ободрался о каменную стену, но 
тут же уперся в нее рукой и стабилизировал спуск. Посмотрев наверх, он 
увидел голову Серой Шкурки, и это его успокоило.
    Питер видел, как внизу под ним полощется на ветру голубая 
занавеска. Когда он очутился прямо против окна, веревка замерла. 
Схватившись за подоконник, Питер подтянулся, забрался на него, 
освободил веревку, которая тотчас же исчезла наверху, и ступил внутрь.
    Серая Шкурка тем временем спрятала веревку в сумку. Заперев дверь 
комнаты, она укрылась за коробками, сваленными в углу. Хотя она и 
убеждала Питера, что ей ничто не угрожает, на самом же деле она очень 
боялась. Боялась собак, лай которых доносился с первого этажа, потому 
что здесь, в закрытом помещении, она не могла использовать свою 
скорость. Скорчившись за коробками, она представляла себя несущейся по 
бушу, и в этот момент топот солдат на лестнице вернул ее мысли к 
запертой комнате и недавним страхам.
    Солдаты забарабанили в дверь чем-то железным и потребовали, чтобы 
ее немедленно открыли. Не получив ответа, они высадили ее и ворвались 
в комнату, выставив вперед длинные копья. Два огромных мастиффа 
кинулись обнюхивать всю рухлядь, и хотя они изо всех сил пытались 
освободиться от поводков, солдаты держали их крепко. Собаки скребли 
пол здоровенными лапами, задыхались от лая, из пасти у них текла 
слюна... - но поводки сдерживали их.
    - Обыскать комнату! - громко скомандовал капитан, верзила с грубым 
небритым лицом. - Выше комнат нет. Он должен быть здесь. Перебросьте 
все барахло в один угол. Он прячется где-то тут.
    Солдаты быстро добрались до сложенных в углу коробок, отодвинули 
их в сторону и увидели Серую Шкурку. Она не говорила и ничем не 
показывала, что кенгуру могут разговаривать с теми, кто несет в себе 
дух буша. Она казалась обычной кенгуру, простым животным, которых люди 
отстреливают, чтобы приготовить из них консервы для собак, а также 
ради того, что они называют спортивным интересом.
    Солдаты схватили Серую Шкурку, набросили ей на шею веревку и 
выволокли на середину комнаты, где стоял разъяренный капитан.
    - Что это? - проревел он. - И где тот фермер, что ее привел?
    - Похоже, он убежал, - сказал один из солдат. - Здесь его нет.
    - Он привел кенгуру в замок, чтобы продать на суп, - пояснил 
другой солдат. - Они бежали вместе. Фермер, наверно, прячется где-то 
еще. Он, очевидно, бросил кенгуру, и зверь забился сюда.
    - Да, так все и было, точно, - согласился капитан. - Как бы то ни 
было, кенгуру мы отведем к королю и узнаем, что он скажет.
    Солдаты потащили Серую Шкурку вниз по лестнице и привели в 
королевские покои, где на украшенном алмазами троне восседал король. 
Вокруг него толпились рыцари и придворные, наперебой старавшиеся 
сказать королю что-нибудь приятное.
    - Где тот фермер, который привел в замок это животное? - вопросил 
король.
    - Мы не смогли его сыскать, Ваше Величество, - ответил капитан. - 
Наверно, он покинул замок вместе с другими фермерами.
    - Мы должны остерегаться принцев и рыцарей, - молвил король и 
резко спросил: - Но ты не думаешь, что это мог быть переодетый рыцарь, 
а?
    - Нет, что вы! - воскликнул капитан, - то был жалкий бродяга в 
залатанных обносках. Он ни за что не мог быть принцем. По-моему, он 
просто хотел посмотреть на Прекрасную Принцессу. Наверно, он очень 
беден, раз пытался продать эту бедную животину на суп.
    - Это так, - подвел итог король, окидывая Серую Шкурку хищным, 
оценивающим взглядом. - На кенгуру можно что-нибудь заработать? - 
поинтересовался он, выпрямляясь и поправляя корону. - За сколько его 
покупают на суп?
    - Очень дешево, Вате Величество, - ответит капитан. - Можно 
выручить больше, если пустить его на консервы для собак. Да и шкура 
чего-то стоит.
    - Чудесно, - обрадовался король. - Построить во дворе для него 
клетку и держать там, пока я не свяжусь с доверенным человеком из 
фирмы "Собачьи консервы".
    Он сделал запись в блокноте: "Продать кенгуру и на вырученные 
деньги купить еще несколько кенгуру. Предположительная прибыль через 
шесть месяцев - сорок долларов".
    {whisp09.gif}
    Эта цифра ему понравилась. Серая Шкурка, даже под мехом 
похолодевшая при упоминании о консервах для собак, все же поняла, что 
по крайней мере несколько дней ее не тронут.
    Ей было очень грустно. Она жаждала мчаться по бушу вместе с 
Питером и Мунлайт, но ей казалось, что те дни никогда не вернутся. Она 
засунула лапу в сумку и сжала Волшебный Лист. Это ее успокоило.
    - Убрать кенгуру отсюда, - распорядился король, закончив подсчеты. 
- Поставить клетку и бросить туда травы.
    Солдаты потащили Серую Шкурку во двор, где по приказу капитана они 
соорудили маленький загон рядом со стеной замка. Они огородили 
небольшой участок высокой металлической сеткой, через которую без 
хорошего разбега Серой Шкурке было не перепрыгнуть. А разбегаться было 
негде. Ни крыши, ни навеса, чтобы укрываться от дождя или солнца, не 
было, как не было и соломы для постели, так что лежать приходилось 
прямо на булыжнике.
    Один солдат кинул ей поесть немного сена, и все ушли.
    Серая Шкурка качалась на хвосте и размышляла, сможет ли Питер 
спасти Прекрасную Принцессу.
    "Надеюсь, они меня не забудут", - думала она.

