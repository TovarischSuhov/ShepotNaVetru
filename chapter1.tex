      Глава 1
        ПОИСКИ НАЧИНАЮТСЯ

    Когда-то, давным-давно, жили-были старик и маленький мальчик. 
Старик был такой скрюченный, что его прозвали Кривой Мик. Мальчика 
звали Питер. В хижине с трубой и двумя окнами, где они жили, день-
деньской не смолкал собачий лай. Дело в том, что заросли буша вокруг 
хижины были настолько густы, что собаке просто не хватало места 
повилять хвостом, а поскольку лаять без этого совершенно невозможно, 
она лаяла в доме.
    В хижине стояли два стула, стол и две кровати. Над открытым очагом 
свисал на цепях с потолка закопченный походный котелок. На деревянных 
гвоздях, вбитых в стену, были развешаны уздечки и седла. На стене 
висела картина, где была изображена брыкающаяся лошадь со всадником. 
Это и был Кривой Мик. На книжной полке стояли две книги: одна о 
лошадях, другая - о Прекрасных Принцессах. Между балок под крышей 
домика бегали опоссумы, а в норках под полом жили бандикуты.
    Кривой Мик был лучшим в мире наездником. Он мог стоять на лошади, 
когда она брыкалась, он мог лежать на лошади, когда она брыкалась, он 
мог сидеть на ней и пить кофе, когда она брыкалась. Он ездил на 
лошадях, которые взбрыкивали так высоко, что он мог шапкой поймать 
несколько звезд, он ездил на лошадях, которые взбрыкивали с такой 
скоростью, что даже сбрасывали с себя выжженное тавро. Он ездил на 
белых лошадях, он ездил на черных лошадях, он ездил на пегих лошадях, 
- но ни одна из них ни разу не сбросила его на землю. Кривой Мик и 
Питера научил ездить верхом.
    У Питера была белая лошадка, способная мчаться быстрее ветра. А 
красотой она напоминала птицу, и казалось даже, что в полете на ногах 
у нее вырастают крылья. Когда бури и грозы ломились через заросли 
буша, а летящие облака с грохотом скрещивали мечи молний, озаряя 
небосвод, Питер несся сквозь непогоду, перескакивая через каждую 
молнию раньше, чем она успевала погаснуть. При этом белая лошадка 
задирала голову и ржала навстречу ветру, а ее длинный хвост и лохматая 
грива развевались как знамена. Она мчала Питера так стремительно, что 
на него не успевала упасть ни единая капля дождя и так легко касалась 
копытами земли, что не приминала ни одного цветка, не сбивала ни одной 
капельки, которые бисером покрывали траву.
    Как-то раз их окликнул Южный Ветер, и голос его прогремел подобно 
грому: "Эй, мальчик на белом коне, что скачет через молнии! 
Остановись-ка! Я хочу поговорить с тобой". Он вышел из бури, 
закутанный в плащ из облаков, и сел на бревно. Это был старик с 
длинной седой бородой, в которой искрились дождевые капли, и молодыми 
глазами - яркими и блестящими, как у юноши.
    Подскакав к Южному Ветру, Питер натянул поводья, и лошадка встала 
на дыбы, изогнула шею.
    Ей, дочери ветра и солнца, хотелось мчаться наперегонки с бурей по 
горам и широким долинам, где она могла взять настоящий разбег. Но 
Питер потрепал ее по холке, приговаривая "постой, постой", и она 
остановилась.
    - А ты прекрасный наездник! - начал Южный Ветер. - Здорово же твоя 
лошадка прыгает через облака! Как ее звать?
    - Мунлайт, Лунное сияние.
    - Мунлайт! - повторил Южный Ветер. - Красивое имя, а главное - 
точное, ведь ее шерсть блестит, как луна. Не хочешь ли ты и Мунлайт 
поработать со мной? Я мою Землю и поддерживаю на ней чистоту: сдуваю 
мертвые листья и поливаю дождем. Если вы согласитесь, жизнь ваша будет 
прекрасной! Я познакомлю вас со своими братьями - Северным Ветром, 
Восточным Ветром и Западным Ветром, и вы будете помогать нам в наших 
странствиях по свету. Вы полетите впереди нас и поведете нас через 
горы и равнины. Мы привяжем к твоему седлу облака, и вы будете 
доставлять их в Мулгийские пустыни, в земли Спинифекса, которые лежат 
за Прибрежными горами. Вы принесете дождь в пустыни, где живут лишь 
песчаные бури, и в пустынях этих расцветут сады.
    - А найду ли я где-нибудь там Прекрасную Принцессу? - спросил 
Питер.
    - Прекрасную Принцессу? - рассмеялся Южный Ветер, да так, что от 
его смеха закачались деревья. - Нет. Я, например, уже тысячу лет не 
видел ни одной Прекрасной Принцессы. Это раньше я встречал их, когда 
они стояли у окон своих замков и расчесывали золотые волосы. Но 
сейчас... Нет, ты не найдешь Принцессы в тех краях, над которыми я 
пролетаю. А зачем она тебе?
    - Я хочу ее спасти.
    - Спасти от кого?
    - От дракона.
    - Ха! Так это, пожалуй, будет труднее всего. Драконы! Дай-ка 
подумать. Когда же я видел драконов в последний раз? Может быть, в 
Китае? Нет, не помню. Прекрасная Принцесса, которую стережет Дракон! 
Вот уж действительно задачка!
    - Кривой Мик сказал мне, что, если я поищу получше, я найду 
Прекрасную Принцессу. Еще он сказал, что всех Прекрасных Принцесс 
стерегут драконы.
    - Кривой Мик! - воскликнул Южный Ветер. - Это не тот ли, что 
вступил в борьбу с бураном и сломал ему шею?
    - Он самый. Он мне об этом рассказывал.
    - Тогда я его знаю, - обрадовался Южный Ветер. - Однажды я нес его 
пятьдесят миль на листе кровельного железа.
    - Да, он как раз чинил крышу овчарни, когда ты его сдул, - сказал 
Питер. - Он вспоминал, что эти пятьдесят миль промчался за десять 
минут.
    - О, то был настоящий ураган! - воскликнул Южный Ветер, потирая 
руки от удовольствия при одном воспоминании. - Мик привязал повозку с 
волами к красному эвкалипту, а я заставил ее трястись, как бумажный 
листок. Уж если он говорит, что ты найдешь Прекрасную Принцессу, 
значит, хоть одна Принцесса в Австралии да есть. Подожди здесь 
немного, я мигом облечу Землю с севера на юг и с запада на восток и 
скажу тебе, где я ее видел.
    Он поднялся и завернулся в свой громадный белый плащ. Раздался 
удар грома, пронесся мощный порыв ветра, и старик исчез. Ветви 
деревьев, мимо которых он промчался, хлестали по воздуху, а упавшие с 
них листья, покачиваясь, опускались на землю. Не успел Питер и глазом 
моргнуть, как старик уже был в облаках и погнал их на Север, крутя и 
вертя.
    Вернулся он минут через двадцать. Сначала в воздухе раздался 
легкий шелест, потом вихрем взметнулись сухие листья; и вот уже старик 
опять сидит на своем бревне.
    - Быстро же вы обернулись! - удивился Питер.
    - Да неужели? Я летел со скоростью ветра. Я пронесся над пустынями 
и долинами, где живут великаны, заглянул во все пещеры и каньоны, 
обыскал пляжи, где растут пальмы, но мне так и не удалось найти 
Прекрасную Принцессу. Потом я встретил своего брата, Северного Ветра, 
и загнал дым от раздутых им лесных пожаров прямо ему в глаза. Затем я 
вонзил в полыхающее пламя струи дождя, а самого братца отогнал в те 
земли, где рождаются ветры. Но нигде я не увидел ни одной Прекрасной 
Принцессы. Я сбросил колдунью с метлы, я заставил волосы в бороде 
великана щелкать, как кнуты, а потом полетел в землю аборигенов. Я 
осыпал брызгами водопада игравших на мелководье детишек, тронул за 
плечо их отца, который собирался пронзить рыбу острогой, и спросил 
его: "Где найти Прекрасную Принцессу, которую стережет Дракон?"
    "Принцессы жили в золотой век моего народа, - ответил он мне, - но 
потом все принцессы стали птицами, а драконы - ящерицами или змеями. 
Во всей Австралии осталась только одна принцесса. Она живет в 
заточении у своего жадного отца-короля, и ее стережет Буньип. Мне о 
ней поведало Солнце. Каждый день оно золотит ее волосы, и они начинают 
так ярко сиять, что в комнате тает темнота, а к окну подлетают птицы, 
надеясь согреться в холодные зимние дни. Она самая прекрасная из 
Прекрасных Принцесс со дня сотворения мира, и, когда она улыбается, 
страх покидает Землю, и все звери оставляют свои убежища и выходят к 
солнцу".
    "Как же я ее прозевал, - не понимаю", - удивился я.
    "Зато твои братья ее знают", - ответил рыбак.
    "А сможет ли найти ее маленький мальчик на белой лошадке?" - 
спросил я.
    "Да, сможет. Но только если он храбрый и добрый. Ему придется 
проехать по Долине Цепляющейся Травы к замку Великана, пересечь лес, 
где живет Бледная Колдунья, миновать Пустыню Одиночества и идти все 
вперед и вперед, пока не дойдет до Последнего Холма. Там растет 
эвкалипт, такой старый, что помнит начало мира. Мальчик должен сесть 
под него и сидеть до тех пор, пока луна не покажется из-за горизонта. 
А когда она целиком усядется на кромке земли, на ее фоне возникнет 
силуэт громадного замка. Тогда дерево заговорит и расскажет мальчику, 
как найти этот замок и Принцессу".
    Потом он что-то вспомнил и добавил: "Но оно расскажет это только 
доброму человеку!"
    Окончив рассказ, Южный Ветер поднялся с бревна и расправил белый 
плащ.
    - Итак, Питер, в путь! - воскликнул он. - Тебя ждет Прекрасная 
Принцесса! Не будь жадиной и эгоистом, думай сперва о других, а после 
- о себе, и ты найдешь ее, какие бы трудности и опасности ни 
встретились тебе. Мы с братьями попросим наших друзей не упускать тебя 
из виду и помогать во всем. А теперь на прощанье я хочу тебе кое-что 
подарить.
    Южный Ветер опустил руку в карман плаща и достал оттуда маленькую 
кожаную сумочку на серебряной цепочке. Склонившись, он повесил ее на 
шею Питеру.
    - В этой сумочке - Волшебный Лист, - сказал Южный Ветер. - Из всех 
подарков, которые я мог бы тебе преподнести, этот - самый ценный. И 
самый полезный для тебя. Он тебе очень поможет. Всякий раз, как ты 
повстречаешься с бедой, увидишь несчастных людей или, наоборот, людей, 
способных помешать тебе, способных убить и других, и тебя, отдай им 
этот Лист, и они изменятся. Расстаться с Листом не бойся. Как только 
ты его отдашь, в твоей сумке тут же появится новый.
    - Спасибо, - сказал Питер, - ты меня очень выручил.
    - А теперь мне пора, - заявил Южный Ветер. - Меня ждут три грозы, 
их нужно перенести в те долины, которые уже давно просят дождя. 
Прощай!
    Он умчался в небо, волоча за собой свое пышное одеяние.
    - Мистер Южный Ветер! - крикнул Питер вверх, в то место, где 
собирались облака. - А что означает этот Лист?
    Из крохотного окошечка ясного неба прогрохотало: "Он означает: 
тебя любят, ты нужен людям".