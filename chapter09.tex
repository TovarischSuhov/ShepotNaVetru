\chapter{БИТВА С КОТАМИ СОМНЕНИЙ}
\par\parЭту ночь Питер и Серая Шкурка провели под деревом на опушке 
Недреманного Леса и утром, едва солнце стало всходить, были на ногах. 
Свои спальные мешки они аккуратно свернули, и Серая Шкурка опустила их 
в сумку. Расположившись на берегу ручья позавтракать, Питер и Серая 
Шкурка наблюдали, как черные утки плыли вверх по течению, направляясь 
к тихим заводям, скрытым в кустах. Время от времени в воздухе 
раздавалось хлопанье крыльев - это пролетали чирок или свиязь.
\par- Вот бы здесь жить, - вздохнула Серая Шкурка.
\par- Да, я бы тоже не отказался, - согласился Питер.
\parВетер шевелил листья старого дерева, под которым они сидели, и 
Питеру показалось, что оно стало что-то нашептывать. Питер поднял 
голову, пытаясь определить, откуда идет звук. Он пристально смотрел на 
колышащиеся листья и ждал, когда шепот усилится.
\par- Послушай, - обратился он к Серой Шкурке, - по-моему, это дерево 
пытается что-то нам сообщить.
\parСерая Шкурка тоже посмотрела наверх и, замерев, стала ждать.
\parШепот усиливался.
\par"Недреманный Лес очень жесток, - шептала крона. - Каждое дерево, 
что растет здесь, подслушивает и подсматривает. Вам будет казаться, 
что за вами наблюдают тысячи глаз. Идите прямо по тропе и не сходите с 
нее, чтобы поглазеть на ручейки и водопады, которые вам встретятся. В 
центре низины на вас набросятся Коты Сомнений. Размером они с 
леопардов, а полосы на шкуре, как у тигров. Они будут выть и реветь и 
испугают вас. Но вы должны сразиться с ними и идти вперед, пока не 
дойдете до Последнего Холма".
\parШелест листвы прекратился, и шепот затих. Питер поднялся, похлопал 
дерево по стволу, затем, обернувшись к Серой Шкурке, сказал:
\par- Пора в путь. Через этот лес нам ехать долго, может быть, 
несколько дней.
\par- Я готова, - сказала Серая Шкурка. Она подождала, пока Питер 
вскочит на Мунлайт, и запрыгала следом. Переходя неглубокий ручей, 
Мунлайт подняла фонтан брызг и остановилась попить. Потом они 
выбрались на противоположный берег и пустились по узкой тропинке, 
которая и привела их в лес.
\parВокруг них недвижно высились огромные стройные стволы эвкалиптов, 
среди которых то тут, то там попадались акация и черное дерево. Лесные 
цветы уже начали пробиваться из-под прошлогодних листьев, вовсю цвел 
вереск. Воздух был настолько чудесен, что Питер сделал несколько 
глубоких вдохов, перевел Мунлайт на легкий галоп и ехал, раскачиваясь 
из стороны в сторону. Серая Шкурка прыгала следом, ее хвост то 
поднимался, то опускался подобно рукоятке водяного насоса.
\parОни двигались весь день, а когда начали опускаться сумерки, нашли 
старый эвкалипт с огромным дуплом, дно которого было устлано сухой 
корой, а поверху лепились сделанные из глины гнезда ласточек. В этом 
дупле Питер и Серая Шкурка провели ночь, пока утром их не разбудили 
солнечные лучи.
\parНа второй день идти стало намного труднее. Местами тропинка совсем 
заросла, а иногда они наталкивались на упавшее дерево, которое 
перегораживало дорогу. Такие преграды они легко преодолевали. Мунлайт 
разгонялась, взвивалась в воздух и перелетала через них, как на 
крыльях. Серая Шкурка прыгала следом, так что иногда в воздухе они 
оказывались бок о бок. Как-то раз в полете Питер протянул руку и 
дотронулся до кенгуру. Так они и приземлились вместе. Это было 
здорово!
\parИ все же некоторое время спустя они ощутили усталость. Питера, 
который обычно ни на что не жаловался, охватило уныние. Стала 
недовольно ворчать и Серая Шкурка.
\par- Похоже, нам ни в жизнь отсюда не выбраться, - захныкала она.
\par- И Принцессу нам не найти, - вторил ей Питер. - Мы все идем и 
идем, а замка все нет и нет.
\par- Я знаю, что надо делать! - заявила Серая Шкурка. - Надо 
поворачивать назад. Я устала от того, что все здесь подслушивает тебя, 
все подсматривает за тобой. Нам надо...
\parЕе прервал ужасающий, невообразимый вой. Он раздался из густых 
зарослей в низине, куда сбегала тропинка. Это был даже не вой, а почти 
вопль, и едва он стих, как ему ответил другой, затем - еще один, и вот 
уже вся долина огласилась таким кошачьим концертом, что Питер и Серая 
Шкурка остановились как вкопанные и в испуге посмотрели друг на друга.
\par- Наверное, это Коты Сомнений, - сказала Серая Шкурка. - Помнишь, 
что говорило дерево? Хорошо бы выпутаться отсюда без всяких сражений.
\par- Хорошо бы выпутаться хоть как-нибудь! - поправил ее Питер.
\parТропа, по которой они спускались, шла наискось по склону холма. 
Справа круто вверх уходил склон, весь покрытый адиантумом и мхом. 
Слева край тропы нависал над плотными колючими зарослями кизила. В 
конце склона дорогу пересекал ручей, с трудом продиравшийся сквозь 
кустарник. Противоположный склон был почти голым, и на нем стоял 
высокий, обожженный эвкалипт, раскинув на фоне неба растопыренные 
ветви.
\parМимо этого дерева пробежал один из Котов, подобный тигру. 
Огромными скачками он мчался к тропинке наперерез Питеру и Серой 
Шкурке. Следом за ним бежали другие Коты, а неистово раскачивающийся 
кустарник говорил о том, что сквозь него продираются новые и новые 
зверюги. Они яростно выли, и Питеру пришлось изо всех сил натянуть 
поводья, чтобы сдержать Мунлайт.
\par- Нас спасут только ноги, - сказала Серая Шкурка. - Вперед! От 
собак я уже бегала, но чтобы удирать от гигантских котов... Вот уж 
никогда не думала, что доживу до такого.
\par- Надеюсь, их когти не достанут до Мунлайт.
\par- А ты угости их Громобоем, - посоветовала кенгуру. - Разверни-ка 
кнут и повращай его. Не подпускай их к лошади. А если кто из котов 
прыгнет, постарайся, чтоб Громобой подпортил ему шкуру. Ну что, готов?
\par- Готов, - ответил Питер. В правой руке он держал кнут, в левой - 
поводья. Он привстал в седле, слегка наклонился вперед и пустился в 
галоп. Мунлайт неслась вниз по тропе с бешеной скоростью, ее уши были 
наставлены вперед, грива развевалась по ветру. Рядом скакала Серая 
Шкурка. При каждом скачке она взлетала так высоко, что равнялась с 
Питером. А опустившись, тут же отталкивалась от земли мощными лапами и 
через мгновение снова была в воздухе.
\parМунлайт бежала настолько уверенно, что ни разу не оступилась в ямы 
и рытвины, которые образовались в земле от дождей. Она то огибала их, 
то перепрыгивала.
\parОдин Кот, с полосами, как у обычной кошки, выскочил на тропу 
впереди них. Глухо рыча, он повернулся и прыгнул на Серую Шкурку, 
которая, однако, успела приготовиться к встрече: задними лапами она 
сильно ударила Кота в грудь да еще вдобавок разодрала ему шкуру своими 
длинными когтями.
\parОт кошачьего воя у всех заложило уши. Коты вылетали из кустов, 
словно полосатые снаряды. Кнут Питера щелкал без остановки, и каждый 
щелчок вырывал из кошачьей шкуры клок шерсти, который плавно опускался 
на землю. Получив удар кнутом, Коты валились, а когда приходили в 
чувство, Питер был далеко. Догнать его они уже были не в силах.
\parНикто из Котов так и не дотянулся до Питера, хотя одному удалось 
зацепить лапой седло и разорвать Питеру брюки. Питер огрел его по 
голове рукояткой кнута, в которую был залит свинец, и оглушенный Кот 
брякнулся на землю. Других Котов это, правда, не остановило, и они 
продолжали выскакивать из кустов. Несколько раз они царапнули когтями 
бока Мунлайт, и та, разъярившись, с такой силой лягала их подкованными 
копытами, что они еще в воздухе нагнали корчиться и вопить, а затем 
исчезали позади.
\parСерая Шкурка тем временем сцепилась с одним особенно свирепым 
животным, которому удалось избежать удара ее задних лап. Обхватив его 
передними лапами, она со страшной силой сдавила его и не отпускала до 
тех пор, пока тот не обмяк. Тогда она бросила его на землю и поскакала 
дальше.
\parПодлетев к ручью, который пересекал дорогу, Мунлайт и Серая Шкурка 
одновременно изо всех сил оттолкнулись от земли, чтобы разом 
перемахнуть через Котов, поджидавших их у берега. Коты тянулись вверх, 
загребая лапами воздух, но достать Питера и Серую Шкурку так и не 
смогли.
\parТеперь дорога пошла в гору, так что Мунлайт и Серой Шкурке 
пришлось напрячь все свои мощные мускулы, чтобы уйти от преследования. 
Справа и слева на них выскакивали новые и новые Коты, но наши друзья 
успевали промчаться мимо них прежде чем те прыгали.
\parСейчас все решала выносливость, а уж по этой части Мунлайт и Серая 
Шкурка могли потягаться с кем угодно. Наконец, вопли их 
преследователей затихли вдали, и Питер перевел Мунлайт на рысь, чтобы 
она могла немного отдохнуть.
\par- Больше они до нас не доберутся, - сказала Серая Шкурка. Теперь, 
когда опасность миновала, она очень гордилась собой.
\parПитер не ответил. Он слишком устал, ему хотелось только одного: 
поскорей найти место для ночной стоянки.
\parОни поднялись из низины, взобрались на гребень невысокого холма, с 
которого открывался вид на многие мили вокруг, и застыли. Перед ними 
высился Последний Холм, а на его вершине рос старый красный эвкалипт. 
Его искореженный ствол, казалось, согнулся под тяжестью своих же 
ветвей, которые переплетались на фоне неба, вытягивая из воздуха 
солнечное тепло и влагу.
\parПитер и Серая Шкурка поспешили к нему. Доскакав до зеленой 
травянистой лужайки, Питер спрыгнул на землю. Серая Шкурка догнала его 
и остановилась. Вот, наконец, оно перед ними, это старое как мир 
дерево, - так назвал его старик-абориген. Мощные корни дерева словно 
могучими пальцами стискивали землю. Ни один ураган не мог повалить 
его. Питер и Серая Шкурка чувствовали, что рядом с этим великаном в 
них самих как бы вливались небывалые силы. Любое дело теперь казалось 
им по плечу. Они найдут Прекрасную Принцессу, обязательно найдут.
\parОни уже успели проголодаться, и потому Серая Шкурка достала из 
сумки стул и стол и накрыла его как для принца. Они принялись за еду, 
и им показалось, что ничего более вкусного они никогда еще не едали. 
Потом Серая Шкурка убрала все обратно в сумку и достала два спальных 
мешка, которые они развернули изголовьем к дереву. Питер расседлал 
Мунлайт, снял с нее уздечку и пустил пастись на лужайке с высокой 
травой.
\parТак лежали они, ожидая восхода луны, ведь именно на ее фоне, как 
сказал Южный Ветер, Питер должен был увидеть замок Прекрасной 
Принцессы.
\parВот на востоке появился еле заметный отсвет. Из тьмы проступили 
силуэты неподвижных деревьев, молча ожидавших взрыва лунного света, 
что возвещало восход луны. Наконец, из-за горизонта показалась 
изогнутая полоска желтого света. Она росла, росла и в конце концов 
превратилась в сияющий диск, который лежал на кромке земли: темным 
силуэтом на нем проступили башни и укрепления огромного замка.
\parИ тут зашелестели листья старого эвкалипта, что-то нашептывая. 
Звуки сливались в один глухой, по мягкий голос. Дерево заговорило.
\par"Перед тобой замок Прекрасной Принцессы, Питер. Твой поиск 
подходит к концу. Впереди еще много трудностей, но ты доказал, что 
храбр и любишь людей. Волшебный Лист и в будущем защитит тебя и 
придаст сил справиться со всеми испытаниями. Он уже изменил жизнь тех, 
кто встретился тебе на пути; те, что были прежде жестокими и злыми, 
сейчас будут помогать путникам, которых встретят, а не убивать их.
\parЗавтра тебе предстоит трудный день, но замок уже недалеко, и путь 
к нему открыт. Ты сам найдешь способ встретиться с Прекрасной 
Принцессой. А теперь спи. Тебе ничто не причинит зла, пока ты спишь 
под моей кроной".
\parСверху склонилась темная ветка и листьями слегка прикоснулась к 
Питеру. Это был дружеский жест.
\parПитеру дерево понравилось.
\par- Ты слышала, что оно сказало? - шепотом спросил он у Серой 
Шкурки. - Оно коснулось меня своими листьями. Оно замечательное!
\par- Похоже, это славное существо, - ответила Серая Шкурка уже в 
полусне.
\parКогда они крепко уснули, тень от дерева полностью накрыла их, и в 
темноте одежда Питера изменилась. Теперь на нем был костюм принца, и 
выглядел он как настоящей принц.
