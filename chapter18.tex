Глава 18
\parФАЕРФАКС
\par\parНа следующее утро Кривой Мик, Питер и Серая Шкурка вновь поднялись 
на вершину холма и там, прячась за деревьями, разработали план поимки 
Фаерфакса. Долина была открытая, ровная. Речка, вдоль которой они шли, 
извивалась вдоль долины и исчезала в тесном ущелье. По берегам речки, 
повторяя каждый ее изгиб, стояла, словно охрана, стена деревьев.
\parТабун лошадей пасся в дальнем конце долины. Кривого Мика, однако, 
больше заинтересовала дорога, которая вела от табуна к началу ущелья. 
Очевидно ее проложили лошади, когда переходили из этой долины на 
другие пастбища. Она-то и надоумила Кривого Мика, как поймать 
Фаерфакса.
\parОн поделился своим планом с Питером и Серой Шкуркой, и скоро они 
отправились в обход к тому ущелью, где исчезала дорога. Кривой Мик 
намеревался погнать лошадей по этой дороге, которая была единственным 
известным им путем к спасению.
\parДостигнув начала ущелья, они вошли в него и шли до тех пор, пока 
оно не сузилось и не превратилось в узкую теснину с уходящими ввысь 
стенами. То тут, то там посреди дороги лежали огромные валуны, 
сорвавшиеся с отвесных стен каньона, а сама дорога была изрыта 
копытами лошадей, проходивших по ней то в одну, то в другую сторону.
\parЗдесь все остановились. Кривой Мик решил перегородить ущелье 
высоким забором, который должен был остановить лошадей, когда они 
помчатся прочь из долины. Речка в этом месте была совсем мелкой и 
перегородить ее не составляло труда.
\par- Прежде всего мы должны построить забор, - сказал Кривой Мик. - А 
для этого пусть Серая Шкурка достанет нам необходимые инструменты. 
Серая Шкурка оправдала надежды Кривого Мика. Ей льстило, что успех 
операции зависел и от нее. Она вытащила из сумки экскаватор для рытья 
ям, потом, улыбаясь, извлекла лом, два мотка колючей проволоки, 
лопату, бензопилу, бурав и краску для железа.
\par- Теперь у нас есть все, что необходимо, - сказал Кривой Мик, не 
замечая довольного вида Серой Шкурки. - Мы спилим вон те молодые 
деревья, что впереди. Ты, Питер, возьмешь пилы и приготовишь шесты. 
Ты, Серая Шкурка, выроешь шесть ям для столбов на дороге и в речке на 
мелководье, а я заберусь на скалу как можно выше и натяну от одной 
стены до другой колючую проволоку.
\parПроработав весь день, они плотно поужинали жареным мясом и 
улеглись в свои спальные мешки. Утром они вернулись немного назад по 
ущелью и построили еще один забор, с мощными, сделанными из бревен 
воротами. Ворота Кривой Мик оставил открытыми, чтобы лошади, убегая из 
долины, могли в них проскочить.
\par- Теперь, Питер, вы с Серой Шкуркой возвращайтесь к тому месту, 
откуда мы вошли в долину, и оттуда гоните лошадей вдоль дороги. Да, и 
поработай как следует своим кнутом, Питер, - пусть несутся в ущелье, - 
когда же попадут в загон, я закрою ворота, а потом заарканю Фаерфакса. 
Вперед!
\parДо начала долины было не очень далеко. Питер прижался к холке 
Мунлайт. и та обогнула долину, пройдя у самых гор. Серая Шкурка 
скакала рядом.
\parУвидев, что к нему галопом приближается всадник, Фаерфакс вскинул 
голову и фыркнул. Скакавшую рядом кенгуру он не опасался - кенгуру для 
лошади не страшны, по всадник, размахивающий кнутом, - совсем другое 
дело.
\parОн рысью пустился навстречу Питеру, высоко вскидывая ноги и тяжело 
ударяя копытами о землю. Ноздри его раздувались от гнева. Он выгнул 
шею, вскинул голову и грозно заржал, бросая вызов непрошенному гостю. 
Рыжая шерсть Фаерфакса золотилась на солнце, и Питер с восхищением 
разглядывал его. Расстояние между ними все сокращалось.
\parРаскрутив над головой кнут, Питер с такой силой щелкнул им, что 
ответное эхо громыхнуло как пушечный выстрел. Фаерфакс отпрянул назад, 
тут же развернулся и помчался к табуну, который от страха сбился в 
кучу. Он ворвался в ее середину и стал кусать лошадей за бока и холки, 
принуждая бежать. Пока они набирали скорость, он нетерпеливо крутился 
вокруг них, подгоняя, но когда они разогнались вовсю, он возглавил 
табун и повел его по дороге к ущелью. Его грива билась на ветру, как 
пламя, а длинный хвост стлался над землей.
\parПитер следил за Мунлайт. Хотя она без труда могла бы перегнать 
любую лошадь из табуна, она, словно понимая свою роль, ровно, без 
усилий бежала за ними.
\parКогда табун вошел в ущелье, Питер догнал задних лошадей и обрушил 
на них град ударов кнута. Лошади обезумели, задние стали наседать на 
передних, а те в свою очередь просто втолкнули Фаерфакса в загон. За 
ним потоком понеслись остальные. Кривой Мик выскочил из-за дерева, за 
которым прятался, и захлопнул ворота.
\parФаерфакс добежал до забора и остановился на всем скаку. Он пытался 
повернуться, но обезумевшие кобылы окружили его, не давая одним мощным 
прыжком перемахнуть через забор, который отделял его от свободы.
\parКривой Мик подождал, пока Фаерфакс пробьется назад к воротам, и 
набросил на него аркан. Конь-великан взревел от ярости. Он рванулся 
вверх, взмахнув передними копытами. Но Кривой Мик для упора намотал 
веревку на шест, так что Фаерфакс только туже затянул петлю на 
собственной шее. Полузадушенный, он рухнул на землю посреди ошалевших 
от страха кобыл, которые теперь шарахались от него во все стороны.
\par- Открывай ворота! - крикнул Кривой Мик.
\parПитер, верхом на Мунлайт сдавший у ворот, спрыгнул на землю, 
бросился к воротам и мгновенно их распахнул. Кобылы, увидев выход, 
кучей ринулись туда, тесня друг друга, и помчались назад к долине с ее 
просторами и травой.
\parВ загоне остались один Кривой Мик и лежащий у его ног жеребец. Мик 
быстро расслабил петлю на его шее и через голову надел ему недоуздок, 
а повод привязал к столбу в заборе. Фаерфакс с трудом встал на ноги и, 
почувствовав узду, стал биться, как рыба на крючке. Но повод держал 
крепко, и через некоторое время конь затих, дрожа всем телом. Повод 
оставался натянутым.
\parКривой Мик набросил ему на голову мешок, и, оказавшись в темноте, 
Фаерфакс перестал тянуть повод и стоял спокойно.
\parПодошли Питер и Серая Шкурка и вместе укрепили на его спине 
тяжелое седло и затянули пахву. Не снимая мешка, надели уздечку. Когда 
все было готово, Кривой Мик скомандовал:
\par- Садись на него, Питер!
\parИ Питер вскочил в седло, прежде чем Фаерфакс успел сообразить, что 
с ним произошло.
\parТогда Кривой Мик сорвал мешок с головы коня, и какую-то секунду 
конь-великан стоял не шелохнувшись. Потом он мощно оттолкнулся от 
земли и одним гигантским прыжком покрыл половину расстояния до ворот, 
но Питер словно прирос к седлу. Он улыбнулся, помахал в воздухе рукой 
и прокричал: "Но! Но!" Выгнув дугой спину и опустив морду между 
передних ног, Фаерфакс сделал второй мощный прыжок. Приземлился он на 
вытянутых, поставленных рядом ногах, и удар, сотрясший его при 
приземлении, перетряхнул все кости в теле Питера.
\parПонемногу Питер привык к прыжкам коня. Он чувствовал под собой это 
могучее тело, предвидел дальнейшие движения задних ног. Питер был 
уверен, что Фаерфакс попытается сбросить его, прыгая вверх и в 
сторону. Так и случилось. Фаерфакс подпрыгнул и затем, уже в воздухе, 
с такой быстротой лягнул задними ногами вбок, что Питер неминуемо 
оказался бы на земле, если бы не ожидал этого фокуса. Он изо всей силы 
сжал бока коня ногами и удержался.
\parПоняв свою неудачу, Фаерфакс прямо-таки завизжал от ярости. Он 
сорвался с места и, перейдя за несколько прыжков в карьер, с такой 
скоростью помчался по ущелью, что стволы деревьев вдоль дороги 
показались Питеру частоколом.
\parКривой Мик вскочил на Мунлайт, и они с Серой Шкуркой поспешили 
следом. Едва успели они добраться до долины, как увидели редчайшую 
картину. Фаерфакс задался целью во что бы то ни ехало сбросить Питера 
и продемонстрировал такой набор трюков, что Питер потом хвастался ими 
десять лет подряд. Фаерфакс взмывал в воздух выше деревьев, вертелся 
волчком, выгибал спину, угрожающе храпел, на что Питер в ответ издавал 
клич погонщика.
\parФаерфакс пытался ободрать Питера о стволы деревьев, но Питер 
рывком высвободил ноги из стремени и подобрал их, так что конь только 
поранил сам себя. Тогда Фаерфакс вернулся на открытое пространство, 
где он мог без помех прыгать и брыкаться. Но он уже стал уставать, его 
гордая голова начала клониться к земле. Оставался еще один прием - 
встать на дыбы и упасть на спину, чтобы раздавить Питера своим весом. 
Это была его последняя, отчаянная попытка избежать поражения.
\parНо Питер предвидел и ее и был наготове, когда огромный конь встал 
на дыбы. На несколько мгновений Питер прилип к спине Фаерфакса, пока 
тот, раскачиваясь на задних ногах, бил передними воздух и ревел широко 
открытым ртом, из которого падала пена. Вот конь на мгновение замер, 
стройный, как дерево, и тут же рухнул спиной назад. Питер, заранее 
освободивший ноги от стремян, успел оттолкнуться и соскочить с седла. 
Он мягко приземлился рядом с Фаерфаксом, держа поводья в руке. Когда 
же конь с трудом поднялся, Питер снова вскочил в седло, готовый к 
следующему трюку.
\parНо сил у Фаерфакса больше не осталось: Питер чувствовал, как тот 
дрожит под седлом. Он осадил его, ласково потрепал по взмыленной шее и 
мягко заговорил. Потом он шагом провел коня по всей долине, пока все 
его опасения не рассеялись, и лишь тогда остановился рядом с Кривым 
Миком и сказал:
\par- Что ж, я покорил его, но я раскаиваюсь в этом. Нельзя укрощать 
такого великолепного коня.
\par- Я знаю, - согласился Кривой Мик. - Но ты не забывай, что сейчас, 
когда он сломлен, он к тебе привяжется. И не будет больше стремиться к 
свободе.
\par- Я рад этому, - ответил Питер.
\parОн спрыгнул на землю и держал уздечку, пока Кривой Мик садился в 
седло.
\par- Я сделаю на нем круг по долине, чтобы посмотреть, как он идет.
\parИ он пустил коня легким галопом. Когда Кривой Мик вернулся, он 
улыбался.
\par- Давай я все же поеду на Мунлайт и поведу гнедого. На Фаерфаксе в 
замок должен въехать ты сам.
\parПитер сел верхом на Фаерфакса, и они тронулись в путь. Все очень 
устали.
