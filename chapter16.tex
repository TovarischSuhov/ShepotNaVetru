Глава 16
\parСОСТЯЗАНИЕ ЛГУНОВ
\par\parНа следующий день в Большом Зале собрались люди со всех четырех 
концов королевства. Пришли фермеры и садоводы, скотоводы и бродяги-
золотоискатели, а Директор акционерной компании по добыче олова, меди, 
цинка, свинца, золота и нефти привел с собой жену ж детей. Король 
всячески его обхаживал, так как надеялся найти нефть на принадлежащих 
компании землях, и предоставил им всем места в первом ряду.
\parДиректор и сам был лгун что надо.
\par- Надеюсь, Ваше Величество, почерпнуть из ваших историй что-нибудь 
полезное, - заявил он.
\par- Мне нечему вас учить, - заскромничал король. - Я всего лишь 
любитель, у меня нет вашего опыта. Но я, без сомнения, одолею 
соперника, кого бы они ни выставили. - И он кивнул в сторону Питера и 
Серой Шкурки, которые вместе с Кривым Миком входили в зал.
\parЛована была не с ними. Она сидела рядом с королевой на одном из 
тронов, поставленных в зале.
\parКороль разговаривал с Буньипом, который был чрезвычайно доволен 
тем, что его хитрость удалась и он сумел тайно впустить своих друзей в 
замок. Он сидел на огромном бревне, которое выкатили на середину 
площадки. Рядом с ним стоял трон, так обильно утыканный алмазами, что 
просто резало глаза. Он предназначался для короля. Кривой Мик сел на 
бревне рядом с Буньипом, который разглядывал какие-то бумаги. Питер 
расположился слева от королевского трона.
\parМгновение спустя король вошел и опустился на трон. Он расправил 
мантию, поправил корону и придал лицу важное выражение.
\par- Удобно ли Вам, Ваше Величество? - спросил Кривой Мик.
\par- Вполне, - ответил король. - Когда я удобно устраиваюсь, я удачно 
лгу. Поэтому на состязании я считаю очень важным сесть как следует, а 
не присесть кое-как.
\par- Согласно правилам, - начал Буньип, открывая состязания и читая 
скрепленный сургучом пергамент, - вы вправе усаживаться и 
присаживаться как пожелаете. Вы вправе лгать с лошади или с любого 
другого животного, избранного вами для передвижения. Вы вправе лгать, 
двигаясь по направлению к судье, - то есть ко мне. Но не вправе лгать, 
двигаясь от судьи, так как это оскорбительно для меня. Вы вправе 
лгать...
\par- Хватит болтать! - вспылил король. - Приступим к состязаниям. 
Правила я знаю - сам писал.
\par- Это уже пошла первая ложь, Ваше Величество? - спросил Кривой 
Мик, который считал короля весьма посредственным лгуном. Он был 
уверен, что сумеет одолеть его с помощью преувеличения, в крайнем 
случае с помощью выдумки или легкой неправды.
\par- Я еще не начинал, - ответил король.
\par- Тишина! - проревел Буньип.
\parСлушатели поудобней уселись в креслах. Все они надеялись на 
поражение короля, поскольку любили Ловану и желали ей выйти замуж за 
Питера.
\par- Состязание объявляется открытым. Первый свою ложь представляет 
король. В зависимости от качества лжи я начисляю от одного очка за 
преувеличение до десяти за ложь наглую и бессовестную.
\parКороль откашлялся и приступил.
\par- В юности я зарабатывал немало денег тем, что вырубал из земли 
заброшенные шахты и продавал их на юге для колодцев.
\par- Как давно это было? - спросил Буньип.
\par- Очень давно. Задолго до того, как я родился, - ответил король.
\par- Что ж, такое красивое начало лжи не часто услышишь, - сказал 
Буньип. - Продолжайте.
\par- Для такой работы требовалось большое искусство, - говорил 
король. - Мне надо было сохранять кристальную честность и иметь 
обширные познания в математике.
\par- Зачем?
\par- Мне приходилось немало складывать, умножать и вычитать.
\par- Понятно. А сколько вы брали за колодец?
\par- Сто долларов.
\par- Одно очко, только одно, - решил Буньип.
\par- Четыреста долларов.
\par- Уже лучше. Еще очко.
\par- Каждый колодец я грузил в фургон, запряженный быками, - 
продолжал король, - и затем трогался в Мельбурн. В общем, задавал я 
быкам работенку.
\par- А сколько их было?
\par- Двести двадцать.
\par- Ничего себе упряжка! - воскликнул Буньип.
\par- Да, не маленькая. Я протянул телефон от фургона до передних в 
упряжке быков, и посадил погонщика управлять ими. Когда я хотел 
сделать привал, я звонил из фургона погонщику и просил его 
остановиться. Тот останавливал передних быков. За ними постепенно 
останавливались другие, когда их цепь ослабевала. В результате задние 
останавливались через полчаса после передних. Такая система всю дорогу 
работала неплохо, пока как-то раз я случайно не набрал неправильный 
номер, а когда исправил ошибку, быки прошли уже десять миль.
\par- Полагаю, что при данных обстоятельствах указанное расстояние 
является весьма вероятным, - сказал Буньип. - Продолжайте.
\par- Самая большая неприятность произошла в то утро, когда мы 
двинулись в путь, - объяснил король. - Передние тронулись и лишь через 
полчаса задние почувствовали, что цепь натянулась и тронулись тоже. Мы 
теряли так много времени на то, чтобы тронуться с места и остановку, 
что до Мельбурна добирались целых шесть месяцев. Мне удалось всю шахту 
продать одному человеку, который хотел соорудить в своем саду колодец. 
И надо сказать, хороший колодец получился. Но я в конце концов 
забросил это дело, поскольку оно не окупало расходы.
\par- Это все? - спросил Буньип.
\par- Да.
\par- Что ж, ложь хорошая, качественная, но лично мне труднее всего 
поверить в то, что вы возили шахты на быках вместо того, чтобы 
отправить их поездом. Почему вы так не сделали?
\par- Шахты находились в необжитых районах, поезда туда не ходили.
\par- А, все ясно. Теперь послушаем, что скажет Кривой Мик.
\parКривой Мик поднялся с бревна и начал.
\par- Однажды я вез на фургоне с быками из Бурка пять тонн оловянных 
дудочек.
\par- Неплохое начало, - отметил Буньип, делая какие-то пометки в 
блокноте.
\par- А куда ты их вез? - спросил Питер, которого этот рассказ вдруг 
заинтересовал.
\par- Никуда, - ответил Кривой Мик.
\par- Но ты ведь должен был везти их куда-нибудь, - встрял король.
\par- Я вез их никуда, - повторил Кривой Мик. - Если бы я вез их куда-
то, я бы в конце концов туда и привез, и тогда оловянных дудок у меня 
больше не было бы. А поскольку я не вез их никуда, они всегда были при 
мне.
\par- Но какая от них польза? - спросил Питер.
\par- Ни малейшей, - ответил Кривой Мик, - хотя они могли бы 
пригодиться для подарков.
\par- Мне почему-то кажется, что ты лжешь, - проговорил Буньип.
\par- А я что должен делать? - поинтересовался Кривой Мик.
\parБуньип выглядел огорошенным.
\par- Да, конечно, - быстро добавил он. - Я стал забывать. Никогда еще 
ложь так тесно не переплеталась с правдой. Продолжай.
\par- Налетел порыв ветра, - рассказывал Кривой Мик, - который 
всколыхнул деревья и поднял пыль.
\par- Утверждение правдоподобно, - перебил его Буньип. - Ветер всегда 
колышет деревья и поднимает пыль. Боюсь, мне придется тебя оштрафовать 
на десять очков. На правду не переходить. Продолжай.
\par- Мне следовало бы сказать, что ветер не колыхал деревья и не 
поднимал пыли, - поправился Кривой Мик, - хотя на расстоянии вытянутой 
руки уже не было ничего видно. Ветер с такой силой дул в 
противоположных направлениях, что он сталкивался сам с собой и 
оставался в полнейшей неподвижности, пытаясь преодолеть сам себя. 
Понятно, что я хочу сказать?
\par- Нет, - ответил Буньип.
\par- На это я и рассчитывал, - сказал Кривой Мик.
\par- Это как два быка, которые столкнулись лбами и не могут друг 
друга сдвинуть? - спросил Питер.
\par- Да, - ответил Кривой Мик.
\parБуньипу это стало надоедать и он строго посмотрел на Питера.
\par- Здесь сужу я. Быки не имеют никакого отношения к рассказу. - 
Затем Кривому Мику: - Пожалуйста, продолжай.
\par- Я тащил быков вверх по песчаному склону, - продолжал Кривой Мик. 
- Я уже находился на уровне колес, а у быков голова вытянулась к 
коленям. Я щелкал кнутом и вопил: "Но-о! Но-о!", как вдруг услышал 
прекраснейшую музыку. Я оперся о кнутовище и прислушался. Музыка 
продолжалась, и я осознал, что слушаю Пятую симфонию Бетховена, 
исполняемую на оловянных дудочках. Понимаете, дудочки лежали 
мундштуками в сторону ветра. Веревки, которыми они были связаны, 
закрывали одни отверстия на дудочках, и открывали другие. Когда фургон 
раскачивался от рывков, одни отверстия открывались, другие 
закрывались, причем таким образом, что ветер, влетающий в мундштуки, 
вылетая, сотворял маленькое чудо. Знаете, меня это изумило.
\par- И меня это изумляет, - угрюмо проворчал король.
\parБуньип в блокноте суммировал очки. Он то бесцельно смотрел на 
крышу, то стучал карандашом по зубам, на какой-то странице нарисовал 
лошадиную голову, тут же ее перечеркнул и произнес:
\par- Приняв во внимание правила состязания, оговоренные в третьем 
параграфе меморандума, а также условия, нижеименуемые... - впрочем, 
довольно. Победителем объявляю Кривого Мика.
\parЗрители захлопали, закричали "Ура!", а одна фрейлина даже упала в 
обморок. Не потому, что для этого был какой-то повод. Просто ей 
нравилось падать в обморок на подобных состязаниях. Она считала, что 
ее утонченное воспитание не позволяет ей выносить ложь.
\parКороль был вне себя. Вскочив, он отбросил мантию вверх и в 
сторону, так что она намоталась на его плечи как крыло. На нижних 
ступеньках лестницы, ведущей к трону, он задержался и гневно произнес:
\par- Возмутительно! Того и гляди, начнут ставить под сомнение 
обещания, которые дают высокопоставленные лжецы! Я обязан извиниться 
перед Директором акционерной компании по добыче олова, меди, цинка, 
свинца, золота и нефти за оскорбление, нанесенное его прозорливости.
\par- Полно, полно, - поспешно отозвался Директор. - Не надо ломать 
комедию.
\parОн нацарапал какую-то записку на бумаге и протянул ее королю.
\par- Вы можете получить работу в нашем совете директоров, как только 
захотите.
\parКороль прочел записку. Настроение его поднялось, по гнев не 
прошел.
\par- К счастью, остаются еще два задания, чтобы испытать этого 
самозванца, который выдает себя за принца. Пусть сначала приведет мне 
Фаерфакса, Дикого коня с Глухих Гор - оседланного, укрощенного и 
годного для езды, а потом готовится к последнему заданию.
\parКороль покинул зал. Последовавшая за ним королева, спускаясь во 
двор, наступила пяткой на собственный шлейф и разорвала его по шву. 
Одна из фрейлин, однако, скрепила края булавкой. Это злополучное 
происшествие случилось как нельзя некстати, поскольку королю пришлось 
стоять и ждать королеву, а делать это с подобающим величием крайне 
затруднительно.
\par{whisp11.gif}
\parПитер пошел проводить принцессу Ловану. Она взяла его под руку, и 
они так и шли по длинным коридорам. Когда кто-нибудь встречался им, 
Питер снимал шляпу с перьями и кланялся, а принцесса улыбалась и 
кланялась тоже. Те, кому она улыбалась, говорили: "Словно запели вдруг 
все лесные птицы".
