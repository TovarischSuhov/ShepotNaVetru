Глава 21
        ПИТЕР ЖЕНИТСЯ НА ПРЕКРАСНОЙ ПРИНЦЕССЕ

    Настал день свадьбы Питера и Лованы, и все жители окрестных 
селений шли или ехали к замку, разукрашенному флагами. Флаги были алые 
и золотые, голубые и желтые, и все они неистово бились на ветру, 
словно пели принцессе Ловане песнь радости.
    Лована в своей комнате готовилась к свадьбе. Вокруг нее суетились 
фрейлины, то затягивая ей платье в одном месте, то отпуская в другом. 
Само же платье, сшитое из наилучшего шелка цвета слоновой кости, 
сверкало алмазами, его украшал и золотой с драгоценными камнями 
медальон, в котором хранился Волшебный Лист. Лицо Лованы было 
настолько прекрасно, что все, кто ее видел, испытывали благоговейный 
восторг.
    Лована не была уже той грустной принцессой, которая дрожала при 
одной только мысли об экзаменах или о том, что король с королевой 
могут ее отругать. Теперь она была счастливой и свободной и выходила 
замуж за человека, которого любила. Не придется ей больше выслушивать 
нотации королевы, не придется подчиняться капризам короля-себялюбца...
    А король в это время сидел в кабинете и занимался арифметикой. 
Сначала он подсчитал общую сумму своего состояния. Оно достигало 
сорока миллионов долларов. Затем дошла очередь до страшного действия - 
вычитания. Ему надо было вычесть стоимость Фаерфакса и золотой с 
алмазами короны. Он давно считал то и другое своей собственностью, и 
мысль о том, что они потеряны для него навсегда, приводила его в 
бешенство.
    За минувшее время король постепенно свыкся с тем, что корона 
навеки погребена на дне озера, поэтому, когда Питер принес ее, он 
просто обомлел. Очищенная от грязи и натертая Лованой, корона с 
алмазами сверкала так ярко, что королю даже пришлось надеть солнечные 
очки, чтобы защититься от блеска. Ему нестерпимо хотелось оставить ее 
себе, и он всячески тянул время, пытаясь придумать Питеру еще какое-
нибудь задание, которое тот ни за что не выполнит.
    Но Питер стоял на своем:
    - Король дал слово, что для женитьбы на принцессе мне нужно 
выполнить всего три задания.
    Король в ярости стукнул кулаком по столу.
    - Проклятье! - вскричал он. - Неужели нельзя относиться к моим 
словам как к словам обычного дельца, а короля оставить в покое?
    - Конечно, нельзя, - ответил Питер.
    - О боже! - воскликнул король. - Ты один стоил мне дороже, чем все 
другие претенденты!
    - Еще бы! - усмехнулся Питер. - Всех остальных вы убивали, а их 
деньги присваивали. Буньип мне рассказал.
    - Этот Буньип - известный лгун и мошенник, - злобно прошипел 
король. - Я его изничтожу.
    - Вряд ли вы это сделаете, - ответил Питер. - Ведь он теперь наш 
друг.
    Больше королю сказать было нечего. Он в последний раз бросил 
завистливый взгляд на корону и пошел сообщить все королеве.
    Она же все это время смотрела в замочную скважину и восхищалась 
Питером. Он выглядел так величественно, что у нее исчезли всякие 
сомнения относительно его хорошего воспитания и того, какое 
впечатление он произведет на соседей. Когда король открыл дверь, она 
быстро отскочила в сторону и бросилась к Питеру, протянув к нему руки 
для объятья.
    - Дорогой мой! - сказала она, - ты выглядишь просто очаровательно. 
А теперь - извини меня - мне надо увидеть любимую дочь и проверить, 
что за платье она собирается надеть. - И королева упорхнула прежде, 
чем король успел отчитать ее за подглядывание.
    Таким образом Питер и Лована получили согласие ее родителей на 
брак и стали быстро готовить свадьбу.
    В день бракосочетания Питер почистил Мунлайт и подготовил ее к 
обратному путешествию. Для себя Лована выбрала в королевской конюшне 
самую красивую кобылку. Она была такой же белоснежной, как Мунлайт, и 
величиной с нее же. Седло сделали из тончайшей кожи и украсили 
драгоценными камнями.
    Кривой Мик собрался ехать на Фаерфаксе. Он надел на него простое 
пастушье седло, заявив, что на всякие "новомодные" седла у него просто 
нет времени.
    Три лошади ждали под большим деревом. К морде каждой была 
подвязана торба с сечкой, и они спокойно ели.
    Люди переходили подъемный мост и стекались во двор замка, где 
должно было происходить бракосочетание. Даже самый большой зал не смог 
бы вместить всех желающих. В основном это были крестьяне, жившие на 
обширных землях короля; они были бедны и поэтому одеты просто. Однако 
в толпе попадались и рыцари и принцы из других королевств, 
прослышавшие, что нашелся наконец такой претендент, который выполнил 
все задания отца принцессы. Въезжая во двор на роскошных лошадях и со 
знаменами в руках, они завидовали Питеру.
    Никто из них, однако, не подходил к Буньипу, который исполнял роль 
церемониймейстера и командовал гостям: "Пожалуйста, не толкайтесь. Не 
спешите. Не стоит торопиться. Сюда, пожалуйста". Ему хотелось 
вспомнить свое искусство и сбить этих зазнаек с коней струями из двух 
ноздрей сразу, но с тех пор, как Питер дал ему Волшебный Лист, он не 
мог больше причинять людям зло.
    В одном конце двора соорудили помост, на который лицом к алтарю 
поставили несколько тронов. Здесь же стояла группа епископов в длинных 
белых одеждах, с митрами на головах. Они говорили о бедности в их 
приходах, и о скудных сборах, которые они получали по воскресеньям.
    По бокам помоста стояли трубачи, и, когда все собрались, они с 
такой силой задули в трубы, что с некоторых гостей чуть не слетели 
шляпы.
    После того, как трубачи протрубили второй раз, Буньип проорал:
    - Король и королева!
    Один из рыцарей, который в свое время еле унес ноги после того, 
как Буньип сбил его наземь, от страха свалился с лошади.
    Король и королева вышли из двери, расположенной за помостом, и 
сели на два лучших трона. Собравшиеся приветствовали их, хотя и без 
особого энтузиазма; в ответ король поднял руки над головой, с 
достоинством сцепив их в рукопожатье.
    Затем Буньип проорал второй раз:
    - Принцесса Лована и принц Питер!
    На этот раз толпа ревела во всю мочь, пока они усаживались на два 
трона поскромнее.
    - Серая Шкурка и Кривой Мик! - проревел опять Буньип, и вошли 
друзья Питера. У Серой Шкурки за ухом был цветок, и выглядела она 
вполне благовоспитанной. Кривой Мик был в штанах наездника и со 
шпорами, и хотя благовоспитанным его назвать было нельзя, зато 
выглядел он величайшим наездником мира. Они сели на два расшатанных 
трона, давным-давно валявшихся среди разной рухляди. В этом ни Питер, 
ни Лована не были виноваты, потому что свадьбу организовывал и троны 
готовил король.
    Буньип уже собирался объявить торжество открытым, когда в воздухе 
что-то сверкнуло и прямо перед помостом неуклюже приземлилась 
колдунья. Волосы у нее были опалены. Со всех сторон на ней висело 
столько фотоаппаратов, что метла, наверно, с трудом ее подняла. Она 
уже сделала дистанционный снимок с Луны, а теперь желала снять ближним 
планом.
    Водрузив перед помостом на треноги несколько фотоаппаратов, 
колдунья стала с невероятной быстротой делать снимки со вспышкой. 
Лампы-вспышки у нее были какие-то громоздкие, и каждый раз, когда она 
нажимала на спуск, издавали громкий хлопок. Своей яркостью они 
затмевали солнце. Многие гости жаловались, что после такой вспышки они 
по пять минут ничего не видят. Это значило, что они вообще ничего не 
видели, поскольку колдунья щелкала не переставая.
    {whisp15.gif}
    Епископы не обращали на нее ни малейшего внимания. Они воздевали 
руки к небу и ходили туда-сюда, вперед-назад, в то время как Питер и 
Лована стояли перед алтарем. Каждый из епископов хотел самолично 
скрепить их брак, поскольку такая возможность предоставлялась нечасто. 
Они оказывались друг у друга на пути, сталкивались, и, как утверждали 
потом некоторые длинные языки, обвенчали Питера и Ловану два раза.
    Когда церемония подошла к концу, Питер взял слово. Он сказал:
    - Я хочу сделать подарок каждому из присутствующих здесь. 
Пожалуйста, подходите все к помосту. Но сначала я хочу преподнести 
подарки королю и королеве.
    Он подошел к подножью тронов и вручил родителям принцессы по 
Волшебному Листу. Сначала королева с презрением его отвергла, сердито 
проворчав: "Ты все превращаешь в балаган", но Питер буквально вложил 
Лист в ее руку, и она мгновенно изменила тон: "О! Как ты добр! - 
воскликнула она. - Спасибо тебе огромное!"
    Она улыбалась и рассыпала всем воздушные поцелуи, и стала просто 
очаровательной. На короля, однако, Лист подействовал иначе. Он по-
настоящему страдал. Всю жизнь он любил копить деньги. Теперь же ему 
хотелось помогать людям, отдать им часть своих богатств - ведь он 
чувствовал, что они его любят. И пока он сидел, скривившись от муки, 
Питер и Лована быстро раздавали Листы, так что вскоре лица у всех 
светились счастьем.
    У края помоста стояло несколько женщин с младенцами. Король 
неожиданно сошел с трона и перецеловал младенцев одного за другим. 
Матери были безмерно счастливы, хотя самим детям это не очень-то 
понравилось.
    Затем король вновь поднялся на помост и призвал стражу:
    - Возьмите эти ключи и откройте мою кладовую. Принесите два 
сундука с золотом. Я хочу подарить каждому из присутствующих по 
пятьсот долларов.
    Стражники в величайшем возбуждении убежали, а король плюхнулся на 
трон и сказал сам себе:
    - Какой комар меня укусил? Я, наверно, сошел с ума!
    Когда стражники принесли сундуки, король открыл их ключом, 
спрятанным у него в кармане, и раздал всем по пятьсот долларов.
    Бедняки обомлели. Некоторые плакали от счастья, поскольку им 
больше не надо было беспокоиться о ценах на фрукты и овощи. У них были 
деньги купить все необходимое.
    Король с каждой минутой чувствовал себя все более счастливым. 
Теперь он обратился к поварам:
    - Идите на кухню и принесите сюда каждому по куску мяса для 
воскресного обеда.
    Даже королева посчитала, что он зашел слишком далеко.
    - Хватит и по полкуска, - прошептала она, прикрыв рот рукой.
    - Нет, по два куска каждому! - крикнул король, не обращая внимания 
на слова королевы. Повара побежали на кухню и скоро стали раздавать 
мясо.
    Лована некоторое время смотрела, как они это делают, а потом 
наклонилась и прошептала Серой Шкурке:
    - Не могла бы ты пойти и принести два фунта сосисок? Я хочу их 
поджарить Питеру и Кривому Мику на ужин.
    - Не беспокойся, - ответила Серая Шкурка. - Нам можно не думать о 
еде. Из своей сумки я могу достать все, что тебе надо.
    - Ах, да! - восторженно воскликнула Лована. - Я и забыла.
    - Нам пора, - сказал Питер, по очереди пожимая всем руки.
    Когда гости увидели, что они уезжают, все проводили их до того 
места, где стояли привязанные лошади. Сюда пришли все - король и 
королева, Буньип, нянюшки, ухаживавшие за принцессой. Лована их всех 
поцеловала. Теперь, когда она уезжала, король с королевой загрустили.
    - Я буду навещать вас раз в месяц, - пообещала она.
    Питер подсадил ее на спину белой кобылки и вскочил на Мунлайт. 
Кривой Мик верхом на храпящем Фаерфаксе повел их через буш. Фаерфакс 
задрал голову и изогнул шею - горячий рыжий жеребец, он шел впереди 
двух белых как снег кобылиц, на которых ехали два прекрасных человека. 
Позади всех скакала Серая Шкурка.
    И вот в конце концов они приехали к избушке Кривого Мика в буше. 
Коровы приветливо мычали, козы блеяли, куры кудахтали, а собаки 
носились и лаяли. Такой счастливой Лована еще никогда не была.
    Она готовила еду, убирала избушку и вместе с Питером носилась по 
бушу. Обе лошадки летали быстрее ветра, и длинные волосы Лованы 
золотым потоком струились за ней.
    Так и жили они вместе с Кривым Миком и Серой Шкуркой и были самыми 
счастливыми на земле.
