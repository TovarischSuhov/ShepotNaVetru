\chapter{ДОЛИНА ЦЕПЛЯЮЩЕЙСЯ ТРАВЫ}
\par\par- До чего же здорово вот так путешествовать, - говорила Серая 
Шкурка, прыгая рядом с Питером, который ехал верхом. - Прыгая вверх, я 
делаю вдох, приземляюсь - выдох, когда я вверху - я вижу все на многие 
мили вокруг. Смотри! - она подпрыгнула так высоко, что оказалась выше 
Питера, который натянул поводья, придерживая лошадь.
\par- У тебя это прекрасно получается! - крикнул он. - А теперь 
поскачем быстрее.
\par- Давай! - согласилась Серая Шкурка. - Мы с тобой летим. Я вижу на 
милю вперед. Я вижу, что там за холмом.
\parОни взлетели на холм и остановились на вершине.
\par- Что-то у меня тут закололо, - пожаловалась Серая Шкурка, держась 
за бок. - Я запыхалась. Надо передохнуть. Мы давали не меньше сорока 
миль в час, и это в гору!
\par- Что за странная долина! - воскликнул Питер, глядя вперед, где 
перед ними лежало широкое поле сухой травы. Трава колыхалась от ветра, 
по ней скользили какие-то тени. До холма, где стояли Питер и Серая 
Шкурка, доносилось ее холодное, режущее слух шуршание, похожее на 
шипение змей или на чей-то неустанный шепот.
\parСерая Шкурка поежилась.
\par- Не нравится мне это место, - сказала она. - Я здесь уже была. 
Это Долина Цепляющейся Травы, которая непрестанно что-то шепчет. Дети 
всего мира, перед тем как стать взрослыми, обязательно должны пройти 
через эту долину. Те, у кого родители умные, проходят ее легко. Но 
большинству здесь приходится туго.
\par- Давай как-нибудь увильнем, - предложил Питер. - Давай обойдем 
ее.
\par- Не получится, - ответила Серая Шкурка. - Прекрасная Принцесса 
живет как раз по ту сторону Долины, так что нам, хочешь, не хочешь, 
придется ее пересечь. Трава будет цепляться за ноги, пытаться повалить 
на землю, а если ты схватишь ее рукой, поранит тебе пальцы. Знаешь, 
есть такая трава, о которую можно порезаться, если попытаться ее 
сорвать. Так вот, эта - такая же. Она ужасная. К тому же вечно что-то 
шепчет.
\par- Что же она шепчет?
\par- Это зависит от того, с кем она говорит. Особенно она жестока к 
детям несчастным, обиженным, одиноким и к тем, кто не верит в себя. 
Она тянет их вниз, режет им коленки, да так, что шрамы остаются на всю 
жизнь. Если прислушаться, можно различить ее шепот: "Ты - слишком 
толстый, ты - слишком худой, ты - дылда, ты - коротышка. Расправь 
плечи. Почему ты такой глупый? Подожди, дома я тебе задам. Вот дойдет 
до отца, тогда узнаешь. Ты лентяй. Ты эгоист. И врун. Почему ты так 
плохо учишься? Почему ты хуже соседских детей? Сделай то, сделай это, 
иди сюда, пойди туда, чтоб тебя было видно, но не слышно, слушай 
старших, подчиняйся, соглашайся..."
\par- Трава начинает шептать, стоит ребенку лишь подойти к ней, - 
продолжала Серая Шкурка. - Она сводит детей с ума, режет им ноги, и 
следы от этих порезов не затягиваются и не исчезают.
\parПока они разговаривали, из-за кустов у подножья холма вышла 
группка детей. Все они были в школьной форме, и у каждого за спиной 
висел ранец с учебниками. Маленьким было лет по семь, старшим - лет по 
четырнадцать. Они остановились у края Долины и прислушивались к шепоту 
травы. Идти по полю они боялись, оно казалось им просто бескрайним.
\par- Знать бы, как им помочь, - сказал Питер, Его внезапно охватила 
ненависть к траве, и он живо представил себе, как скосил бы 
газонокосилкой все это поле, чтобы трава никогда больше не цеплялась 
за ноги ребят.
\parСерая Шкурка догадалась, о чем он думает,
\par- Ее не скосить, - сказала она. - Я однажды пыталась, но трава 
вырастает быстрее, чем ты ее срезаешь, а уж я-то умею обращаться с 
косой.
\par- Стоя тут, мы вряд ли им поможем, - сказал Питер.
\parОн подбежал к лошади, вскочил на нее и галопом погнал вниз по 
склону. Серая Шкурка - за ним. Они мчались меж деревьев и скал, 
перепрыгнули через мелкий ручей и устремились к детям, стоявшим на 
краю поля под красным эвкалиптом. Многие дети плакали. Самый маленький 
мальчик порезал об острую траву руку и обмотал ее носовым платком: на 
платке виднелись пятна крови. Другой мальчик сильно порезал себе 
коленки, а у одной девочки на лбу был синяк - она ударилась, 
споткнувшись о невидимый в траве камень. Дети были напуганы и стояли, 
сбившись в кучку. Питер осадил лошадь возле них; они смотрели на 
кенгуру, как на врага.
\parВдалеке, посреди долины виднелась еще одна группа детей. Те 
решились проложить себе путь через Цепляющуюся Траву, но, чем дальше 
они углублялись в сплетенную гущу стеблей и листьев, тем труднее 
становилось им идти. Трава неистово колыхалась вокруг них, из ее 
сплошной переплетенной массы вытягивались маленькие серые пальцы и 
раздирали детям руки и ноги. Постоянный шепот все нарастал, пока не 
перешел в громкое шипение, как у тысячи змей, когда они то бросаются 
вперед, то отступают.
\parПо траве, поднимаясь и опускаясь серыми волнами, проносились 
темные тени, напоминающие тени от облаков. Становилось дурно от одного 
взгляда на эти сухопутные волны, которые, в отличие от волн морских, 
не освежали душу.
\par"Почему ты не можешь сдать экзамены? - шептала детям потревоженная 
трава. - Надо больше заниматься и меньше играть. Надо трудиться 
упорней. Кто не сдаст экзамены - останется без работы. Ты уже слишком 
большая, чтобы играть в куклы и прочие детские игры. Подумай о 
будущем".
\parТрава становилась все злее и злее, и дети стали падать и 
барахтаться в объятиях листьев и стеблей. Питер не мог больше ждать. 
Детям нужно было дать Волшебный Лист. Мальчик пустил лошадь в галоп, и 
она помчалась вперед длинными прыжками. В руке Питер держал Волшебный 
Лист, и трава, тянувшаяся к нему, съеживалась и увядала. Колышащиеся 
стебли расступались под копытами Мунлайт, увядая и шипя, словно их 
что-то сжигало.
\parКогда Питер нагнал детей, трава отпрянула от них и безжизненными 
плетьми легла у их ног. Ее шепот умолк.
\parПитер соскочил с лошади и вручил каждому ребенку по Волшебному 
Листу. Дети прекратили плакать и стали улыбаться.
\par- Вам больше нечего бояться, - сказал Питер. - Продолжайте свой 
путь. Пока у вас в руках будут эти Волшебные Листики, ничего плохого с 
вами не случится.
\parДети побежали вперед, смеясь и танцуя. Питер проводил их взглядом, 
пока они не достигли края ноля, а потом вскочил на Мунлайт и вернулся 
к детям, которых оставил под красным эвкалиптом.
\par- Скажите, вы добрые, и ты, и кенгуру? - шепотом спросила одна 
девочка, державшая за руку брата.
\par- Пожалуй, да, - ответил Питер. - Во всяком случае, мы пришли сюда 
помочь вам.
\parТут все дети заулыбались и перестали бояться.
\par- А мне ты дашь такой лист? - спросила девочка, у которой все лицо 
было покрыто веснушками.
\par- Конечно. Я всем дам по листу.
\par- Скажи мне, - спросила Серая Шкурка девочку, - мама тебя любит?
\par- Ну, конечно! Но она любила бы меня еще больше, если бы не эти 
веснушки. Она все время мне говорит: "Как жалко, что у тебя веснушки!"
\par- Ах, вот оно что! - сказала Серая Шкурка, и, немного подумав, 
добавила: - По-моему, веснушки - это даже очень красиво.
\par- По-моему, тоже, - согласился Питер и дал девочке Волшебный Лист.
\parОна зажала Лист в руке. И тут же веснушки на ее лице стали совсем 
незаметными, а само оно так изменилось, будто с души слетела тень, и 
вместо нее засверкало солнце, отражаясь в глазах.
\par- И я хочу такой лист, - произнес мальчик, который стоял, понурив 
голову, - он стеснялся смотреть на Питера.
\par- А что говорит твой отец?
\par- Он все время твердит, что я неудачник, а я не знаю, что это 
такое. По-моему, он жалеет, что я расту не таким, как он. Я хочу быть 
художником, а он твердит, что все художники не от мира сего.
\par- О боже! - воскликнула Серая Шкурка и зашептала Питеру на ухо: - 
Именно так и про меня все говорили, когда я вынула из сумки пианино. 
Понимаешь, я люблю играть на пианино. Меня уверяли, что я играю Шопена 
с большим чувством, уверяли люди, которые любят музыку.
\par- А я умею играть "Дом родной" на губной гармошке, - сказал Питер. 
- Старина Мик говорил, что у него от моей игры даже слезы на глаза 
навертываются.
\par- Вот это успех! - воскликнула Серая Шкурка. - Я бы тоже хотела 
делать что-нибудь такое, что заставляло бы людей смеяться, плакать или 
танцевать.
\par- А лист-то мне дадите? - спросил мальчик, который уже начал 
беспокоиться.
\par- О, прости меня, - сказал Питер. Он вынул из маленькой сумки 
Волшебный Лист и вручил его мальчику, который вдруг поднял голову и 
улыбнулся. - Больше ты не будешь ходить, понурив голову.
\par- А у меня для тебя тоже есть подарок, - сказала Серая Шкурка. Она 
засунула лапку в карман и вытащила оттуда альбом и набор красок. В нем 
были краски двадцати четырех цветов и еще две кисточки.
\par- Теперь, когда у тебя есть Волшебный Лист, - сказала она, - ты 
нарисуешь удивительные картины. Покажи их завтра отцу, они ему 
понравятся.
\parВсе дети просили Волшебный Лист. Питер быстро раздал листья, и 
вскоре ребята почувствовали, что больше не боятся Цепляющейся Травы.
\par- Теперь вы сможете пройти по этому полю, - сказал Питер. - Я 
провожу вас.
\parС гиканьем дети побежали по траве, и бледные стебли в ужасе 
расступались перед ними, образуя проход. Он протянулся через все поле, 
а там, по ту сторону поля, высились зеленые холмы, протекали речки и 
светило яркое солнце. Дети бросились бежать по свободному проходу, 
потом остановились и помахали Питеру и Серой Шкурке.
\parИменно в этот момент Питер заметил, что у него на ногах появились 
замечательные сапоги, каких он никогда в жизни не видел. Они были из 
тончайшей кожи, а широкие отвороты доходили до колен. Питер с 
удивлением их разглядывал.
\par- Не вижу ничего удивительного, - сказала Серая Шкурка. - 
Волшебство постепенно превращает тебя в принца. Это Волшебный Лист 
награждает тебя, когда ты приносишь людям добро. Интересно, что бы 
случилось, если бы ты вдруг стал зазнайкой и эгоистом?
\parПитеру тоже было интересно - что.
