Глава 20
\parПОСЛЕДНЕЕ ЗАДАНИЕ
\par\parУтром Питер и его друзья собрались под большим деревом обсудить, 
каким образом достать золотую корону со дна озера.
\parБуньип предложил выпить всю воду из озера и затем выпустить ее в 
речку, но Кривой Мик заметил, что речка выйдет из берегов и затопит 
много земли. Поэтому от его предложения пришлось отказаться.
\parПитер все время думал о колдунье, которую обидела принцесса. Он не 
сомневался, что это та самая колдунья, которая возила его на Луну, и 
стал ломать голову, как бы ее позвать. Если бы она показала точное 
место на озере, куда она бросила корону, они могли бы попросить 
великана Ярраха войти в воду и своей ручищей пошарить по дну.
\parПитер поделился планом с Серой Шкуркой и Кривым Миком, и он им 
чрезвычайно понравился. Но как известить колдунью и великана?
\par- Я на Фаерфаксе могу домчаться до избушки колдуньи и потом до 
замка великана, - предложил Кривой Мик. - Этой ночью конь передохнул и 
к утру должен быть свежим. Если он пойдет галопом, то к вечеру я уже 
снова буду здесь.
\par- А вдруг не успеешь? - засомневался Питер. Он-то знал, сколько 
миль предстояло покрыть Кривому Мику.
\par- Непременно успею. Если сам я к вечеру и не вернусь, то великан и 
колдунья придут точно. Метла колдуньи дает хорошую скорость, и она, 
наверно, прилетит первой, хотя и великан одним шагом перемахивает 
несколько миль. Оба они обязательно явятся. Но почему тебе так важно 
заполучить их именно к вечеру?
\par- Хочу, чтобы они работали ночью, - ответил Питер. - Ведь днем 
великан всех перепугает, да и колдунье люди тоже не обрадуются.
\par- Понятно. Значит, мне надо отправляться как можно скорее.
\parКривой Мик недолго седлал Фаерфакса. Он вскочил на коня, махнул на 
прощанье рукой и устремился по дороге. Фаерфакс шел широким галопом. 
Кривой Мик натягивал поводья, не давая коню опустить голову, Фаерфакс 
грыз удила. Они быстро скрылись из виду.
\parВесь этот день Питер провел, ухаживая за Мунлайт. Буньип храпел 
под деревом. Ближе к вечеру, когда в ложбинах и под деревьями стали 
сгущаться тени. Серая Шкурка пошла в замок позвать Ловану. Питеру 
хотелось, чтобы она была с ним, когда придут ведьма и великан.
\parЛована все еще боялась колдуньи. Ей не верилось, что та стала 
добродушной старушкой, какой обрисовал ее Питер.
\par- Волшебный Лист изменил ее совершенно, - убеждал он ее.
\parНо Лована не могла забыть, как та скреблась в окно, чтобы попасть 
в комнату, и сомнения продолжали мучить ее.
\parКогда Серая Шкурка привела Ловану к дереву, она села рядом с 
Питером и спросила:
\par- А что мне сказать колдунье, когда я ее увижу? Она ведь помнит, 
что я спихнула ее с подоконника.
\par- Я не удивлюсь, даже если она тебя... поцелует. Она ведь уверена, 
что ты ее любишь. Вот что делает Волшебный Лист.
\parОн взял ее руку в свою и не отпускал. И она позабыла о колдунье.
\par- Мне нравится, когда ты держишь мою руку, - сказала она.
\par- Мне тоже нравится держать ее. Ты выйдешь за меня, когда я 
выполню третье задание?
\par- Да. Я буду всегда вместе с тобой и Кривым Миком, готовить вам 
завтраки и убирать дом. Я хочу делать все, что мне не разрешали делать 
здесь. Это так замечательно! - И Лована опустила голову на плечо 
Питеру.
\parБуньипу стало противно.
\par- Никогда такого не видывал, - бурчал он. - Буньипы никогда не 
держат девочек за руки. Так просто не принято. - Он резко запрокинул 
голову: - Что это?
\parПолоска голубого пламени дугой прочертила небо. На мгновение она 
осветила призрачным светом деревья и нырнула вниз. Голубое пламя 
облетело вокруг дерева, сделало разворот, затем петлю и ударилось в 
землю рядом с Буньипом, который от испуга плюхнулся на спину.
\par- Как поживаете? - спросила колдунья.
\par- Мы очень рады тебя видеть, -- сказал Питер.
\par- Вам повезло, что вы вообще меня видите, - ответила колдунья. - Я 
иногда увлекаюсь скоростью и, входя в атмосферу, имею массу 
неприятностей. А к вам я так торопилась, что при входе в плотные слои 
у меня образовался огненный хвост, как у кометы. Я немного опалилась, 
но ничего, пройдет.
\par- Ты помнишь Ловану? - спросил Питер. - Принцессу, у которой ты 
украла корону?
\par- Помню, помню, - сказала колдунья, подходя к принцессе и обнимая 
ее. - Мне всегда хотелось снова с тобой встретиться и извиниться за 
то, что я пыталась к тебе влезть. Да еще сорвала с тебя корону - это 
было ужасно! Кривой Мик сказал мне, что ты хочешь получить ее назад, и 
я тебе помогу. Я покажу тебе точное место, куда я ее зашвырнула.
\par- Мы с Лованой собираемся пожениться, как только достанем со дна 
корону, - объяснил ей Питер.
\par- Чудесно! - воскликнула колдунья. - Никогда не забуду, что я 
подумала, когда первый раз увидела тебя. А подумала я, какое из тебя 
получится прекрасное жаркое... - она вдруг остановилась и 
сконфузилась. - Ой-ой-ой! Что же это я говорю? Я хотела сказать, я 
подумали, какой замечательный муж получился бы для Прекрасной 
Принцессы.
\par- А ты больше не сердишься на меня, что я столкнула тебя с 
подоконника? - спросила Лована.
\par- Нет, дорогая, что ты! Мне не страшно, когда меня сталкивают с 
окон. У меня ведь всегда при себе метла, и я на ней улетаю. А теперь 
пойдем к озеру, и я покажу вам, где лежит корона. Я спешу, мне надо 
успеть на луну, пока она не зашла. Там опустился новый аппарат, и, 
может быть, я пополню свою коллекцию одной-двумя камерами. Пошли.
\parИ колдунья по петляющей тропинке повела всех к озеру, которое в 
лунном свете выглядело серебряным диском. Подойдя ближе, они увидели 
стаи уток, плавающих посредине озера.
\parКолдунья только собиралась указать место рядом с одной из стай, 
как вдруг земля у них под ногами начала содрогаться, и они услышали 
глухие удары шагов великана, идущего по бушу. Это подходил великан 
Яррах. Было слышно, как трещат и ломаются деревья, которые ноги 
великана подминали, словно траву.
\parЗатем они увидели, как нога раздавила несколько деревьев не так 
далеко от них. Потом она поднялась, исчезла во тьме, и следующий шаг 
пришелся совсем рядом с ними.
\par- Надо что-то предпринять, а то он на нас наступит, - сказала 
колдунья, глядя наверх во тьму, где огромная круглая тень заслонила 
луну. - Я поднимусь на метле и поговорю с ним.
\parОна вскочила на метлу и рванулась во тьму. Секунду спустя где-то в 
вышине раздался ее сердитый голос:
\par- Смотри, куда ступаешь. Ишь, вымахал, а ума-то с гулькин нос. Ты 
чуть меня не раздавил!
\par- Прошу прощения, мадам, - извинился великан Яррах. - Я иду помочь 
своим друзьям.
\par- Если тебе нужны Питер и Прекрасная Принцесса, то они у тебя под 
ногами.
\par- Правда? Я хорошо его помню и рад, что он нашел свою принцессу, Я 
с ним поговорю и узнаю, что он от меня хочет.
\parВеликан оставил звездную вышину и наклонился. Его могучее тело 
показалось Питеру надвигающейся на небо безбрежной тенью. Питер вышел 
на открытое место, и пальцы великана охватили его со всех сторон. Его 
поднимали все выше и выше, и в конце концов он оказался на уровне лица 
великана, сидя на его ладони. Великан открыл рот, и Питер увидел 
громадный язык, который шевелился, как крыло дракона в пещере. Питер 
испугался и захотел снова очутиться на земле, рядом с Лованой. Но 
голос великана звучал мягко.
\par- С того времени, как ты дал мне Волшебный Лист, - сказал он, - я 
сделался самым счастливым великаном на свете. Ты знаешь - скольким 
людям я помог с тех пор, как ты посетил мой замок.
\par- На самом деле ты всегда был добрым, тебя только приучили быть 
злым. А Лист всего лишь сделал тебя тем, кем ты был с самого начала. Я 
знал, что ты мне поможешь.
\par- Что я должен сделать?
\parКолдунья, сидевшая на метле чуть выше Питера, не дала ему и рта 
раскрыть.
\par- Дай я все ему расскажу, потому что я тоже хочу тебе помочь, - 
заявила она.
\par- Судя по тому, как ты меня приветствовала, не похоже, что ты 
очень хочешь кому-то помочь, - тихо проговорил великан.
\par- Иногда меня тянет на старое, - сконфузилась колдунья. - Забудь, 
что я сказала. Когда я была злой, я украла у принцессы корону и 
забросила в озеро. И еще произнесла заклятье, что без нее она не 
выйдет замуж. Так что видишь, они с Питером никогда не смогут 
пожениться, если мы не достанем корону со дна озера, а оно очень 
глубокое. Питер хочет, чтобы ты вошел в воду и достал корону - я 
покажу тебе точное место.
\par- Это нетрудно, - сказал Великан, - когда начнем?
\par- Сначала поставь меня на землю, - сказал Питер. Ему не хотелось 
путешествовать до середины озера на ладони, да если еще вдобавок 
именно она будет шарить под водой.
\parВеликан Яррах нагнулся и опустил Питера на землю. Огромные пальцы, 
как решетка, окружили Питера, а когда он уже прочно стоял на земле, 
разомкнулись и освободили его.
\par- Я достану корону для Лованы, - пообещал великан.
\parЯррах стал выпрямляться, его огромная голова поднималась все выше 
и выше над деревьями, туда, где его уже ждала колдунья, чтобы указать 
точное место.
\par- Видишь, - прямо посреди озера плавает стая уток, - сказала она. 
- Как черные точки на серебре.
\par- Отлично вижу, - ответил великан.
\par- Вот там и лежит корона. Прямо под ними. Они, конечно, взлетят, 
когда ты к ним приблизишься, но ты заметь место на воде и пощупай дно. 
Там в иле и лежит корона.
\par- А какая там глубина? - спросил великан.
\par- Тебе примерно до пояса, но я боюсь, что ты еще немного 
погрузишься в ил. Может быть, возьмешь что-нибудь вроде палки - ствол 
дерева, например?
\par- Не нужна мне никакая палка!
\parВеликан Яррах сделал большой шаг, и его громадный ботинок ступил в 
воду. Он двигался вперед, пока вода не дошла ему до пояса. Стая уток с 
кряканьем перелетела на другое место.
\parОн на мгновение замер, его силуэт резко обозначился на фоне 
залитой лунным светом воды. Потом он наклонился и опустил руку в воду. 
Но глубина была слишком большой, и дна он не достал. Тогда он сделал 
глубокий вдох и с головой ушел под воду. Минуту спустя он резко 
вырвался на поверхность и стал с такой силой отфыркиваться, что в 
воздух поднялись миллионы капель воды.
\parСначала Питеру даже показалось, что пошел дождь. Серая Шкурка 
бросила Питеру и Ловане плед, который она выхватила из своей сумки, и 
они в него закутались.
\par- Ох! - воскликнула Серая Шкурка, глядя на великана. - 
Оказывается, это всего лишь отфыркивается наш большой друг. А сейчас 
он снова ушел под воду.
\parНа этот раз великан оставался под водой минуты две, а когда, 
наконец, вынырнул на поверхность, то сделал такой выдох, что все 
деревья вокруг озера закачались, словно сквозь них пронесся порыв 
ветра.
\parПотом великан немного передвинулся и стал ощупывать дно в другом 
месте. На этот раз он нашел корону, выпрямился и поднял ее над 
головой. Корона сверкала, как скопление звезд. Выйдя на берег, он 
вручил ее Ловане, а она от волнения даже не смогла ее как следует 
надеть. Питер поправил на ней корону и отступил на шаг, чтобы 
посмотреть со стороны.
\parКорона излучала собственный свет, и все пространство вокруг озера 
озарялось ее лучами. Но красивое лицо Лованы словно освещалось и 
внутренним огнем. А Волшебный Лист, спрятанный в медальоне, придавал 
ей чувство собственного достоинства и величественную осанку. Она 
больше не была маленькой запуганной девочкой. Она стала принцессой 
Лованой.
\parДаже деревья, и те притихли и больше не шелестели листьями, как 
будто любое движение могло помешать им любоваться принцессой.
\parВ ярком свете короны изменился и Питер. Он вырос на два дюйма, 
раздался в плечах и превратился в статного юношу. Обернувшись к 
великану, он сказал:
\par- От имени Лованы и от меня позволь поблагодарить тебя, Яррах! 
Скоро мы навестим тебя - я хочу показать Ловане твою кухню.
\par- Вы всегда желанные гости в моем доме, - ответил великан. - 
Выберите только день, когда в кухне будет хорошая погода. А теперь 
давайте прощаться. Мне предстоит до рассвета проделать большой путь.
\parВскоре он зашагал прочь, и было слышно, как под его ногами трещат 
деревья. Когда же он переступил через ближнюю гряду гор, звук его 
шагов постепенно затих.
\parЗасобиралась и колдунья. Ей не терпелось до рассвета подмести чуть 
ли не пол-Луны.
\par- Слушай, Питер! - обратилась она к нему. - Когда у вас с Лованой 
свадьба?
\par- Послезавтра, - ответил Питер. - Мы надеемся, что ты почтишь нас 
своим присутствием.
\par- Конечно! Но вот что я хочу предложить: если вы еще не наняли 
фотографа, я хотела бы быть вашим официальным фотографом... если вы, 
конечно, не возражаете. У меня лучшая в мире коллекция фотоаппаратов. 
Правда, некоторые из них годятся лишь на то, чтобы снимать звезды и 
тому подобную чепуху. Но я могла бы сгонять на Луну и сделать оттуда 
дистанционный снимок. Представляете: вы с Лованой идете по подъемному 
мосту! Это была бы самая оригинальная фотография в мире!
\par- Мы с удовольствием назначим тебя нашим официальным фотографом, - 
сказала Лована, - и я буду рада иметь фотоснимок, сделанный с Луны.
\par- Что ж, договорились! - обрадовалась колдунья. - И, оставив 
позади себя голубую полоску пламени, стартовала в сторону Луны.
\par- Такой скоростной метлы я еще не видел, - проворчал Буньип, 
сладко спавший на берегу озера. - Настоящая спортивная модель, 
преемистость у нее просто страшная.
\parТут он огляделся и понял, что слушать его некому: Питер и Лована 
уже шли по дороге к замку. Бурча себе что-то под нос, Буньип поплелся 
следом.
\parКогда они подходили к большому дереву, их догнал Кривой Мик. 
Фаерфакс был весь в мыле, но голову держал высоко.
\parКривому Мику не терпелось узнать, как дела с великаном и 
колдуньей. И пока он чистил коня, Питер рассказал ему обо всем, что 
случилось, и о том, как они достали корону.
\par- Теперь ты выполнил все задания, - сказал Кривой Мик. - Осталось 
только сыграть свадьбу, а перед этим мне надо хорошенько выспаться.
\parОн отвел Фаерфакса в конюшню, завернулся в одеяло и лег прямо под 
звездами, не обращая внимания на храп Буньипа. Питер проводил Ловану в 
ее комнату, тоже забрался в спальный мешок, который ему дала Серая 
Шкурка, и вскоре заснул.
