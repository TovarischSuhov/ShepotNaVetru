sp07.gif}
    - Кто бы это мог быть? - в изумлении спросил Питер.
    - Это Буньип, который день и ночь сторожит Прекрасную Принцессу. 
Нам про него говорили.
    - Ах да, я и забыл.
    - Он очень злобный, - продолжала Серая Шкурка, но в голосе ее 
появились нотки неуверенности, так как спокойное похрапывание 
продолжалось. - По крайней мере, так говорят... Рыцарей и принцев он 
убивает десятками. Он их подпаливает и поджаривает, обдавая огнем из 
ноздрей. Посмотри-ка: не ноздри, а печные трубы. Давай лучше отойдем, 
да обсудим, как с ним бороться. Не нравятся мне его ноздри. Вмиг 
сделают из тебя поджарку. Идем.
    Она попятилась, но Буньип неожиданно вскочил и в изумлении 
уставился на них.
    - Что вы тут делаете? - проревел он. - Ваши имена? Кто вы такие? 
Стоять смирно! Произнесите по буквам "фантасмагория". Проводимые за 
обычную плату экскурсии для туристов начинаются в два часа. Вы входите 
через передние ворота, а выносят вас через заднюю калитку на носилках: 
а там уже наготове доктор! Теперь отвечайте, или вы умолкнете навеки!
    - Ну, - начал Питер, - если говорить о твоем замечательном 
приветствии, то оно показалось мне несколько путанным.
    - Твоя правда, - согласился Буньип. - Такой уж я путаник. Что 
дальше?
    - Ты стережешь Прекрасную Принцессу?
    - Да. А в чем дело?
    - Я пришел спасти ее!
    - Вот как! Жаль. Очень, очень жаль... Не люблю и никогда не любил 
убивать дружелюбных юношей. Но это моя работа. Я убиваю рыцарей на 
черных конях и на белых конях, убиваю любых принцев, - для меня нет 
разницы - убиваю без малейшего снисхождения. Прекрасная Принцесса 
должна быть защищена от всякого, кто захочет ее спасти.
    Она - если можно употребить такое словцо, - неспасабельна. И не 
забудьте, - поспешил он добавить, - что принцы и рыцари все равно 
погибнут, ведь самым настойчивым из них король дает три задания, и ни 
одно из них невыполнимо. А ваша смерть будет совершенно 
безболезненной. Гарантирую, вы ничего не почувствуете. Надеюсь, у вас 
не останется ко мне дурных чувств. Я вас опрысну, и все. Чистая, 
здоровая смерть, и никакого беспорядка.
    - Что значит - "опрыснешь"? Что это ты задумал? - возмутилась 
Серая Шкурка. - Ни я, ни Питер не дадим себя опрыскивать. И вообще - 
будь поосторожнее, если решил угрожать нам... Кое-кто из моих знакомых 
за это уже получил по носу.
    - Нет, вы только ее послушайте! - презрительно воскликнул Буньип. 
- Одна струя из моей ноздри, и ты отлетишь по этой дороге на сотню 
ярдов. О, женщина! Я в своем деле - чемпион мира, а ты говоришь, что я 
получу щелчок по носу. Ха-ха-ха-ха-а! - Буньип запрокинул голову и 
затрясся от смеха.
    - Я думала, ты выдыхаешь огонь, - сказала Серая Шкурка смущенно. - 
Как же ты можешь сторожить Прекрасную Принцессу, если ты не 
огнедышащий?
    - Что ж, сделаем так, - сказал Буньип. - Садитесь сюда и 
отдыхайте. Я вас не трону, пока не позавтракаю, так что можете 
расслабиться и чувствовать себя как дома. Я же пока обегу вокруг замка 
и прогоню людей и зверей, если кто-нибудь подошел слишком близко. Я 
вернусь быстро - минут через двадцать, а потом мы вместе позавтракаем. 
У меня приготовлены печеные лягушки - пальчики оближете. - Он 
облизнулся. - Когда вернусь, я расскажу вам о себе, а затем вы сможете 
бежать. Терпеть не могу убивать людей, которые не пытаются убежать. Я 
даю каждому фору в пятьдесят ярдов, прежде чем пускаю воду. Нельзя 
требовать от меня большего.
    Буньип поднялся и принялся подпрыгивать, чтобы разогреть мускулы. 
Выглядел он действительно очень необычно. Длинная шея поднимала его 
голову на высоту двадцать футов над землей. Он мог вертеть ею по 
сторонам, и ему не составляло труда даже так развернуть ее, чтобы 
посмотреть назад. Тело его было громоздким и тяжелым. Свой 
кенгуруподобный хвост при ходьбе он держал на весу, а во время бега 
покачивался и переваливался с ноги на ногу, как перегруженный корабль 
во время сильного шторма.
    Сейчас он пустился галопом по той широкой дороге, что огибала 
замок. Когда он поднимался в воздух, его шея подавалась вперед, а 
когда лапы снова касались земли, шея возвращалась в прежнее положение. 
Он бежал подобно жирафу, хотя ноги у него были короткими, как у 
медведя. На бегу Буньип выстреливал из ноздрей водяные струи и громко 
ревел. Эти струи сверкающими дугами вылетали из его ноздрей и 
подбрасывали в воздух случайно забредших сюда коров. При желании 
Буньип был способен посылать струи из любой ноздри, причем с такой 
силой, что коров по несколько раз переворачивало, прежде чем они, 
чихая и отплевываясь, могли подняться на дрожащие ноги.
    Если бы он продолжал выпускать свои струи еще некоторое время, 
животные бы просто утонули, но он удовлетворялся тем, что только валил 
их с ног. Когда же против него выходили рыцари и принцы, он делал вдох 
поглубже и посылал более длинную струю, которая ударяла во всадника, и 
тот с грохотом и лязганьем падал на землю. После этого можно было 
разглядеть только копья, мечи и дергающиеся ноги - все это торчало из 
воды, которая быстро спадала и растекалась. Мало кто из рыцарей 
выдерживал такие водяные удары. Они тонули вместе со всем своим 
снаряжением.
    Но в этот пригожий день ни принцев, ни рыцарей Буньипу не 
встретилось, и вскоре он тяжело подбежал к Питеру и Серой Шкурке. На 
его лице играла самодовольная ухмылка, а где-то сзади отфыркивались 
несколько коров.
    - Ну, как я? - крикнул он, едва остановившись. - Видели, как я 
подбросил вон ту корову одной несильной струей из правой ноздри?
    - По-моему, это жестоко, - сказал Питер. - Та корова не 
представляла ни для кого никакой опасности.
    - Опасность представляет всякий, кто приближается к замку, - 
заявил Буньип. - Если бы вы знали, сколько рыцарей и принцев мечтают 
жениться на Прекрасной Принцессе, вы бы ужаснулись. Сегодня еще 
спокойно. Лишь вы двое и подвернулись. А то этих спасителей набегает 
со всего света! Прекрасные Принцессы ныне - редкость. Наши могильщики 
копают для убитых могилы все дни напролет, кроме субботнего вечера и 
воскресенья, когда они ходят в церковь.
    Буньип уселся под деревом и развернул свой завтрак.
    - Нет ли у тебя перца и соли? - спросил он Серую Шкурку. - 
Королевская кухарка так невнимательна. А печеные лягушки без соли 
просто в горло не лезут.
    Серая Шкурка достала из сумки солонку и перечницу и передала 
Буньипу.
    - Удобная у тебя сумка, - похвалил он и продолжал: - Нет смысла 
предлагать вам разделить со мной трапезу, потому что я убью вас 
примерно через час. Если же у вас есть свой завтрак, то оставьте его 
мне: я съем его с чаем в три часа. А теперь я буду завтракать и 
рассказывать вам о своей жизни.
