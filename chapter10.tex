\parГлава 10
\parПОЯВЛЕНИЕ БУНЬИПА
\parРано утром Питер и Серая Шкурка собрались и тронулись в путь. 
Деревья, подсвеченные первыми лучами солнца, отбрасывали длинные тени, 
которые, играя, падали на тропинку, на Питера и Серую Шкурку.
\parПройдя около двух миль, они увидели замок: за поворотом тропинки 
открылось могучее строение с башнями, бастионами и зубчатыми стенами. 
В каменных стенах были прорезаны узкие окна-бойницы, зарешеченные, как 
в тюрьме. Без решетки оставалось лишь одно окно - под самой крышей 
огромной башни, оно было узким и скругленным сверху. Сотни птиц сидели 
на его подоконнике и кружились рядом, пытаясь добраться до еды, 
насыпанной на каменном карнизе. Питер подумал, что там-то, наверное, и 
находится комната Прекрасной Принцессы. Но ее он так и не увидел, 
должно быть, она была чем-то занята и не подходила к окну.
\parЗамок окружал глубокий ров с темной, неподвижной водой. 
Перебраться через него можно было только по подъемному мосту, который 
соединял ворота замка с дорогой, по которой пришли Питер и Серая 
Шкурка. В этот момент мост был поднят и удерживался наверху двумя 
мощными цепями, которые уходили в стену замка и где-то внутри 
соединялись с механизмом, опускавшим и поднимавшим мост.
\parПозади моста можно было разглядеть огромные сводчатые ворота, 
окованные медью. Металлические шарниры выделялись на фоне деревьев как 
солнечные шары. Ворота были такой ширины, что в них могли бы въехать 
бок о бок четыре всадника, и такой высоты, что сквозь них можно было 
пронести поднятые флаги и знамена. Сейчас ворота были заперты и 
прошибить их невозможно было никаким тараном.
\parВокруг всего замка шла вытоптанная темная дорожка. Иногда она 
отходила ото рва и исчезала среди кустарников, потом появлялась снова 
и шла все дальше и дальше по кругу, пока не соединялась со своим 
началом у ворот замка. У обочины этой дороги, примерно в ста ярдах от 
моста, рос старый эвкалипт: его широкие, раскидистые ветви 
образовывали пятно густой тени посреди жаркого света уже поднявшегося 
солнца.
\parПод деревом, сложив когтистые лапы на жирном брюхе, лежало на 
спине самое удивительное животное, какое когда-либо видел Питер. Если 
бы оно встало, то, наверное, всем своим видом и размером напомнило бы 
динозавра, но сейчас, спящее, оно выглядело не очень страшным.
\parПитер и Серая Шкурка вышли из кустов и, подойдя поближе к 
чудовищу, принялись его рассматривать. Оно было все, от носа до 
хвоста, покрыто шерстью. Тело его напоминало громадного вомбата, 
толстый, малоподвижный хвост - кенгуру, длинная шея - жирафа, а голова 
- дракона. Но вместо панцирной чешуи и острого гребня, какие бывают у 
драконов, у него была густая шерсть. На голове шерсть была длинной и 
неопрятной, она даже свисала ему на глаза, и было видно, что ее ни 
разу не расчесывали. Во сне чудовище храпело, и в такт храпу 
поднимались и опускались его скрещенные на брюхе передние лапы.
\par- Кто бы это мог быть? - в изумлении спросил Питер.
\par- Это Буньип, который день и ночь сторожит Прекрасную Принцессу. 
Нам про него говорили.
\par- Ах да, я и забыл.
\par- Он очень злобный, - продолжала Серая Шкурка, но в голосе ее 
появились нотки неуверенности, так как спокойное похрапывание 
продолжалось. - По крайней мере, так говорят... Рыцарей и принцев он 
убивает десятками. Он их подпаливает и поджаривает, обдавая огнем из 
ноздрей. Посмотри-ка: не ноздри, а печные трубы. Давай лучше отойдем, 
да обсудим, как с ним бороться. Не нравятся мне его ноздри. Вмиг 
сделают из тебя поджарку. Идем.
\parОна попятилась, но Буньип неожиданно вскочил и в изумлении 
уставился на них.
\par- Что вы тут делаете? - проревел он. - Ваши имена? Кто вы такие? 
Стоять смирно! Произнесите по буквам "фантасмагория". Проводимые за 
обычную плату экскурсии для туристов начинаются в два часа. Вы входите 
через передние ворота, а выносят вас через заднюю калитку на носилках: 
а там уже наготове доктор! Теперь отвечайте, или вы умолкнете навеки!
\par- Ну, - начал Питер, - если говорить о твоем замечательном 
приветствии, то оно показалось мне несколько путанным.
\par- Твоя правда, - согласился Буньип. - Такой уж я путаник. Что 
дальше?
\par- Ты стережешь Прекрасную Принцессу?
\par- Да. А в чем дело?
\par- Я пришел спасти ее!
\par- Вот как! Жаль. Очень, очень жаль... Не люблю и никогда не любил 
убивать дружелюбных юношей. Но это моя работа. Я убиваю рыцарей на 
черных конях и на белых конях, убиваю любых принцев, - для меня нет 
разницы - убиваю без малейшего снисхождения. Прекрасная Принцесса 
должна быть защищена от всякого, кто захочет ее спасти.
\parОна - если можно употребить такое словцо, - неспасабельна. И не 
забудьте, - поспешил он добавить, - что принцы и рыцари все равно 
погибнут, ведь самым настойчивым из них король дает три задания, и ни 
одно из них невыполнимо. А ваша смерть будет совершенно 
безболезненной. Гарантирую, вы ничего не почувствуете. Надеюсь, у вас 
не останется ко мне дурных чувств. Я вас опрысну, и все. Чистая, 
здоровая смерть, и никакого беспорядка.
\par- Что значит - "опрыснешь"? Что это ты задумал? - возмутилась 
Серая Шкурка. - Ни я, ни Питер не дадим себя опрыскивать. И вообще - 
будь поосторожнее, если решил угрожать нам... Кое-кто из моих знакомых 
за это уже получил по носу.
\par- Нет, вы только ее послушайте! - презрительно воскликнул Буньип. 
- Одна струя из моей ноздри, и ты отлетишь по этой дороге на сотню 
ярдов. О, женщина! Я в своем деле - чемпион мира, а ты говоришь, что я 
получу щелчок по носу. Ха-ха-ха-ха-а! - Буньип запрокинул голову и 
затрясся от смеха.
\par- Я думала, ты выдыхаешь огонь, - сказала Серая Шкурка смущенно. - 
Как же ты можешь сторожить Прекрасную Принцессу, если ты не 
огнедышащий?
\par- Что ж, сделаем так, - сказал Буньип. - Садитесь сюда и 
отдыхайте. Я вас не трону, пока не позавтракаю, так что можете 
расслабиться и чувствовать себя как дома. Я же пока обегу вокруг замка 
и прогоню людей и зверей, если кто-нибудь подошел слишком близко. Я 
вернусь быстро - минут через двадцать, а потом мы вместе позавтракаем. 
У меня приготовлены печеные лягушки - пальчики оближете. - Он 
облизнулся. - Когда вернусь, я расскажу вам о себе, а затем вы сможете 
бежать. Терпеть не могу убивать людей, которые не пытаются убежать. Я 
даю каждому фору в пятьдесят ярдов, прежде чем пускаю воду. Нельзя 
требовать от меня большего.
\parБуньип поднялся и принялся подпрыгивать, чтобы разогреть мускулы. 
Выглядел он действительно очень необычно. Длинная шея поднимала его 
голову на высоту двадцать футов над землей. Он мог вертеть ею по 
сторонам, и ему не составляло труда даже так развернуть ее, чтобы 
посмотреть назад. Тело его было громоздким и тяжелым. Свой 
кенгуруподобный хвост при ходьбе он держал на весу, а во время бега 
покачивался и переваливался с ноги на ногу, как перегруженный корабль 
во время сильного шторма.
\parСейчас он пустился галопом по той широкой дороге, что огибала 
замок. Когда он поднимался в воздух, его шея подавалась вперед, а 
когда лапы снова касались земли, шея возвращалась в прежнее положение. 
Он бежал подобно жирафу, хотя ноги у него были короткими, как у 
медведя. На бегу Буньип выстреливал из ноздрей водяные струи и громко 
ревел. Эти струи сверкающими дугами вылетали из его ноздрей и 
подбрасывали в воздух случайно забредших сюда коров. При желании 
Буньип был способен посылать струи из любой ноздри, причем с такой 
силой, что коров по несколько раз переворачивало, прежде чем они, 
чихая и отплевываясь, могли подняться на дрожащие ноги.
\parЕсли бы он продолжал выпускать свои струи еще некоторое время, 
животные бы просто утонули, но он удовлетворялся тем, что только валил 
их с ног. Когда же против него выходили рыцари и принцы, он делал вдох 
поглубже и посылал более длинную струю, которая ударяла во всадника, и 
тот с грохотом и лязганьем падал на землю. После этого можно было 
разглядеть только копья, мечи и дергающиеся ноги - все это торчало из 
воды, которая быстро спадала и растекалась. Мало кто из рыцарей 
выдерживал такие водяные удары. Они тонули вместе со всем своим 
снаряжением.
\parНо в этот пригожий день ни принцев, ни рыцарей Буньипу не 
встретилось, и вскоре он тяжело подбежал к Питеру и Серой Шкурке. На 
его лице играла самодовольная ухмылка, а где-то сзади отфыркивались 
несколько коров.
\par- Ну, как я? - крикнул он, едва остановившись. - Видели, как я 
подбросил вон ту корову одной несильной струей из правой ноздри?
\par- По-моему, это жестоко, - сказал Питер. - Та корова не 
представляла ни для кого никакой опасности.
\par- Опасность представляет всякий, кто приближается к замку, - 
заявил Буньип. - Если бы вы знали, сколько рыцарей и принцев мечтают 
жениться на Прекрасной Принцессе, вы бы ужаснулись. Сегодня еще 
спокойно. Лишь вы двое и подвернулись. А то этих спасителей набегает 
со всего света! Прекрасные Принцессы ныне - редкость. Наши могильщики 
копают для убитых могилы все дни напролет, кроме субботнего вечера и 
воскресенья, когда они ходят в церковь.
\parБуньип уселся под деревом и развернул свой завтрак.
\par- Нет ли у тебя перца и соли? - спросил он Серую Шкурку. - 
Королевская кухарка так невнимательна. А печеные лягушки без соли 
просто в горло не лезут.
\parСерая Шкурка достала из сумки солонку и перечницу и передала 
Буньипу.
\par- Удобная у тебя сумка, - похвалил он и продолжал: - Нет смысла 
предлагать вам разделить со мной трапезу, потому что я убью вас 
примерно через час. Если же у вас есть свой завтрак, то оставьте его 
мне: я съем его с чаем в три часа. А теперь я буду завтракать и 
рассказывать вам о своей жизни.
