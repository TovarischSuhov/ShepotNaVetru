Глава 14
        ПРЕКРАСНАЯ ПРИНЦЕССА

    Спрыгнув с подоконника, Питер оказался в просторной комнате, 
роскошное убранство которой потрясло его. Такого он еще не видал. На 
стенах висели бесценные гобелены и картины. Ковер с персидским 
орнаментом полностью покрывал пол. У стены стоял стол, на котором 
лежали прекрасные камни: опал и яшма, дымчатый кварц и болотный агат, 
янтарь, хризопраз и окаменевшее дерево. По ковру гонялись друг за 
другом два кролика. Под потолком свили гнездо ласточки и сновали туда-
сюда, нося червячков своим птенцам.
    У дальней стены стоял книжный шкаф. Питер разглядел три тома 
"Повелителя Колец", всю серию о Нарнии, "Алису в стране чудес", 
"Остров сокровищ" и несколько книг о лошадях. На шкафу стояла ваза с 
полевыми цветами, а рядом, свернувшись в клубок, спал опоссум. На 
диване - тоже спал - мальтийский терьер, похожий на шерстяной коврик.
    В центре комнаты за столом сидела самая красивая девушка, каких 
Питер когда-либо видел. Ее гладкие длинные волосы опускались почти до 
пояса и обрамляли лицо двумя золотистыми локонами, а голубые глаза 
были настолько глубоки, красивы и излучали столько света, что в 
комнате не оставалось ни одного темного уголка, и вся она освещалась 
ими, как солнцем.
    Серый дрозд, клевавший на подоконнике крошки, пытался воспеть ее 
красоту, но ему явно не хватало звуков. Даже дикие лебеди, голова к 
голове летевшие в вечернем небе, не могли сравниться красотой с 
принцессой. Не могли с ней сравниться и застенчивые орхидеи, 
пускавшиеся танцевать в лунном свете при малейшем дуновении ветерка.
    Один взгляд на принцессу делал человека счастливым.
    Принцесса писала сочинение, и на столе перед ней лежала раскрытая 
книга. Ни один человек ни разу не входил в ее комнату без 
предварительного доклада, и появление Питера, да еще из окна, поразило 
ее до чрезвычайности. Она поднялась и смотрела на него широко 
раскрытыми глазами. На ней было шелковое платье, сверкавшее голубым 
огнем, как крылья вьюрка и одновременно как лунное отражение на воде. 
Украшали платье алмазы. Только принцесса могла носить такой наряд.
    Питер не мог вымолвить ни слова. Он вообще не умел разговаривать с 
девчонками, а с принцессой и подавно. Он просто не знал, что сказать.
    Наконец, принцесса улыбнулась и спросила:
    - Ты - принц, который продирался сквозь леса и горы, чтобы меня 
спасти?
    - Я еще не совсем принц, - отвечал Питер. - Но уже наполовину им 
стал.
    - Одет ты как принц.
    - Да, и мой наряд с каждым днем становится все лучше. Как тебе 
нравится моя шляпа? Со страусиным пером?
    - Шляпа замечательная. Как хорошо, что перья именно страусиные. У 
некоторых принцев на шляпе перья цапли. Камилла, моя служанка, 
объяснила мне, что из-за этих перьев цапель убивают, и я поклялась, 
что никогда не буду иметь дело с тем, кто носит перья цапли. А теперь 
расскажи мне о себе.
    Питер рассказал ей, как он жил с Кривым Миком, и как он отправился 
спасать Прекрасную Принцессу. Рассказал о Серой Шкурке и ее волшебной 
сумке. Он говорил и говорил, а принцесса слушала, подавшись вперед и 
боясь пропустить хоть слово. Рассказ Питера чрезвычайно ее изумил, а 
когда она услышала, как Питеру пришлось ради нее бросить Серую Шкурку, 
она вскочила и воскликнула: "Мы должны ее снасти!" А подумав, 
добавила: "Но я ведь здесь в заточении и выхожу отсюда только под 
охраной. Все равно я что-нибудь придумаю, надо только чуточку 
потерпеть".
    - Можно и потерпеть, лишь бы быть уверенным, что ей ничто не 
грозит, - сказал Питер.
    - Ей ничто не будет грозить, - пообещала принцесса.
    - А почему тебя держат как в тюрьме? - спросил Питер. - Я ведь 
пришел спасти тебя. Я должен знать твое имя и вообще все о тебе.
    - Меня зовут Лована. Тебе нравится? На языке аборигенов оно 
означает "первая дочь".
    - По-моему, чудесное имя.
    - Мне тоже нравится.
    - Расскажи о себе еще.
    - О, мне столько надо тебе рассказать! Как замечательно, когда 
есть кому рассказывать. Сначала я расскажу, как я ощущаю мир.
    И она поведала Питеру о том, что мир полон тайн и красоты, о том, 
что ей хочется окунуться в него, словно сразу за окном начинается 
волшебный бассейн.
    - Я знала, что только в юности можно открыть свою душу внешнему 
миру, - говорила принцесса. - Мне хотелось все подержать в руках, 
прислониться щекой ко всему, что растет. Но в юности надо и учиться. 
Меня заперли в четырех стенах и не давали смотреть, как в лесу растут 
орхидеи. Мне пришлось склоняться в этой комнате над книгой и учиться, 
но страниц я почти не видела - мысленно я танцевала среди деревьев и 
подбрасывала в воздух цветы. И вот король и королева - а они очень 
строгие родители - решили не выпускать меня отсюда, пока я не сдам 
экзамены и не получу матрикул, и каждое утро они читают мне нотации. 
Кривой Мик читает тебе нотации?
    - Нет, - ответил Питер. - Он меня любит. Он ни за что не стал бы 
так делать. Когда я был маленьким, он показал мне всех птиц в лесу, 
всех зверей, все цветы. О жизни в буше я узнал все, что мог. Потом он 
читал мне книги обо всем на свете, а когда я подрос, я стал читать 
сам. Вот так я и выучился.
    - Мой папа все делает иначе, - сказала Лована. - Он любит стоять 
надо мной и говорить: "Я не желаю, чтобы моя дочь была более 
невежественной, чем другие девочки. Все дети в моем замке сдали 
экзамены и получили матрикул. Мне хочется тобой гордиться, а когда 
другие говорят мне, что их дочки уже сдали экзамены, мне от стыда 
некуда глаза девать. Я хотел бы иметь право ответить им, что моя дочь 
тоже получила матрикул. Как ты сможешь существовать в этом мире, если 
не будешь знать больше других? Твоя беда в том, что ты ленива и не 
хочешь заниматься. Мы с твоей матерью не можем на равных общаться с 
другими людьми, зная, что у нас глупая дочь. Как только ты сдашь 
экзамены, я тебя освобождаю. А до тех пор принцам и рыцарям бесполезно 
добиваться твоей руки. Я поставлю перед ними невыполнимые задачи и 
отрублю им головы".
    "Именно так, - говорит и моя мать - королева. - Подумать только, 
дочь сэра Реджинальда Малтраверса уже получила матрикул, а ей всего 
пятнадцать лет! И теперь весьма состоятельный принц предлагает ей руку 
и сердце. А он принадлежит к одной из лучших семей".
    "Это позор! - подводит итог король".
    - Как я рыдала, - продолжала Лована. - Мне никогда не сдать этот 
экзамен, и никогда не суждено выйти отсюда, если ты меня не спасешь.
    - По-моему, ты совсем не глупа, - сказал Питер. - Стоит на тебя 
взглянуть, чтобы это понять. Я тебя заберу отсюда, и, когда ты станешь 
взрослой, мы поженимся. Ты будешь жить у нас с Кривым Миком, помогать 
нам готовить завтрак, кормить птиц и животных. Мы будем убирать дом, 
застилать постели и угощать гостей чаем. Мы купим тебе небольшую 
лошадку, такую, как Мунлайт, и каждый день мы будем ездить по бушу. Я 
научу тебя колоть дрова для печки и жарить на углях котлеты. И еще мы 
будем читать сотни книг.
    - Как чудесно! - воскликнула Лована. - Мне здесь не разрешают 
делать ничего похожего, а мне бы хотелось этого больше всего на свете. 
Я с удовольствием буду колоть дрова и жарить на углях котлеты. А 
сейчас, если я проголодалась, мне стоит позвонить в колокольчики. 
Видишь, вон они стоят.
    Рядом с ее постелью висела полка, и на ней Питер увидел ряд 
колокольчиков.
    - Для разных людей разные колокольчики, - продолжала Лована. - 
Большой вызывает кухарку, следующий - горничную, и так до самого 
маленького, которым я зову учителя. Сегодня я должна ему позвонить 
после того как выучу исследователей Австралии. Утром мне надо будет 
перечислить их всех моему отцу. Скажи, если мы сможем пожениться, ты 
действительно разрешишь готовить вам с Кривым Миком еду, ухаживать за 
вами, когда вы будете болеть, кормить птиц и животных?
    - Ты сможешь все делать, но мы будем тебе помогать. Мы всегда 
помогаем друг другу. Если я заболею, то помогать не смогу. Тогда уже 
тебе придется делать все самой.
    - Как вы, наверно, замечательно живете! - сказала Лована. - Вы, 
наверно, такие счастливые!
    - Да, мы счастливы, - ответил Питер. - Но сейчас я беспокоюсь за 
Серую Шкурку. Хорошо бы ты сдала экзамен прямо сегодня. Тогда тебя бы 
отпустили, и мы смогли бы найти ее.
    - Я не очень умна, - грустно сказала Лована, - в этом вся беда. 
Эти экзамены я никогда не сдам.
    - Нет, сдашь, - сказал Питер. - Гляди, у меня для тебя подарок. - 
Он вынул из маленькой сумочки, висевший у него на шее, Волшебный Лист 
и протянул его принцессе. - Возьми его. Сожми в руке. Ну, пожалуйста. 
Это удивительный Лист. Вот увидишь.
    Лована взяла Лист и крепко его сжала. И понемногу она начала 
меняться. Она стала еще более прекрасной, чем раньше. Улыбка, 
озарившая ее лицо, была так непередаваемо хороша, что Питер чуть не 
закричал.
    Подарок изменил и Питера: он стал благороднее и мужественнее.
    - Ты вдруг так изменился... - сказала Лована.
    - Мне тоже это показалось... А ты сама что-нибудь чувствуешь?
    - Невероятно, но мне хочется сдавать экзамены прямо сейчас! Не 
понимаю, что со мной случилось?
    - Тебя любят, ты нужна людям, - ответил Питер. - Ты сдашь любой 
экзамен. Звони своему учителю.
    - Звоню. Спрячься в моей спальне, и я его позову.
    Она закрыла за Питером дверь спальни и позвонила в самый маленький 
колокольчик. Учитель явился мгновенно. Он уже стучал в дверь, когда 
принцесса еще не кончила звонить.
    - Войдите! - отозвалась принцесса. В комнату вошел сутулый человек 
в перепачканном мелом черном кителе и в очках со стальной оправой, 
причем очки сидели на кончике его носа так, что он мог смотреть поверх 
них.
    - Ваше Высочество закончили заниматься? - спросил он.
    - Да, закончила, и закончила навсегда, - ответила Лована. - Я хочу 
прямо сейчас сдавать экзамены. Принесите мне вопросы, пожалуйста.
    - Но, Ваше Высочество, - пытался возразить учитель. - Вы отлично 
знаете, что три раза вы уже проваливались. Вы еще совсем не готовы 
сдавать ни один экзамен. Понимаете, сидеть над раскрытой книгой - еще 
не значит выучить ее. Необходимо собраться. Кроме того, я уверен, что 
вы не помните дату казни Карла I. Или Уота Тайлера. Или разницу между 
полуостровом и мысом.
    - Я знаю абсолютно точно, в каком году он был казнен, и знаю, что 
казнить людей жестоко. Я помню все до единой книги, которые читала о 
мысах и об Уоте Тайлере. Немедленно дайте мне экзаменационные вопросы.
    Лована верила в Волшебный Лист.
    - Хорошо, хорошо, - проворчал учитель. - Но король будет страшно 
недоволен, если вы снова провалитесь.
    Он ушел за вопросами и через некоторое время вернулся, держа в 
руке несколько больших листов.
    - Итак, перед вами экзаменационные вопросы, - высокопарно начал 
он. - Садитесь за стол и отвечайте на них. Разговаривать запрещается. 
Я буду сидеть и следить за временем.
    Лована взяла вопросы и разложила перед собой на столе. Она ни 
капли не волновалась, как раньше. Она нисколько не боялась, поскольку 
знала, что сможет ответить на любой вопрос.
    Учитель достал из кармана жилетки массивные часы и посмотрел на 
них.
    - Я даю вам два часа на всю работу, - сказал он. - Приступайте к 
работе, когда я скажу "Начали".
    Несколько секунд он неотрывно смотрел на часы и затем скомандовал: 
"Начали!"
    Никогда еще Лована так быстро не писала. Ее перо скользило по 
бумаге без остановки. Ни один вопрос не поставил ее в тупик. Она то и 
дело бросала взгляд на Лист, сжатый в ее левой руке. Гора исписанных 
листков рядом с ней быстро росла, и, когда учитель крикнул: "Время!", 
Лована успела ответить на все вопросы.
    Учитель собрал листки и сел их проверять, а Лована стала наблюдать 
за его лицом. Возгласы, которые он издавал при чтении, не тревожили 
ее: "А-а! О-о! Ох!" - восклицал он.
    Закончив, учитель положил листы на стол и сказал:
    - Я просто не могу поверить. Вы сдали экзамен с наивысшими 
оценками. Поразительно! Вы были совершенной тупицей, Ваше Высочество! 
Я думал, у вас только красота и совсем нет ума. Что с вами произошло? 
Я обязан доложить королю. Теперь мне полагается повышение.
    Он убежал искать короля, а Лована пошла в спальню сказать Питеру о 
своем успехе.
    - Если это и есть образование, то я его получила, - сказала она. - 
Там не было ни одного вопроса о доброте и бескорыстии. От меня не 
требовалось проявить свое внимание и заботу. Ни слова о полевых цветах 
и птицах. Ни слова о жизни. Всему этому мне придется учиться у тебя и 
Кривого Мика.
    - Ты уже выучилась, - ответил Питер. А затем спросил: - Теперь ты 
свободна? - Он спешил спасти Серую Шкурку.
    - Да. Всем известно, почему я сидела взаперти, так что королю 
придется держать свое слово. Они с учителем сейчас придут. Встань 
рядом со мной, и я не буду бояться. Он попытается избавиться от тебя, 
дав три невыполнимых задания.
    - Мы не побоимся, - ответил Питер. - Волшебный Лист нам поможет.
