Глава 6
        ПИТЕР И СЕРАЯ ШКУРКА
        В ПЛЕНУ У БЛЕДНОЙ КОЛДУНЬИ

    Некоторое время спустя, проезжая по широкой, поросшей кустарниками 
долине, Питер почувствовал усталость. Он оглянулся и увидел, что Серая 
Шкурка немного отстала.
    - Что случилось? - крикнул он ей.
    Серая Шкурка допрыгала по тропе до дерева, стоявшего на краю 
поляны, и села под ним.
    - Сама не знаю, - ответила она. - Мне не хочется ничего делать. Я 
хочу лишь одного: сидеть под этим деревом и играть на гитаре.
    Она достала из сумки гитару и принялась наигрывать песенку. Питер 
соскочил на землю и сел рядом, опершись спиной о лежащее бревно.
    - У меня тоже пропало желание что-либо делать, - сказал он. - Я не 
хочу искать Прекрасную Принцессу. Мне бы хотелось остаться здесь и 
ничего не делать, мне бы хотелось поставить под этими деревьями хижину 
и просто проводить в ней все время. Интересно, что это на нас нашло?
    Вдруг из кустов вылетела кукабарра и уселась на ветку над головой 
Питера. Она начала смеяться, и Питеру и Серой Шкурке показалось, что 
она смеется над ними.
    - Почему ты смеешься? - спросил Питер.
    - Хм, - усмехнулась кукабарра. - Многие люди шли этой дорогой, и 
со всеми случалось одно и то же. Я сама видела. Им не хотелось 
работать. Не хотелось идти вперед. Они просто садились на землю и 
ничто их больше не волновало. И все оттого, что вы вступили во 
владения Бледной Колдуньи, и она вас околдовала. Теперь Питер 
позабудет о Прекрасной Принцессе. Вы перестанете к чему-либо 
стремиться. А потом придет Бледная Колдунья. Сначала она покажется вам 
доброй. Она пригласит вас к себе в гости, а затем убьет вас обоих и 
съест на ужин. Ясно вам? Она поедает людей. Это очень злая колдунья.
    Серая Шкурка швырнула гитару обратно в сумку и вскочила. Она очень 
сильно рассердилась.
    - Не желаю быть съеденной на ужин кем бы то ни было! Мы должны 
поскорее избавиться от этого ужасного чувства, которое нашло на нас. 
Давай поделаем упражнения для дыхания и попрыгаем, чтобы снова 
почувствовать силу. Нам необходимо сосредоточиться.
    - А я не хочу сосредоточиваться, - ответил Питер. - Я вообще 
ничего не хочу. Сейчас слишком жарко.
    Он снял бархатную куртку и бросил ее рядом на землю. Потом снял и 
бросил около бревна цепочку, на которой висела кожаная сумочка с 
Волшебным Листом. Сумочка упала в высокую траву.
    Лишившись Волшебного Листа, который придавал ему силы, Питер стал 
раздражительным, ему все теперь действовало на нервы. Он сердито 
смотрел на Серую Шкурку, которая прыгала взад-вперед, пытаясь 
избавиться от непонятной усталости.
    - Почему бы тебе не посидеть? - раздраженно сказал Питер. - Не 
будь дурой, это прыганье ничем не поможет.
    - Если я сяду, то засну, - объяснила Серая Шкурка. - А я не хочу, 
чтобы Бледная Колдунья застала нас врасплох.
    - Прыгай - не прыгай, все равно толку не будет, - пробормотал 
Питер, уже начиная клевать носом.
    Серая Шкурка с беспокойством посмотрела на него, потом пошла к 
лошади и отвела на крошечный островок зеленой травы. Погладив ее, 
сказала: "Никуда не уходи отсюда, Мунлайт. Что бы ни случилось, 
оставайся где-нибудь поблизости".
    Затем она вернулась к Питеру, который уже спал крепким сном.
    Серая Шкурка понимала, что колдунья должна быть где-то рядом, если 
чары ее усиливаются. Сама Серая Шкурка решила бодрствовать во что бы 
то ни стало. Она принялась трясти Питера, пытаясь его разбудить, потом 
выпрямилась и застыла, прислушиваясь. Она услышала, как кто-то 
пробирается к ним сквозь буш.
    Кукабарра, увидев с высоты колдунью, улетела прочь и растаяла в 
кронах деревьев. Колдунья, наклонив голову, медленно шла по узкой 
тропинке, проложенной в буше. Подойдя к Питеру и Серой Шкурке, она 
подняла голову и посмотрела на них. Глаза ее горели злобой. На ней 
была высокая черная шляпа и черный плащ. Нос был крючком, подбородок 
выгнут навстречу носу. В руках она держала метлу, а на плече у нее 
сидела ворона.
    - О, мои прелестные, - тихо сказала она, - пойдемте ко мне, я 
покажу вам кое-что очень интересное, приготовленное специально для 
вас.
    Колдунья внимательно их осмотрела. Пощупав руки и ноги Питера, она 
прошептала:
    - Очень пухленький, пойдет на пирог с мясом...
    - Идемте же, идемте, пообедаем вместе, - уговаривала она.
    - Неохота нам с тобой обедать, - сказала Серая Шкурка.
    - А я уверена, что твой приятель был бы не прочь, - ответила 
колдунья. Она коснулась плеча Питера, в тот открыл глаза. Сначала он 
смотрел ничего не понимающим взглядом и не замечая склонившейся над 
ним колдуньи. Потом у него в глазах появился страх, и он рывком сел.
    - Поднимайся, - заговорила колдунья вкрадчивым голосом, - 
поднимайся и иди за мной. Я твой друг. Я не причиню тебе зла. Идем же.
    Питер встал на ноги и стоял, глядя на колдунью. Его слегка 
покачивало.
    - Иди! Иди за мной! - приказала колдунья.
    Питер пошел за колдуньей словно через силу. Сначала он качался, 
как пьяный, но потом настойчивый голос колдуньи словно бы успокоил 
его, и он пошел увереннее и тверже.
    Серая Шкурка колебалась. Она не поддалась чарам колдуньи и поэтому 
могла бы остаться на месте, но ей не хотелось оставлять Питера одного, 
и поэтому она пошла тоже. По пути Серая Шкурка оставляла отметины, по 
которым можно было бы найти дорогу назад. Она не сомневалась, что 
Мунлайт никуда не уйдет и будет пастись недалеко от бревна.
    Следуя за колдуньей, Питер и кенгуру подошли к маленькой хижине, 
прятавшейся в зарослях акаций. Она выглядела довольно непритязательно: 
острая крыша, крошечные грязные оконца, на одной стороне - труба, на 
другой - крыльцо с навесом.
    Поднявшись на крыльцо, Питер вслед за колдуньей вошел в 
единственную комнату хижины. Это была странная комната. У огня сидел 
черный кот, на балке под потолком - сова, над огнем на цепях висел 
котел, из которого поднимался пар.
    В углу комнаты стояла большая клетка с распахнутой дверцей. 
Колдунья вошла в клетку, Питер - за ней. Серая Шкурка, оцепенев и 
словно не понимая, что делает, вошла тоже.
    Именно этого колдунья и ждала. Быстро повернувшись, она 
выскользнула за дверь и захлопнула ее за собой. Потом она заперла ее 
на замок, и Питер с Серой Шкуркой оказались в плену.
    Они, должно быть, сразу же заснули, потому что когда они вновь 
открыли глаза, был уже вечер. Колдунья сидела на стуле и чинила свою 
метлу. Питер и Серая Шкурка поняли, что чары колдуньи перестали 
действовать, потому что оба они соображали ясно.
    - Ага, проснулись, - сказала колдунья, поднимая голову, чтобы 
посмотреть на них. - Очень хорошо. Вы только не волнуйтесь. Чувствуйте 
себя, как дома. Я съем вас не раньше, чем через несколько дней, а пока 
хочу, чтобы вы хорошо ели и поправлялись. Сейчас мне надо лететь на 
Луну. Мне приходится подметать ее каждую ночь, потому что американцы и 
русские продолжают забрасывать туда всякую дрянь, работы там ужасно 
много. А пока сова, кот и ворона составят вам компанию. Я постараюсь 
обернуться побыстрей.
    С этими словами колдунья сунула метлу под мышку и вышла из 
комнаты. И тут же с улицы раздался свист рассекаемого воздуха, - это 
она полетела на Луну.
    - Ну и попали мы с тобой в переделку, - сказал Питер. - Я оставил 
Волшебный Лист рядом с курткой, и теперь мы не сможем превратить эту 
колдунью в добрую, пока не достанем его.
    - Отсюда не выбраться, - сказала Серая Шкурка. - Нам суждено 
набрать вес, а потом быть съеденными. Колдунья нас ни за что не 
выпустит, пока не раскормит, а тогда отправит прямехонько в котел.
    Питер и Серая Шкурка стали обсуждать разные способы спасения, но 
все они сводились к тому, чтобы завоевать дружбу колдуньи, а ведь это 
было невозможно...
    - Зачем тебе понадобилось снимать с шеи цепочку? - раздраженно 
спросила Серая Шкурка.
    - Даже не знаю. Я, наверное, спятил, - ответит Питер. - А почему 
ты меня не остановила? Тебе ничего не стоило это сделать, - сердито 
набросился он на Серую Шкурку.
    - Лист-то ведь был дан тебе, а не мне!
    - Ну и что! Все равно ты виновата не меньше!
    Серая Шкурка ничего не ответила.
    Через некоторое время Питер снова заговорил:
    - Интересно, откуда она взяла все эти фотоаппараты? Посмотри, вон 
сколько их валяется у стены.
    У стены на полу лежало свыше десятка фотоаппаратов и кинокамер, 
причем большая часть - со штативами. Тут же был и маленький 
механический экскаватор, какие обычно работают на батарейках.
    - Она, наверное, увлекается фотографией, - предположила Серая 
Шкурка. - Я спрошу, когда она вернется. Нам надо стараться быть с ней 
поласковей. Мне бы очень не хотелось разозлить ее.
    Колдунья вернулась лишь под утро. Когда она прилетела, Питер и 
Серая Шкурка спали. Она забегала по комнате, готовя им еду, но Питер и 
Серая Шкурка уже поужинали, воспользовавшись сумкой кенгуру, и поэтому 
от еды отказались. Это разозлило колдунью. Она пробормотала что-то 
вроде того, что добьется от них послушания, даже если ей придется 
морить их голодом.
    Серая Шкурка попыталась наладить контакт.
    - Вы интересуетесь фотографией? - спросила она.
    - Вовсе нет, - ответила колдунья. - Все эти фотоаппараты я 
подобрала на Луне. Их туда забрасывают ракеты, а я, когда ее подметаю, 
беру их себе. Этот экскаватор, когда я его нашла, сам рыл яму. Я 
принесла его сюда, но здесь он почему-то не работает. Зато у меня есть 
замечательная коллекция фотографий. Если хотите, могу показать. Я 
проявляла пленки, которые были в фотоаппаратах, и печатала фотографии.
    - Как-нибудь в другой раз, - сказала Серая Шкурка, она терпеть не 
могла разглядывать фотографии.
    Питер все время сидел, подперев голову руками, и думал, как бы 
обмануть колдунью. Неожиданно его осенило.
    - Какое блюдо вы собираетесь из меня приготовить? - спросил он.
    - А как ты сам хочешь быть приготовленным? - вопросом на вопрос 
ответила колдунья.
    - Пожалуй, я бы хотел быть поджаренным с приправой из травы бо-бо.
    - Я о такой не слыхала, - бросила колдунья, однако травой 
заинтересовалась.
    - Не слыхала о траве бо-бо! - воскликнул Питер. - Да ведь за этой 
травой люди гоняются по всему миру! Когда ею посыпают жаркое, оно 
приобретает удивительный аромат. Мне рассказал о ней Старина Мик, и он 
же научил меня, как ее найти. У нее такой изумительный вкус, что люди 
отдавали жизнь, чтобы только найти ее. Я бы не хотел быть изжаренным 
без травы бо-бо.
    - А где она растет? - заинтересовалась колдунья.
    - Я видел ее недалеко от того места, где ты нас захватила. Если 
хочешь, могу принести ее.
    - Скажи, как она выглядит, я принесу сама.
    - Ее трудно описать. Лучше я покажу ее тебе.
    Колдунья колебалась. Она была жадной и очень хотела заполучить эту 
травку, но боялась, как бы Питер не обманул ее.
    - Ты уверен, что от этой травки ты станешь вкуснее?
    - Абсолютно! Посыпь меня ею, и у тебя получится самое прекрасное 
блюдо, какое тебе только доводилось есть.
    - Хорошо, - согласилась колдунья, наконец. - Отведи меня туда, где 
она растет, но если попытаешься меня обмануть, я превращу тебя в жабу, 
и ты будешь ею до конца дней своих!
    Она открыла замок и выпустила Питера. Серая Шкурка, не 
перестававшая удивляться всему этому разговору, осталась в клетке 
одна. Но Питер, уходя, подмигнул ей, и она успокоилась.
    Оказавшись на свободе, Питер пошел к тому месту, где колдунья 
впервые напустила на них свои чары. Он хотел найти сумочку с Волшебным 
Листом, он точно знал, где ее оставил: около бревна, у которого они с 
Серой Шкуркой отдыхали.
    Питер шел впереди, - колдунья - за ним. Вот он увидел Мунлайт, 
которая паслась на островке зеленой травы, потом разглядел и бревно. 
Питер пошел медленнее, осматривая землю, словно выискивал траву бо-бо.
    - Здесь ли она? - забеспокоилась колдунья.
    - Где-то здесь.
    Около бревна Питер заметил блеснувшую в траве цепочку и шагнул к 
ней. Но колдунья тоже ее увидела. Она бросилась вперед и, обогнав 
Питера, склонилась над ней.
    Питер рассвирепел - из-за этой поганой колдуньи рушился его 
замечательный план! Колдунья, нагнувшись за цепочкой, оказалась перед 
ним в позе, которая вдохновила его. Он со всей силы дал ей пинка.
    {whisp03.gif}
    Удар оторвал ошарашенную колдунью от земли. Она перелетела через 
бревно и приземлилась на голову. Питер схватил сумочку, надел на шею 
цепочку и достал Волшебный Лист.
    Колдунья была вне себя от ярости. С диким криком, по-кошачьи она 
вскочила на ноги. Лицо ее исказила злоба. Растопырив руки и загребая 
воздух скрюченными пальцами, она завизжала, что навсегда превратит 
Питера в жабу.
    Питер испугался, но тем не менее протянул ей Лист и сказал:
    - Вот травка бо-бо. Возьми ее, прежде чем меня заколдовать.
    Колдунья замешкалась. Проклятье, уже готовое сорваться с ее губ, 
замерло. Она вырвала Волшебный Лист из рук Питера.
    - Давай сюда! - взвизгнула она.
    Как только ее рука коснулась Листа, с ней стали происходить 
поразительные перемены. Однако она так долго пребывала во власти зла, 
что смогла воспротивиться волшебной силе Листа. Она не хотела быть 
нужной, не хотела, чтобы ее любили. Она хотела только одного - 
ненавидеть.
    Колдунья бросилась на землю и, казалось, стала с кем-то сражаться. 
На губах у нее выкупила пена. Она молотила по земле сжатыми кулаками.
    - Не хочу! Не хочу! - кричала она. - Не хочу, чтобы меня любили!
    Но сила Волшебного Листа была все-таки больше силы колдуньи. Она 
долго каталась по земле, наконец, замерла, и еще через несколько 
секунд медленно встала. Отвратительное выражение исчезло с ее лица, на 
нем появилось сострадание и доброта. Она ласково посмотрела на Питера.
    - Прости меня, - сказала она совсем другим голосом.
    А Питер, вернув Волшебный Лист, только и думал, как бы помочь 
Колдунье. Он обнял ее за плечи и сказал:
    - Теперь ты станешь счастливее. Забудь прошлое. Ты научишься 
любить людей, а они полюбят тебя.
    - Наверное, я привыкну быть хорошей, - сказала колдунья, - но 
поначалу мне будет трудно. Обратно к хижине они шли рядом.
    - Хочешь, я сегодня помогу тебе подметать Луну? - предложил Питер.
    - Хочу, - обрадовалась колдунья.
    Когда они вошли в хижину, Серая Шкурка поразилась происшедшей 
перемене. Колдунья выпустила ее из клетки и попросила чувствовать себя 
как дома.
    Питер рассказал Серой Шкурке, как все произошло, а потом добавил:
    - Завтра мы с тобой отправимся дальше, а сегодня я хочу помочь 
колдунье подмести Луну. Чтобы доказать, что отныне мы - друзья.
    - Неплохая мысль, - одобрила Серая Шкурка и шепотом добавила:
    - А бархатную куртку ты взял?
    - Нет, она пропала.
    - Так я и думала, - огорчилась Серая Шкурка.
