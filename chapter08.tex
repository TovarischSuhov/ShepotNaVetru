Глава 8
\parЧЕЛОВЕК-СМЕРЧ ВИЛЛИ-ВИЛЛИ
\par\parВ эту ночь Питер и Серая Шкурка устроили привал на берегу реки. 
Шершавые стволы эвкалиптов словно разглядывали себя в водной глади; ее 
зеркальную неподвижность нарушали только платипусы - их горбатые спины 
то и дело высовывались на поверхность. Серая Шкурка достала из сумки 
жареное мясо, пирог и пудинг.
\parУтром друзья двинулись дальше по неровной, холмистой местности. По 
тропам меж скал медленно передвигались вомбаты, а с дерева на дерево 
перелетали малиновые попугаи. Возле одной каменистой горы заросли буша 
вдруг поредели, и жаркое солнце принялось без помех палить 
путешественников. Мунлайт вспотела: в том месте, где сбруя терлась о 
шею, выступила испарина. Ручьи и речки сбегали с противоположной 
стороны горы, а здесь перед путешественниками простиралась только 
голая равнина, слепящая глаза.
\par- Это Пустыня Одиночества, - объяснила Серая Шкурка. - Мой папа 
рассказывал, что люди, переходящие ее, часто умирают от жажды. Здесь 
никого не встретишь. Это глухое место, и перейти его можно только 
вместе с друзьями. Когда люди идут компанией, они разговаривают, 
смеются и забывают, в какую даль забрели и как трудно тут добыть воду. 
Как ни странно, именно они часто ее находят. Те же, кто идет в 
одиночку, не находят никогда.
\parНеожиданно на горизонте появился высокий пылевой столб. Вращаясь в 
воздухе, он двигался прямо на Питера и Серую Шкурку. Когда он 
приблизился еще, они разглядели листья и целые веточки, втянутые в 
столб. Если он проходил по сухой траве, то она, недвижно лежавшая на 
земле, вдруг бешено подскакивала и начинала носиться по кругу, а затем 
ныряла в столб и выбрасывалась наверх, где продолжала судорожно 
вращаться. Сухие листья и трава образовывали нечто вроде шапки над 
безумно пляшущим пылевым столбом, который поднимал вверх руки и 
опирался на землю ногами.
\par- Это человек-смерч Вилли-Вилли! - воскликнула Серая Шкурка, 
которая, похоже, знала все на свете. - Он - дядя четырех ветров и 
частенько здесь носится. Он любит принимать зримый облик, поэтому 
живет в Пустыне Одиночества, где много пыли и сухой травы, которые 
легко сорвать с места и поднять в воздух. А если он пляшет по зеленой 
траве, его почти не видно, - только трава машет, как бы приветствует 
его, зная, что это он идет.
\par{whisp05.gif}
\parСмерч обошел их вокруг. Поднятые песчинки больно хлестали Питера и 
Серую Шкурку, так что им пришлось прикрыть глаза и отступить. Тогда 
вращающийся пылевой столб дернулся и остановился, и откуда-то из его 
глубины выпрыгнул маленький человечек. Он было пошатнулся, по тут же 
выпрямился.
\par- Добрый день! - заговорил он. - Простите, что меня немного 
качает. Просто я двигался со слишком большой скоростью, чтобы успеть к 
вам.
\parОсвобожденный от хватки человека-смерча, пылевой столб распался, 
превратившись в легкий туман, и скоро совсем исчез.
\parУ человека-смерча Вилли-Вилли было улыбающееся загорелое лицо. 
Носил он куртку и брюки цвета красной земли, как песок пустыни, как 
краска, которой разрисовывают себя аборигены. Он приподнялся на носках 
в своих гибких ботинках и, раскинув руки в стороны, несколько раз 
обернулся вокруг себя.
\par- Предпочитаю останавливаться постепенно, - пояснил он. - Вращение 
с такой скоростью, какая была у меня, требует большого внимания. - Он 
потер руки, очищая их от пыли, и продолжал: - Я уже говорил, вращение 
- это счастье движения. Вопрос лишь в том, танцую ли я потому, что 
счастлив, или же счастлив потому, что танцую!
\par- А кому ты это говорил? - удивился Питер.
\par- Одному дубу. Он вечно вздыхает. Он вздыхает, когда нет ветра, он 
вздыхает, когда есть ветер... Я посоветовал ему танцевать и махать 
своими руками-ветками. Любой из нас ведь может загрустить, когда 
просто стоит и ничего не делает, верно?
\par- Пожалуй, - согласился Питер. - Но я никогда не слышал, чтобы 
дерево танцевало.
\par- Ну как же! - воскликнула Серая Шкурка, шокированная невежеством 
друга. - Что же, по-твоему, деревья делают, когда ты спишь? А спишь ты 
треть своей жизни. Деревья же в это время танцуют.
\par- Никогда не думал об этом, - признался Питер.
\par- Они танцуют, покачивая ветвями, - продолжал Вилли-Вилли. - Они 
раскачиваются вместе с тенью и скользят вместе с Луной. Между прочим, 
о тебе мне тоже рассказало дерево. - Маленький человечек улыбнулся. - 
Оно сказало: "Когда Питер подойдет к Пустыне Одиночества, он будет 
нуждаться в друге. Встреть его и перенеси через пустыню, хорошо?" И 
вот я здесь.
\par- Разве она так велика, что мы не сможем перейти ее сами? - 
удивился Питер. - Я поеду верхом, а Серая Шкурка поскачет рядом.
\par- Конечно, на самом деле пустыня не такая уж большая. Но тем 
детям, которые идут по ней в одиночку, без друга, она представляется 
бесконечной, и они легко сбиваются с пути. Я тебя перенесу через нее, 
не беспокойся, и для этого сделаю не простой смерч, а смерч-громадину! 
Это будет здорово!
\par- Меня при вращении тошнит, - заявил Питер.
\par- Меня тоже, - поддержала Серая Шкурка. - Больше шести вращений я 
не выдерживаю. После этого я отключаюсь.
\par- Правда? - удивился Вилли-Вилли. - Но внутри смерча будет 
совершенно спокойно, и вы совсем не будете вращаться. Вы будете сидеть 
на воздушной подушке, а пыль и листья будут вращаться вокруг вас. - Он 
огляделся. - Я сейчас попробую завестись и протанцую несколько миль по 
пустыне, пока не наберу двадцати ярдов в диаметре, а потом вернусь и 
подхвачу вас.
\par- Тревожно мне что-то, - шепнула Серая Шкурка. - Я понимаю, что 
друзья необходимы, но не такие, от которых на тебя накатывает дурнота.
\par- Вилли-Вилли ведь обещал, что мы не будем вращаться, - успокоил 
ее Питер. - Он наш друг, а друзья никогда не делают ничего во вред и 
не пугают.
\par- Ладно уж, рискнем, - согласилась Серая Шкурка. - Когда мы 
отправляемся? - обратилась она к Вилли-Вилли.
\par- Жди, я скажу, - ответил человек-смерч. - Ты, Питер, садись на 
лошадь, а ты, - он обернулся к Серой Шкурке, - стой рядом и держись за 
стремя.
\parОни так и сделали: Питер сел верхом, а Серая Шкурка стала рядом. 
Она крепко сжала кожаный ремень и закрыла глаза.
\parПитеру стало страшно. Он позвал Вилли-Вилли:
\par- Послушай, прежде чем мы отправимся, я хотел бы подарить тебе 
этот Лист, - и он вручил Вилли-Вилли Волшебный Лист.
\parМаленький человечек был тронут.
\par- Какой чудесный подарок! - воскликнул он. - Не беспокойся, я буду 
осторожен и внимателен, ведь ты - мой лучший друг. Я не стану делать 
смерч таким огромным, как говорил. Если честно, то я чуть-чуть 
прихвастнул, двадцать ярдов в диаметре мне не осилить.
\parОн сунул Лист в карман, приподнялся на носки и, вытянув руки в 
стороны, начал вращаться. Где-то внутри пего несколько раз чихнул 
мотор, но пыль в воздух не поднялась. Человек-смерч остановился и сел.
\par- Видите ли, - стал он оправдываться. - Мотор у меня двухтактный, 
а двухтактные всегда заводятся плохо. Знаете, как трудно завести 
газонокосилку? У вас случайно нет при себе шнура?
\par- Сейчас дам, - отозвалась Серая Шкурка. Она пошарила в сумке и 
вытащила пусковой шнур с деревянной ручкой на конце.
\par- Ничего себе! - изумился Вилли-Вилли. - Будь у меня такая сумка, 
я сколотил бы целое состояние. В любом приличном магазине такой шнур 
продается не меньше, чем за полдоллара. Ты не могла бы поставлять их 
мне крупными партиями? Беру двенадцать дюжин, естественно, с 
десятипроцентной скидкой за оплату наличными.
\par- Не говори глупостей, - рассердилась Серая Шкурка. - У меня тут 
не лавка запчастей. Я люблю людей и просто дарю то, что им нужно.
\par- Конечно, конечно, - сконфузился Вилли-Вилли. - Прошу простить 
меня, - и он погладил Лист.
\parЧеловек-смерч попытался еще раз запустить себя. Он обмотал 
пусковой шнур вокруг талии и попросил Питера дернуть за него.
\par- Лучше всего я завожусь, если дернуть сильно и резко - одним 
рывком, но изо всей силы, - объяснил он.
\parПитер дергал несколько раз, затем попробовала Серая Шкурка, но 
мотор все не подхватывал и даже ни разу не чихнул. Человек-смерч 
вращался, пока не кончался шнур, а потом останавливался.
\par- Не понимаю, что с ним случилось, - озабоченно произнес Вилли-
Вилли. - Я прошел капитальный ремонт всего месяц назад. Надо серьезно 
подумать о переходе на четырехтактный.
\par- Наверно, загрязнилась свеча зажигания, - предположил Питер. - А 
как зазор между контактами - нормальный?
\parОн все знал о двухтактных моторах, потому что у Кривого Мика была 
газонокосилка, которая тоже плохо заводилась.
\par- Сейчас посмотрим, - с надеждой сказал Вилли-Вилли.
\parОн сунул руку в карман, достал оттуда свечу и принялся пристально 
ее разглядывать. Потом лизнул ее, потер о рукав, снова лизнул, снова 
потер о рукав, и после этого поднес к глазам.
\par- Кажется, чистая... - решил он.
\par- А зазор между контактами? - снова спросил Питер.
\parВилли-Вилли протянул свечу Питеру и тот постучал ею о камень.
\par- Попробуй еще раз.
\parВилли-Вилли отправил ее в карман, и Питер снова дернул за шнур.
\par- Та-та-та, - затянул человек-смерч, - та-та-тр-тр-тр-тр, - и он 
начал вращаться.
\parПитер, все еще со шнуром в руке, отскочил в сторону, и человек-
смерч пронесся мимо, увлекая за собой пыль.
\par- Стойте вместе, я через минуту вернусь и заберу вас! - крикнул он 
на ходу.
\parОн понесся по пустыне, подбирая на ходу пыль. Она вилась в 
воздухе, напоминая спущенную с небес гигантскую веревку, которая к 
тому же раскачивалась туда-сюда.
\parПотом он начал гудеть, гул разрастался и наконец вся пустыня 
наполнилась неистовым ревом. Всосанная пыль бешено крутилась огромными 
кругами, быстро поднимаясь и превращая веревку в столб, а столб - в 
кошмарную движущуюся башню. Чахлые деревца, над которыми она 
проносилась, неистово вскидывали головы. Смерч срывал с них листья и 
бросал вверх, чтобы они кружились вместе с пылью на высоте пятьсот 
футов. Смерч отошел примерно на милю, а потом по кругу стал 
возвращаться назад. От его грома дрожала земля.
\parПеред собой смерч гнал стаю кенгуру, которые затем свернули куда-
то в сторону. С его пути удирали, также на полной скорости, несколько 
эму, вытянув вперед длинные, похожие на копья, шеи, и поднимая лапками 
фонтанчики пыли.
\par- Не нравится мне это, - сказала Серая Шкурка. - Совсем не 
нравится.
\par- Мне тоже, - признался Питер. - Мне даже немножко страшно.
\parОн вскочил на Мунлайт, а Серая Шкурка стала рядом, держась за 
стремя.
\par- Я же дал ему Волшебный Лист, - сказал Питер, - так что зла он, 
нам, конечно, не причинит. Но я все равно волнуюсь
\par- А я не боюсь, - решительно заявила Серая Шкурка, - просто Вилли-
Вилли себя не контролирует. Он плохой водитель, вот что! Держись! - 
неожиданно крикнула она.
\parСмерч развернулся и двигался теперь прямо на них. Воздух вокруг 
заколыхался и грива Мунлайт забилась на ветру. Серая Шкурка спрятала 
голову в седло. Питер нагнулся и уткнулся в гриву Мунлайт.
\parЗавывающий столб пыли возвышался над ними. Листья и веточки 
кружились в вихре, как мотыльки вокруг света. Смерч грохотал, 
скрежетал, мотаясь из стороны в сторону на всей своей страшной высоте, 
словно пораженный болью гигантский удав. Внезапно он накрыл Питера и 
Серую Шкурку, после чего они ощущали только вой ветра.
\parПотом Питер и Серая Шкурка почувствовали, как невидимые руки 
подхватили их и втянули внутрь. Мунлайт рванулась, встала на дыбы, 
вместе с ней поднялась и Серая Шкурка, инстинктивно сжимавшая стремя. 
Она изо всех сил держалась за него, хотя лапы ее вдруг ослабли и она 
чуть не разжала пальцы. Питеру казалось, будто он сидит на 
необъезженной лошади, но он лишь сжал колени, и продолжал сидеть в 
прежней позе; он чувствовал, как вертит под ним лошадь. Неожиданно их 
обоих выхватило из тьмы и пыли и перенесло в спокойное недвижное место 
- это было высоченное пространство с гладкими вращающимися стенами. 
Впечатление было такое, словно они попали в середину гибкой трубы, 
уходящей в необозримую высоту и теряющуюся среди грозовых туч. Здесь 
царил покой, снаружи сюда еле доносился, словно из-за стены, рев 
могучего смерча.
\parСам Вилли-Вилли сидел на стуле и пил чай, налив его из фляги. Его 
стул неподвижно висел в воздухе. Рядом стояли еще два.
\par- Садитесь, - пригласил Вилли-Вилли, жестом указывая на стулья. - 
Я всегда держу их под открытым небом, чтобы они были рядом, если мне 
вдруг захочется проехаться с комфортом.
\par- А ты уверен, что они нас выдержат? - спросила Серая Шкурка. - 
Они же ни на чем не держатся.
\par- А в самолетах? - вопросом на вопрос ответил Вилли-Вилли. - В 
самолетах ведь есть кресла.
\par- Да, но там все иначе, - не соглашалась Серая Шкурка.
\par- Да эти стулья слона выдержат! - уверил ее Вилли-Вилли.
\par- Что ж, - сказала Серая Шкурка, - посмотрим. - И она выдернула из 
сумки слона. Это был тот самый слон, которого она вытащила, когда 
впервые встретила Питера. Слон был взбешен.
\par- Послушай, ты! Уж не собираешься ли ты вытаскивать меня из своей 
проклятой сумки каждый раз, когда вам приспичит разрешить спор? - 
раздраженно взревел он. - Это уже переходит всякие границы! Мне это 
надоело!
\par- Я хочу лишь, чтобы ты сел на этот стул, - спокойно объяснила 
Серая Шкурка. - Давай, дружище, садись!
\parСлон, недовольно ворча, сел.
\par- Ну вот видите! - воскликнул Вилли-Вилли. - Не шелохнулся!
\par- Ты прав, - сдалась Серая Шкурка. Затем обратилась к слону: - 
Спасибо, дружище. Извини, что побеспокоили.
\par- Чтоб это было в последний раз! - по-прежнему сердился слон. Он 
хотел добавить что-то еще, но Серая Шкурка схватила его и головой вниз 
бросила в сумку, где он и исчез. Потом она опустилась на стул.
\par- Имея такой талант, следует соблюдать осторожность, - сказал 
Вилли-Вилли. - Ведь смерч не предназначен для переноски слонов. Он мог 
бы вообще развалиться, и мы сломали бы себе шеи. Да и так у нас упала 
скорость. Надо перейти на пониженную передачу. Сейчас я переключу на 
вторую.
\parОн достал из кармана рычаг переключения передач и прикрепив к 
ботинку, стал крутить туда-сюда, пока смерч вновь не начал набирать 
обороты.
\par- Тр-тр, тр-тр, - громыхал он. Вилли-Вилли опять поднял рычаг в 
верхнее положение и смерч стал вращаться быстрее и быстрее.
\parПитер сел на другой стул и устроился поудобней. Мунлайт стояла 
рядом, равнодушная и к окружающему гулу, и к тому, что стоит на 
воздухе. Она только задирала голову и настороженно шевелила ушами.
\parПитер погладил и похлопал ее по шее.
\par- Внимание! - неожиданно произнес Вилли-Вилли подчеркнуто 
серьезным тоном. - Желаю вам приятного путешествия! Мы летим на высоте 
пятьсот футов, подгоняемые попутным ветром. Координаты нашего 
местонахождения: 65 минут долготы и 23 минуты широты. К югу и северу 
от гор местами туман. Температура в Дарвине 89 градусов, в Мельбурне 
52 градуса. В аварийной ситуации пристегните ремни безопасности и 
надуйте плот, находящийся в сиденье под вами. Сохраняйте его в надутом 
виде во время приземления. Непосредственно перед посадкой поместите 
плот под себя. Благодарю за участие и сотрудничество.
\parНапоследок он с важным видом кашлянул, чтобы все поняли, какой он 
блестящий оратор.
\par- Ничего не понимаю. Что-то у тебя все с ног на голову поставлено, 
- отозвался Питер. Речь эта показалась ему верхом глупости.
\par- Никто тебя и не просит искать ноги и головы в том, что я сказал, 
- обиделся Вилли-Вилли. - Новые головы и ноги никому не нужны - у всех 
есть свои. А говорил я от чистого сердца, - с чувством закончил он.
\parНа Серую Шкурку речь, наоборот, произвела впечатление.
\par- Очень интересно, что в Дарвине температура 89 градусов, - 
сказала она. - Но меня беспокоит надувание плотов при приземлении. Это 
очень ответственно, особенно в случае аварии.
\par- Все это ерунда! - гнул свое Питер.
\parНеожиданно труба, в которой они сидели, наполнилась сеном. Оно 
летало над головами, кружилось внизу под их стульями, - там его было 
особенно много. Оно щекотало им кожу, вилось вокруг ног, а потом 
уходило вверх и в конце концов припечатывалось к стенам. Где-то далеко 
внизу, на земле, сердито кричали люди, и труба усиливала их крики.
\par- О боже! - воскликнул Вилли-Вилли. - Мы переехали стог сена! Оно 
принадлежит четырем карликам, что живут в пещере. Слышите, как они 
кричат? Надо будет вернуть сено, когда я вас высажу. Ой-ей-ей! 
Настоящий грабеж!
\par- Отличное сено, - сказал Питер, разглядывая колосок. - В нем 
много овса, то, что нужно для лошадей. - Ему вдруг очень захотелось 
показать это сено Кривому Мику.
\par- Надо определить, где мы находимся, - озабоченно сказал Вилли-
Вилли. - Я несколько отклонился от курса.
\parОн вытащил из-под стула подзорную трубу и раздвинул ее - она 
оказалась довольно-таки длинной. Просунув один конец сквозь стену 
смерча, он приложился к окуляру.
\par- Мы идем на северо-юго-восток со скоростью сто узлов, - объявил 
он. - В заливе Порт-Филип ветер свежий.
\par- Но отсюда до Порт-Филипа тысячи миль, - сказал Питер. Он уже 
начал сомневаться, знает ли вообще Вилли-Вилли, куда несется.
\par- Но я и вижу на тысячи миль в свою подзорную трубу, - ответил 
Вилли-Вилли. - Сейчас мы проходим мимо высоких скал. Вижу огромного 
кенгуру-валлаби. А сейчас вижу одинокого человека, бредущего по 
пустыне, он сбился с пути.
\parВилли-Вилли резко сложил трубу и повел смерч по дуге, чтобы 
выравнять курс и взять заблудившегося. Смерч подхватил его, как 
листок, втащил наверх и поставил рядом со всеми.
\par- Хочу пить, - сказал человек. - Я сбился с пути, и не мог идти 
дальше.
\parСерая Шкурка достала из сумки бутылку лимонада и протянула ему; он 
жадно его выпил,
\par- Нельзя переходить Пустыню Одиночества без друга, - сказал Вилли-
Вилли. - Тебе повезло, что мы оказались неподалеку.
\par- Это верно. Я увидел смерч, но никак не подумал, что в нем может 
кто-нибудь находиться. Признаться, я даже испугался, когда он меня 
подхватил. - Внезапно он замолк, прислушиваясь, как постукивает смерч, 
затем заявил: - Надо подрегулировать мотор. Он неправильно дает искру.
\par- А ты разбираешься в двухтактных? - изумился Вилли-Вилли. - Кто 
ты по профессии?
\par- Механик-моторист.
\par- Двухтактные заводить можешь?
\par- Запросто.
\par- Тогда могу предложить тебе работу, - сказал Вилли-Вилли, 
довольный собой. - Как тебя звать?
\par- Том.
\par- Хорошее имя. Короткое и простое. Мы с тобой будем друзьями, Том.
\parНеожиданно труба выбросила вверх листья, которые покружились 
вокруг и прилипли к стене.
\par- Мы подошли к опушке Недреманного Леса, - объяснил Вилли-Вилли. - 
Я, наверное, зацепил эвкалипт. Но я ему сейчас все возмещу. Готовьтесь 
к посадке. Пристегните ремни!
\par- Но на стульях нет никаких ремней! - воскликнул Питер, торопливо 
шаря по стулу.
\par- Ты прав, их нет, - нехотя согласился Вилли-Вилли. - Тогда не 
пристегивайтесь. Держитесь - опускаемся! Вууу-у! - И он закрыл глаза.
\parСмерч сделал круг и остановился. Вращение прекратилось, и все 
стали медленно, как листья, опускаться. Они опускались все ниже и 
ниже, пока не почувствовали под ногами землю. В неподвижном воздухе на 
них начало падать сено, листья и пыль, и скоро вокруг них вырос целый 
стог.
\parОчутились они на опушке Недреманного Леса. Здесь густо росли 
эвкалипты, которые умели подслушивать мысли и шепотом передавать 
дальше, так что вскоре уже весь лес знал о появлении гостей и об их 
планах.
\par- Только добрые люди могут пересечь этот лес, - начал человек-
смерч. - Волшебный Лист защитит вас. Я перенес вас через Пустыню 
Одиночества, как и обещал. Больше вы никогда не будете одиноки. Тем, 
кто хоть раз пересек эту пустыню, одиночество не страшно. Видите вон 
ту гору? - Он указал в сторону пика, торчавшего из-за деревьев на 
горизонте. - Это Последняя Гора. Идите к ней. Там вы узнаете ответы на 
все ваши вопросы. - Он повернулся к Тому. - А теперь заводи меня, Том. 
Такого человека, как ты, я искал всю жизнь. Обмотай меня вокруг пояса 
шнуром и дерни.
\par- Где шнур? - спросил Том.
\par- Ой-ей-ей, я, наверно, потерял его! - воскликнул Вилли-Вилли, 
ощупывая карманы.
\parСерая Шкурка вытащила связку шнуров и протянула Тому:
\par- Следи за ними, Том, он вечно их теряет. И за ним тоже следи. Он 
хороший человек, но ему необходим друг и помощник.
\par- Не беспокойся, я послежу за ним, - пообещал Том.
\parВсе вместе они пролезли сквозь стог и выбрались на чистое место. 
Мунлайт по пути жадно хватала сено, поэтому Питер взял две охапки и 
отнес ей в сторонку под дерево, где она могла спокойно поесть.
\parЧеловек-смерч стоял на клочке свободной земли, среди пыли, и Том 
обматывал его шнуром.
\par- Я вернусь за тобой, Том, - сказал Вилли-Вилли. - Встань на копну 
и жди меня. Нам надо отдать сено карликам.
\par- Понятно, всем отойти! - скомандовал Том и дернул за шнур. 
Человек-смерч несколько раз кашлянул, чихнул и завращался в облаке 
пыли. Он вернулся в пустыню, там начал раскачиваться и расти, пока не 
поднялся до небес. Вскоре он развернулся и устремился туда, где Том 
ждал его, стоя на сене. Поднимаясь в воздух на огромной копне, Том на 
прощанье махнул рукой Питеру и Серой Шкурке и исчез в пыли. Смерч 
полетел прочь, становясь все меньше и меньше, пока не исчез совсем.
