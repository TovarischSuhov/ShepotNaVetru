Глава 15
        ТРИ ЗАДАНИЯ

    Крики команд и топот охраны возвестили о приближении короля. Дверь 
рывком распахнулась, и вошел король, а следом за ним и королева. Она 
тут же бросилась обнимать Ловану, - ее прямо-таки распирало от 
гордости.
    - Дорогая, - восклицала она. - Я так счастлива. Леди Малтраверс 
умрет от зависти, когда узнает о твоих оценках. Ты сдала лучше, чем 
обе ее дочери.
    Но король был взбешен. Он увидел Питера и отметил его великолепную 
одежду и гордую осанку.
    - Кто посмел сюда вторгнуться? - вскричал он. - Взять его!
    Он махнул охране, солдаты рванулись с места, но Лована шагнула 
вперед и подняла руку.
    - Я призываю вас выполнить ваше же обещание, - обратилась она к 
королю. - Вы объявили, что когда я получу матрикул и стану свободной, 
я буду вправе сама выбирать друзей.
    - Пропади ты пропадом! - выругался король. - Разве я такое обещал? 
Наверно, у меня помутился рассудок.
    - А слово короля - закон, - напомнила принцесса.
    - Все ясно, - быстро произнес король, - но как бы то ни было, 
нельзя требовать от короля так много. Прошу тебя никогда больше не 
напоминать мне ни о чем подобном. Матрикул ты, может быть, и получила, 
но вот тактичности тебе явно не хватает. Как человеку деловому, мне бы 
хотелось, чтобы переговоры велись на моих условиях. Не волнуйся, я не 
трону этого принца, или кто он там есть, но ты должна знать, что 
согласие на брак с тобой я дам только тому, кто сможет выполнить три 
задания, которые я назову.
    - Я их выполню, какие бы они ни были, - сказал Питер. - Назовите 
их.
    - Хорошо, - ответил король. - Во-первых, ты должен привести 
человека, который сможет солгать лучше меня.
    - Боже мой! - прошептала Лована на ухо Питеру. - Мой отец - 
величайший лгун всех времен. Нет такого человека, кто мог бы его 
превзойти.
    - Такой человек есть, - прошептал Питер в ответ. А королю он 
заявил: - Я приведу такого человека, при условии, что судьей будет 
Буньип.
    Королю это понравилось. Буньип всегда был верен ему, и он не 
сомневался, кого тот объявит победителем в состязании лгунов.
    - Идет, - ответил король. - Я согласен назначить Буньипа судьей.
    - А какими будут другие задания? - спросил Питер.
    - Если тебе удастся найти человека, который превзойдет меня во 
лжи, твоим следующим заданием будет усмирить Фаерфакса, дикого коня из 
Глухих Гор. Я хочу, чтобы его привели ко мне, оседланного, 
взнузданного и пригодного для езды.
    - Не слыхал о таком, - сказал Питер.
    Король рассмеялся.
    - Услышишь, - сказал он, потирая руки. - Он опаснее дракона, 
быстрее, чем удар молнии, и подпрыгивает выше деревьев. Еще никому и 
никогда не удавалось набросить на него уздечку. Если он тебя и не 
убьет, то подбросит так высоко, что друзья успеют подготовиться к 
твоим похоронам, прежде чем ты упадешь на землю.
    Питер не мог представить такого коня, но он знал, что с помощью 
Кривого Мика он его усмирит.
    - А третье задание? - спросил он.
    - Третье же задание, - сказал король, - будет самым главным. 
Корона, которую должна надеть Лована, если ей когда-нибудь суждено 
выйти за тебя замуж, лежит на дне озера. Туда ее забросила колдунья, 
которая подметает луну. Она ее украла, когда меня не было дома. Со 
мной бы у нее такие штучки не прошли, это точно. В наши дни никому 
нельзя верить.
    - Это та самая колдунья, что летает с кинокамерой?
    - Да, с чрезвычайно дорогой камерой.
    - Это последнее задание, которое я должен выполнить, чтобы 
получить руку принцессы? - спросил Питер, сообразив, что речь идет о 
той самой колдунье, с которой он встретился на пути в замок.
    - Да, последнее. Боюсь, однако, что тебе и жизни не хватит, чтобы 
достать золотую корону со дна озера, даже если тебе удастся выполнить 
первые два. Ну, а коли ты все выполнишь... принцесса - твоя.
    Король смахнул пушинку со своего камзола и объявил:
    - Отныне вы оба имеете право на свободное передвижение в пределах 
замка до тех пор, пока не придет время начать первое испытание.
    Он кашлянул в кулак, чтобы подчеркнуть собственную значительность, 
и приказал страже удалиться.
    - Идем, - позвал он королеву. - Надо обговорить с Буньипом условия 
состязания лгунов.
    Обернувшись к Питеру, он бросил:
    - Состязание состоится завтра днем в Большом Зале. До этого срока 
покидать замок тебе не разрешается.
    Когда король с королевой ушли, Лована сказала:
    - А теперь пойдем искать Серую Шкурку.
    Они побежали наверх, где Питер оставил Серую Шкурку, но там никого 
не было. Тогда они спустились и пошли по коридорам, минуя стражу, 
которая получила приказание их не трогать, и потому беспрепятственно 
пропускала. Серую Шкурку они нашли в саду, запертую в загончике из 
металлической сетки, натянутой на столбах. Выглядела она вполне 
сносно, поскольку верила, что Питер и принцесса ее спасут.
    - Я знала, что вы придете, - сказала она. - И надеялась, что это 
случится скоро. Кормят меня здесь ужасно - плесневелым сеном. И я не 
могла разговаривать - с тех самых пор, как меня схватили. Через день-
другой меня бы пустили на консервы для собак. А я пыталась думать о 
чем-нибудь другом. Но ответь, почему и ты, и принцесса свободно 
разгуливаете?
    - Ее зовут Лована. - сказал Питер.
    - Это имя мне правится, - сказала Серая Шкурка. - Оно звучит, 
словно дуб что-то шепчет на ветру...
    - Я рада, что оно тебе понравилось, - отозвалась Лована.
    Питер поведал Серой Шкурке обо всем, что произошло. Он рассказал о 
трех заданиях, которые ему предстоит выполнить, и о том, как он 
собирается ударом Громобоя вызвать на помощь Кривого Мика.
    - Что ж, не стоит терять время, - посоветовала Серая Шкурка. - 
Если состязание лгунов состоится завтра, то Кривому Мику надо дать 
время придумать какую-нибудь потрясающую историю, которая бы короля 
припечатала.
    Они вышли на середину двора и Питер развернул длинную плеть 
Громобоя. Собравшись с силами, он начал описывать им над головой 
круговые движения, пока тот не образовал вращающееся в воздухе колесо. 
Тогда Питер резко бросил руку вниз и назад, и в ответ громыхнуло так, 
словно выстрелила пушка. Солдаты и офицеры, которые стояли оперевшись 
о стену, подпрыгнули от испуга, но когда обернулись, чтобы узнать, в 
чем дело, то увидели только морщинистого погонщика, который стоял 
перед Лованой и ее другом.
    {whisp10.gif}
    Кривой Мик был удивлен не меньше солдат. Он сидел у дома и чинил 
седло, как вдруг в одно мгновение очутился во дворе замка. Но Питер 
быстро ему все объяснил, и Мик сел на скамьи у стены обдумать 
создавшееся положение.
    - В состязании лгунов я без труда уложу короля на обе лопатки, - 
промолвил он. - Но этот Фаерфакс меня беспокоит. Как мы его поймаем, 
ума не приложу.
    - Я догоню его на Мунлайт, - сказал Питер. - Не может он бежать 
быстрее нее. А догнав, я вскочу на него.
    - А где он обычно пасется?
    - Живет он в Глухих Горах, - ответил Питер. - Мы его запросто 
найдем.
    - Это может оказаться не так-то просто, - ответил Кривой Мик. - Но 
пока отложим это. Как насчет третьего задания? Озеро, наверно, 
глубокое, а ныряльщик я никудышний.
    - Об этом не беспокойся, - сказал Питер. - Я знаю, кто мне 
достанет корону. Двух первых заданий я боюсь больше.
    - Состязание лгунов, считай, у нас в кармане, - заверил его Кривой 
Мик. - Куда королю со мной тягаться! Завтра щелкни кнутом, когда надо 
будет идти в Большой Зал, и я буду тут как тут. А пока я вернусь домой 
и дочиню седло.
    Он встал и исчез, подняв за собой вихрь из пылинок.
