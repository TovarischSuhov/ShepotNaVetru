\chapter{ВОЗВРАЩЕНИЕ В ЗАМОК}
\par\parНа обратном пути Серая Шкурка обогнала Кривого Мика и ускакала 
вперед. Она хотела предупредить короля об их приходе и сообщить 
Ловане, что Фаерфакс, Дикий конь из Глухих Гор, пойман и усмирен. 
Питеру хотелось, чтобы Лована первой узнала эту новость, поэтому Серая 
Шкурка, прискакав к замку, растолкала храпящего Буньипа и попросила 
отвести ее к принцессе.
\parБуньип зевнул и сонно пробурчал:
\par- Мне приказано убивать с наименьшей болью всех желающих увидеть 
Прекрасную Принцессу. Готовься, пока я вскипячу котелок.
\par- Не дури! - рассердилась Серая Шкурка. - Со сном никак не 
расстаться? Давай, просыпайся поскорей.
\parБуньип рывком сел.
\par- А, это ты! - воскликнул он. - А где Питер? Поймал он дикого 
коня?
\par- Об этом я расскажу только принцессе, - твердо сказала Серая 
Шкурка. - Отведи меня к ней.
\parБуньип медленно встал и, еле переставляя ноги, побрел к замку, 
что-то бормоча себе под нос. У рва он позвонил в колокольчик, и 
подъемный мост, погромыхивая цепями, тут же опустился, глухо 
ударившись о берег.
\parБуньип пересек мост, перебросился несколькими Фразами с 
приветствовавшей его охраной и повел Серую Шкурку к комнатам, где жила 
Прекрасная Принцесса. Они постучали в дверь и стали ждать.
\parКогда Лована открыла дверь, на ее устах играла улыбка. Она слышала 
прыжки Серой Шкурки по коридору и догадалась, что та несет вести от 
Питера. Принцесса протянула к ней руки, и они крепко обнялись.
\parБуньип смотрел на них с отвращением. Он не верил во всякие там 
поцелуи и объятья и потому пробурчал:
\par- Женщины есть женщины! Когда они встречаются, они выставляют себя 
на посмешище. Только на нервы мне действуют, а на нервной почве у меня 
разыгрывается мигрень.
\parУ него и в самом деле разыгралась мигрень, и он поглаживал голову, 
входя в комнату следом за Лованой и Серой Шкуркой.
\parКогда все расселись. Серая Шкурка рассказала с начала до конца 
историю укрощения Фаерфакса. И закончила следующими словами:
\par- Они прибудут в замок с минуты на минуту. Зови короля, Лована. Он 
должен видеть их приезд, чтобы лично убедиться в успехе Питера. 
Наверно, он придет в ярость...
\parЛована позвала слуг и отправила их в покои короля. Буньипу она 
велела погромче прореветь приказ: всем собраться под большим деревом.
\par- Пусть твой голос разнесется на многие мили, - закончила она.
\parБуньип остался доволен. Он вышел к подъемному мосту, откашлялся и 
с таким жаром несколько раз прокричал объявление, что разбудил всех 
обитателей замка.
\parЛована надела свое лучшее платье, и они с Серой Шкуркой тоже 
вышли.
\par- Это наверняка одно из лучших твоих платьев, - сказала Серая 
Шкурка.
\par- Ты думаешь, Питеру оно понравится? - спросила Лована.
\par- Да, - ответила Серая Шкурка.
\parКогда они шли по мосту, люди уже собирались под деревом. Для 
короля и королевы поставили переносные троны, Лована и Серая Шкурка 
сидели на траве рядом с ними. Король с королевой выходили из замка, их 
приветствовали хлопками, но когда подъехали Кривой Мик и Питер, им 
устроили настоящую овацию.
\parПри выезде из буша Кривой Мик придержал Мунлайт, и последние сто 
ярдов Питер проехал на Фаерфаксе один. Жеребец выглядел как нельзя 
лучше. Его огненная шерсть, которую Питер тщательно расчесал при 
приближении к замку, сверкала, как начищенная медь. Фаерфакс шел 
легким галопом, раскачиваясь из стороны в сторону и выгнув шею, как на 
параде, и поднялся на дыбы, когда Питер остановил его перед королем. 
Король был заворожен его красотой. Он шепотом спросил у одного из 
придворных:
\par- Сколько такая лошадь может стоить на свободном рынке?
\par- Тысячу долларов, - ответил придворный почти не открывая рта. 
Король радостно потер руки.
\par- Так значит, ты привел мне Дикую лошадь из Глухих Гор, укрощенную 
и пригодную к езде.
\par- Да, он укрощен, Ваше Величество, - ответил Питер. - Именно в 
этом и было мое задание, но я не привел его вам в качестве подарка. Он 
принадлежит Кривому Мику, который его поймал.
\par- Что за вздор! По какому такому праву конь принадлежит Кривому 
Мику?
\par- По праву того, кто его поймал. Если бы не Кривой Мик, мы бы 
никогда его не поймали. Вы дали слово, что второе задание будет 
считаться выполненным, когда Фаерфакс будет приведен к вам оседланный, 
укрощенный и пригодный к езде. О том, что он будет принадлежать вам, 
ничего не говорилось.
\par"Боже милостивый! - воскликнул про себя король. - Они так 
произносят это "слово короля", как будто мы не люди. Но ничего, третье 
задание его доконает, я отрублю ему голову и конь будет моим".
\parА вслух он произнес:
\par- Второе задание ты выполнил удовлетворительно, но на принцессе ты 
сможешь жениться только когда выполнишь третье и последнее задание, а 
если не выполнишь - я тебя обязательно казню, или, по крайней мере, 
это сделает Буньип, - добавил он самодовольно.
\par- Э-э... да... конечно, - запинаясь, выдавил из себя Буньип, придя 
в замешательство. - Это запросто, да... да. Одной струи из правой 
ноздри и двух из левой ему должно хватить, - и Буньип подмигнул 
Питеру.
\par- Ты должен выполнить третье задание, - продолжал король. - Ты 
должен принести мне золотую с алмазами корону, которую колдунья, 
подметающая Луну, забросила в озеро после того, как принцесса ее 
обидела.
\par- Я только столкнула ее с подоконника, - оправдывалась принцесса. 
- Она царапалась в стекло и напугала меня.
\par- Знаю, знаю, - раздраженно проговорил король. - Но она сорвала 
драгоценную корону с твоей головы, и прежде чем швырнуть ее в озеро, 
произнесла заклятье.
\par- Какое заклятье? - спросил Питер. Лована смотрела на него с 
беспокойством.
\par- А такое... Принцесса сможет выйти замуж, только если у нее на 
голове будет та самая корона с алмазами. - Вид у короля был печальный: 
корона стоила по меньшей мере полмиллиона долларов. Он вздохнул и, 
наклонившись к уху придворного, спросил: - А сколько бы она стоила 
сейчас?
\par- Миллион долларов, - ответил тот.
\par- Боже милостивый! - воскликнул король.
\parШатаясь, он сошел с трона и побрел в замок, оплакивая свою потерю.
