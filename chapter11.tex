Глава 11
        РАССКАЗ БУНЬИПА

    - Вы удивляетесь, почему я не выдыхаю дым и огонь, - начал Буньип, 
проглотив порцию лягушачьих лапок. - Потому что это не нравится, вот и 
все.
    Понимаете, дело вот в чем. Я родился в болоте - там рождаются все 
буньипы. Мои родители были обычными буньипами без каких-либо особых 
талантов. Папа любил реветь по ночам, пугая людей до смерти. В 
остальное время он ловил рыбу и охотился на животных, спускавшихся к 
водопою. Мы жили хорошо. Но папу волновало наше общественное 
положение. Другие буньипы жили в просторных пещерах, пол которых от 
стены до стены был покрыт болотной травой. А мы жили в норе, где на 
полу всегда стояла слякоть. Папа и мама считали, что единственный 
способ поднять наше общественное положение - эхо добиться для меня 
должности охранника Прекрасной Принцессы. Это была ответственная 
работа, и отец смог бы этим хвастаться.
    И вот он послал меня в Школу Драконов, где драконят обучали 
охранять Прекрасных Принцесс. Директором школы был старый рыцарь по 
имени Святой Георгий. Вы случайно не слыхали о таком? Когда-то он 
здорово сразился с драконом и якобы прикончил его. Я не верил ни 
одному его слову начиная с того момента, как он в первый же раз бросил 
нам "Привет".
    - Я о нем слышал, - сказал Питер. - Он действительно убил дракона. 
Есть такой рассказ - "Святой Георгий и Дракон". Это был замечательный 
человек.
    - Да, именно так он себя и называет, - презрительно бросил Буньип.
    Питер решил с ним не спорить. Не потому, что боялся, - ведь в руке 
он держал Волшебный Лист, - просто не любил спорить.
    - Нашими учителями были рыцари, которые прежде сражались с 
драконами, - продолжал Буньип. - Они должны были обучить нас бороться 
с рыцарями так, чтоб никто из них спасти Прекрасную Принцессу не мог.
    Это была не школа, а чистый ужас. Нас поднимали ни свет ни заря, 
выстраивали колонной и вели на завтрак. Подавали нам расплавленную 
лаву и горящие угли. Считалось, что это самая подходящая еда для 
драконят, которым предстоит научиться пускать на врагов пламя с дымом. 
Но у меня от такой еды было только несварение желудка.
    После завтрака нас выстраивали на площадке в шеренгу, напротив 
становились рыцари на конях. Они были закованы в лучшие доспехи, 
которые так сияли на солнце, что слепили нам глаза. Их кони радостно 
ржали, задирали морды и били копытами. Рыцари наставляли свои копья и 
мчались на нас с криками: "Ату их, ату! Защищайтесь, сэр! Иду на вас, 
мошенник! Руки вверх!" - и тому подобными. Конечно, мы старались. Мы 
выдыхали пламя и страшно много дыма. Но мы ведь были еще дети и могли 
дохнуть пламенем только футов на десять. После хорошего завтрака кое-
кому из учеников удавалось достичь и двенадцати, но у меня дальше 
шести никогда не выходило. Так что оставалось просто увертываться от 
коней. О, боже! Как мы вертелись! Мы путались у них под ногами, 
хлестали их хвостами и рычали, скалили зубы, кусались и даже делали 
сальто, лишь бы увернуться от копья. Из-за моего меха мне доставалось 
больше всех. Драконята целиком покрыты чешуйками, и копья с них просто 
соскальзывали. Иногда им удавалось даже перекусить копье пополам. Мне 
было хуже. Острие копья вырывало у меня шерсть клочьями и продирало 
между ребрами целые борозды. Я ревел как бык, опустивший голову под 
воду - такой боевой клич буньипов - но на рыцарей он не производил ни 
малейшего впечатления.
    Если мы ломали им копья, они лупили нас мечом плашмя. От нашего 
огня их спасали доспехи. А чтобы в них не изжариться, они с нами 
подолгу не возились и в конце концов прогоняли с площадки.
    Рыцари любили греть походный котелок и, поливая чай из жестяных 
кружек, хвастаться друг перед дружкой тем, по скольку драконов они 
уничтожили, прежде чем стали учителями. Они сидели на траве и 
потешались, а нам приходилось выслушивать эти россказни из-за забора.
    Меня они просто бесили. И однажды ночью, наслушавшись стонов и 
тяжелых вздохов драконят в нашей спальне, я кое-что придумал. На 
следующее утро, когда мы собрались на площадке, я бросился к озеру 
перед нашей школой и втянул в себя столько воды, что уровень озера 
упал на два фута. Знаете, как иногда булькает в животе? Вот так 
булькало и у меня. Это шипели раскаленные угли, когда вода гасила во 
мне огонь. Я ничего не сказал, а вернулся и встал в строй вместе с 
другими драконятами. Когда же рыцари на нас поскакали, я послал в них 
такую струю воды из каждой ноздри, что всех их повалило в одну кучу. 
Дело в том, что я сумел пустить одну длинную струю. Они решили, что 
прорвалась плотина. Они вопили, бранились, пытались снова взобраться 
на коней, но вода сбивала их с ног и они снова падали, растеряв в 
конце концов все свое оружие.
    Драконята ликовали. Все хотели научиться пускать воду вместо огня, 
но я не мог их этому научить. У драконов желудки изнутри выложены 
железом, чтобы там мог гореть огонь. От воды оно бы стало ржаветь, и у 
них образовалась бы язва. А согласитесь, что дракон с язвой от воды - 
жалкое зрелище...
    Как бы то ни было, один из учителей кинулся в школу и привел 
Святого Георгия. Как же! Ученики посмели бросать вызов установленному 
порядку. Это же бунт! Если бы драконы переделали свои желудки и стали 
извергать воду вместо дыма и огня, Святой Георгий остался бы без 
работы. Поэтому он надел новый шлем с таким огромным забралом, что 
когда оно было опущено, он уже ничего ни видел, ни слышал... Он надел 
новые латы со специальным приспособлением для закрепления конца копья 
и сел на боевого коня, на котором было навешано столько доспехов, что 
он скакал с превеликим трудом.
    Святой Георгий опустил забрало и крикнул нам:
    - Отправляйтесь по своим комнатам и выдыхайте дым два часа.
    Никто не шелохнулся.
    У меня в желудке еще оставалась тысяча галлонов воды, так что я 
молчал и ждал, когда он нападет, но он не спешил. Он повторил свое 
требование: всем разойтись по комнатам и выдыхать дым. Но драконята 
сгрудились вокруг меня. Они были напуганы, и один из них признался: 
"Во мне не осталось ни одной искорки. Я не смогу дохнуть огнем. Он же 
убьет нас всех, чтоб другим было неповадно. Ты точно сможешь его 
остановить?"
    - Увидишь, - ответил я.
    И вот Святой Георгий двинулся на меня. Его могучий конь медленно 
разгонялся и, подгоняемый шпорами, громыхнул галопом. Святой Георгий 
выпрямил в стременах ноги и подался вперед в седле. Привстав на 
стременах и закрепив копье, он медленно опускал его до тех пор, пока 
оно не оказалось направленным прямо в меня. Все рыцари его 
подбадривали. "Берегись! Это сам Святой Георгий, победитель Дракона!"
    Когда между нами оставалось пятьдесят ярдов, я попробовал дальний 
удар и пустил ему в наколенники две струи из каждой ноздри. Потом я 
немного поднял голову и пустил несколько струй ему в набедренники, 
наручники, грудной панцирь, оплечье и забрало. Он покачнулся. В конце 
концов я выпустил ему в нагрудник все полтораста галлонов. О, видели 
бы вы это зрелище! Он отлетел к лошадиному хвосту, раскорячив ноги и 
молотя по воде руками, затем с треском и грохотом бухнулся наземь. 
Гром раздался такой, будто я перевернул повозку с жестяными котелками. 
Конь же его продолжал приближаться. Я ударил из правой ноздри прямой 
наводкой ему в грудь, он встал на дыбы и понесся прочь.
    {whisp08.gif}
    Видели бы вы Святого Георгия! Он поднимался по частям, словно 
складной метр, и грозил мне железным кулаком. "Ты исключен!" - крикнул 
он. Его так и подмывало броситься на меня даже без коня, но он был 
трусоват. "Пошел вон из моих владений!" - продолжал он вопить.
    Так я и сделал и отправился к тому месту, где забор шел вплотную к 
дороге. Но я не мог не думать о папе с мамой. Родители, которые 
надеются улучшить свое положение с помощью успехов детей, очень 
сердятся, когда с детьми происходит нечто подобное. Ты просто не 
имеешь права их разочаровывать, вот в чем беда. Я никогда об этом не 
забывал. Понятно, что мне было невесело, и я загрустил.
    Буньип и в самом деле загрустил даже при воспоминании об этом.
    - Теперь все позади, - сказал Питер. - Сейчас мы с Серой Шкуркой 
вскипятим котелок. Чай пить будешь?
    - Буду что? - поразился Буньип.

