\parЯ написал "Шепот на ветру" после того, как один старый русский 
писатель объяснил мне, что я уже достаточно вырос, чтобы писать для 
детей. Ему не нравилось, когда молодые писатели начинали с детских 
произведений, а потом переходили к взрослым. По его мнению, все должно 
быть наоборот. Это был Самуил Маршак, очень известный писатель, 
который, однако, сам свою первую книжку написал именно для детей.
\parПо возвращении домой мне пришла в голову одна идея. Я задумал 
написать сказку на австралийском материале, в которой перемешались бы 
колдуньи, эльфы и драконы - и, конечно, ее героем не мог быть никто 
другой, кроме этого ужасного лжеца, Кривого Мика.
\parЯ никогда ничего не писал специально для детей, мои произведения 
никогда не были только плодом воображения, я всегда придерживался 
реальной жизни. Но когда я начал работать над сказкой и почувствовал 
всю свою свободу, я подумал, как это замечательно. От работы над 
книгой я получил истинное удовольствие...
\par\parАлан Маршалл
\par(Из книги "Алан Маршалл рассказывает")
