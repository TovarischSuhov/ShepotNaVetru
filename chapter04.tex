Глава 4
\parСЕРАЯ ШКУРКА СРАЖАЕТСЯ С ВЕЛИКАНОМ
\par\parСерая Шкурка и Питер прошли совсем немного, когда далеко на юге 
стали собираться черные тучи. Раздались отдаленные раскаты грома, 
усилился ветер. Когда тучи приблизились, ударила молния. Огненные 
зигзаги ринулись в провалы между облаками и устремились к деревьям. За 
этим последовали оглушительные удары грома, и деревья в смятении 
зашумели кронами.
\par- Дождь может пойти с минуты на минуту, - произнес Питер. - 
Поскакали вон к тем скалам впереди. Может быть, там мы найдем пещеру.
\par- А не найдем, так придется помокнуть, - сказала Серая Шкурка.
\parОни свернули с дороги и, продираясь через густые заросли 
вечнозеленых кустарников, напрямик помчались к стеной возвышающимся 
скалам. К счастью, им удалось обнаружить тропу, проложенную вомбатами. 
По ней они дошли до прогалины у подножья скал и прямо перед собой 
увидели огромную пещеру, в которой, наверное, жили многие поколения 
вомбатов.
\parКогда они уже входили в укрытие, начали падать первые крупные 
капли дождя. В пещере было сухо, на песчаном полу валялись камни, на 
которых можно было сидеть.
\par- Мы успели как раз вовремя, - сказала Серая Шкурка, встряхиваясь. 
- Пожалуй, здесь можно заночевать.
\parМощный удар грома заглушил ее слова, и молния осветила пещеру, 
словно в нее вдруг прорвалось солнце. В пещеру залетали капли дождя, и 
вместе с ними влетел Южный Ветер, стряхивая брызги с волос.
\par- Я так и думал, что найду вас где-нибудь здесь, - сказал он, 
усаживаясь.
\par- Честное слово, я рад тебя видеть! - воскликнул Питер.
\par- Я следил за тобой, - сказал Южный Ветер. - Да и мои братья тоже 
о тебе рассказывали. Деревья в этих местах давно уже жаждут дождя, вот 
я и решил навестить тебя, заодно напоив буш. Видал, какой я принес 
ливень, а? Деревьям он понравится. Но скоро мне придется уйти. Нужно 
промчаться еще сотни миль, прежде чем эта громада облаков иссякнет. 
Послушай-ка, я скажу тебе кое-что. Ты вступил в земли великана Ярраха.
\par- А кто это?
\par- Это исполин, о котором я тебе говорил. Он нападает на всех, кто 
проходит по его землям.
\par- И что же он с ними делает - убивает?
\par- Нет, не убивает. Он всех хватает и запихивает в маленькие 
коробки, а продержав там несколько месяцев, выпускает. Однако к этому 
времени все люди становятся одинаковыми. Они разучиваются думать сами, 
и могут только повторять, что они читали или слышали. Нет ничего 
ужаснее, чем стать такими, поэтому я пришел предупредить тебя: будь 
осторожен.
\par- А он очень большой?
\par- Ростом он с гору, а ботинки у него длиной в пять футов.
\par- А что он делает с кенгуру? - поинтересовалась Серая Шкурка. - 
Неужели их тоже запихивает в коробки?
\par- К сожалению, да.
\par- Но меня нельзя запихнуть в маленькую коробку, - не соглашалась 
Серая Шкурка, - ведь тогда сломается мой хвост. - Ее стала беспокоить 
подобная перспектива. - Никакому великану не позволено ломать мой 
хвост.
\par- Можно сделать отверстие внизу коробки и просунуть его туда, - 
предложил Питер.
\par- Мне это совсем не нравится, - ответила Серая Шкурка. - Хорошо 
тебе говорить. Ты можешь запросто подтянуть колени к подбородку и 
легко поместиться в коробке, а у меня нет коленей.
\par- Что ж, придется как-нибудь обмануть этого великана, - сказал 
Питер и обратился к Южному Ветру:
\par- А Волшебный Лист разве не поможет нам превратить этого великана 
в доброго?
\par- Помочь-то он поможет, - ответил Южный Ветер, - но тебе будет не 
просто заставить великана взять этот лист. Вспомни о его росте. А 
когда ты окажешься у него в коробке, так и подавно не сможешь это 
сделать. Тебе надо постараться поставить его в такое положение, чтобы 
он не смог отказаться от твоего подарка.
\par- Попробую, - сказал Питер.
\par- Да, попробуй, и по-моему, ты сумеешь это сделать. Конечно, 
сумеешь. Тот, кто дерзает и не сдается, всегда своего добивается. А 
теперь мне пора.
\parУдарил гром, блеснула молния, Южный Ветер снова оказался среди 
облаков и погнал их на север. Через несколько минут уже светило 
солнце.
\parИ Питер, и кенгуру проголодались. Мунлайт тоже нужен был отдых и 
еда. Серая Шкурка достала из сумки несколько охапок сена и бросила их 
лошадке, которая тут же принялась его жевать. Питеру Серая Шкурка 
достала кусок мяса и пирог с почками, себе - тоже немного сена.
\parЧерез час они собрались покинуть пещеру, чтобы продолжить 
путешествие. Питер подвел свою лошадку к выходу, но вдруг остановился. 
Две мощные ноги загородили ему путь.
\par- Великан! - прошептала Серая Шкурка, тоже увидевшая ноги. Они 
отступили в глубь пещеры и затаились. Было слышно, как великан топает 
снаружи. Он нюхал воздух и говорил сам с собой.
\par- Я чую, чую людей. Они наверняка прячутся где-то здесь. Следы, 
ведущие в пещеру, принадлежат лошади. А вот следы кенгуру.
\parБыло слышно, как великан сопит, опускаясь та землю. Его пальцы 
начали отдирать куски земли, чтобы увеличить отверстие, потом в пещеру 
просунулась огромная рука, она начала медленно шарить в пустоте. 
Могучие пальцы ощупывали потолок и стены пещеры, словно что-то 
отыскивая.
\parРаньше Питер часто засовывал руку в кроличьи норы и ловил 
кроликов. Теперь он понял, что должны были чувствовать кролики. Он 
приготовил и распрямил уложенный кольцами Громобой. Описав им над 
головой несколько кругов, Питер резко опустил кнут, так что он щелкнул 
прямо по большому пальцу великана, содрав с него кожу. Показалась 
кровь.
\parВеликан взвыл от боли и быстро убрал руку. Было слышно, как он 
сосет палец и шепчет: "Наверное, там змеи".
\parПотом стало слышно, что великан опускается на землю, чтобы 
заглянуть в пещеру, и вот уже его глаз загородил весь проем. Глаз был 
огромный, его ресницы напоминали камыши у берега озера. Питер мог бы 
выбить ему глаз одним ударом Громобоя, но он не мог заставить себя 
поступить так жестоко. Глаз, не мигая, смотрел прямо на них.
\par- Ага! - с удовлетворением произнес великан.
\parНо это радостное восклицание вдруг перешло в крик от боли. Это 
Питер, раскрутив кнут, ударил им великана по щеке, так что выступила 
кровь. Великан отпрянул, затем, судя по звукам, поднялся и поплелся к 
ручью.
\par- Он хочет промыть щеку, - сказала Серая Шкурка.
\par- Наверное, - согласился Питер, - но ручей отсюда недалеко. Мы 
пройдем в глубь пещеры и будем устраиваться на ночлег. Становится 
поздно, а в темноте вручить великану Волшебный Лист все равно 
невозможно.
\parОни крепко заснули в глубине пещеры. Когда рассвело и запели 
первые птицы, мальчик и кенгуру уже завтракали, собираясь продолжить 
путешествие.
\parСерая Шкурка волновалась.
\par- Побудь здесь, пока я огляжусь, - сказала она. Выбравшись на 
солнце, она остановилась, принюхалась, потом пропрыгала взад-вперед 
перед пещерой и вернулась.
\par- Кажется, все в порядке, - сказала она. - Пожалуй, можно рискнуть 
идти дальше.
\parПитер вскочил в седло, и они тронулись. Они уже почти прошли 
поляну, как вдруг услышали глухой удар и в испуге оглянулись. Великан 
Яррах спрыгнул со скалы, за которой он прятался всю ночь. Увидев 
беглецов, он захохотал от удовольствия. Великан отрезал им путь в 
пещеру, а спасаться в зарослях буша было бесполезно - великан догнал 
бы их, сделав всего несколько шагов. Так Питер и Серая Шкурка попали в 
ловушку. Великан подошел к ним, расставив руки и скрючив пальцы.
\par- Кто посмел войти в мой лес? - прогремел он. Схватив за корень 
древовидную акацию, он вырвал ее из земли и швырнул в Питера и Серую 
Шкурку. Дерево просвистело в воздухе, крутясь и переворачиваясь. 
Ударившись о землю недалеко от них, оно разлетелось на куски, щепки и 
ветки. Большая ветка пролетела над головой кенгуру. Серая Шкурка едва 
успела пригнуться, и ветка воткнулась в землю.
\par- Видишь? - крикнула она Питеру. - Он хотел убить нас. - Кенгуру 
не на шутку разозлилась. - Смотри за мной. Я ему сейчас покажу.
\parБольшими прыжками она поскакала к великану, каждый раз подпрыгивая 
на тридцать футов. Приблизившись настолько, что ее почти доставали 
скрюченные руки великана, она с огромной силой оттолкнулась от земли. 
Она взлетала все выше и выше, пока не допрыгнула до его груди, и тогда 
она вытянула вперед свои могучие задние лапы и со страшной силой 
ударила великана в грудь, причем ее пальцы глубоко вонзились в плотную 
ткань его рубашки. Оттолкнувшись от мощной груди великана, как от 
трамплина, она еще раз подпрыгнула и сделала в воздухе сальто. При 
этом ее тяжелый хвост, описав вокруг нее дугу, ударил великана по лицу 
с силой брошенного бревна.
\parОт удара голова великана качнулась назад. Он взвыл от боли и 
растопырил руки, пытаясь удержать равновесие, потерянное еще после 
первого удара в грудь, но не смог устоять на ногах и рухнул назад. Он 
повалился, как валится могучее дерево, и, ударившись огромной головой 
о вершину скалы, растянулся на лужайке лицом вверх. С вершины с 
грохотом скатились два валуна, выбитые со своих мест ударом мощного 
затылка. Сам великан неподвижно лежал на земле, оглушенный ударом.
\par- Надо сковать ему руки, пока он не пришел в себя, - крикнула 
Серая Шкурка.
\parПитер бросился ей помогать. Соскочив с седла, он схватился за 
гигантские наручники, которые Серая Шкурка как раз вытаскивала из 
своей сумки. Наручники со звоном упали на землю и тут же, на глазах, 
стали так быстро увеличиваться в размерах, что вскоре ни Питер, ни 
Серая Шкурка уже не могли их поднять и надеть на запястья великана, - 
такими тяжелыми они стали.
\parПитер посмотрел на могучую руку, лежащую в пяти ярдах от него. Она 
напоминала ствол дерева, и была такой же тяжелой. Ладонь была 
раскрыта, и пальцы торчали вверх, как шесты. По такой ладони Питер мог 
бы проехать на коне.
\parПоложение было безнадежным. Казалось, не было никакой возможности 
сковать великана, пока он не придет в себя, а до тех пор Волшебный 
Лист ему никак не вручить.
\par- Нам может помочь только один человек на свете, - сказал Питер, - 
крановщик. Он знает, как обращаться с тяжелыми грузами, и сумеет 
заковать великана в кандалы.
\par- А это идея! - воскликнула Серая Шкурка. Она засунула лапку в 
карман и рывком вытащила оттуда портового крановщика. Он был весь в 
муке: очевидно, его выхватили прямо из трюма.
\par- Как меня сюда занесло? - сердито заговорил он. - Почему какая-то 
кенгуру дергает меня за ногу?
\par- Сейчас некогда заниматься объяснениями, - ответила Серая Шкурка. 
- Ты вернешься на свой корабль, как только сделаешь для нас одно 
дельце. Вон видишь лежащего великана?
\par- Я в великанов не верю, - сказал крановщик. - И этот меня не 
убеждает.
\par- Посмотри, вон на земле рядом с тобой лежит кисть его руки. А 
дальше - его локоть.
\par- Это пластик, - упорствовал крановщик. - Вы, наверное, сделали 
этого великана для какого-нибудь представления или чего-то в этом 
роде.
\par- А ты дотронься до его руки, - не унималась Серая Шкурка.
\parКрановщик подошел к руке великана и дотронулся до нее.
\par- Разрази меня гром, - воскликнул он, - но этот пластик теплый!
\parВ это время великан глубоко вздохнул и снова замер.
\par- Что это? - спросил крановщик.
\par- Это великан просыпается, - ответил Питер. - Если ты не поспешишь 
надеть на него наручники, он всех нас убьет.
\par- Тогда за дело, - сказал крановщик. - Мне нужен кран.
\parСерая Шкурка вытащила из кармана маленький кран. Он начал быстро 
увеличиваться в размерах, пока его стрела не оказалась выше великана.
\par- Подними обе его руки и сложи их на груди, - скомандовала Серая 
Шкурка. - Но сначала подними туда меня.
\parКрановщик обвязал Серую Шкурку веревкой и поднял ее так, что она 
смогла дотянуться до рубашки великана и по ней залезть ему на грудь. 
Потом крановщик одну за другой поднял обе руки великана и сложил их на 
груди, после чего доставил туда же наручники, которые Серая Шкурка тут 
же защелкнула на гигантских запястьях.
\parТем же способом они сковали кандалами и ноги великана. Затем Серая 
Шкурка достала из сумки цепь и затянула ее петлей на его шее. Цепь 
сразу же стала увеличиваться в размерах и в весе. Свободный конец 
крановщик обмотал вокруг скалы, а Серая Шкурка продела в одно из 
звеньев огромный замок и закрыла его.
\parВеликан Яррах шевельнулся и попытался сесть.
\par- Мне пора, - сказал крановщик, прыгнул к Серой Шкурке в сумку и 
тотчас исчез. Кран стал уменьшаться и последовал туда же.
\parВеликан попытался освободиться.
\par- Откуда взялись эти наручники? - ревел он. - Снимите их с меня.
Он дергал руками в разные стороны, пытаясь разорвать наручники. Он 
хотел подняться, но цепь удерживала его в лежачем положении. Он дергал 
руками и ногами, вопил, бился о землю - и так до тех пор, пока силы не 
покинули его. Тогда он затих и только глубоко дышал.
\parЭтого-то момента Питер и дожидался. Он положил Волшебный Лист на 
ладонь великану, отскочил и стал смотреть. Великан сжал ладонь и взял 
Лист. И тут же с ним начали происходить перемены. Свирепое выражение 
его лица смягчилось, а когда он заговорил, то голос уже не вызывал ни 
у кого страха.
\par- Вы меня боитесь? - обратился он к Питеру.
\par- Да.
\par- Почему?
\par- Потому что ты пытался нас убить.
\par- Наверное, на меня что-то нашло. Я никогда больше никого не убью. 
Сейчас я хочу помочь вам, вот и все.
\par- А засунуть нас в маленькие коробки ты больше не хочешь, нет? - 
спросил Питер.
\par- С этим я покончил навсегда, - пообещал великан. - Пойдем вместе 
в замок и выпустим всех, кто там томится.
\parСерая Шкурка отомкнула наручники и кандалы, сняла с великана цепь. 
Он сел, потянулся, размял руки и ноги, затем медленно встал и 
осмотрелся.
\par- "Как же нам добраться до замка?" - пробормотал он про себя, а 
вслух предложил: - Прыгай на коня, Питер. Я вас всех понесу. Самим вам 
никогда на эти скалы не забраться.
\parКогда Питер вскочил на Мунлайт, великан наклонился, поднял их 
обоих и опустил в карман рубахи. Потом он поднял Серую Шкурку и 
опустил ее рядом с ними. Поднявшись на скалу, великан пошел по дороге, 
которая вела к его замку.
