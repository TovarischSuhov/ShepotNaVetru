 Глава 5
        ГРОЗА В ЗАМКЕ

     Великан шел, тяжело переваливаясь с ноги на ногу. Он был неуклюж 
и время от времени спотыкался. Его корпус раскачивался из стороны в 
сторону, и у Питера и Серой Шкурки возникло ощущение, что они плывут 
на корабле по бурному морю. Они сидели в кармане рубахи, сшитой из 
грубого, как брезент, и такого же прочного материала.
    Питер продолжал сидеть на Мунлайт, которая крепко упиралась всеми 
четырьмя копытами в дно кармана. Серая Шкурка стояла рядом, держась за 
кожаное стремя.
    Питер привстал в стременах, ухватился за верхний край кармана, и, 
оттянув его вниз, смог высунуть голову наружу. Вдали виднелся замок 
великана, вознесший к самому небу свои мощные башни и зубчатые стены. 
Деревья рядом с ним казались кустами.
    Когда они подошли к замку, Питер увидел ворота, к которым они как 
раз направлялись. Ворота имели форму арки, причем эта арка была такой 
огромной, что вполне могла бы служить мостом через морской пролив.
    Питер отцепился от верхней кромки кармана и опустился в седло.
    - Мы подходим к такому огромному замку, какого я в жизни не 
встречал, - сказал он Серой Шкурке. - Когда ты его увидишь, ты не 
поверишь, что это замок.
    - Я уже видела, каков великан Яррах, и теперь поверю чему угодно.
    Пока они так разговаривали, великан вошел в замок, прошел по 
коридору на кухню, где достал из кармана своих пассажиров и опустил на 
пол. Потом он сунул в карман Волшебный Лист, который держал в руке.
    Помещение, в котором они оказались, было похоже на огромную 
пещеру. Как темное небо, раскинулся над ними сводчатый потолок. 
Кухонная плита была размером с дом, в ее топке горели стволы деревьев. 
На полке, сложенной из огромных кусков скалы, обтесанных и уложенных 
как кирпичи, стояли гигантские кастрюли и чайники.
    Мебель в кухне была под стать огромному росту великана. Ножки 
стульев были намного выше Питера, хотя он все еще сидел верхом. Стол, 
сколоченный из струганных стволов деревьев, стоял на ножках толщиной в 
три фута.
    Питер слез с Мунлайт и привязал поводок к одной из этих ножек.
    - Посидите на стуле, пока я готовлю вам еду, - сказал великан. - Я 
хочу потушить говядину с картошкой, и уже припас необходимые продукты 
- двух отличнейших быков и полтонны картошки.
    Серая Шкурка брезгливо поморщилась.
    - Обо мне не беспокойся, - сказала она. - Мне хватит охапки травы.
    - Хорошо, - сказал великан. - Я брошу охапку в жаркое.
    Серой Шкурке чуть было не стало дурно.
    - Я предпочитаю траву в свежем виде, - объяснила она. - Хорошо бы 
только обрызнуть ее холодной водой.
    - А мне жаркого совсем чуть-чуть, - попросил Питер, аппетит 
которого стал понемногу пропадать.
    Питер был бы не прочь сесть на стул, но он просто не мог на него 
забраться. Серая Шкурка догадалась, о чем он думает. Она достала из 
сумки небольшую лесенку и прислонила ее к ножке стула. По ней они оба 
поднялись на сиденье, обитое красным плюшем. Им показалось, что они 
погрузились в шерстяное море или в высокую траву.
    - Этот стул пора косить, - сказала Серая Шкурка. - Мне так и 
хочется достать из сумки газонокосилку и хорошенько здесь пройтись.
    - Смотри, - прошептал Питер. - Великан помешивает жаркое. Меня от 
этого тошнит.
    Великан поставил на плиту огромный котел и помешивал в нем 
деревянным черпаком. Скоро из котла начали подниматься клубы пара. 
Вытяжной трубы в кухне не было, и пар собирался в тучи под сводами 
потолка. Постепенно облачный покров окутал весь потолок, и тучи стали 
яростно носиться по кухне взад-вперед, ища выход.
    Великан Яррах с тревогой посмотрел наверх.
    - Похоже, погода у нас на кухне портится, - объявил он. - В 
комнате сухо, но мы не можем перейти туда, пока мясо не дотушится. 
Правда, - добавил он, - дождя, по-моему, не будет.
    Грянул гром. Замок содрогнулся. Ударившись друг о друга, зазвенели 
чашки на кухонном шкафу. Одновременно с громом блеснул зигзаг молнии и 
ударил в каменную полку рядом с великаном. От удара вывалился один из 
огромных камней, сверкнуло, затем погасло голубое пламя.
    Питер испугался, что молния может попасть в стул, на котором они с 
Серой Шкуркой сидели, и крикнул великану: "У тебя в кухне есть 
громоотвод?"
    - Нет, - ответил великан. - Я никак не ожидал, что из-за моего 
жаркого начнется гроза. Но громоотвод сейчас будет.
    Он отошел от плиты и крикнул в коридор:
    - Джордж, по-моему, я опять вызвал грозу! Ты можешь сейчас же 
придти сюда?
    - Конечно, - ответил низкий голос из какой-то дальней комнаты. - 
Уже иду.
    Яррах вернулся к котлу и гигантской ложкой попробовал жаркое. "Еще 
не вполне готово", - пробормотал он и снова принялся помешивать в 
котле, отчего вверх метнулись новые клубы пара.
    Из коридора послышались тяжелые шаги, и в кухню вошел великан 
среднего роста, в парадном вечернем костюме. Он был в белых резиновых 
перчатках и держал длинный медный стержень с тремя разветвлениями на 
концах.
    - Это Джордж, - представил его Яррах. - Он и есть мой громоотвод. 
Когда я занимаюсь стряпней, он всегда готов придти и немного 
поотводить молнии.
    Джордж поклонился, улыбаясь. Гром загрохотал как барабанная дробь, 
и Джордж стал размахивать ему в такт своим медным стержнем, как 
дирижерской палочкой. Когда из туч под потолком ударила молния, Джордж 
успел поймать ее медным стержнем. Пройдя через стержень, молния с 
треском вышла с другого конца, вылетела за дверь, пронеслась по 
коридору, повернула за угол и, оказавшись на свободе, уткнулась в 
землю.
    Джордж, улыбаясь, повернулся и отвесил поклон Питеру и Серой 
Шкурке, которые смотрели на него с изумлением.
    - Как ты думаешь, надо ему похлопать? - спросила Серая Шкурка.
    - Похоже, он этого ждет, - ответил Питер.
    Они оба захлопали в ладоши, и Джордж поклонился еще раз.
    {whisp02.gif}
    Вдруг блеснула новая молния и опалила Джорджу брюки. Он быстро 
повернулся и стал с бешеной скоростью отводить молнии, которые 
посыпались одна за другой. Джордж рывками переводил свой инструмент из 
стороны в сторону, перехватывая молнии, которые, не касаясь пола, 
вылетали в дверь, проносились по коридору и, повернув за угол, уходили 
в землю.
    - Как он вам нравится? - прокричал великан Яррах, перекрывая 
грохот грома. - Я еще не встречал никого, кто отводил бы молнии лучше 
Джорджа.
    От этой похвалы Джордж пришел в полный восторг. Гроза затихала. В 
качестве заключительного аккорда Джордж взмахнул своим мерным стержнем 
и одним широким движением поймал две молнии сразу. Они голубым 
пламенем блеснули на конце стержня и исчезли в коридоре.
    Упало несколько капель дождя. Если Джордж чего и боялся на свете, 
так это намочить свой вечерний костюм. Он посмотрел на небо, 
повернулся и побежал.
    - Держитесь! - крикнул он Питеру и Серой Шкурке. - Сейчас хлынет 
ливень.
    Он промчался по коридору с такой скоростью, что, наверное, обогнал 
бы молнию, хлопнул дверью своей комнаты и заперся изнутри.
    Из туч хлынули целые потоки. Вода лилась на плиту, в котел с 
мясом, в чашки на кухонном шкафу. Сбегая со стола, она образовывала 
четыре водопада. Вскоре по всему полу уже несся бурный поток, 
стремившийся к выходу.
    - Мунлайт утонет! - вскричал Питер. - Она привязана к ножке стола.
    Он быстро полез вниз по лестнице. Серая Шкурка бросилась за ним. 
Вода на полу уже доходила Питеру до пояса, и ему пришлось бороться с 
течением, чтобы добраться до лошади, которая стояла рядом с одним из 
водопадов. Под столом вода неслась особенно быстро, и Мунлайт было 
очень трудно удержаться на месте.
    Питер прошел сквозь льющийся со стола водопад и на секунду 
прижался к ножке стола, чтобы перевести дыхание. Поток воды несся мимо 
него, разбивался о другую ножку на два потока и снова сливался вместе 
под кухонным шкафом. В воду падали смытые с открытых полок кастрюли и 
миски и тоже неслись к двери, словно какой-то затейливый флот.
    Питеру и Серой Шкурке удалось добраться до Мунлайт, которую поток 
воды уже сбил с ног. Она погружалась в воду и била передними ногами, 
отчаянно пытаясь разорвать поводок. Серая Шкурка достала из сумки нож 
и, перерезав поводок, освободила лошадь.
    Теперь они плыли все. Сверху гремел голос великана: "Где вы? Я 
спасу вас. Я не дам вам утонуть". Волшебный Лист настолько изменил 
его, что он думал о безопасности своих друзей больше, чем о 
собственной. "Где вы? Где?" - не переставая кричал он.
    Питера и его друзей течение унесло под стол, и там великан не мог 
их увидеть. Ему самому приходилось несладко. Огонь в топке погас, и 
великан все время натыкался на разные затопленные предметы. Он 
рассадил себе ногу о помойное ведро и ударился головой о прибитую к 
стене кофемолку. Несколько раз он спотыкался и падал, сердито ругая 
эту проклятую кухонную погоду.
    Питер окликнул его, но крик потонул в реве водяного потока. Вода 
носила Питера и его друзей по всей кухне, швыряла из стороны в 
сторону, мчала по каким-то речным порогам. Вдруг они очутились в 
гигантском водовороте и стали вращаться по огромному кругу, который с 
каждым витком становился все меньше, приближаясь к центру воронки, 
откуда доносилось ужасающее бульканье.
    - Мы в раковине! - крикнул Питер, сообразив, что бульканье 
исходило из сливного отверстия. Тут волна отбросила их к краю 
раковины, и Мунлайт, зацепившись за него своими мощными ногами, сумела 
выкарабкаться сама и вытащить за собой и Питера, и Серую Шкурку, 
которые ухватились за кожаные стремена. Они спрыгнули в воду по другую 
сторону раковины и были подхвачены течением. Оно потащило их к двери, 
куда со всех сторон стекалась вода, устремившаяся в коридор.
    Теперь это был даже не поток, а могучая река, тяжелой массой 
переливающаяся через порог кухни. Мунлайт, высоко задрав голову и 
отфыркиваясъ, стремительно протащила Питера и Серую Шкурку через 
водоворот и увлекла в коридор. Подгоняемые разъяренной рекой, они 
ринулись вперед и уже через мгновение оказались выброшенными на ровную 
землю, по которой вода растекалась и исчезала в траве, пробивая себе 
путь к ручью.
    Серая Шкурка встряхнулась, освобождаясь от воды, которая пропитала 
ее шубку. Питер промок насквозь, но, едва его коснулись солнечные 
лучи, мгновенно высох. На нем появилась голубая бархатная куртка с 
золотыми пуговицами, на голове снова оказалась шляпа с перьями, на 
ногах - сапоги, сухие и начищенные, хотя Питер и опасался, что вода 
может испортить их.
    Такое изменение внешнего вида Питера нисколько не удивило Серую 
Шкурку.
    - Тебе надо еще обзавестись новыми брюками, - сказала она, оглядев 
его. - Эти старые портят весь вид. Тебе предстоит еще долгий путь, 
прежде чем ты станешь принцем.
    Ее речь прервало появление великана, который вышел из замка, 
выжимая бороду. В его ботинках хлюпала вода. Он стал извиняться за 
причиненное беспокойство. Он жалел о погубленном мясе и хотел 
приготовить что-нибудь другое.
    - Не беспокойся о нас, - сказала Серая Шкурка. - Нашу еду я ношу в 
своей сумке. Уже становится темно, и нам пора идти.
    - Но прежде чем мы уйдем, - сказал Питер, - ты должен освободить 
всех, кого засунул в коробки. Теперь, когда ты стал добрым великаном, 
ты не должен больше никого захватывать. Ты должен помогать всем, кто 
идет этой дорогой, а не пытаться делать всех одинаковыми.
    - Я хотел освободить всех сразу после обеда, - сказал великан 
Яррах. - Мы сделаем это сейчас. Идите за мной.
    Он подошел к маленькой двери в стене замка. Сразу за ней вниз 
уходила узкая лестница. Она привела в огромное подземелье, где рядами 
стояли коробки. Мальчики, девочки, мужчины и женщины были втиснуты в 
них в самых неудобных позах. Увидев Питера и Серую Шкурку, пленники 
стали кричать: "Помогите нам! Вызволите нас отсюда!"
    - Сейчас я вас освобожу, - сказал великан. - Потом я всех накормлю 
и выведу отсюда.
    Питер и кенгуру помогали великану открывать ящички. Они бегали от 
одного ящичка к другому, отодвигали задвижки и помогали людям ступить 
на землю. Сначала люди не могли ни согнуться, ни разогнуться, словно 
они одеревенели, и им пришлось долго махать руками и топать ногами, 
прежде чем они почувствовали, что снова могут ходить. Великан вывел их 
по лестнице на воздух. Люди смотрели на солнце, на облака, на вершины 
деревьев, и мысль о свободе возвращала румянец на их бледные щеки.
    - Мы должны вас сейчас покинуть, - сказал Питер. - Нам еще 
предстоит долгий путь. Великан Яррах позаботится о вас. Он накормит 
вас и проводит.
    - Да, - подтвердил великан. - Вам больше незачем меня бояться.
    Питер вскочил на Мунлайт и обернулся, чтобы помахать на прощанье 
великану и стоящим вокруг него людям.
    - Прощайте! - крикнул он.
    - Прощайте! - ответили они.
    - Будь осторожен! - добавил великан Яррах. Питер тронул поводья, и 
Мунлайт потрусила прочь, сопровождаемая Серой Шкуркой.
