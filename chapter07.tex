Глава 7
\parПИТЕР И КОЛДУНЬЯ ЛЕТЯТ НА ЛУНУ
\par\parКогда Питер и колдунья собрались лететь, на небе висела полная 
Луна. Они стояли у порога хижины и смотрели вверх, на светящийся 
лунный диск, по которому им вскоре предстояло прогуляться.
\par- Отсюда Луна выглядит чистой, - сказала колдунья. - Но когда ты 
попадешь туда, ты обомлеешь, - столько там мусора. Просто ужас!
\par- На Луне уже побывали люди, - сказал Питер.
\par- Знаю, знаю. После них я целую неделю наводила порядок. А ты бы 
видел, как этот парень Армстронг делал первый шаг! Смех! Я чуть живот 
не надорвала!
\parКолдунья села верхом на метлу, Питер пристроился сзади. Он 
вцепился в ее черный плащ и закрыл глаза. Тонкими, костлявыми пальцами 
колдунья сжала метлу, и она ракетой взвилась в небо. Земля таяла 
позади, по скорости они не замечали. Вскоре они почувствовали 
невесомость, и сидеть на метле стало легко.
\par- Сегодня метла хорошо летит, - сообщила колдунья. - Пожалуй, с 
двумя пассажирами на борту она идет лучше. Мы уже превысили вторую 
космическую скорость, семь миль в секунду, но я разгоню ее до 240 
тысяч миль в час. Значит на Луне мы будем через 55 минут. Держись 
крепче!
\parМетла действительно летела быстрее. Лунный диск впереди быстро 
увеличивался.
\parВдруг колдунья так резко вильнула в сторону, что Питер чуть не 
слетел с метлы. Мимо них промелькнул какой-то космический корабль с 
освещенными иллюминаторами.
\par- Куда тебя несет? Смотреть надо! - закричала колдунья. Потом 
обернулась к Питеру и сказала:
\par- Хорошенькое дело! Этот дурак чуть меня не сбил. Сейчас я ему 
покажу!
\parОна резко бросила метлу в вираж, развернулась и погналась за 
кораблем, летавшим вокруг земли. Сманеврировав, она пристроилась прямо 
к его иллюминатору. Из-за толстого стекла на них ошарашенно смотрел 
человек.
\par- Эй, балбес, права-то у тебя есть? - вопила колдунья, радуясь, 
что снова может быть грубой. - Ты нас чуть не сбил. Коляской тебе надо 
управлять, детской коляской - и ничем больше!
\parКосмонавт смотрел на них широко раскрытыми от изумления глазами. 
Он пытался что-то сказать в микрофон, но не мог выдавить ни слова. 
Вдруг он заговорил с необыкновенной скоростью, словно хотел сообщить 
что-то необычайно срочное.
\par- Земля, Земля! Прием.
\par- Что случилось? - спросил голос из приемника.
\par- В иллюминатор на меня смотрят ведьма и мальчик. Они верхом на 
метле. Прием.
\par- Ты болен. Повторяю: ты болен. Слишком устал. Подбавь кислорода. 
Потяни рычаг Б, потом задвинь его назад. Выпей кофе из термоса 3-а. 
Сфотографируй ведьму и немедленно отправь фотографию нам. Повторяю: ты 
нездоров. Возьми упаковку таблеток на полке слева. Прими сразу две 
таблетки. После этого ты уснешь. Повторяю: ты болен.
\parКолдунья постучала рукой по стеклу.
\par- Не обращай на них внимания! - крикнула она космонавту. - Ты 
вполне здоров.
\parНо космонавт уже проглотил две таблетки и заснул.
\par- Готов, - сказала колдунья. - Теперь он проспит несколько 
часов... - Она заглянула в иллюминатор. - Что там перед ним в термосе? 
Не кофе ли?
\par- Наверное.
\par- Как ты думаешь, для космонавта будет большая потеря, если мы его 
выпьем? Вон сверху люк. Через него мы могли бы легко попасть внутрь.
\par- Лично я не отказался бы от чашечки кофе.
\parКолдунья подогнала метлу к люку корабля. Там они совершили 
посадку, открыли люк и быстро скользнули вниз, причем колдунья 
прихватила с собой метлу. Питер закрыл люк. Колдунья налила две 
чашечки кофе из термоса и передала одну Питеру. Они уселись и 
принялись пить маленькими глотками.
\par- Хороший кофе у них в Америке, - сказала колдунья. - Но нам надо 
спешить: на Луне еще уйма работы.
\par- Бедняга этот парень, - заговорил Питер. - Ты думаешь, на Земле 
ему кто-нибудь поверит, что он видел в иллюминаторе ведьму и мальчика 
верхом на метле?
\par- Конечно, не поверят. Ему скажут, что все это ему приснилось или 
просто он все это выдумал. Когда ты встречаешь что-то удивительное и 
рассказываешь об этом другим, тебе никто не верит. Если же ты хочешь, 
чтобы тебе верили - говори неправду. Скажи, что во время путешествия 
не случилось ничего интересного, и тебе поверят.
\par- Может быть, мы оставим ему записку, подпишемся и приколем к его 
куртке?
\par- Давай, только побыстрее. У тебя есть бумага и ручка?
\par- Я вырву листок из его блокнота, - сказал Питер. Блокнот и ручка 
лежали рядом.
\parПитер взял листок и написал: "Удостоверяем, что ведьма и мальчик 
посетили этот корабль и выпили кофе космонавта".
\parОн передал листок колдунье: "Подписывай!" Та нацарапала внизу свое 
имя, Питер подписался тоже. Они прикололи листок к куртке космонавта и 
вылезли через люк наружу.
\par- Хотелось бы мне видеть его лицо, когда он проснется и прочтет 
записку, - мечтательно произнесла колдунья.
\parОни снова сели верхом на метлу.
\par- Оттолкнись ногами от корабля, когда я дам знак, - сказала 
колдунья. - Это придаст нам начальное ускорение. - Она уселась 
поудобней, скомандовала: "Давай!", Питер оттолкнулся, и они взлетели.
\parНекоторое время метлу водило из стороны в сторону, но потом она 
понеслась ровно, как машина. Колдунья гнала свою метлу быстрее, чем 
обычно. Они мчались с такой скоростью, что казалось, будто Луна 
несется им навстречу. Скоро уже можно было различить ее высокие крутые 
горы и огромные сухие моря. Колдунья и Питер ринулись вниз с 
безоблачного неба и нырнули в огромный кратер, на дне которого лежало 
множество камней.
\parТам они остановились. Колдунья слезла с метлы и принялась яростно 
подметать. Она двигалась с такой скоростью, что почти исчезла из вида. 
Питер понял, что помочь ей ничем не сможет и принялся бродить вокруг, 
перепрыгивая через скалы и перемахивая через расселины. На Луне это 
оказалось очень легко, ведь его вес не превышал там нескольких фунтов
\parОн услышал, как колдунья, стоя где-то внизу на ровной площадке, 
зовет его. Оказывается, она обнаружила кинокамеру, которая 
поворачивалась на 180 градусов туда и обратно.
\par- Давай позировать, - объявила она Питеру, как только он 
спустился. - Становись рядом, и когда камера повернется к нам, говори 
"Сы-ыр", и растягивай губы в стороны. Тогда получится, будто мы 
улыбаемся.
\parКамера медленно поворачивалась.
\par- Давай! - скомандовала колдунья.
\par- Сы-ыр! - сказали они разом.
\par{whisp04.gif}
\par- То-то на земле будут ошарашены, - сказала колдунья. - Это у них 
станет главной сенсацией дня. А камеру мы оставим здесь. У меня их и 
так скопилось слишком много. Слышишь попискивание? Это, наверное, наше 
изображение передается на землю.
\parВзяв метлу, колдунья подмела вокруг, передвинула несколько камней, 
чтобы был опрятный вид. Потом села и принялась копаться в метле, 
готовя ее к полету.
\par- Сегодня мы опаздываем, - объяснила она, - и мне хотелось бы в 
этом полете установить рекорд. Если замерзнешь, обмотайся моим плащом. 
Стартовать, пожалуй, будем вон с той скалы, - она указала на 
остроконечную вершину, которая возвышалась над ними.
\parОни легко полезли наверх, - ведь на Луне их вес уменьшился, - и 
уже через несколько минут вскарабкались на вершину. Питеру не давала 
покоя мысль о камнях, которые привозили с Луны астронавты, и он сунул 
в карман горсть камешков, что валялись под ногами. "Раздам их 
друзьям", - подумал он.
\par- Послушай, - обратился он к колдунье, - а почему Армстронг и 
Олдрин не могли ходить по Луне нормально? Им надо было всего лишь 
набить карманы свинцом. Поехали!
\parНе успели они оторваться от скалы, как метла провалилась вниз, 
нырнула в широкую впадину и понеслась над одним из сухих морей. Питер 
видел, как внизу, на ровной поверхности, громоздились скалы. В глубине 
одной из них был виден яркий красный отсвет: там горел вулкан.
\parМетла разогналась, повернула к Земле и вскоре достигла скорости 
полутора миль в секунду, необходимой, чтобы преодолеть притяжение 
Луны. Скорость ее нарастала, пока не остановилась на какой-то 
постоянной цифре. К метле был привинчен спидометр, и колдунья часто с 
ним сверялась.
\par- Сейчас мы даем 250 тысяч миль в час, - сообщила она. - Плюс-
минус полмили.
\parПитер стал замерзать. Он замотался в плащ колдуньи и закрыл глаза.
\parВдруг он услышал, как она крикнула: "Держись!" Метла стала 
дергаться, словно брыкающаяся лошадь. Хорошо еще, что Питер был 
неплохим наездником, а то его бы сбросило.
\par- Мы входим в земную атмосферу, - объяснила колдунья, - а скорость 
слишком большая. Приходится тормозить, иначе сгорим. Вот мы и 
подпрыгиваем, как камешек на поверхности пруда.
\parПереднюю часть метлы покрыла полоска голубого пламени. Огонь 
опалил брови колдуньи. Питер чувствовал вокруг себя жар, но он крепко 
прижался к спине колдуньи, и огонь миновал его. Метла прыгала подобно 
дельфину.
\par- Мне придется сделать несколько витков вокруг Земли, чтобы сбить 
скорость! - крикнула колдунья.
\parОни обогнули Землю три раза и, когда погасили скорость, колдунья 
расстегнула пуговицу на плаще, который раздулся позади них, словно 
парашют.
\par- Вот коридор. Мы спускаемся.
\parМетла, как метеор, пронзила атмосферу. Позади нее тянулся длинный 
огненный след. Колдунья промчалась над крышей хижины, блестяще 
выполнила петлю и приземлилась у порога.
\par- Вот что получается от увлечения скоростью, - сказала она, ступая 
на Землю. - Никогда не думала, что так получится, больше я этого 
делать не буду. Желание пофасонить дорого обошлось мне: я сожгла верх 
шляпы и опалила брови. Тебе еще повезло, что ты сидел сзади.
\par- Жар чувствовался и там, - сказал Питер. - Мне обожгло палец, он 
и сейчас немного болит.
\parНавстречу из хижины выпрыгнула Серая Шкурка.
\par- Я подумала, что это падает комета, - сказала она. - Что 
случилось?
\par- Тормоза отказали, - объяснила колдунья.
\par- Эта метла для космоса не годится, - сказала Серая Шкурка. - Надо 
ее поменять на новую модель.
\par- Чепуха! - бросила колдунья. Она вошла в хижину, где уже был 
готов чай.
\parРассвет только разгорался, небо на востоке порозовело.
\par- Надо трогаться в путь, как только попьем чаю, - сказал Питер.
\par- А куда вы торопитесь? - забеспокоилась колдунья. - Можете 
оставаться у меня сколько хотите, вы же знаете.
\par- Я ищу Прекрасную Принцессу, - объяснил Питер. - Путь еще очень 
длинный, нам надо поторапливаться. Я надеюсь, мы с тобой когда-нибудь 
встретимся.
\par- Я тоже надеюсь. - Колдунья выглядела сейчас добродушной пожилой 
женщиной, так сильно изменил ее Волшебный Лист.
\par- А ты когда-нибудь слышала о Прекрасной Принцессе? - обратился к 
ней Питер.
\par- Нет. В своей жизни я встретила только одну принцессу. Она 
пыталась столкнуть меня с подоконника, поэтому я сорвала у нее с 
головы корону и забросила в озеро. И она не сможет выйти замуж, пока 
не найдет корону.
\par- Как это жестоко!
\par- Я же тогда была злой колдуньей и ужас что вытворяла. Это было не 
так давно, но сейчас я стала совсем другой. Я никогда не буду делать 
ничего подобного. Если же я что-нибудь узнаю о твоей Прекрасной 
Принцессе, я дам тебе знать.
\parНа прощанье Питер поцеловал колдунью в щеку и вдруг почувствовал, 
что бархатная куртка снова на нем. Серая Шкурка тоже ее заметила и 
улыбнулась.
\par- Мунлайт привязана неподалеку, - сообщила она.
